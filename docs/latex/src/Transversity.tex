%% LyX 2.0.3 created this file.  For more info, see http://www.lyx.org/.
%% Do not edit unless you really know what you are doing.
\documentclass[twoside,english]{paper}
\usepackage{lmodern}
\renewcommand{\ttdefault}{lmodern}
\usepackage[T1]{fontenc}
\usepackage[latin9]{inputenc}
\usepackage[a4paper]{geometry}
\geometry{verbose,tmargin=3cm,bmargin=2.5cm,lmargin=2cm,rmargin=2cm}
\usepackage{color}
\usepackage{babel}
\usepackage{float}
\usepackage{bm}
\usepackage{amsthm}
\usepackage{amsmath}
\usepackage{amssymb}
\usepackage{graphicx}
\usepackage{esint}
\usepackage[unicode=true,pdfusetitle,
 bookmarks=true,bookmarksnumbered=false,bookmarksopen=false,
 breaklinks=false,pdfborder={0 0 0},backref=false,colorlinks=false]
 {hyperref}
\usepackage{breakurl}
\usepackage{mathrsfs}

\makeatletter

%%%%%%%%%%%%%%%%%%%%%%%%%%%%%% LyX specific LaTeX commands.
%% Because html converters don't know tabularnewline
\providecommand{\tabularnewline}{\\}

%%%%%%%%%%%%%%%%%%%%%%%%%%%%%% Textclass specific LaTeX commands.
\numberwithin{equation}{section}
\numberwithin{figure}{section}

%%%%%%%%%%%%%%%%%%%%%%%%%%%%%% User specified LaTeX commands.
\usepackage{babel}

\@ifundefined{showcaptionsetup}{}{%
 \PassOptionsToPackage{caption=false}{subfig}}
\usepackage{subfig}
\makeatother

\begin{document}

\title{Transversity Distributions}

\maketitle

\section{Perturbative evolution}

In this section we discuss the structure of the DGLAP evolution
equations for the transversity distributions. The same structure holds
for both PDFs and FFs. Therefore we first discuss the structure of the
evolution equations in terms of distributions in the so-called
``evolution'' basis and then report the splitting functions up to
$\mathcal{O}(\alpha_s^2)$, \textit{i.e.} next-to-leading order (NLO),
separately for PDFs and FFs. Contrary to unpolarised and
longitudinally polarised collinear distributions, no transversely
polarised gluon distribution exists. This simplifies the structure of
the evolution equations that, when written in the evolution basis, are
completely decoupled. 

As a first step we define the evolution basis. Given a set of quark
distributions in the more familiar ``physical'' basis, \textit{i.e.}
$\{\overline{t},\overline{b},\overline{c},\overline{s},\overline{u},\overline{d},d,u,s,c,b,t\}$,
and defining $q^{\pm}\equiv q\pm \overline{q}$, the ``evolution''
basis is defined as follows:
\begin{equation}
\begin{array}{rcl}
\Sigma&=& \sum_{q}q^+\,,\\
V &=& \sum_{q}q^-\,,\\
T_3&=& u^+-d^+\,,\\
V_3&=& u^--d^-\,,\\
T_8&=& u^++d^+-2s^+\,, \\
V_8&=& u^-+d^- -2s^-\,, \\
T_{15}&=& u^++d^++s^+-3c^{+}\,, \\
V_{15}&=& u^-+d^- +s^--3c^{-}\,, \\
T_{24}&=& u^++d^++s^++c^{+}-4b^+\,, \\
V_{24}&=& u^-+d^- +s^-+c^{-}-4b^-\,, \\
T_{35}&=& u^++d^++s^++c^{+}+b^+-5t^{+}\,, \\
V_{35}&=& u^-+d^- +s^-+c^{-}+b^--5t^{-}\,.
\end{array}
\end{equation}
It is possible to show that in this basis the general form of the
DGLAP evolution equations for the transversity distribution reduces to
the following set of \textit{decoupled} integro-differential equation:
\begin{equation}\label{eq:evoleqs}
\begin{array}{rcl}
\displaystyle \mu^2\frac{d\Sigma}{d\mu^2} &=& \displaystyle P_{qq}\otimes \Sigma\,,\\
\\
\displaystyle \mu^2\frac{dV}{d\mu^2} &=& \displaystyle P^{V}\otimes V\,,\\
\\
\displaystyle \mu^2\frac{dT_i}{d\mu^2} &=& \displaystyle P^+\otimes T_i\,,\\
\\
\displaystyle \mu^2\frac{dV_i}{d\mu^2} &=& \displaystyle P^-\otimes V_i\,,\\
\\
\end{array}
\end{equation}
with $i=3,8,15,24,35$ and where the Mellin convolution symbol
$\otimes$ is defined as:
\begin{equation}
f(x)\otimes g(x)\equiv \int_x^1\frac{dy}{y}f(y)g\left(\frac{x}{y}\right)=\int_x^1\frac{dy}{y}f\left(\frac{x}{y}\right)g(y)\,.
\end{equation}
The splitting functions $P_{qq}$, $P^V$, $P^+$, and $P^-$ are usually
decomposed as follows:
\begin{equation}
\begin{array}{l}
\displaystyle P^\pm \equiv P_{qq}^V \pm P_{q\overline{q}}^V\,, \\
\\
\displaystyle P_{qq} \equiv P^+ + n_f (P_{qq}^S + P_{q\overline{q}}^S)\,,\\
\\
\displaystyle P^V \equiv P^- + n_f (P_{qq}^S - P_{q\overline{q}}^S)\,,
\end{array}\,,
\end{equation}
where $n_f$ is the number of active flavours at a given scale $\mu$
and the splitting functions $P_{qq}^V$, $P_{q\overline{q}}^V$,
$P_{qq}^S$, $P_{q\overline{q}}^S$ have the usual perturbative
expansion:
\begin{equation}
P(x,\mu)=\sum_{n=0}\left(\frac{\alpha_s(\mu)}{4\pi}\right)^{n+1}P^{(n)}(x)\,.
\end{equation}
Given the expansion above, one can show that at
$\mathcal{O}(\alpha_s)$, \textit{i.e.} leading order, all coefficients
but $P_{qq}^{V,(0)}$ vanish. It is then easy to see that:
\begin{equation}
P_{qq}^{(0)} = P^{V,(0)} = P ^{+,(0)}= P^{-,(0)}=P_{qq}^{V,(0)}\,.
\end{equation}
This means that the evolution equations in Eq.~(\ref{eq:evoleqs}) have
all the same evolution kernel.

If one wants to include NLO corrections, one finds that the
$\mathcal{O}(\alpha_s^2)$ coefficients $P_{qq}^{V,(1)}$ and
$P_{q\overline{q}}^{V,(1)}$ are different from zero while
$P_{qq}^{S,(1)}$ and $P_{q\overline{q}}^{S,(1)}$ vanish. This
immediately implies that $P_{qq} = P^+$ and $P^V = P^-$. Therefore, at
NLO, the evolution equations are fully determined by the functions
$P_{qq}^{V,(0)}$, $P_{qq}^{V,(1)}$, and
$P_{q\overline{q}}^{V,(1)}$. We are now in the position to discuss the
specific expressions of these functions for both PDFs and FFs. A
further simplification is given by the fact that the function
$P_{qq}^{V,(0)}$ is the same for both PDFs and FFs. However, this is
no longer the case for $P_{qq}^{V,(1)}$ and
$P_{q\overline{q}}^{V,(1)}$ whose form differs between PDFs and FFs.

In order to carry out the implementation of the splitting functions in
{\tt APFEL}, it is necessary to make sure that the expressions of the
single coefficients of the perturbative expansions have the following
structure:
\begin{equation}\label{eq:splittingfuncs}
P(y) = R(y) + S \left(\frac1{1-x}\right)_++ L\delta(1-y)\,,
\end{equation}
where $R$ is a regular function in $x=1$, and $S$ and $L$ are
numerical coefficients. The plus prescription used above has the
following definition upon integration with a test function $f$:
\begin{equation}
\int_0^1dy\left(\frac1{1-y}\right)_+f(y)\equiv \int_0^1dy\frac{f(y)-f(1)}{1-y}\,.
\end{equation}
An important detail to notice is that the definition above is strictly
true only if the lower integration bound is equal to zero. In actual
facts, this is never the case because Mellin convolutions involving
plus-prescripted functions have the following structure:
\begin{equation}
\int_x^1dy\left(\frac1{1-y}\right)_+f(y)\,.
\end{equation}
with $0<x<1$. This integral can be manipulated as follows:
\begin{equation}
\begin{array}{rcl}
\displaystyle \int_x^1dy\left(\frac1{1-y}\right)_+f(y) &=&
                                                           \displaystyle
                                                           \int_0^1dy\left(\frac1{1-y}\right)_+f(y)
                                                           -
                                                           \displaystyle
                                                           \int_0^xdy\left(\frac1{1-y}\right)_+f(y)\\
\\
&=& 
                                                           \displaystyle
                                                           \int_0^1dy \frac{f(y)-f(1)}{1-y}
                                                           -
                                                           \displaystyle
                                                           \int_0^xdy\frac{f(y)}{1-y}\\
\\
&=& 
                                                           \displaystyle
                                                           \int_x^1dy \frac{f(y)-f(1)}{1-y}
                                                           -
                                                           \displaystyle
                                                           f(0)\int_0^x\frac{dy}{1-y}\\
\\
&=& 
                                                           \displaystyle
                                                           \int_x^1dy \left(\frac{1}{1-y}\right)_{\oplus}f(y)
                                                           +f(0)\ln(1-x)\\
\\
&=& 
                                                           \displaystyle
                                                           \int_x^1dy \left[\left(\frac{1}{1-y}\right)_{\oplus}+\ln(1-x)\delta(1-y)\right]f(y)\,,
\end{array}
\end{equation}
where we have defined a ``generalised'' plus prescription that holds
in its form regardless of the lower integration bound:
\begin{equation}
\int_x^1dy\left(\frac1{1-y}\right)_\oplus f(y)\equiv \int_x^1dy\frac{f(y)-f(1)}{1-y}\,.
\end{equation}
Therefore, a Mellin-like convolution of the splitting function in
Eq.~(\ref{eq:splittingfuncs}) with the test function $f$ will take the
form:
\begin{equation}
\int_x^1 dy P(y) f(y) = \int_x^1 dy \left[ R(x)
  +S\left(\frac1{1-y}\right)_\oplus + \left(L+S\ln(1-x)\right)\delta(1-y)\right] f(y)\,.
\end{equation}
This provides a suitable expression for the implementation in {\tt
  APFEL}. Therefore, one just needs to manipulate the expressions
given in the literature to reduce them to the form of
Eq.~(\ref{eq:splittingfuncs}). This is typically an easy task.

Let us start with $P_{qq}^{V,(0)}$ that we take from Eq.~(38) of
Ref.~\cite{Vogelsang:1997ak}.\footnote{A factor 2 is introduced to
  account for the different expansion parameter, here $\alpha_s/4\pi$
  rather than $\alpha_s/2\pi$} After a simple manipulation,
it takes the form:
\begin{equation}\label{eq:LOsplitting}
P_{qq}^{V,(0)}(y) = 2C_F\left[-2+2\left(\frac{1}{1-y}\right)_++\frac32\delta(1-y)\right]\,.
\end{equation}
In this form it is easy to identify the elements introduced in
Eq.~(\ref{eq:splittingfuncs}). In particular, we find that
$R(x)=-4C_F$, $S=4C_F$, and $L=3C_F$. As mentioned above,
$P_{qq}^{V,(0)}$ is the same for PDFs and FFs, therefore the
expression in Eq.~(\ref{eq:LOsplitting}) is all one needs to implement
the LO evolution of both transversity PDFs and FFs

We now consider the NLO corrections. In order to distinguish between
PDFs and FF we will use the symbols $\mathcal{P}$ and $\mathbb{P}$,
respectively, for the splitting functions. We first consider the PDF
splitting functions that we again take from
Ref.~\cite{Vogelsang:1997ak}. We observe that
$\mathcal{P}_{q\overline{q}}^{V,(1)}$, taken from Eq.~(44) of this
paper, is a purely regular functions with no plus-prescripted and
$\delta$-function terms. Therefore, it needs no manipulation:
\begin{equation}\label{eq:PDFLOsplittingqqb}
\mathcal{P}_{q\overline{q}}^{V,(1)} (y) =
4C_F\left(C_F-\frac12C_A\right)\left[ - 1 + y -\frac{4S_2(y)}{1+y}\right]\,,
\end{equation}
with:
\begin{equation}
S_2(y) = -2\mbox{Li}_2(-y) - 2\ln y\ln(1+y)+\frac12\ln^2y-\frac{\pi^2}{6}\,.
\end{equation}

The function $\mathcal{P}_{qq}^{V,(1)}$ from Eq.~(43) of
Ref.~\cite{Vogelsang:1997ak} is instead more complicated but can be
recasted in the form of Eq.~(\ref{eq:LOsplitting}) as:
\begin{equation}\label{eq:PDFLOsplittingqq}
\begin{array}{rcl} 
  \mathcal{P}_{qq}^{V,(1)} (y) &=& \Bigg\{\displaystyle 4C_F^2 \left[ 1-y -\left( \frac{3}{2} + 
                                   2 \ln (1-y) \right) \frac{2y\ln
                                   y}{1-y}\right]\\
  \\
                               &+& \displaystyle 2C_F C_A\left[
                                   - \frac{143}{9} +
                                   \frac{2\pi^2}{3} +y + \left( \frac{11}{3} 
                                   + \ln y  \right) \frac{2y\ln
                                   y}{1-y}\right] + \displaystyle \frac{8}{3} n_f C_F
                                   T_R  \left[ - \frac{2y\ln y}{1-y}+ \frac{10}{3} \right]\Bigg\}\\
  \\
                               &+& \displaystyle \Bigg\{2C_F C_A\left(\frac{134}{9} -
                                   \frac{2\pi^2}{3} \right) - \frac{80}{9} n_f C_F
                                   T_R  \Bigg\}\left(\frac{1}{1-y}\right)_+\\
  \\
                               &+& \displaystyle\Bigg\{ 4C_F^2 \left( \frac{3}{8} -\frac{\pi^2}{2} + 6\zeta (3) 
                                   \right) + 2C_FC_A \left( \frac{17}{12} + \frac{11 \pi^2}{9} -
                                   6 \zeta (3) \right) - \frac{8}{3} n_f C_F
                                   T_R\left( \frac{1}{4} + \frac{\pi^2}{3} \right)\Bigg\} \delta(1-y)\,,
\end{array}
\end{equation}
where the regular, plus-prescripted, and $\delta$-function terms are
enclosed between curly brackets.

We can now turn to consider the splitting functions for FFs. In this
case we take the expressions from Ref.~\cite{Stratmann:2001pt}. From
Eq.~(17) of this paper we immediately see that:
\begin{equation}
\mathbb{P}_{q\overline{q}}^{V,(1)} (y)=\mathcal{P}_{q\overline{q}}^{V,(1)} (y)\,,
\end{equation}
where $\mathcal{P}_{q\overline{q}}^{V,(1)}$ is given in
Eq.~(\ref{eq:PDFLOsplittingqqb}). For $\mathbb{P}_{qq}^{V,(1)}$ we
instead find:
\begin{equation}
\begin{array}{rcl} 
  \mathbb{P}_{qq}^{V,(1)} (y) &=& \Bigg\{\displaystyle 4C_F^2 \left[ 1-y +\left( \frac{3}{2} + 
                                   2 \ln (1-y) -2\ln y\right) \frac{2y\ln
                                   y}{1-y}\right]\\
  \\
                               &+& \displaystyle 2C_F C_A\left[
                                   - \frac{143}{9} +
                                   \frac{2\pi^2}{3} +y + \left( \frac{11}{3} 
                                   + \ln y  \right) \frac{2y\ln
                                   y}{1-y}\right] + \displaystyle \frac{8}{3} n_f C_F
                                   T_R  \left[ - \frac{2y\ln y}{1-y}+ \frac{10}{3} \right]\Bigg\}\\
  \\
                               &+& \displaystyle \Bigg\{2C_F C_A\left(\frac{134}{9} -
                                   \frac{2\pi^2}{3} \right) - \frac{80}{9} n_f C_F
                                   T_R  \Bigg\}\left(\frac{1}{1-y}\right)_+\\
  \\
                               &+& \displaystyle\Bigg\{ 4C_F^2 \left( \frac{3}{8} -\frac{\pi^2}{2} + 6\zeta (3) 
                                   \right) + 2C_FC_A \left( \frac{17}{12} + \frac{11 \pi^2}{9} -
                                   6 \zeta (3) \right) - \frac{8}{3} n_f C_F
                                   T_R\left( \frac{1}{4} + \frac{\pi^2}{3} \right)\Bigg\} \delta(1-y)\,,
\end{array}
\end{equation}
that is just a small difference in the regular term as compared to
$\mathcal{P}_{qq}^{V,(1)}$ in Eq.~(\ref{eq:PDFLOsplittingqq}).

\newpage
\bibliographystyle{ieeetr}
\bibliography{bibliography}

\end{document}
