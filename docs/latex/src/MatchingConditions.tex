\documentclass[10pt,a4paper]{article}
\usepackage{amsmath,amssymb,bm,graphicx,makeidx,subfigure}
\usepackage[italian,english]{babel}
\usepackage[center,small]{caption}[2007/01/07]
\usepackage{fancyhdr}

\oddsidemargin = 12pt
\topmargin = 0pt
\textwidth = 440pt
\textheight = 650pt

\makeindex

\begin{document}

\tableofcontents
\newpage

\section{PDF matching conditions}

If the (Zero Mass) Variable Flavour Number Scheme (ZM$-$VFNS) at NNLO
is considered, matching conditions for the PDFs and the coupling
constant at the heavy quarks thresholds ($m_c^2$, $m_b^2$ and $m_t^2$)
must be implemented. This is due to the fact that we are working with
an ``effective theory'' where before a certain threshold, say $m_h^2$,
the heavy quark flavour $h$ is treated as infinitely massive, while
after the crossing of the same threshold the same flavour is treated
as massless. This results in discontinuities of PDFs and coupling
constant in correspondence of the thresholds. Such discontinuities can
be evaluated in perturbation theory, and in particular one can see
that they start at NNLO, so that PDFs and $a_s$ are continuous at LO
and NLO~\cite{Buza:1996wv}.

The discontinuity of the PDF $l$ (in the Mellin space) of a light
quark(\footnote{Note that the light quarks run from $1$ to $n_f$.})
just beyond the threshold $m_h^2$($=m_c^2,m_b^2,m_t^2$), where the
effective flavour number passes from $n_f$ to $n_f+1$, is given as a
function of the same PDF just before the threshold by the following
relation \cite{Vogt:2004ns}:
\begin{equation}\label{eq:lightqmc}
l^{(n_f+1)}(N,m_h^2)=[1+a_s^2(m_h^2)A_{qq,h}^{N\!S,(2)}(N)]l^{(n_f)}(N,m_h^2)\,.
\end{equation}
with $l=u,\overline{u},d,\overline{d},\dots$, while the gluon
distribution function is given by:
\begin{equation}\label{gluon}
\displaystyle g^{(n_f+1)}(N,m_h^2)=[1+a_s^2(m_h^2)A_{gg,h}^{S,(2)}(N)]g^{(n_f)}(N,m_h^2)+a_s^2(m_h^2)A^{S,(2)}_{gq,h}(N)\Sigma^{(n_f)}(N,m^2_h)
\end{equation}
and, in the end, the sum of heavy quark $h$ and its anti-quark
$\overline{h}$, which are going to be produced after the threshold
$m_h^2$, is:
\begin{equation}\label{eq:heavyqmc}
(h^{(n_f+1)}+\overline{h}^{(n_f+1)})(N,m_h^2)=a_s^2(m_h^2)[\tilde{A}^{S,(2)}_{hq}(N)\Sigma^{(n_f)}(N,m_h^2)+\tilde{A}^{S,(2)}_{hg}(N)g^{(n_f)}(N,m_h^2)]\,.
\end{equation}
Of course, we have $h=\overline{h}$. 

Now, since:
\begin{equation}
\Sigma^{(n_f+1)}=\sum_{l=1}^{n_f}(l^{(n_f+1)}+\overline{l}^{(n_f+1)})+(h^{(n_f+1)}+\overline{h}^{(n_f+1)})
\end{equation}
we find that the matching condition for the singlet is:
\begin{equation}\label{singlet}
\begin{array}{c}
\displaystyle\Sigma^{(n_f+1)}(N,m_h^2)=[1+a_s^2(m_h^2)A_{qq,h}^{N\!S,(2)}(N)]\Sigma^{(n_f)}(N,m_h^2)+\\
\\
\displaystyle a_s^2(m_h^2)[\tilde{A}^{S,(2)}_{hq}(N)\Sigma^{(n_f)}(N,m_h^2)+\tilde{A}^{S,(2)}_{hg}(N)g^{(n_f)}(N,m_h^2)]
\end{array}
\end{equation}
so that, from eqs. (\ref{gluon}) and (\ref{singlet}):
\begin{equation}\label{couple1}
\begin{array}{c}
\displaystyle {\Sigma^{(n_f+1)} \choose g^{(n_f+1)}}=\begin{pmatrix}1+a_s^2[A_{qq,h}^{N\!S,(2)}+\tilde{A}^{S,(2)}_{hq}] & a_s^2\tilde{A}^{S,(2)}_{hg}\\
a_s^2A^{S,(2)}_{gq,h} & 1+a_s^2A_{gg,h}^{S,(2)}\end{pmatrix}{\Sigma^{(n_f)} \choose g^{(n_f)}}\\
\\
\displaystyle =\left[\begin{pmatrix} 1 & 0 \\ 0 & 1\end{pmatrix}+a_s^2A_{qq,h}^{N\!S,(2)}\begin{pmatrix} 1 & 0 \\ 0 & 0\end{pmatrix}+a_s^2\begin{pmatrix} \tilde{A}^{S,(2)}_{hq} & \tilde{A}^{S,(2)}_{hg} \\A^{S,(2)}_{gq,h} & A_{gg,h}^{S,(2)}\end{pmatrix}\right]{\Sigma^{(n_f)} \choose g^{(n_f)}}
\end{array}
\end{equation}
where we have omitted all the dependencies. So we have obtained the
matching conditions for the singlet and the gluon distribution
functions.

Now we consider the other distributions. The valence distribution $V$
for $n_f+1$ active (light) flavours just beyond the threshold $m_h^2$
is defined as:
\begin{equation}
V^{(n_f+1)}=\sum_{l=1}^{n_f}(l^{(n_f+1)}-\overline{l}^{(n_f+1)})+(h^{(n_f+1)}-\overline{h}^{(n_f+1)})
\end{equation}
but, since $h=\overline{h}$, the last term vanish and we are left with:
\begin{equation}\label{valence}
\begin{array}{c}
\displaystyle V^{(n_f+1)}=\sum_{l=1}^{n_f}(l^{(n_f+1)}-\overline{l}^{(n_f+1)})=\\
\\
\displaystyle [1+a_s^2A_{qq,h}^{N\!S,(2)}]\sum_{l=1}^{n_f}(l^{(n_f)}-\overline{l}^{(n_f)})=[1+a_s^2A_{qq,h}^{N\!S,(2)}]V^{(n_f)}\,.
\end{array}
\end{equation}
So, this is the matching condition for the valence distribution $V$.

Now we consider the valence distribution $V_3$ and $V_8$ which are
both composed only by light quarks, namely:
\begin{equation}
V_3=(u-\overline{u})-(d-\overline{d})\quad\mbox{and}\quad V_8=(u-\overline{u})+(d-\overline{d})-2(s-\overline{s})
\end{equation}
so that, in these cases, the matching conditions work as in the case
of $V$, i.e.:
\begin{equation}\label{v38}
V_{3,8}^{(n_f+1)}=[1+a_s^2A_{qq,h}^{N\!S,(2)}]V^{(n_f)}_{3,8}\,.
\end{equation}

The same holds for $T_{3}$ and $T_8$, which are defined as:
\begin{equation}
T_3=(u+\overline{u})-(d+\overline{d})\quad\mbox{and}\quad V_8=(u+\overline{u})+(d+\overline{d})-2(s+\overline{s})
\end{equation}
so:
\begin{equation}\label{t38}
T_{3,8}^{(n_f+1)}=[1+a_s^2A_{qq,h}^{N\!S,(2)}]T^{(n_f)}_{3,8}\,.
\end{equation}

The remaining valence distribution $V_{15}$, $V_{24}$ and $V_{35}$ are
defined as:
\begin{equation}
\begin{array}{l}
V_{15}=(u-\overline{u})+(d-\overline{d})+(s-\overline{s})-3(c-\overline{c})\\
V_{24}=(u-\overline{u})+(d-\overline{d})+(s-\overline{s})+(c-\overline{c})-4(b-\overline{b})\\
V_{35}=(u-\overline{u})+(d-\overline{d})+(s-\overline{s})+(c-\overline{c})+(b-\overline{b})-5(t-\overline{t})
\end{array}
\end{equation}
and since in each one of them the heavy quarks appear always as
difference between quark and anti-quark, they cancel exactly. For
example, at the $m_b^2$ threshold, $V_{15}$ is entirely composed by
light quarks so there is no problem, while $V_{24}$ and $V_{35}$ have
also a $b$-quark contribution, given by $-4(b-\overline{b})$ and
$(b-\overline{b})$ respectively (of course, the $t(\overline{t})$
distribution is zero). Anyway, this terms give no matching condition
since the $b$ contribution is exactly equal to the $\overline{b}$
contribution, so that they cancel. So:
\begin{equation}\label{v152435}
V_{15,24,35}^{(n_f+1)}=[1+a_s^2A_{qq,h}^{N\!S,(2)}]V^{(n_f)}_{15,24,35}\,.
\end{equation}

In the end, to deal with $T_{15}$, $T_{24}$ and $T_{35}$, we have to
specify the threshold. Indeed, in these cases the heavy quark
contribution does not cancel. Their definition is:
\begin{equation}
\begin{array}{l}
T_{15}=(u+\overline{u})+(d+\overline{d})+(s+\overline{s})-3(c+\overline{c})\\
T_{24}=(u+\overline{u})+(d+\overline{d})+(s+\overline{s})+(c+\overline{c})-4(b+\overline{b})\\
T_{35}=(u+\overline{u})+(d+\overline{d})+(s+\overline{s})+(c+\overline{c})+(b+\overline{b})-5(t+\overline{t})\,.
\end{array}
\end{equation}
Just before the threshold $m_c^2$ we have only 3 active light flavours
($u$, $d$ and $s$), while just beyond $m_c^2$ we have 4 active
flavours and among them the flavour $c$ is considered to be heavy. Of
course, we have no $b$ and $t$ contribution (so $T_{24}$ and $T_{35}$
are equal). So the matching conditions are:
\begin{equation}
\begin{array}{c}
\displaystyle T_{15}^{(4)}=[1+a_s^2A_{qq,c}^{N\!S,(2)}]\underbrace{\sum_{l=u,d,s}(l^{(3)}+\overline{l}^{(3)})}_{\Sigma^{(3)}}-3a_s^2[\tilde{A}^{S,(2)}_{cq}\Sigma^{(3)}+\tilde{A}^{S,(2)}_{cg}g^{(3)}]=\\
\\
\displaystyle \begin{pmatrix} 1+a_s^2[A_{qq,c}^{N\!S,(2)}-3\tilde{A}^{S,(2)}_{cq}] & -3a_s^2\tilde{A}^{S,(2)}_{cg}\end{pmatrix}{\Sigma^{(3)} \choose g^{(3)}}
\end{array}
\end{equation}
while:
\begin{equation}
T_{24,35}^{(4)}=\begin{pmatrix} 1+a_s^2[A_{qq,c}^{N\!S,(2)}+\tilde{A}^{S,(2)}_{cq}] & a_s^2\tilde{A}^{S,(2)}_{cg}\end{pmatrix}{\Sigma^{(3)} \choose g^{(3)}}\,.
\end{equation}
We can put the above relation in a matricial form:
\begin{equation}\label{pippo1}
\begin{pmatrix} T_{15}^{(4)} \\ T_{24}^{(4)} \\ T_{35}^{(4)} \end{pmatrix} = \begin{pmatrix}  1+a_s^2[A_{qq,c}^{N\!S,(2)}-3\tilde{A}^{S,(2)}_{cq}] & -3a_s^2\tilde{A}^{S,(2)}_{cg}\\  
1+a_s^2[A_{qq,c}^{N\!S,(2)}+\tilde{A}^{S,(2)}_{cq}] & a_s^2\tilde{A}^{S,(2)}_{cg} \\
1+a_s^2[A_{qq,c}^{N\!S,(2)}+\tilde{A}^{S,(2)}_{cq}] & a_s^2\tilde{A}^{S,(2)}_{cg} \end{pmatrix}{\Sigma^{(3)} \choose g^{(3)}}\,.
\end{equation}
But now it is easy to generalize. At $m_b^2$, $T_{15}$ does not
contain, so:
\begin{equation}\label{pippo2}
T_{15}^{(5)}=[1+a_s^2A_{qq,b}^{N\!S,(2)}]T_{15}^{(4)}
\end{equation}
while:
\begin{equation}\label{pippo3}
\begin{pmatrix} T_{24}^{(5)} \\ T_{35}^{(5)} \end{pmatrix} = \begin{pmatrix}  1+a_s^2[A_{qq,b}^{N\!S,(2)}-4\tilde{A}^{S,(2)}_{bq}] & -4a_s^2\tilde{A}^{S,(2)}_{bg}\\  
1+a_s^2[A_{qq,b}^{N\!S,(2)}+\tilde{A}^{S,(2)}_{bq}] & a_s^2\tilde{A}^{S,(2)}_{bg} \end{pmatrix}{\Sigma^{(4)} \choose g^{(4)}}\,.
\end{equation}
Finally, at $m_t^2$ we have:
\begin{equation}\label{pippo4}
\begin{array}{l}
\displaystyle T_{15}^{(6)}=[1+a_s^2A_{qq,t}^{N\!S,(2)}]T_{15}^{(5)}\\
\\
\displaystyle T_{24}^{(6)}=[1+a_s^2A_{qq,t}^{N\!S,(2)}]T_{25}^{(5)}
\end{array}
\end{equation}
and:
\begin{equation}\label{pippo5}
T_{35}^{(6)} = \begin{pmatrix}  1+a_s^2[A_{qq,t}^{N\!S,(2)}-5\tilde{A}^{S,(2)}_{tq}] & -5a_s^2\tilde{A}^{S,(2)}_{tg}\end{pmatrix}{\Sigma^{(5)} \choose g^{(5)}}\,.
\end{equation}

An explicit calculation for the coefficients $A^{(2)}$ in the
$x$-space can be found in [hep-ph/9612398]. Anyhow, that calculation
is performed more generally in the case $m_h^2\neq\mu_F^2$. This
results in extra-terms proportional to $\ln(m_h^2/\mu_F^2)$, which
vanish if, as we do, one takes the factorisation scale $\mu^2$
coinciding with the scale of the process $Q^2$. Moreover, in that case
also NLO ($\propto a_s$) appear in the matching conditions.

To summarise the PDF matching conditions at the threshold $m_h^2$, we
have that:
\begin{itemize}
\item singlet and gluon couple as follows:
\begin{equation}\label{couple}
{\Sigma^{(n_f+1)} \choose g^{(n_f+1)}}=\left[\begin{pmatrix} 1 & 0 \\ 0 & 1\end{pmatrix}+a_s^2A_{qq,h}^{N\!S,(2)}\begin{pmatrix} 1 & 0 \\ 0 & 0\end{pmatrix}+a_s^2\begin{pmatrix} \tilde{A}^{S,(2)}_{hq} & \tilde{A}^{S,(2)}_{hg} \\A^{S,(2)}_{gq,h} & A_{gg,h}^{S,(2)}\end{pmatrix}\right]{\Sigma^{(n_f)} \choose g^{(n_f)}}
\end{equation}
\item from eqs. (\ref{valence}), (\ref{v38}), (\ref{v152435}) and (\ref{t38}), one can see that $V$, $V_{3,8,\dots,35}$ and $T_{3,8}$ behave in the same way, i.e.:
\begin{equation}\label{ciao}
P^{(n_f+1)}=[1+a_s^2A_{qq,h}^{N\!S,(2)}]P^{(n_f)}\quad\mbox{with}\quad P=V,V_3,\dots,V_{35},T_3,T_8
\end{equation}
\item $T_{15}$, $T_{24}$ and $T_{35}$ have different matching
  conditions depending on the threshold. In particular: for
  $m_h^2=m_c^2$ they are given by eq. (\ref{pippo1}), for
  $m_h^2=m_b^2$ they are given by eqs. (\ref{pippo2}) and
  (\ref{pippo3}) and for $m_h^2=m_t^2$ they are given by
  eqs. (\ref{pippo4}) and (\ref{pippo5})
\end{itemize}

In the following Sections we will discuss how to write the evolution
kernels in the presence of the matching conditions. We will explicitly
consider only the forward evolution, i.e. the final scale $Q^2$
greater than the initial one $Q_0^2$. Anyway the backward evolution
($Q_0^2>Q^2$) can be easily obtained from the forward one. In fact,
given the evolution kernel $\Gamma$, the following relation holds:
\begin{equation}
\Gamma(Q^2,Q_0^2)\Gamma(Q_0^2,Q^2)=1\quad\Longrightarrow\quad\Gamma(Q^2,Q_0^2)=\Gamma^{-1}(Q_0^2,Q^2).
\end{equation}
so, if $Q^2>Q_0^2$ the code computes directly $\Gamma(Q^2,Q_0^2)$, else if $Q^2_0>Q^2$ the code evaluates first the forward evolution $\Gamma(Q_0^2,Q^2)$ and then, to get $\Gamma(Q^2,Q_0^2)$, it calculates $\Gamma^{-1}(Q_0^2,Q^2)$.

\subsection{Matching Conditions on the Evolution Kernels: 0 Thresholds Crossing}

Before to discuss the crossing of the thresholds, it would be useful
to write down the evolution kernels in the ``trivial'' situation of no
threshold crossing. There are 4 particular cases: 1) $Q_0^2<Q^2<m_c^2$
with $n_f=3$ active flavours, 2) $m_c^2<Q_0^2<Q^2<m_b^2$ with $n_f=4$
active flavours, 3) $m_b^2<Q_0^2<Q^2<m_t^2$ with $n_f=5$ active
flavours and 4) $m_t^2<Q_0^2<Q^2$ with $n_f=6$ active flavours, which
do not need the introduction of the matching conditions. So, in the
following Subsections we will write down the evolution of the whole
PDF set the these cases.

\subsubsection{$Q_0^2<Q^2<m_c^2$}
\begin{itemize}
\item \textbf{Singlet} and \textbf{gluon}:
\begin{equation}
{\Sigma^{(3)}(Q^2) \choose g^{(3)}(Q^2)} =\underbrace{\begin{pmatrix} \Gamma_{qq}& \Gamma_{qg} \\ \Gamma_{gq}& \Gamma_{gg}\end{pmatrix}}_{(Q^2,Q_0^2)}{\Sigma^{(3)}(Q_0^2) \choose g^{(3)}(Q_0^2)}
\end{equation}
\item $\mathbf{V}$:
\begin{equation}
V^{(3)}(Q^2)=\Gamma^{v}(Q^2,Q_0^2)V^{(3)}(Q^2_0)
\end{equation}
\item $\mathbf{V_3}$ and $\mathbf{V_8}$:
\begin{equation}
V^{(3)}_{3,8}(Q^2)=\Gamma^{-}(Q^2,Q_0^2)V^{(3)}_{3,8}(Q^2_0)
\end{equation}
\item $\mathbf{V_{15}}$, $\mathbf{V_{24}}$ and $\mathbf{V_{35}}$:
\begin{equation}
V_{15,24,35}^{(3)}(Q^2)=\Gamma^{v}(Q^2,Q^2_0)V^{(3)}(Q_0^2)
\end{equation}
\item $\mathbf{T_3}$ and $\mathbf{T_8}$:
\begin{equation}
T^{(3)}_{3,8}(Q^2)=\Gamma^{+}(Q^2,Q_0^2)T^{(3)}_{3,8}(Q^2_0)
\end{equation}
\item $\mathbf{T_{15}}$, $\mathbf{T_{24}}$ and $\mathbf{T_{35}}$:
\begin{equation}
T_{15,24,35}^{(3)}(Q^2) = \underbrace{\begin{pmatrix} \Gamma_{qq} & \Gamma_{qg}\end{pmatrix}}_{(Q^2,Q_0^2)}{\Sigma^{(3)}(Q_0^2) \choose g^{(3)}(Q_0^2)}
\end{equation}
\end{itemize}

\subsubsection{$m_c^2<Q_0^2<Q^2<m_b^2$}
\begin{itemize}
\item \textbf{Singlet} and \textbf{gluon}:
\begin{equation}
{\Sigma^{(4)}(Q^2) \choose g^{(4)}(Q^2)} =\underbrace{\begin{pmatrix} \Gamma_{qq}& \Gamma_{qg} \\ \Gamma_{gq}& \Gamma_{gg}\end{pmatrix}}_{(Q^2,Q_0^2)}{\Sigma^{(4)}(Q_0^2) \choose g^{(4)}(Q_0^2)}
\end{equation}
\item $\mathbf{V}$:
\begin{equation}
V^{(4)}(Q^2)=\Gamma^{v}(Q^2,Q_0^2)V^{(4)}(Q^2_0)
\end{equation}
\item $\mathbf{V_3}$, $\mathbf{V_8}$ and $\mathbf{V_{15}}$:
\begin{equation}
V^{(4)}_{3,8,15}(Q^2)=\Gamma^{-}(Q^2,Q_0^2)V^{(4)}_{3,8,15}(Q^2_0)
\end{equation}
\item $\mathbf{V_{24}}$ and $\mathbf{V_{35}}$:
\begin{equation}
V_{24,35}^{(4)}(Q^2)=\Gamma^{v}(Q^2,Q^2_0)V^{(4)}(Q_0^2)
\end{equation}
\item $\mathbf{T_3}$, $\mathbf{T_8}$ and $\mathbf{T_{15}}$:
\begin{equation}
T^{(4)}_{3,8,15}(Q^2)=\Gamma^{+}(Q^2,Q_0^2)T^{(4)}_{3,8,15}(Q^2_0)
\end{equation}
\item $\mathbf{T_{24}}$ and $\mathbf{T_{35}}$:
\begin{equation}
T_{24,35}^{(4)}(Q^2) = \underbrace{\begin{pmatrix} \Gamma_{qq} & \Gamma_{qg}\end{pmatrix}}_{(Q^2,Q_0^2)}{\Sigma^{(4)}(Q_0^2) \choose g^{(4)}(Q_0^2)}
\end{equation}
\end{itemize}

\subsubsection{$m_b^2<Q_0^2<Q^2<m_t^2$}
\begin{itemize}
\item \textbf{Singlet} and \textbf{gluon}:
\begin{equation}
{\Sigma^{(5)}(Q^2) \choose g^{(5)}(Q^2)} =\underbrace{\begin{pmatrix} \Gamma_{qq}& \Gamma_{qg} \\ \Gamma_{gq}& \Gamma_{gg}\end{pmatrix}}_{(Q^2,Q_0^2)}{\Sigma^{(5)}(Q_0^2) \choose g^{(5)}(Q_0^2)}
\end{equation}
\item $\mathbf{V}$:
\begin{equation}
V^{(5)}(Q^2)=\Gamma^{v}(Q^2,Q_0^2)V^{(5)}(Q^2_0)
\end{equation}
\item $\mathbf{V_3}$, $\mathbf{V_8}$, $\mathbf{V_{15}}$ and $\mathbf{V_{24}}$:
\begin{equation}
V^{(5)}_{3,8,15,24}(Q^2)=\Gamma^{-}(Q^2,Q_0^2)V^{(5)}_{3,8,15,24}(Q^2_0)
\end{equation}
\item $\mathbf{V_{35}}$:
\begin{equation}
V_{35}^{(5)}(Q^2)=\Gamma^{v}(Q^2,Q^2_0)V^{(5)}(Q_0^2)
\end{equation}
\item $\mathbf{T_3}$, $\mathbf{T_8}$, $\mathbf{T_{15}}$ and $\mathbf{T_{24}}$:
\begin{equation}
T^{(5)}_{3,8,15,24}(Q^2)=\Gamma^{+}(Q^2,Q_0^2)T^{(5)}_{3,8,15,24}(Q^2_0)
\end{equation}
\item $\mathbf{T_{35}}$:
\begin{equation}
T_{35}^{(5)}(Q^2) = \underbrace{\begin{pmatrix} \Gamma_{qq} & \Gamma_{qg}\end{pmatrix}}_{(Q^2,Q_0^2)}{\Sigma^{(5)}(Q_0^2) \choose g^{(5)}(Q_0^2)}
\end{equation}
\end{itemize}

\subsubsection{$m_t^2<Q_0^2<Q^2$}
\begin{itemize}
\item \textbf{Singlet} and \textbf{gluon}:
\begin{equation}
{\Sigma^{(6)}(Q^2) \choose g^{(6)}(Q^2)} =\underbrace{\begin{pmatrix} \Gamma_{qq}& \Gamma_{qg} \\ \Gamma_{gq}& \Gamma_{gg}\end{pmatrix}}_{(Q^2,Q_0^2)}{\Sigma^{(6)}(Q_0^2) \choose g^{(6)}(Q_0^2)}
\end{equation}
\item $\mathbf{V}$:
\begin{equation}
V^{(6)}(Q^2)=\Gamma^{v}(Q^2,Q_0^2)V^{(6)}(Q^2_0)
\end{equation}
\item $\mathbf{V_3}$, $\mathbf{V_8}$, $\mathbf{V_{15}}$, $\mathbf{V_{24}}$ and $\mathbf{V_{35}}$:
\begin{equation}
V^{(6)}_{3,8,15,24,35}(Q^2)=\Gamma^{-}(Q^2,Q_0^2)V^{(6)}_{3,8,15,24,35}(Q^2_0)
\end{equation}
\item $\mathbf{T_3}$, $\mathbf{T_8}$, $\mathbf{T_{15}}$, $\mathbf{T_{24}}$ and $\mathbf{T_{35}}$:
\begin{equation}
T^{(6)}_{3,8,15,24,35}(Q^2)=\Gamma^{+}(Q^2,Q_0^2)T^{(6)}_{3,8,15,24,35}(Q^2_0)
\end{equation}
\end{itemize}


\subsection{Matching Conditions on the Evolution Kernels: 1 Threshold Crossing}

In order to implement the matching conditions in our code, we will
show how to transfer them from the PDFs to the evolution kernels.

In this Section we suppose that the evolution crosses only one
threshold. We will show how the matching conditions on the PDFs modify
the form of the evolution kernels in the cases: 1)
$Q_0^2<m_c^2\leq Q^2$, 2) $Q_0^2<m_b^2\leq Q^2$ and
$Q_0^2<m_c^2\leq Q^2$.

\subsubsection{$Q_0^2<m_c^2\leq Q^2$}

In order to evolve PDFs from the scale $Q_0^2$ to $Q^2$ passing
through the threshold $m_c^2$, we have to: first evolve them from
$Q^2_0$ to $m_c^2$, where there are 3 active flavours, then increase
the number of active flavour from 3 to 4 by imposing the matching
conditions, and in the end evolve the PDFs, now having 4 active
flavours, from $m_c^2$ to the scale $Q^2$.

In what follows we will work only in the Mellin space, so we will drop
any dependence on $N$.

Let's start with singlet and gluon. We have:
\begin{equation}\label{couple4Qmc}
{\Sigma^{(4)} \choose g^{(4)}}(Q^2) = \begin{pmatrix} \Gamma_{qq} & \Gamma_{qg} \\ \Gamma_{gq}& \Gamma_{gg}\end{pmatrix}(Q^2,m_c^2){\Sigma^{(4)} \choose g^{(4)}}(m_c^2)\,.
\end{equation}
From eq. (\ref{couple}):
\begin{equation}
\displaystyle {\Sigma^{(4)} \choose g^{(4)}}(m_c^2)=\left[\begin{pmatrix} 1 & 0 \\ 0 & 1\end{pmatrix}+a_s^2(m_c^2)A_{qq,c}^{N\!S,(2)}\begin{pmatrix} 1 & 0 \\ 0 & 0\end{pmatrix}+a_s^2(m_c^2)\begin{pmatrix} \tilde{A}^{S,(2)}_{cq} & \tilde{A}^{S,(2)}_{cg} \\A^{S,(2)}_{gq,c} & A_{gg,c}^{S,(2)}\end{pmatrix}\right]{\Sigma^{(3)} \choose g^{(3)}}(m_c^2)
\end{equation}
now, substituting the above relation into the eq. (\ref{couple4Qmc}),
we get:
\begin{equation}\label{couple43Qmc}
\begin{array}{l}
\displaystyle {\Sigma^{(4)} \choose g^{(4)}}(Q^2) = \begin{pmatrix} \Gamma_{qq} & \Gamma_{qg} \\ \Gamma_{gq}& \Gamma_{gg}\end{pmatrix}(Q^2,m_c^2)\Bigg[\begin{pmatrix} 1 & 0 \\ 0 & 1\end{pmatrix}+a_s^2(m_c^2)A_{qq,c}^{N\!S,(2)}\begin{pmatrix} 1 & 0 \\ 0 & 0\end{pmatrix}+\\
\\
\hspace{180pt}\displaystyle a_s^2(m_c^2)\begin{pmatrix} \tilde{A}^{S,(2)}_{cq} & \tilde{A}^{S,(2)}_{cg} \\A^{S,(2)}_{gq,c} & A_{gg,c}^{S,(2)}\end{pmatrix}\Bigg]{\Sigma^{(3)} \choose g^{(3)}}(m_c^2)\,.
\end{array}
\end{equation}
But:
\begin{equation}\label{couple3mcQ0}
{\Sigma^{(3)} \choose g^{(3)}}(m_c^2) = \begin{pmatrix} \Gamma_{qq} & \Gamma_{qg} \\ \Gamma_{gq}& \Gamma_{gg}\end{pmatrix}(m_c^2,Q_0^2){\Sigma^{(3)} \choose g^{(3)}}(Q_0^2)\,.
\end{equation}
So, in the end:
\begin{equation}\label{couple43QQ0}
\begin{array}{l}
\displaystyle {\Sigma^{(4)} \choose g^{(4)}}(Q^2) = \Bigg\{\begin{pmatrix} \Gamma_{qq} & \Gamma_{qg} \\ \Gamma_{gq}& \Gamma_{gg}\end{pmatrix}(Q^2,m_c^2)\Bigg[\begin{pmatrix} 1 & 0 \\ 0 & 1\end{pmatrix}+a_s^2(m_c^2)A_{qq,c}^{N\!S,(2)}\begin{pmatrix} 1 & 0 \\ 0 & 0\end{pmatrix}+\\
\\
\hspace{70pt}\displaystyle a_s^2(m_c^2)\begin{pmatrix} \tilde{A}^{S,(2)}_{cq} & \tilde{A}^{S,(2)}_{cg} \\A^{S,(2)}_{gq,c} & A_{gg,c}^{S,(2)}\end{pmatrix}\Bigg]\begin{pmatrix} \Gamma_{qq} & \Gamma_{qg} \\ \Gamma_{gq}& \Gamma_{gg}\end{pmatrix}(m_c^2,Q_0^2)\Bigg\}{\Sigma^{(3)} \choose g^{(3)}}(Q_0^2)
\end{array}\,.
\end{equation}

Now we consider the distributions $V$, $V_{3,8}$ and $T_{3,8}$ which
evolve re\-spe\-cti\-ve\-ly through $\Gamma^v$, $\Gamma^-$ and
$\Gamma^+$, but which obey the same matching conditions. So:
\begin{equation}
P^{(4)}(Q^2)=\Gamma^{(P)}(Q^2,m_c^2)P^{(4)}(m_c^2)
\end{equation}
so that:
\begin{equation}
P=\left\{
\begin{array}{ll}
\displaystyle V &\rightarrow \Gamma^{(P)}=\Gamma^v\\
\displaystyle V_{3,8} &\rightarrow \Gamma^{(P)}=\Gamma^-\\
\displaystyle T_{3,8} &\rightarrow \Gamma^{(P)}=\Gamma^+
\end{array}\right.
\end{equation}
but:
\begin{equation}
P^{(4)}(m_c^2)=[1+a_s^2(m_c^2)A_{qq,c}^{N\!S,(2)}]P^{(3)}(m_c^2)
\end{equation}
and:
\begin{equation}
P^{(3)}(m_c^2)=\Gamma^{(P)}(m_c^2,Q_0^2)P^{(3)}(Q^2_0)
\end{equation}
so that:
\begin{equation}
P^{(4)}(Q^2)=\left\{\Gamma^{(P)}(Q^2,m_c^2)[1+a_s^2(m_c^2)A_{qq,c}^{N\!S,(2)}]\Gamma^{(P)}(m_c^2,Q_0^2)\right\}P^{(3)}(Q^2_0)
\end{equation}

Now we consider $V_{15}$, which before $m_c^2$ evolves as:
\begin{equation}
V_{15}^{(3)}(m_c^2)=\Gamma^{v}(m_c^2,Q_0^2)V^{(3)}(Q_0^2)
\end{equation}
while after $m_c^2$ it evolves as:
\begin{equation}
V_{15}^{(4)}(Q^2)=\Gamma^{-}(Q_0^2,m_c^2)V^{(4)}_{15}(m_c^2)\,.
\end{equation}
From eq. (\ref{ciao}), we find that the matching condition at $m_c^2$ is:
\begin{equation}
V^{(4)}_{15}(m_c^2)=[1+a_s^2(m_c^2)A_{qq,c}^{N\!S,(2)}] V^{(3)}_{15}(m_c^2)
\end{equation}
so:
\begin{equation}
V_{15}^{(4)}(Q^2)=\left\{\Gamma^{-}(Q^2,m_c^2)[1+a_s^2(m_c^2)A_{qq,c}^{N\!S,(2)}]\Gamma^{v}(m_c^2,Q_0^2)\right\}V^{(3)}(Q_0^2)
\end{equation}

Instead, for $Q_0^2<m_c^2\leq Q^2$, both $V_{24}$ and $V_{35}$ evolve as: 
\begin{equation}
V_{24,35}^{(4)}(Q^2)=\left\{\Gamma^{v}(Q^2,m_c^2)[1+a_s^2(m_c^2)A_{qq,c}^{N\!S,(2)}]\Gamma^{v}(m_c^2,Q_0^2)\right\}V^{(3)}(Q_0^2)
\end{equation}

Now, we consider $T_{15}$. After $m_c^2$, it evolves as:
\begin{equation}
T_{15}^{(4)}(Q^2)=\Gamma^{+}(Q^2,m_c^2)T^{(4)}_{15}(m_c^2)\,.
\end{equation}
This time the matching condition is given by the first line of eq. (\ref{pippo1}):
\begin{equation}
T_{15}^{(4)}(m_c^2)= \begin{pmatrix}  1+a_s^2(m_c^2)[A_{qq,c}^{N\!S,(2)}-3\tilde{A}^{S,(2)}_{cq}] & -3a_s^2(m_c^2)\tilde{A}^{S,(2)}_{cg} \end{pmatrix}{\Sigma^{(3)}(m_c^2) \choose g^{(3)}(m_c^2)}\,.
\end{equation}
But:
\begin{equation}
{\Sigma^{(3)}(m_c^2) \choose g^{(3)}(m_c^2) }=\begin{pmatrix} \Gamma_{qq}(m_c^2,Q_0^2) & \Gamma_{qg}(m_c^2,Q_0^2)\\ \Gamma_{gq}(m_c^2,Q_0^2) & \Gamma_{gg}(m_c^2,Q_0^2)\end{pmatrix}{\Sigma^{(3)}(Q_0^2) \choose g^{(3)}(Q_0^2)}
\end{equation}
In the end, one finds that:
\begin{equation}
\begin{array}{rcl}
\displaystyle T_{15}^{(4)}(Q^2)&=&\Bigg\{\Gamma^{+}(Q^2,m_c^2)\begin{pmatrix} 1+a_s^2(m_c^2)[A_{qq,c}^{N\!S,(2)}-3\tilde{A}^{S,(2)}_{cq}] & -3a_s^2(m_c^2)\tilde{A}^{S,(2)}_{cg}\end{pmatrix}\times\\
\\
& &\displaystyle \begin{pmatrix}\Gamma_{qq}(m_c^2,Q_0^2) & \Gamma_{qg}(m_c^2,Q_0^2)\\ \Gamma_{gq}(m_c^2,Q_0^2) & \Gamma_{gg}(m_c^2,Q_0^2)\end{pmatrix}\Bigg\}{\Sigma^{(3)}(Q_0^2) \choose g^{(3)}(Q_0^2)}
\end{array}
\end{equation} 

Now we are left only with $T_{24}$ and $T_{35}$, which evolve as the
single before and after the threshold, so they evolve exactly as the
first line of eq. (\ref{couple43QQ0}), i.e:
\begin{equation}
\begin{array}{l}
\displaystyle T_{24,35}^{(4)}(Q^2)= \Bigg\{\begin{pmatrix} \Gamma_{qq} & \Gamma_{qg} \end{pmatrix}(Q^2,m_c^2)\Bigg[\begin{pmatrix} 1 & 0 \\ 0 & 1\end{pmatrix}+a_s^2(m_c^2)A_{qq,c}^{N\!S,(2)}\begin{pmatrix} 1 & 0 \\ 0 & 0\end{pmatrix}+\\
\\
\hspace{70pt}\displaystyle a_s^2(m_c^2)\begin{pmatrix} \tilde{A}^{S,(2)}_{cq} & \tilde{A}^{S,(2)}_{cg} \\A^{S,(2)}_{gq,c} & A_{gg,c}^{S,(2)}\end{pmatrix}\Bigg]\begin{pmatrix} \Gamma_{qq} & \Gamma_{qg} \\ \Gamma_{gq}& \Gamma_{gg}\end{pmatrix}(m_c^2,Q_0^2)\Bigg\}{\Sigma^{(3)} \choose g^{(3)}}(Q_0^2)
\end{array}\,.
\end{equation}

Now, let us summarise what happens to the evolution kernels, by
introducing the matching conditions, if one crosses the $m_c^2$
threshold. We remind that the matching conditions appear only from the
NNLO.
\begin{itemize}
\item \textbf{Singlet} and \textbf{gluon}:
\begin{equation}
{\Sigma^{(4)}(Q^2) \choose g^{(4)}(Q^2)} = \Bigg\{\underbrace{\begin{pmatrix} \Gamma_{qq} & \Gamma_{qg} \\ \Gamma_{gq}& \Gamma_{gg}\end{pmatrix}}_{(Q^2,m_c^2)}\begin{pmatrix} M_{11}^c & M_{12}^c \\ M_{21}^c & M_{22}^c\end{pmatrix}\underbrace{\begin{pmatrix} \Gamma_{qq}& \Gamma_{qg} \\ \Gamma_{gq}& \Gamma_{gg}\end{pmatrix}}_{(m_c^2,Q_0^2)}\Bigg\}{\Sigma^{(3)}(Q_0^2) \choose g^{(3)}(Q_0^2)}
\end{equation}
where:
\begin{equation}
\begin{pmatrix} M_{11}^c & M_{12}^c \\ M_{21}^c & M_{22}^c\end{pmatrix}=\begin{pmatrix} 1 & 0 \\ 0 & 1\end{pmatrix}+a_s^2(m_c^2)A_{qq,c}^{N\!S,(2)}\begin{pmatrix} 1 & 0 \\ 0 & 0\end{pmatrix}+a_s^2(m_c^2)\begin{pmatrix} \tilde{A}^{S,(2)}_{cq} & \tilde{A}^{S,(2)}_{cg} \\A^{S,(2)}_{gq,c} & A_{gg,c}^{S,(2)}\end{pmatrix}
\end{equation}
\item $\mathbf{V}$:
\begin{equation}
V^{(4)}(Q^2)=\left\{\Gamma^{v}(Q^2,m_c^2)[1+a_s^2(m_c^2)A_{qq,c}^{N\!S,(2)}]\Gamma^{v}(m_c^2,Q_0^2)\right\}V^{(3)}(Q^2_0)
\end{equation}
\item $\mathbf{V_3}$ and $\mathbf{V_8}$:
\begin{equation}
V^{(4)}_{3,8}(Q^2)=\left\{\Gamma^{-}(Q^2,m_c^2)[1+a_s^2(m_c^2)A_{qq,c}^{N\!S,(2)}]\Gamma^{-}(m_c^2,Q_0^2)\right\}V^{(3)}_{3,8}(Q^2_0)
\end{equation}
\item $\mathbf{V_{15}}$:
\begin{equation}
V_{15}^{(4)}(Q^2)=\left\{\Gamma^{-}(Q^2,m_c^2)[1+a_s^2(m_c^2)A_{qq,c}^{N\!S,(2)}]\Gamma^{v}(m_c^2,Q_0^2)\right\}V^{(3)}(Q_0^2)
\end{equation}
\item $\mathbf{V_{24}}$ and $\mathbf{V_{35}}$:
\begin{equation}
V_{24,35}^{(4)}(Q^2)=\left\{\Gamma^{v}(Q^2,m_c^2)[1+a_s^2(m_c^2)A_{qq,c}^{N\!S,(2)}]\Gamma^{v}(m_c^2,Q_0^2)\right\}V^{(3)}(Q_0^2)
\end{equation}
\item $\mathbf{T_3}$ and $\mathbf{T_8}$:
\begin{equation}
T^{(4)}_{3,8}(Q^2)=\left\{\Gamma^{+}(Q^2,m_c^2)[1+a_s^2(m_c^2)A_{qq,c}^{N\!S,(2)}]\Gamma^{+}(m_c^2,Q_0^2)\right\}T^{(3)}_{3,8}(Q^2_0)
\end{equation}
\item $\mathbf{T_{15}}$:
\begin{equation}
\begin{array}{rcl}
\displaystyle T_{15}^{(4)}(Q^2)&=&\Bigg\{\Gamma^{+}(Q^2,m_c^2)\begin{pmatrix} 1+a_s^2(m_c^2)[A_{qq,c}^{N\!S,(2)}-3\tilde{A}^{S,(2)}_{cq}] & -3a_s^2(m_c^2)\tilde{A}^{S,(2)}_{cg}\end{pmatrix}\times\\
\\
 & & \displaystyle \underbrace{\begin{pmatrix}\Gamma_{qq}& \Gamma_{qg} \\ \Gamma_{gq} & \Gamma_{gg}\end{pmatrix}}_{(m_c^2,Q_0^2)}\Bigg\}{\Sigma^{(3)}(Q_0^2) \choose g^{(3)}(Q_0^2)}
\end{array}
\end{equation} 
\item $\mathbf{T_{24}}$ and $\mathbf{T_{35}}$:
\begin{equation}
T_{24,35}^{(4)}(Q^2) = \Bigg\{\underbrace{\begin{pmatrix} \Gamma_{qq} & \Gamma_{qg}\end{pmatrix}}_{(Q^2,m_c^2)}\begin{pmatrix} M_{11}^c & M_{12}^c \\ M_{21}^c & M_{22}^c\end{pmatrix}\underbrace{\begin{pmatrix} \Gamma_{qq}& \Gamma_{qg} \\ \Gamma_{gq}& \Gamma_{gg}\end{pmatrix}}_{(m_c^2,Q_0^2)}\Bigg\}{\Sigma^{(3)}(Q_0^2) \choose g^{(3)}(Q_0^2)}
\end{equation}
\end{itemize}

Now, it is very easy to rewrite the above summary for the crossing of
the remaining thresholds $m_b^2$ ($Q_0^2<m_b^2\leq Q^2$) and $m_t^2$
($Q_0^2<m_t^2\leq Q^2$).

\subsubsection{$Q_0^2<m_b^2\leq Q^2$}
\begin{itemize}
\item \textbf{Singlet} and \textbf{gluon}:
\begin{equation}
{\Sigma^{(5)}(Q^2) \choose g^{(5)}(Q^2)} = \Bigg\{\underbrace{\begin{pmatrix} \Gamma_{qq} & \Gamma_{qg} \\ \Gamma_{gq}& \Gamma_{gg}\end{pmatrix}}_{(Q^2,m_b^2)}\begin{pmatrix} M_{11}^b & M_{12}^b \\ M_{21}^b & M_{22}^b\end{pmatrix}\underbrace{\begin{pmatrix} \Gamma_{qq}& \Gamma_{qg} \\ \Gamma_{gq}& \Gamma_{gg}\end{pmatrix}}_{(m_b^2,Q_0^2)}\Bigg\}{\Sigma^{(4)}(Q_0^2) \choose g^{(4)}(Q_0^2)}
\end{equation}
where:
\begin{equation}
\begin{pmatrix} M_{11}^b & M_{12}^b \\ M_{21}^b & M_{22}^b\end{pmatrix}=\begin{pmatrix} 1 & 0 \\ 0 & 1\end{pmatrix}+a_s^2(m_b^2)A_{qq,b}^{N\!S,(2)}\begin{pmatrix} 1 & 0 \\ 0 & 0\end{pmatrix}+a_s^2(m_b^2)\begin{pmatrix} \tilde{A}^{S,(2)}_{bq} & \tilde{A}^{S,(2)}_{bg} \\A^{S,(2)}_{gq,b} & A_{gg,b}^{S,(2)}\end{pmatrix}
\end{equation}
\item $\mathbf{V}$:
\begin{equation}
V^{(5)}(Q^2)=\left\{\Gamma^{v}(Q^2,m_b^2)[1+a_s^2(m_b^2)A_{qq,b}^{N\!S,(2)}]\Gamma^{v}(m_b^2,Q_0^2)\right\}V^{(4)}(Q^2_0)
\end{equation}
\item $\mathbf{V_3}$, $\mathbf{V_8}$ and $\mathbf{V_{15}}$:
\begin{equation}
V^{(5)}_{3,8,15}(Q^2)=\left\{\Gamma^{-}(Q^2,m_b^2)[1+a_s^2(m_b^2)A_{qq,b}^{N\!S,(2)}]\Gamma^{-}(m_b^2,Q_0^2)\right\}V^{(4)}_{3,8,15}(Q^2_0)
\end{equation}
\item $\mathbf{V_{24}}$:
\begin{equation}
V_{24}^{(5)}(Q^2)=\left\{\Gamma^{-}(Q_0^2,m_b^2)[1+a_s^2(m_b^2)A_{qq,b}^{N\!S,(2)}]\Gamma^{v}(m_b^2,Q_0^2)\right\}V^{(4)}(Q_0^2)
\end{equation}
\item $\mathbf{V_{35}}$:
\begin{equation}
V_{35}^{(5)}(Q^2)=\left\{\Gamma^{v}(Q_0^2,m_b^2)[1+a_s^2(m_b^2)A_{qq,b}^{N\!S,(2)}]\Gamma^{v}(m_b^2,Q_0^2)\right\}V^{(4)}(Q_0^2)
\end{equation}
\item $\mathbf{T_3}$, $\mathbf{T_8}$ and $\mathbf{T_{15}}$:
\begin{equation}
T^{(5)}_{3,8,15}(Q^2)=\left\{\Gamma^{+}(Q^2,m_b^2)[1+a_s^2(m_b^2)A_{qq,b}^{N\!S,(2)}]\Gamma^{+}(m_b^2,Q_0^2)\right\}T^{(4)}_{3,8,15}(Q^2_0)
\end{equation}
\item $\mathbf{T_{24}}$:
\begin{equation}
\begin{array}{rcl}
\displaystyle T_{24}^{(5)}(Q^2)&=&\Bigg\{\Gamma^{+}(Q^2,m_b^2)\begin{pmatrix} 1+a_s^2(m_b^2)[A_{qq,b}^{N\!S,(2)}-4\tilde{A}^{S,(2)}_{bq}] & -4a_s^2(m_b^2)\tilde{A}^{S,(2)}_{bg}\end{pmatrix}\times\\
\\
 & & \displaystyle \underbrace{\begin{pmatrix}\Gamma_{qq}& \Gamma_{qg} \\ \Gamma_{gq} & \Gamma_{gg}\end{pmatrix}}_{(m_b^2,Q_0^2)}\Bigg\}{\Sigma^{(4)}(Q_0^2) \choose g^{(4)}(Q_0^2)}
\end{array}
\end{equation} 
\item $\mathbf{T_{35}}$:
\begin{equation}
T_{35}^{(5)}(Q^2) = \Bigg\{\underbrace{\begin{pmatrix} \Gamma_{qq} & \Gamma_{qg}\end{pmatrix}}_{(Q^2,m_b^2)}\begin{pmatrix} M_{11}^b & M_{12}^b \\ M_{21}^b & M_{22}^b\end{pmatrix}\underbrace{\begin{pmatrix} \Gamma_{qq}& \Gamma_{qg} \\ \Gamma_{gq}& \Gamma_{gg}\end{pmatrix}}_{(m_b^2,Q_0^2)}\Bigg\}{\Sigma^{(4)}(Q_0^2) \choose g^{(4)}(Q_0^2)}
\end{equation}
\end{itemize}

\subsubsection{$Q_0^2<m_t^2\leq Q^2$}
\begin{itemize}
\item \textbf{Singlet} and \textbf{gluon}:
\begin{equation}
{\Sigma^{(6)}(Q^2) \choose g^{(6)}(Q^2)} = \Bigg\{\underbrace{\begin{pmatrix} \Gamma_{qq} & \Gamma_{qg} \\ \Gamma_{gq}& \Gamma_{gg}\end{pmatrix}}_{(Q^2,m_t^2)}\begin{pmatrix} M_{11}^t & M_{12}^t \\ M_{21}^t & M_{22}^t\end{pmatrix}\underbrace{\begin{pmatrix} \Gamma_{qq}& \Gamma_{qg} \\ \Gamma_{gq}& \Gamma_{gg}\end{pmatrix}}_{(m_t^2,Q_0^2)}\Bigg\}{\Sigma^{(5)}(Q_0^2) \choose g^{(5)}(Q_0^2)}
\end{equation}
where:
\begin{equation}
\begin{pmatrix} M_{11}^t & M_{12}^t \\ M_{21}^t & M_{22}^t\end{pmatrix}=\begin{pmatrix} 1 & 0 \\ 0 & 1\end{pmatrix}+a_s^2(m_t^2)A_{qq,t}^{N\!S,(2)}\begin{pmatrix} 1 & 0 \\ 0 & 0\end{pmatrix}+a_s^2(m_t^2)\begin{pmatrix} \tilde{A}^{S,(2)}_{tq} & \tilde{A}^{S,(2)}_{tg} \\A^{S,(2)}_{gq,t} & A_{gg,t}^{S,(2)}\end{pmatrix}
\end{equation}
\item $\mathbf{V}$:
\begin{equation}
V^{(6)}(Q^2)=\left\{\Gamma^{v}(Q^2,m_t^2)[1+a_s^2(m_t^2)A_{qq,t}^{N\!S,(2)}]\Gamma^{v}(m_t^2,Q_0^2)\right\}V^{(5)}(Q^2_0)
\end{equation}
\item $\mathbf{V_3}$, $\mathbf{V_8}$, $\mathbf{V_{15}}$ and $\mathbf{V_{24}}$:
\begin{equation}
V^{(6)}_{3,8,15,24}(Q^2)=\left\{\Gamma^{-}(Q^2,m_t^2)[1+a_s^2(m_t^2)A_{qq,t}^{N\!S,(2)}]\Gamma^{-}(m_t^2,Q_0^2)\right\}V^{(5)}_{3,8,15,24}(Q^2_0)
\end{equation}
\item $\mathbf{V_{35}}$:
\begin{equation}
V_{35}^{(6)}(Q^2)=\left\{\Gamma^{-}(Q_0^2,m_t^2)[1+a_s^2(m_t^2)A_{qq,t}^{N\!S,(2)}]\Gamma^{v}(m_t^2,Q_0^2)\right\}V^{(5)}(Q_0^2)
\end{equation}
\item $\mathbf{T_3}$, $\mathbf{T_8}$, $\mathbf{T_{15}}$ and $\mathbf{T_{24}}$:
\begin{equation}
T^{(6)}_{3,8,15,24}(Q^2)=\left\{\Gamma^{+}(Q^2,m_t^2)[1+a_s^2(m_t^2)A_{qq,t}^{N\!S,(2)}]\Gamma^{+}(m_t^2,Q_0^2)\right\}T^{(5)}_{3,8,15,24}(Q^2_0)
\end{equation}
\item $\mathbf{T_{35}}$:
\begin{equation}
\begin{array}{rcl}
\displaystyle T_{35}^{(6)}(Q^2)&=&\Bigg\{\Gamma^{+}(Q^2,m_t^2)\begin{pmatrix} 1+a_s^2(m_t^2)[A_{qq,t}^{N\!S,(2)}-5\tilde{A}^{S,(2)}_{tq}] & -5a_s^2(m_t^2)\tilde{A}^{S,(2)}_{tg}\end{pmatrix}\times\\
\\
 & & \displaystyle \underbrace{\begin{pmatrix}\Gamma_{qq}& \Gamma_{qg} \\ \Gamma_{gq} & \Gamma_{gg}\end{pmatrix}}_{(m_t^2,Q_0^2)}\Bigg\}{\Sigma^{(5)}(Q_0^2) \choose g^{(5)}(Q_0^2)}
\end{array}
\end{equation} 
\end{itemize}

\subsection{Matching Conditions on the Evolution Kernels: 2 Thresholds Crossing}

In this Section we will discuss the case in which the evolution
crosses two thresholds. Therefore there are only two situations: 1)
$Q_0^2<m_c^2<m_b^2\leq Q^2$ and 2) $Q_0^2<m_b^2<m_t^2\leq Q^2$.
Anyway, there is nothing new, indeed to obtain such evolution kernels
we have just to ``merge'' together what we have already done in the
previous Section.

\subsubsection{$Q_0^2<m_c^2<m_b^2\leq Q^2$}

\begin{itemize}
\item \textbf{Singlet} and \textbf{gluon}:
\begin{equation}
\begin{array}{rcl}
\displaystyle {\Sigma^{(5)}(Q^2) \choose g^{(5)}(Q^2)} &=& \displaystyle \Bigg\{\underbrace{\begin{pmatrix} \Gamma_{qq} & \Gamma_{qg} \\ \Gamma_{gq}& \Gamma_{gg}\end{pmatrix}}_{(Q^2,m_b^2)}\begin{pmatrix} M_{11}^b & M_{12}^b \\ M_{21}^b & M_{22}^b\end{pmatrix}\underbrace{\begin{pmatrix} \Gamma_{qq}& \Gamma_{qg} \\ \Gamma_{gq}& \Gamma_{gg}\end{pmatrix}}_{(m_b^2,m_c^2)}\times\\
\\
 & &\displaystyle\begin{pmatrix} M_{11}^c & M_{12}^c \\ M_{21}^c & M_{22}^c\end{pmatrix}\underbrace{\begin{pmatrix} \Gamma_{qq}& \Gamma_{qg} \\ \Gamma_{gq}& \Gamma_{gg}\end{pmatrix}}_{(m_c^2,Q_0^2)}\Bigg\}{\Sigma^{(3)}(Q_0^2) \choose g^{(3)}(Q_0^2)}
\end{array}
\end{equation}
\item $\mathbf{V}$:
\begin{equation}
\begin{array}{rcl}
V^{(5)}(Q^2)&=&\displaystyle \Big\{\Gamma^{v}(Q^2,m_b^2)[1+a_s^2(m_b^2)A_{qq,b}^{N\!S,(2)}]\times\\
\\
 & &\displaystyle \Gamma^{v}(m_b^2,m_c^2)[1+a_s^2(m_c^2)A_{qq,c}^{N\!S,(2)}]\Gamma^{v}(m_c^2,Q_0^2)\Big\}V^{(3)}(Q^2_0)
\end{array}
\end{equation}
\item $\mathbf{V_3}$ and $\mathbf{V_8}$:
\begin{equation}
\begin{array}{rcl}
V^{(5)}_{3,8}(Q^2)&=&\displaystyle \Big\{\Gamma^{-}(Q^2,m_b^2)[1+a_s^2(m_b^2)A_{qq,b}^{N\!S,(2)}]\times\\
\\
& & \displaystyle \Gamma^{-}(m_b^2,m_c^2)[1+a_s^2(m_c^2)A_{qq,c}^{N\!S,(2)}]\Gamma^{-}(m_c^2,Q_0^2)\Big\}V^{(3)}_{3,8}(Q^2_0)
\end{array}
\end{equation}
\item $\mathbf{V_{15}}$:
\begin{equation}
\begin{array}{rcl}
V^{(5)}_{15}(Q^2)&=&\displaystyle \Big\{\Gamma^{-}(Q^2,m_b^2)[1+a_s^2(m_b^2)A_{qq,b}^{N\!S,(2)}]\times\\
\\
& & \displaystyle \Gamma^{-}(m_b^2,m_c^2)[1+a_s^2(m_c^2)A_{qq,c}^{N\!S,(2)}]\Gamma^{v}(m_c^2,Q_0^2)\Big\}V^{(3)}(Q^2_0)
\end{array}
\end{equation}
\item $\mathbf{V_{24}}$:
\begin{equation}
\begin{array}{rcl}
V^{(5)}_{24}(Q^2)&=&\displaystyle \Big\{\Gamma^{-}(Q^2,m_b^2)[1+a_s^2(m_b^2)A_{qq,b}^{N\!S,(2)}]\times\\
\\
& & \displaystyle \Gamma^{v}(m_b^2,m_c^2)[1+a_s^2(m_c^2)A_{qq,c}^{N\!S,(2)}]\Gamma^{v}(m_c^2,Q_0^2)\Big\}V^{(3)}(Q^2_0)
\end{array}
\end{equation}
\item $\mathbf{V_{35}}$:
\begin{equation}
\begin{array}{rcl}
V^{(5)}_{35}(Q^2)&=&\displaystyle \Big\{\Gamma^{v}(Q^2,m_b^2)[1+a_s^2(m_b^2)A_{qq,b}^{N\!S,(2)}]\times\\
\\
& & \displaystyle \Gamma^{v}(m_b^2,m_c^2)[1+a_s^2(m_c^2)A_{qq,c}^{N\!S,(2)}]\Gamma^{v}(m_c^2,Q_0^2)\Big\}V^{(3)}(Q^2_0)
\end{array}
\end{equation}

\item $\mathbf{T_3}$ and $\mathbf{T_8}$:
\begin{equation}
\begin{array}{rcl}
T^{(5)}_{3,8}(Q^2)&=&\displaystyle \Big\{\Gamma^{+}(Q^2,m_b^2)[1+a_s^2(m_b^2)A_{qq,b}^{N\!S,(2)}]\times\\
\\
& & \displaystyle \Gamma^{+}(m_b^2,m_c^2)[1+a_s^2(m_c^2)A_{qq,c}^{N\!S,(2)}]\Gamma^{+}(m_c^2,Q_0^2)\Big\}T^{(3)}_{3,8}(Q^2_0)
\end{array}
\end{equation}

\item $\mathbf{T_{15}}$:
\begin{equation}
\begin{array}{rcl}
T^{(5)}_{15}(Q^2)&=&\displaystyle \Bigg\{\Gamma^{+}(Q^2,m_b^2)[1+a_s^2(m_b^2)A_{qq,b}^{N\!S,(2)}]\Gamma^{+}(m_b^2,m_c^2)\times\\
\\
& & \displaystyle \begin{pmatrix} 1+a_s^2(m_c^2)[A_{qq,c}^{N\!S,(2)}-3\tilde{A}^{S,(2)}_{cq}] & -3a_s^2(m_c^2)\tilde{A}^{S,(2)}_{cg}\end{pmatrix}\underbrace{\begin{pmatrix}\Gamma_{qq}& \Gamma_{qg} \\ \Gamma_{gq} & \Gamma_{gg}\end{pmatrix}}_{(m_c^2,Q_0^2)}\Bigg\}{\Sigma^{(3)}(Q_0^2) \choose g^{(3)}(Q_0^2)}
\end{array}
\end{equation}

\item $\mathbf{T_{24}}$:
\begin{equation}
\begin{array}{rcl}
\displaystyle T_{24}^{(5)}(Q^2)&=&\Bigg\{\Gamma^{+}(Q^2,m_b^2)\begin{pmatrix} 1+a_s^2(m_b^2)[A_{qq,b}^{N\!S,(2)}-4\tilde{A}^{S,(2)}_{bq}] & -4a_s^2(m_b^2)\tilde{A}^{S,(2)}_{bg}\end{pmatrix}\times\\
\\
 & & \displaystyle \underbrace{\begin{pmatrix}\Gamma_{qq}& \Gamma_{qg} \\ \Gamma_{gq} & \Gamma_{gg}\end{pmatrix}}_{(m_b^2,m_c^2)}\begin{pmatrix} M^c_{11} & M^c_{12} \\ M_{21}^c & M_{22}^c \end{pmatrix}\underbrace{\begin{pmatrix}\Gamma_{qq}& \Gamma_{qg} \\ \Gamma_{gq} & \Gamma_{gg}\end{pmatrix}}_{(m_c^2,Q_0^2)}\Bigg\}{\Sigma^{(3)}(Q_0^2) \choose g^{(3)}(Q_0^2)}
\end{array}
\end{equation}
\item $\mathbf{T_{35}}$:
\begin{equation}
\begin{array}{rcl}
T_{35}^{(5)}(Q^2) &=&\displaystyle \Bigg\{\underbrace{\begin{pmatrix} \Gamma_{qq} & \Gamma_{qg}\end{pmatrix}}_{(Q^2,m_b^2)}\begin{pmatrix} M_{11}^b & M_{12}^b \\ M_{21}^b & M_{22}^b\end{pmatrix}\underbrace{\begin{pmatrix} \Gamma_{qq} & \Gamma_{qg} \\ \Gamma_{gq}& \Gamma_{gg}\end{pmatrix}}_{(m_b^2,m_c^2)}\times\\
\\
& & \displaystyle \begin{pmatrix} M^c_{11} & M^c_{12} \\ M_{21}^c & M_{22}^c \end{pmatrix}\underbrace{\begin{pmatrix}\Gamma_{qq}& \Gamma_{qg} \\ \Gamma_{gq} & \Gamma_{gg}\end{pmatrix}}_{(m_c^2,Q_0^2)}\Bigg\} {\Sigma^{(3)}(Q_0^2) \choose g^{(3)}(Q_0^2)}
\end{array}
\end{equation}
\end{itemize}

\subsubsection{$Q_0^2<m_b^2<m_t^2\leq Q^2$}

\begin{itemize}
\item \textbf{Singlet} and \textbf{gluon}:
\begin{equation}
\begin{array}{rcl}
\displaystyle {\Sigma^{(6)}(Q^2) \choose g^{(6)}(Q^2)} &=& \displaystyle \Bigg\{\underbrace{\begin{pmatrix} \Gamma_{qq} & \Gamma_{qg} \\ \Gamma_{gq}& \Gamma_{gg}\end{pmatrix}}_{(Q^2,m_t^2)}\begin{pmatrix} M_{11}^t & M_{12}^t \\ M_{21}^t & M_{22}^t\end{pmatrix}\underbrace{\begin{pmatrix} \Gamma_{qq}& \Gamma_{qg} \\ \Gamma_{gq}& \Gamma_{gg}\end{pmatrix}}_{(m_t^2,m_b^2)}\times\\
\\
 & &\displaystyle\begin{pmatrix} M_{11}^b & M_{12}^b \\ M_{21}^b & M_{22}^b\end{pmatrix}\underbrace{\begin{pmatrix} \Gamma_{qq}& \Gamma_{qg} \\ \Gamma_{gq}& \Gamma_{gg}\end{pmatrix}}_{(m_b^2,Q_0^2)}\Bigg\}{\Sigma^{(4)}(Q_0^2) \choose g^{(4)}(Q_0^2)}
\end{array}
\end{equation}
\item $\mathbf{V}$:
\begin{equation}
\begin{array}{rcl}
V^{(6)}(Q^2)&=&\displaystyle \Big\{\Gamma^{v}(Q^2,m_t^2)[1+a_s^2(m_t^2)A_{qq,t}^{N\!S,(2)}]\times\\
\\
 & &\displaystyle \Gamma^{v}(m_t^2,m_b^2)[1+a_s^2(m_b^2)A_{qq,b}^{N\!S,(2)}]\Gamma^{v}(m_b^2,Q_0^2)\Big\}V^{(4)}(Q^2_0)
\end{array}
\end{equation}
\item $\mathbf{V_3}$, $\mathbf{V_8}$ and $\mathbf{V_{15}}$:
\begin{equation}
\begin{array}{rcl}
V^{(6)}_{3,8,15}(Q^2)&=&\displaystyle \Big\{\Gamma^{-}(Q^2,m_t^2)[1+a_s^2(m_t^2)A_{qq,t}^{N\!S,(2)}]\times\\
\\
& & \displaystyle \Gamma^{-}(m_t^2,m_b^2)[1+a_s^2(m_b^2)A_{qq,b}^{N\!S,(2)}]\Gamma^{-}(m_b^2,Q_0^2)\Big\}V^{(4)}_{3,8,15}(Q^2_0)
\end{array}
\end{equation}
\item $\mathbf{V_{24}}$:
\begin{equation}
\begin{array}{rcl}
V^{(6)}_{24}(Q^2)&=&\displaystyle \Big\{\Gamma^{-}(Q^2,m_t^2)[1+a_s^2(m_t^2)A_{qq,t}^{N\!S,(2)}]\times\\
\\
& & \displaystyle \Gamma^{-}(m_t^2,m_b^2)[1+a_s^2(m_b^2)A_{qq,b}^{N\!S,(2)}]\Gamma^{v}(m_b^2,Q_0^2)\Big\}V^{(4)}(Q^2_0)
\end{array}
\end{equation}
\item $\mathbf{V_{35}}$:
\begin{equation}
\begin{array}{rcl}
V^{(6)}_{35}(Q^2)&=&\displaystyle \Big\{\Gamma^{-}(Q^2,m_t^2)[1+a_s^2(m_t^2)A_{qq,t}^{N\!S,(2)}]\times\\
\\
& & \displaystyle \Gamma^{v}(m_t^2,m_b^2)[1+a_s^2(m_b^2)A_{qq,b}^{N\!S,(2)}]\Gamma^{v}(m_b^2,Q_0^2)\Big\}V^{(4)}(Q^2_0)
\end{array}
\end{equation}

\item $\mathbf{T_3}$, $\mathbf{T_8}$ and $\mathbf{T_{15}}$:
\begin{equation}
\begin{array}{rcl}
T^{(6)}_{3,8,15}(Q^2)&=&\displaystyle \Big\{\Gamma^{+}(Q^2,m_t^2)[1+a_s^2(m_t^2)A_{qq,t}^{N\!S,(2)}]\times\\
\\
& & \displaystyle \Gamma^{+}(m_t^2,m_b^2)[1+a_s^2(m_b^2)A_{qq,b}^{N\!S,(2)}]\Gamma^{+}(m_b^2,Q_0^2)\Big\}T^{(4)}_{3,8,15}(Q^2_0)
\end{array}
\end{equation}

\item $\mathbf{T_{24}}$:
\begin{equation}
\begin{array}{rcl}
T^{(6)}_{24}(Q^2)&=&\displaystyle \Bigg\{\Gamma^{+}(Q^2,m_t^2)[1+a_s^2(m_t^2)A_{qq,t}^{N\!S,(2)}]\Gamma^{+}(m_t^2,m_b^2)\times\\
\\
& & \displaystyle \begin{pmatrix} 1+a_s^2(m_b^2)[A_{qq,b}^{N\!S,(2)}-4\tilde{A}^{S,(2)}_{bq}] & -4a_s^2(m_b^2)\tilde{A}^{S,(2)}_{bg}\end{pmatrix}\underbrace{\begin{pmatrix}\Gamma_{qq}& \Gamma_{qg} \\ \Gamma_{gq} & \Gamma_{gg}\end{pmatrix}}_{(m_b^2,Q_0^2)}\Bigg\}{\Sigma^{(4)}(Q_0^2) \choose g^{(4)}(Q_0^2)}
\end{array}
\end{equation}

\item $\mathbf{T_{35}}$:
\begin{equation}
\begin{array}{rcl}
\displaystyle T_{35}^{(6)}(Q^2)&=&\Bigg\{\Gamma^{+}(Q^2,m_t^2)\begin{pmatrix} 1+a_s^2(m_t^2)[A_{qq,t}^{N\!S,(2)}-5\tilde{A}^{S,(2)}_{tq}] & -5a_s^2(m_t^2)\tilde{A}^{S,(2)}_{tg}\end{pmatrix}\times\\
\\
 & & \displaystyle \underbrace{\begin{pmatrix}\Gamma_{qq}& \Gamma_{qg} \\ \Gamma_{gq} & \Gamma_{gg}\end{pmatrix}}_{(m_t^2,m_b^2)}\begin{pmatrix} M^b_{11} & M^b_{12} \\ M_{21}^b & M_{22}^b \end{pmatrix}\underbrace{\begin{pmatrix}\Gamma_{qq}& \Gamma_{qg} \\ \Gamma_{gq} & \Gamma_{gg}\end{pmatrix}}_{(m_b^2,Q_0^2)}\Bigg\}{\Sigma^{(4)}(Q_0^2) \choose g^{(4)}(Q_0^2)}
\end{array}
\end{equation}
\end{itemize}

\subsection{Matching Conditions on the Evolution Kernels: 3 Thresholds Crossing}

In this Section we will discuss the case in which the evolution
crosses three thresholds. Therefore there ais only one situation:
$Q_0^2<m_c^2<m_b^2,m_t^2\leq Q^2$

\subsubsection{$Q_0^2<m_c^2<m_b^2<m_t^2\leq Q^2$}
\begin{itemize}
\item \textbf{Singlet} and \textbf{gluon}:
\begin{equation}
\begin{array}{rcl}
\displaystyle {\Sigma^{(6)}(Q^2) \choose g^{(6)}(Q^2)} &=& \displaystyle \Bigg\{\underbrace{\begin{pmatrix} \Gamma_{qq} & \Gamma_{qg} \\ \Gamma_{gq}& \Gamma_{gg}\end{pmatrix}}_{(Q^2,m_t^2)}\begin{pmatrix} M_{11}^t & M_{12}^t \\ M_{21}^t & M_{22}^t\end{pmatrix}\underbrace{\begin{pmatrix} \Gamma_{qq}& \Gamma_{qg} \\ \Gamma_{gq}& \Gamma_{gg}\end{pmatrix}}_{(m_t^2,m_b^2)}\times\\
\\
 & &\displaystyle\begin{pmatrix} M_{11}^b & M_{12}^b \\ M_{21}^b & M_{22}^b\end{pmatrix}\underbrace{\begin{pmatrix} \Gamma_{qq}& \Gamma_{qg} \\ \Gamma_{gq}& \Gamma_{gg}\end{pmatrix}}_{(m_b^2,m_c^2)}\begin{pmatrix} M_{11}^c & M_{12}^c \\ M_{21}^c & M_{22}^c\end{pmatrix}\\
\\
& &\displaystyle \underbrace{\begin{pmatrix} \Gamma_{qq}& \Gamma_{qg} \\ \Gamma_{gq}& \Gamma_{gg}\end{pmatrix}}_{(m_c^2,Q_0^2)}\Bigg\}{\Sigma^{(3)}(Q_0^2) \choose g^{(3)}(Q_0^2)}
\end{array}
\end{equation}
\item $\mathbf{V}$:
\begin{equation}
\begin{array}{rcl}
V^{(6)}(Q^2)&=&\displaystyle \Big\{\Gamma^{v}(Q^2,m_t^2)[1+a_s^2(m_t^2)A_{qq,t}^{N\!S,(2)}]\times\\
\\
 & &\displaystyle \Gamma^{v}(m_t^2,m_b^2)[1+a_s^2(m_b^2)A_{qq,b}^{N\!S,(2)}]\Gamma^{v}(m_b^2,m_c^2)\\
\\
& & \displaystyle[1+a_s^2(m_c^2)A_{qq,c}^{N\!S,(2)}]\Gamma^{v}(m_c^2,Q_0^2)\Big\}V^{(3)}(Q^2_0)
\end{array}
\end{equation}
\item $\mathbf{V_3}$ and $\mathbf{V_8}$:
\begin{equation}
\begin{array}{rcl}
V^{(6)}_{3,8}(Q^2)&=&\displaystyle \Big\{\Gamma^{-}(Q^2,m_t^2)[1+a_s^2(m_t^2)A_{qq,t}^{N\!S,(2)}]\times\\
\\
& & \displaystyle \Gamma^{-}(m_t^2,m_b^2)[1+a_s^2(m_b^2)A_{qq,b}^{N\!S,(2)}]\Gamma^{-}(m_b^2,m_c^2)\\
\\
& & \displaystyle [1+a_s^2(m_c^2)A_{qq,c}^{N\!S,(2)}]\Gamma^{-}(m_c^2,Q_0^2)\Big\}V^{(3)}_{3,8}(Q^2_0)
\end{array}
\end{equation}
\item $\mathbf{V_{15}}$:
\begin{equation}
\begin{array}{rcl}
V^{(6)}_{15}(Q^2)&=&\displaystyle \Big\{\Gamma^{-}(Q^2,m_t^2)[1+a_s^2(m_t^2)A_{qq,t}^{N\!S,(2)}]\times\\
\\
& & \displaystyle \Gamma^{-}(m_t^2,m_b^2)[1+a_s^2(m_b^2)A_{qq,b}^{N\!S,(2)}]\Gamma^{-}(m_b^2,m_c^2)\\ 
\\
& & \displaystyle [1+a_s^2(m_c^2)A_{qq,c}^{N\!S,(2)}]\Gamma^{v}(m_c^2,Q_0^2)\Big\}V^{(3)}(Q^2_0)
\end{array}
\end{equation}
\item $\mathbf{V_{24}}$:
\begin{equation}
\begin{array}{rcl}
V^{(6)}_{15}(Q^2)&=&\displaystyle \Big\{\Gamma^{-}(Q^2,m_t^2)[1+a_s^2(m_t^2)A_{qq,t}^{N\!S,(2)}]\times\\
\\
& & \displaystyle \Gamma^{-}(m_t^2,m_b^2)[1+a_s^2(m_b^2)A_{qq,b}^{N\!S,(2)}]\Gamma^{v}(m_b^2,m_c^2)\\ 
\\
& & \displaystyle [1+a_s^2(m_c^2)A_{qq,c}^{N\!S,(2)}]\Gamma^{v}(m_c^2,Q_0^2)\Big\}V^{(3)}(Q^2_0)
\end{array}
\end{equation}
\item $\mathbf{V_{35}}$:
\begin{equation}
\begin{array}{rcl}
V^{(6)}_{15}(Q^2)&=&\displaystyle \Big\{\Gamma^{-}(Q^2,m_t^2)[1+a_s^2(m_t^2)A_{qq,t}^{N\!S,(2)}]\times\\
\\
& & \displaystyle \Gamma^{v}(m_t^2,m_b^2)[1+a_s^2(m_b^2)A_{qq,b}^{N\!S,(2)}]\Gamma^{v}(m_b^2,m_c^2)\\ 
\\
& & \displaystyle [1+a_s^2(m_c^2)A_{qq,c}^{N\!S,(2)}]\Gamma^{v}(m_c^2,Q_0^2)\Big\}V^{(3)}(Q^2_0)
\end{array}
\end{equation}
\item $\mathbf{T_3}$ and $\mathbf{T_8}$:
\begin{equation}
\begin{array}{rcl}
T^{(6)}_{3,8}(Q^2)&=&\displaystyle \Big\{\Gamma^{+}(Q^2,m_t^2)[1+a_s^2(m_t^2)A_{qq,t}^{N\!S,(2)}]\times\\
\\
& & \displaystyle \Gamma^{+}(m_t^2,m_b^2)[1+a_s^2(m_b^2)A_{qq,b}^{N\!S,(2)}]\Gamma^{+}(m_b^2,m_c^2)\\
\\
& & \displaystyle [1+a_s^2(m_c^2)A_{qq,c}^{N\!S,(2)}]\Gamma^{+}(m_c^2,Q_0^2)\Big\}T^{(3)}_{3,8}(Q^2_0)
\end{array}
\end{equation}

\item $\mathbf{T_{15}}$:
\begin{equation}
\begin{array}{rcl}
T^{(6)}_{15}(Q^2)&=&\displaystyle \Big\{\Gamma^{+}(Q^2,m_t^2)[1+a_s^2(m_t^2)A_{qq,t}^{N\!S,(2)}]\times\\
\\
& & \displaystyle \Gamma^{+}(m_t^2,m_b^2)[1+a_s^2(m_b^2)A_{qq,b}^{N\!S,(2)}]\Gamma^{+}(m_b^2,m_c^2)\\
\\
& & \displaystyle \begin{pmatrix} 1+a_s^2(m_c^2)[A_{qq,c}^{N\!S,(2)}-3\tilde{A}^{S,(2)}_{cq}] & -3a_s^2(m_c^2)\tilde{A}^{S,(2)}_{cg}\end{pmatrix}\underbrace{\begin{pmatrix}\Gamma_{qq}& \Gamma_{qg} \\ \Gamma_{gq} & \Gamma_{gg}\end{pmatrix}}_{(m_c^2,Q_0^2)}\Bigg\}{\Sigma^{(3)}(Q_0^2) \choose g^{(3)}(Q_0^2)}
\end{array}
\end{equation}
\item $\mathbf{T_{24}}$:
\begin{equation}
\begin{array}{rcl}
T^{(6)}_{24}(Q^2)&=&\displaystyle \Big\{\Gamma^{+}(Q^2,m_t^2)[1+a_s^2(m_t^2)A_{qq,t}^{N\!S,(2)}]\Gamma^{+}(m_t^2,m_b^2)\times\\
\\
& & \displaystyle \begin{pmatrix} 1+a_s^2(m_b^2)[A_{qq,b}^{N\!S,(2)}-4\tilde{A}^{S,(2)}_{bq}] & -4a_s^2(m_b^2)\tilde{A}^{S,(2)}_{bg}\end{pmatrix}\underbrace{\begin{pmatrix}\Gamma_{qq}& \Gamma_{qg} \\ \Gamma_{gq} & \Gamma_{gg}\end{pmatrix}}_{(m_b^2,m_c^2)}\times\\
\\
 & &\displaystyle\begin{pmatrix} M_{11}^c & M_{12}^c \\ M_{21}^c & M_{22}^c\end{pmatrix}\underbrace{\begin{pmatrix} \Gamma_{qq}& \Gamma_{qg} \\ \Gamma_{gq}& \Gamma_{gg}\end{pmatrix}}_{(m_c^2,Q_0^2)}\Bigg\}{\Sigma^{(3)}(Q_0^2) \choose g^{(3)}(Q_0^2)}
\end{array}
\end{equation}
\item $\mathbf{T_{35}}$:
\begin{equation}
\begin{array}{rcl}
T^{(6)}_{35}(Q^2)&=&\displaystyle \Big\{\Gamma^{+}(Q^2,m_t^2)\begin{pmatrix} 1+a_s^2(m_b^2)[A_{qq,t}^{N\!S,(2)}-5\tilde{A}^{S,(2)}_{tq}] & -5a_s^2(m_t^2)\tilde{A}^{S,(2)}_{tg}\end{pmatrix}\times\\
\\
& &\displaystyle\begin{pmatrix} M_{11}^b & M_{12}^b \\ M_{21}^b & M_{22}^b\end{pmatrix}\underbrace{\begin{pmatrix} \Gamma_{qq}& \Gamma_{qg} \\ \Gamma_{gq}& \Gamma_{gg}\end{pmatrix}}_{(m_b^2,m_c^2)}\\
\\
& &\displaystyle\begin{pmatrix} M_{11}^c & M_{12}^c \\ M_{21}^c & M_{22}^c\end{pmatrix}\underbrace{\begin{pmatrix} \Gamma_{qq}& \Gamma_{qg} \\ \Gamma_{gq}& \Gamma_{gg}\end{pmatrix}}_{(m_c^2,Q_0^2)}\Bigg\}{\Sigma^{(3)}(Q_0^2) \choose g^{(3)}(Q_0^2)}
\end{array}
\end{equation}
\end{itemize}

\section{Matching conditions with intrinsic distributions}

The (quite outdated) discussion presented above concerning the PDF
matching conditions is restricted to forward evolution (\textit{i.e.}
$Q>Q_0$) and under the assumption that possible intrinsic heavy-quark
contributions are absent.\footnote{In fact, the discussion also
  assumes that the PDF matching takes place at the heavy quark-mass
  value. This is also a non-necessary assumption that we will however
  not discuss here.} This corresponds to assuming that heavy-quark
PDFs are entirely dynamically generated at the respective threshold
and are identically vanishing below threshold. While this turns out to
be a good approximation for PDFs, this is not the case for
fragmentation functions (FFs). Indeed, intrinsic heavy-quark FFs are
sizeable. In addition, assuming that there are no heavy-quark PDFs
makes backward evolution problematic. The basic problem is that the
matching-function matrix is non-squared which forbids its inversion
thus complicating backward evolution.

The purpose of this section is the relaxation of the no-intrinsic-PDF
assumption. This will imply the introduction of new matching functions
due to the presence match heavy-quark PDFs below
threshold. Importantly, the perturbative expansion of these additional
functions is different from zero at $\mathcal{O}(\alpha_s)$ even
choosing $\mu_h=m_h$, with $\mu_h$ the matching scale, producing a
discontinuity already at NLO.  However, in what follows we will assume
that matching conditions are not computed beyond
$\mathcal{O}(\alpha_s^2)$, \textit{i.e.}  NNLO, and that intrinsic
heavy-quark and heavy-antiquark contributions are equal below
threshold. These assumptions lead to:
\begin{equation}
h^{(n_f)}(\mu_h) = \overline{h}^{(n_f)}(\mu_h)\quad\mbox{and}\quad h^{(n_f+1)}(\mu_h) = \overline{h}^{(n_f+1)}(\mu_h)\,,
\end{equation}
where we dropped the non-scale dependence. It immediately follows that
the valence distributions $\{V,V_3,V_8,V_{15},V_{24},V_{35}\}$ match
multiplicatively as:\footnote{Note that ``1'' in
  Eq.~(\ref{eq:nsingletmc}) is to be understood as $\delta(1-x)$ in
  $x$ space.}
\begin{equation}\label{eq:nsingletmc}
V_i^{(n_f+1)}(\mu_h) = [1+K_{ll}](\mu_h) V^{(n_f)}(\mu_h) = [1+K_{ll}(\mu_h)] V_i^{(n_f)}(\mu_h)\,,\quad i = 3,8,15,24,35\,,
\end{equation}
with $K_{ll}$ some given function that starts at
$\mathcal{O}(\alpha_s^2)$. This equivalently amounts to say that no
non-perturbative quark-antiquark asymmetry is originally present nor
it is generated by the matching procedure.\footnote{The assumption of
  absence of non-perturbative quark-antiquark asymmetry is badly
  violated for light-valence distributions, such as up and down in the
  proton. However, we will never realistically need to match PDFs or
  FFs at the light-quark scales. On the other hand, this assumption
  helps us keep the discussion general.}

This allows us to concentrate the discussion on the singlet sector. We
start by generalising Eqs.~(\ref{eq:lightqmc}), (\ref{gluon}),
and~(\ref{eq:heavyqmc}) (changing and simplifying a bit the notation)
as:
\begin{equation}\label{eq:IntMatchingConds}
\begin{array}{rcl}
  \displaystyle g^{(n_f+1)}&=&\displaystyle
                               [1+K_{gg}]g^{(n_f)}+K_{gl}\sum_l
                               l^{+(n_f+1)} + K_{gh}h^{+(n_f)}\,,\\
  \\
  \displaystyle l^{+(n_f+1)}&=&\displaystyle [1+K_{ll}]l^{+(n_f)}\,,\\
  \\
  \displaystyle h^{+(n_f+1)}&=&\displaystyle K_{hg}g^{(n_f)} + K_{hl}\sum_l
                                l^{+(n_f+1)}+[1+K_{hh}]h^{+(n_f)}\,,\\
  \\
  H^{+(n_f+1)} &=& H^{+(n_f)}\,,
\end{array}
\end{equation}
where $q^+=q+\overline{q}$ and $l$ runs between $1$ and $n_f$. We
remind that $h$ is the $(n_f+1)$-th heavier flavour, \textit{i.e.}
the one that becomes active at the threshold under consideration. We
have also introduced the ``super'' heavy quark flavour $H$, heavier
than $h$, that is not affected by the matching at the $n_f$-th
threshold. Note that there are two new functions coming into play,
namely $K_{gh}$ and $K_{hh}$, both multiplying the intrinsic
heavy-quark contribution $h^{+(n_f)}$. Moreover, we have written
Eq.~(\ref{eq:IntMatchingConds}) in such a way that the coefficients
$K$ start at least at $\mathcal{O}(\alpha_s)$. In fact, we have
already seen that $K_{ll}=\mathcal{O}(\alpha_s^2)$; the same applies
to $K_{hl}$ and $K_{gl}$. Defining the column vector $P$ with
components $(g,d^+,u^+,s^+,c^+,b^+,t^+)$, we can represent the
matching in a matricial form as:
\begin{equation}
P^{(n_f+1)} = \left[\mathbb{I}+\mathbb{K}^{(n_f)}\right]P^{(n_f)}\,,
\end{equation}
where $\mathbb{I}$ is the $7\times7$ unity matrix and
$\mathbb{K}^{(n_f)}$ are given, depending on the value of $n_f$, by:
\begin{equation}\label{eq:MathcingMatrices}
\begin{array}{ll}
\mathbb{K}^{(0)} = 
\begin{pmatrix}
K_{gg} & K_{gh} & 0  & 0  & 0  & 0  & 0 \\
K_{hg} & K_{hh} & 0  & 0  & 0  & 0  & 0 \\
0 & 0 & 0  & 0  & 0  & 0  & 0 \\
0 & 0 & 0  & 0  & 0  & 0  & 0 \\
0 & 0 & 0  & 0  & 0  & 0  & 0 \\
0 & 0 & 0  & 0  & 0  & 0  & 0 \\
0 & 0 & 0  & 0  & 0  & 0  & 0
\end{pmatrix}
&
\mathbb{K}^{(1)} = 
\begin{pmatrix}
K_{gg} & K_{gl} & K_{gh} & 0  & 0  & 0  & 0 \\
0 & K_{ll} & 0 & 0  & 0  & 0  & 0  \\
K_{hg} & K_{hl} & K_{hh}  & 0  & 0  & 0  & 0 \\
0 & 0 & 0  & 0  & 0  & 0  & 0 \\
0 & 0 & 0  & 0  & 0  & 0  & 0 \\
0 & 0 & 0  & 0  & 0  & 0  & 0 \\
0 & 0 & 0  & 0  & 0  & 0  & 0
\end{pmatrix}
\\ \\
\mathbb{K}^{(2)} = 
\begin{pmatrix}
K_{gg} & K_{gl} & K_{gl} & K_{gh} & 0  & 0  & 0 \\
0 & K_{ll} & 0 & 0  & 0  & 0  & 0  \\
0 & 0 & K_{ll} & 0  & 0  & 0  & 0  \\
K_{hg} & K_{hl} & K_{hl} & K_{hh}  & 0  & 0  & 0 \\
0 & 0 & 0  & 0  & 0  & 0  & 0 \\
0 & 0 & 0  & 0  & 0  & 0  & 0 \\
0 & 0 & 0  & 0  & 0  & 0  & 0
\end{pmatrix}
&
\mathbb{K}^{(3)} = 
\begin{pmatrix}
K_{gg} & K_{gl} & K_{gl} & K_{gl} & K_{gh}  & 0  & 0 \\
0 & K_{ll} & 0 & 0  & 0  & 0  & 0  \\
0 & 0 & K_{ll} & 0  & 0  & 0  & 0  \\
0 & 0 & 0 & K_{ll} & 0  & 0  & 0  \\
K_{hg} & K_{hl} & K_{hl} & K_{hl} & K_{hh}  & 0  & 0 \\
0 & 0 & 0  & 0  & 0  & 0  & 0 \\
0 & 0 & 0  & 0  & 0  & 0  & 0
\end{pmatrix}
\\ \\
\mathbb{K}^{(4)} = 
\begin{pmatrix}
K_{gg} & K_{gl} & K_{gl} & K_{gl} & K_{gl} & K_{gh} & 0  \\
0 & K_{ll} & 0 & 0  & 0  & 0  & 0  \\
0 & 0 & K_{ll} & 0  & 0  & 0  & 0  \\
0 & 0 & 0 & K_{ll} & 0  & 0  & 0  \\
0 & 0 & 0 & 0 & K_{ll}  & 0  & 0  \\
K_{hg} & K_{hl} & K_{hl} & K_{hl} & K_{hl} & K_{hh}  & 0 \\
0 & 0 & 0  & 0  & 0  & 0  & 0
\end{pmatrix}
&
\mathbb{K}^{(5)} = 
\begin{pmatrix}
K_{gg} & K_{gl} & K_{gl} & K_{gl} & K_{gl} & K_{gl} & K_{gh} \\
0 & K_{ll} & 0 & 0 & 0  & 0  & 0  \\
0 & 0 & K_{ll} & 0 & 0  & 0  & 0  \\
0 & 0 & 0 & K_{ll} & 0  & 0  & 0  \\
0 & 0 & 0 & 0 & K_{ll} & 0  & 0  \\
0 & 0 & 0 & 0 & 0 & K_{ll} & 0  \\
K_{hg} & K_{hl} & K_{hl} & K_{hl} & K_{hl} & K_{hl} & K_{hh} \\
\end{pmatrix}
\end{array}
\end{equation}
The algorithm to determine the matrix $\mathbb{K}^{(n_f)}$, indexing
the gluon distribution with 0 and the quark distributions from 1 to 6,
reads:
\begin{equation}\label{eq:algorithm}
\mathbb{K}_{ij}^{(n_f)} =
\left\{
\begin{array}{ll}
K_{gg} & \quad i = j = 0\\
K_{gl} & \quad i = 0\,,\;1\leq j \leq n_f\\
K_{gh} & \quad i = 0\,,\;j = n_f+1\\
K_{ll} & \quad i = j\,,\;1\leq i,j \leq n_f\\
K_{hg} & \quad i = n_f+1\,,\;j = 0\\
K_{hl} & \quad i = n_f+1\,,\;1\leq j \leq n_f\\
K_{hh} & \quad i = j = n_f+1\\
0 & \quad\mbox{elsewhere}
\end{array}
\right.\,.
\end{equation}

Now, in order to apply the matching condition to the distributions in
the evolution basis $E=(g,\Sigma,-T_3,T_8,T_{15},T_{24},T_{35})$, it is
necessary to rotate the matching matrices from the physical basis to
the evolution basis through $T\mathbb{K}^{(n_f)}T^{-1}$, where $T$
transforms the vector $P$ into the evolution-basis vector $E$:
\begin{equation}\label{eq:RotationMatrix}
T = 
\begin{pmatrix}
1 & 0 & 0  & 0  & 0  & 0  & 0 \\
0 & 1 & 1  & 1  & 1  & 1  & 1 \\
0 & 1 & -1  & 0  & 0  & 0  & 0 \\
0 & 1 & 1  & -2  & 0  & 0  & 0 \\
0 & 1 & 1  & 1  & -3  & 0  & 0 \\
0 & 1 & 1  & 1  & 1  & -4  & 0 \\
0 & 1 & 1  & 1  & 1  & 1  & -5
\end{pmatrix}\,.
\end{equation}
The resulting matching matrices are much more complicated and
generally less sparse than those in
Eq.~(\ref{eq:MathcingMatrices}). However, we can still compute them
algorithmically depending on the number of light flavours $n_f$. To do
so, we define the following linearly independent combinations:
\begin{equation}
\footnotesize
\begin{array}{lcl}
M_1 &=& K_{gg}\,,\\
M_2 &=& K_{gh} + n_f K_{gl}\,,\\
M_3 &=& K_{gh}-K_{gl}\,,\\
M_4 &=& K_{hg}\,,\\
M_5 &=& K_{hh}+n_f(K_{hl}+K_{ll}) \,,\\
M_6 &=& K_{hh}-(K_{hl}+K_{ll})\,,\\
M_7 &=& K_{ll}\,,
\end{array}
\end{equation}
such that:
\begin{equation}\label{eq:algorithmEv}
\footnotesize
\left[T\mathbb{K}^{(n_f)}T^{-1}\right]_{ij} =
\left\{
\begin{array}{lll}
M_1 & \quad i = 0 &\quad j = 0\\
\frac{1}{6}M_2 & \quad i = 0 &\quad  j = 1\\
0 & \quad i = 0 &\quad  2\leq j \leq n_f\\
-\frac{1}{n_f+1}M_3 & \quad i = 0 &\quad  j = n_f+1\\
\frac{1}{j(j-1)}M_2 & \quad i = 0 &\quad  n_f+ 2\leq j \leq 6\\
\\
M_4 & \quad i = 1&\quad  j=0\\
\frac16 M_5 & \quad i = 1 &\quad  j=1\\
0 & \quad i = 1 &\quad  2\leq j \leq n_f\\
-\frac{1}{n_f+1}M_6 & \quad i = 1 &\quad  j = n_f+1\\
\frac{1}{j(j-1)}M_5 & \quad i = 1 &\quad  n_f+
                                                            2\leq j
                                                            \leq 6\\
\\
\delta_{ij}M_7&\quad 2\leq i \leq n_f &\quad 0\leq j
                                                         \leq 6\\
\\
-n_fM_4 &\quad i = n_f+1 &\quad j =0\\
\frac{n_f}6\left(-M_5+(n_f+1)M_7\right) &\quad i = n_f+1 &\quad j =1\\
0 & \quad i = n_f+1 &\quad  2\leq j \leq n_f\\
\frac1{n_f+1}\left(n_f M_6+(n_f+1)M_7\right) &\quad i = n_f+1 &\quad j =n_f+1\\
\frac{n_f}{j(j-1)}\left( -M_5+(n_f+1)M_7\right) &\quad i = n_f+1
                  &\quad n_f+2 \leq j \leq 6\\
\\
M_4 & \quad n_f+2\leq i \leq 6 &\quad  j=0\\
\frac16 M_5 & \quad n_f+2\leq i \leq 6 &\quad  j=1\\
0 & \quad n_f+2\leq i \leq 6 &\quad  2\leq j \leq n_f\\
-\frac{1}{n_f+1}M_6 & \quad n_f+2\leq i \leq 6 &\quad  j = n_f+1\\
\frac{1}{j(j-1)}M_5 & \quad n_f+2\leq i \leq 6 &\quad  n_f+
                                                            2\leq j
                                                            \leq 6\\
\end{array}
\right.
\end{equation}

In order to be able to perform the backward evolution, it is necessary
to invert the matching matrix $\mathbb{I}+\mathbb{K}^{(n_f)}$. This is
conveniently done perturbatively by expanding the matrix
$\left[\mathbb{I}+\mathbb{K}^{(n_f)}\right]^{-1}$ in powers of
$\alpha_s$ and truncating the expansion at the appropriate
order. Given the assumption discussed above, we are not interested in
going beyond $\mathcal{O}(\alpha_s^2)$. In addition, we know that:
\begin{equation}
\mathbb{K}^{(n_f)} = a_s \mathbb{K}^{(1)(n_f)}+a_s^2 \mathbb{K}^{(2)(n_f)}+\mathcal{O}(\alpha_s^3)\,,
\end{equation}
so that:
\begin{equation}\label{eq:inversion}
\left[\mathbb{I}+\mathbb{K}^{(n_f)}\right]^{-1} = 1 - a_s \mathbb{K}^{(1)(n_f)}-a_s^2\left[\mathbb{K}^{(2)(n_f)}-\left(\mathbb{K}^{(1)(n_f)}\right)^2\right]+\mathcal{O}(\alpha_s^3)\,.
\end{equation}
But the matrix $\mathbb{K}^{(1)(n_f)}$ is particularly simple because
at $\mathcal{O}(\alpha_s)$ the matching functions $K_{ll}$, $K_{hl}$,
and $K_{gl}$ vanish. Therefore, using Eq.~(\ref{eq:algorithm}) we
find:
\begin{equation}
\mathbb{K}_{ij}^{(1)(n_f)} =\delta_{i,0}\left(\delta_{j,0}K_{gg}^{(1)} +
\delta_{j,n_f+1}K_{gh}^{(1)}\right)+ \delta_{i,n_f+1}\left(\delta_{j,0}K_{hg}^{(1)}
+ \delta_{j,n_f+1}K_{hh}^{(1)}\right)\,,
\end{equation}
such that its square has the same structure and can be written as:
\begin{equation}\label{eq:squareK1}
\begin{array}{rcl}
\left(\mathbb{K}^{(1)(n_f)}\right) _{ij}^2 &=&\displaystyle \delta_{i,0}\left(\delta_{j,0}\sum_{k=g,h}K_{gk}^{(1)} K_{kg}^{(1)} +
\delta_{j,n_f+1}\sum_{k=g,h}K_{gk}^{(1)} K_{kh}^{(1)}\right)\\
\\
  &+&\displaystyle  \delta_{i,n_f+1}\left(\delta_{j,0}\sum_{k=g,h}K_{hk}^{(1)} K_{kg}^{(1)}
+ \delta_{j,n_f+1}\sum_{k=g,h}K_{hk}^{(1)} K_{kh}^{(1)}\right)\,,
\end{array}
\end{equation}
where, in $x$ space, the product of two $K$ functions is to be
understood as a convolution. We report below the expression in $x$
space of the functions $K_{gh}^{(1)}$ and $K_{gh}^{(1)}$ for the
matching of PDFs:
\begin{equation}
\begin{array}{rcl}
\displaystyle K_{gh}^{(1)}(x) &=&\displaystyle 
2C_F\frac{1+(1-x)^2}{x}\left(\ln\frac{\mu_h^2}{m_h^2}-1-2\ln x\right)\,,\\
\\
\displaystyle K_{hh}^{(1)}(x) &=&\displaystyle 
2C_F\left[\frac{1+x^2}{1-x}\left(\ln\frac{\mu_h^2}{m_h^2}-1-2\ln(1-x)\right)\right]_+\,.
\end{array}
\end{equation}
We notice that the $\mathcal{O}(\alpha_s^2)$ correction to these
functions is currently unknown, therefore we only use the
$\mathcal{O}(\alpha_s)$ expressions. For the sake of completeness, we
also report here the remaining $\mathcal{O}(\alpha_s)$ matching
functions:
\begin{equation}
\begin{array}{rcl}
\displaystyle K_{gg}^{(1)}(x) &=&\displaystyle -\frac{4}{3}T_R\delta(1-x)\ln\frac{\mu_h^2}{m_h^2}\,,\\
\\
\displaystyle K_{hg}^{(1)}(x) &=&\displaystyle 4T_R\left[x^2+(1-x)^2\right]\ln\frac{\mu_h^2}{m_h^2}\,,
\end{array}
\end{equation}
Of course, the matrix
$\left[\mathbb{I}+\mathbb{K}^{(n_f)}\right]^{-1}$ in
Eq.~(\ref{eq:inversion}) has to be rotated into the evolution basis by
means of the transformation matrix $T$ in
Eq.~(\ref{eq:RotationMatrix})

In order to implement the possibility to compute the evolution
operator evolution in the VFNS, we need to know how the evolution
operator for a given number of active flavours $n_f$ matches at the
following threshold. The first step is to generalise the structure of
the DGLAP evolution equations in the QCD evolution basis allowing for
the presence of (scale-independent) intrinsic contributions. It is
important to realise that DGLAP equations govern the evolution in
$\mu$ of the collinear distributions but do not give us any
information on the non-dynamic part of these distributions,
\textit{i.e.} its intrinsic contribution. The starting point is the
DGLAP equation for the quark distribution $q_k^+$\footnote{We remain
  under the assumption that intrinsic contributions are symmetric upon
  charge conjugation. This allow us to forget about the valence sector
  that will always evolve multiplicatively.} in its basic form:
\begin{equation}
\frac{dq_k^+}{d\ln\mu^2} =\frac{dq_k^+}{dt}= 2P_{qg}\otimes g+
(P_{qq}^V+P_{q\overline{q}}^V)\otimes
q_k^++(P_{qq}^S+P_{q\overline{q}}^S)\otimes \sum_{i=1}^{n_f} q_i^+\,.
\end{equation}
It is crucial to notice that the sum on the r.h.s. of the equation
above runs over the $n_f$ active flavours only. Now, we sum the index
$k$ up to $n_f$ and we get:
\begin{equation}
\frac{d}{dt}\sum_{k=1}^{n_f} q_k^+= 2n_f P_{qg}\otimes g+
P_{qq}\otimes \sum_{k=1}^{n_f} q_k^+\,,
\end{equation}
where we have defined:
\begin{equation}
P_{qq} = (P_{qq}^V+P_{q\overline{q}}^V)+n_f(P_{qq}^S+P_{q\overline{q}}^S).
\end{equation}
In the presence of intrinsic heavy-quark distributions and after some
algebra, one finds that:
\begin{equation}\label{eq:singletDGLAP}
\frac{d\Sigma}{dt}= 2n_f P_{qg}\otimes g+n_fP_{qq}\otimes \left(\frac{\Sigma}{6}+\sum_{j=n_f+1}^6\frac{E_j}{j(j-1)}\right)\,,
\end{equation}
with $E_j\in \{\Sigma,-T_3,T_8,T_{15},T_{24},T_{35}\}$. Therefore, the
singlet distribution, on top of the gluon, couples also to the other
quark distributions. Notice that:
\begin{equation}
\sum_{j=n_f+1}^6\frac{1}{j(j-1)} = \frac{1}{n_f}-\frac16\,,
\end{equation}
such that, for $E_j = \Sigma$, Eq.~(\ref{eq:singletDGLAP}) reproduces
the usual form of the DGLAP equation for the singlet. We now need to
determine how the distributions $E_j$, for $j>1$,
evolve. We find:
\begin{equation}\label{eq:nonsingletDGLAP}
  \frac{dE_j}{dt}= (1-\theta_{n_fj})\left[2n_f P_{qg}\otimes
    g+n_fP_{qq}\otimes
    \left(\frac{\Sigma}{6}+\sum_{i=n_f+1}^6\frac{E_i}{i(i-1)}\right)\right]+\theta_{n_fj}P^+\otimes
  E_j\,,
\end{equation}
with:
\begin{equation}
\theta_{n_fj}=\left\{
\begin{array}{ll}
1   & \quad n_f\geq j\\
0 & \quad n_f < j
\end{array}
\right.\,. 
\end{equation}
Therefore, as expected, for a given $n_f$ (and thus a given energy)
the distribution $E_j$ evolves multiplicatively through $P^+$ if
$j\leq n_f$ and evolves exactly like the singlet if $j>n_f$. As an
example, for $n_f=4$, that is to say for energies between the charm
and the bottom thresholds, $E_2=T_3$, $E_3=T_8$, and $E_4=T_{15}$
evolve multiplicatively, while $E_5=T_{24}$ and $E_6=T_{35}$ evolve
like the singlet.

Finally, the gluon distribution evolves as:
\begin{equation}
\frac{dg}{dt}= P_{gg}\otimes g+
P_{gq}\otimes \sum_{i=1}^{n_f} q_i^+\,,
\end{equation}
that translates into:
\begin{equation}\label{eq:gluonDGLAP}
\frac{dg}{dt}= P_{gg}\otimes g+n_fP_{gq}\otimes \left(\frac{\Sigma}{6}+\sum_{j=n_f+1}^6\frac{E_j}{j(j-1)}\right)\,.
\end{equation}
Eqs.~(\ref{eq:singletDGLAP}), (\ref{eq:nonsingletDGLAP}), and
(\ref{eq:gluonDGLAP}) fully define the splitting kernel matrix in the
general case of possible presence of intrinsic heavy-quark
contributions. Noticeably, these equations correctly reduce to the
more familiar ones if these contributions are set to zero.

Therefore, the evolution of the vector of distributions $E$ with $n_f$
active flavours and possible intrinsic heavy-quark contributions takes
the matricial form:
\begin{equation}\label{eq:DGLAPE}
\frac{dE}{dt} = \mathbb{P}^{(n_f)}\otimes E\,,
\end{equation}
where the entries of the splitting-function matrix
$\mathbb{P}^{(n_f)}$ can be defined algorithmically as:
\begin{equation}\label{eq:splittingalg}
\mathbb{P}_{ij}^{(n_f)}=
\left\{
\begin{array}{lll}
P_{gg} & \quad i = 0 &\quad j = 0\\
\frac{n_f}{6}P_{gq} & \quad i = 0 & \quad j=1\\
0 &\quad i = 0 & \quad 2 \leq j \leq n_f\\
\frac{6}{j(j-1)}\frac{n_f}{6}P_{gq} &\quad i = 0 & \quad n_f+1 \leq j \leq 6\\
\\
2n_fP_{qg} & \quad i = 1 &\quad j = 0\\
\frac{n_f}{6}P_{qq} & \quad i = 1 & \quad j=1\\
0 &\quad i = 1 & \quad 2 \leq j \leq n_f\\
\frac{6}{j(j-1)}\frac{n_f}{6}P_{qq} &\quad i = 1 & \quad n_f+1 \leq j \leq 6\\
\\
\delta_{ij}P^+ & \quad 2 \leq i \leq n_f & \quad 0 \leq j \leq 6\\
\\
2n_fP_{qg} & \quad n_f+1 \leq i \leq 6  &\quad j = 0\\
\frac{n_f}{6}P_{qq} & \quad n_f+1 \leq i \leq 6  & \quad j=1\\
0 &\quad n_f+1 \leq i \leq 6 & \quad 2 \leq j \leq n_f\\
\frac{6}{j(j-1)}\frac{n_f}{6}P_{qq} &\quad n_f+1 \leq i \leq 6 & \quad n_f+1 \leq j \leq 6\\
\end{array}
\right.\,.
\end{equation}

\subsection{Evolution operator}

We assume that the set of distributions $E$ evolves through the
operator ${\bm\Gamma}$ as:
\begin{equation}
  E(t) = {\bm\Gamma}(t,t_0)\otimes E(t_0)\,.
\end{equation}
In order to achieve an efficient computation of the evolution
operator, we need to identify the general structure of ${\bm \Gamma}$.
Given the possible presence of heavy-quark thresholds in the evolution
interval $[t_0,t]$, the operator ${\bm\Gamma}$ is in fact a product of
operators. More specifically. it can be written as:
\begin{equation}\label{eq:conbevop}
{\bm\Gamma}(t,t_0)= {\bm\Gamma}^{(N)}(t,t_N) \otimes\prod_{n_f=N-1}^{0} \mathcal{M}^{(n_f)}\otimes{\bm\Gamma}^{(n_f)}(t_{n_f+1}-\epsilon,t_{n_f})
\end{equation}
with
$\mathcal{M}^{(n_f)} =
T\left[\mathbb{I}+\mathbb{K}^{(n_f)}\right]T^{-1}$
given in Eq.~(\ref{eq:algorithmEv}) and $t>t_N>\dots>t_1>t_0$, being
$t_1,\dots,t_N$ the heavy-quark thresholds enclosed in the evolution
interval. Using Eq.~(\ref{eq:DGLAPE}), each evolution operator
${\bm\Gamma}^{(n_f)}$ obeys the evolution equation:
\begin{equation}\label{eq:DGLAPG}
  \frac{d}{dt}{\bm\Gamma}^{(n_f)}(t,t_{n_f}) = \mathbb{P}^{(n_f)}(t)\otimes {\bm\Gamma}^{(n_f)}(t,t_{n_f})\,,
\end{equation}
with boundary condition
${\bm\Gamma}^{(n_f)}(t_{n_f},t_{n_f})=\mathbb{I}$. The solution of the
equation above can then be written as:
\begin{equation}
  {\bm\Gamma}^{(n_f)}(t,t_{n_f}) = \mathcal{P}\exp\left[\int_{t_{n_f}}^tdt'\,\mathbb{P}^{(n_f)}(t')\right]\,,
\end{equation}
where $\mathcal{P}$ symbolises the path ordering. Given the structure
of the splitting-function matrix in Eq.~(\ref{eq:splittingalg}), one
can infer the structure of ${\bm \Gamma}^{(n_f)}$. It turns out that
the structure of ${\bm \Gamma^{(n_f)}}$ is almost the same as that of
$\mathbb{P}^{(n_f)}$. Specifically:
\begin{equation}\label{eq:evolopalg}
  {\bm\Gamma}^{(n_f)}_{ij}=\mathbb{I}_{ij}+
  \left\{
\begin{array}{lll}
\Gamma_{gg} & \quad i = 0 &\quad j = 0\\
\Gamma_{gq} & \quad i = 0 & \quad j=1\\
0 &\quad i = 0 & \quad 2 \leq j \leq n_f\\
\frac{6}{j(j-1)}\Gamma_{gq} &\quad i = 0 & \quad n_f+1 \leq j \leq 6\\
\\
\Gamma_{qg} & \quad i = 1 & \quad j = 0\\
\Gamma_{qq} & \quad i = 1 & \quad j = 1\\
0 &\quad i = 1 & \quad 2 \leq j \leq n_f\\
\frac{6}{j(j-1)}\Gamma_{qq} &\quad i = 1 & \quad n_f+1 \leq j \leq 6\\
\\
\delta_{ij}\Gamma^+ & \quad 2 \leq i \leq n_f & \quad 0 \leq j \leq 6\\
\\
\Gamma_{qg} & \quad n_f+1 \leq i \leq 6  &\quad j = 0\\
\Gamma_{qq} & \quad n_f+1 \leq i \leq 6  & \quad j=1\\
0 &\quad n_f+1 \leq i \leq 6 & \quad 2 \leq j \leq n_f\\
\frac{6}{j(j-1)}\Gamma_{qq} &\quad n_f+1 \leq i \leq 6 & \quad n_f+1 \leq j \leq 6\\
\end{array}
\right.\,.
\end{equation}
% Using Eqs.~(\ref{eq:splittingalg}) and~(\ref{eq:evolopalg}), we can
% work out the product $\mathbb{P}^{(n_f)}{\bm\Gamma}^{(n_f)}$ required
% for the computation of the evolution operator. This reads:
% \begin{equation}\label{eq:PGamma}
% \left[\mathbb{P}^{(n_f)}{\bm\Gamma}^{(n_f)}\right]_{ij}=\mathbb{P}^{(n_f)}_{ij}+
% \left\{
% \begin{array}{lll}
% P_{gg}\Gamma_{gg}+P_{gq}\Gamma_{qg} & \quad i = 0 &\quad j = 0\\
% P_{gg}\Gamma_{gq}+P_{gq}\Gamma_{qq} & \quad i = 0 & \quad j=1\\
% 0 &\quad i = 0 & \quad 2 \leq j \leq n_f\\
% \frac{6}{j(j-1)}\left[ P_{gg}\Gamma_{gq}+P_{gq}\Gamma_{qq}\right]&\quad i = 0 & \quad n_f+1 \leq j \leq 6\\
% \\
% 2n_fP_{qg}\Gamma_{gg}+P_{qq}\Gamma_{qg} & \quad i = 1 &\quad j = 0\\
% 2n_fP_{qg}\Gamma_{gq}+P_{qq}\Gamma_{qq} & \quad i = 1 & \quad j=1\\
% 0 &\quad i = 1 & \quad 2 \leq j \leq n_f\\
% \frac{6}{j(j-1)} \left[2n_fP_{qg}\Gamma_{gq}+P_{qq}\Gamma_{qq}\right]&\quad i = 1 & \quad n_f+1 \leq j \leq 6\\
% \\
% \delta_{ij}P^+\Gamma^+ & \quad 2 \leq i \leq n_f & \quad 0 \leq j \leq 6\\
% \\
% \frac{6}{j(j-1)}\left[2n_fP_{qg}\Gamma_{gq}+P_{qq}\Gamma_{qq} \right]&\quad n_f+1 \leq i \leq 6 & \quad n_f+1 \leq j \leq 6\\
% \end{array}
% \right.\,.
% \end{equation}

The next step is the computation of the product
$\mathcal{M}^{(n_f)}{\bm\Gamma}^{(n_f)}=T\left[\mathbb{I}+\mathbb{K}^{(n_f)}\right]T^{-1}{\bm\Gamma}^{(n_f)}$
appearing in Eq.~(\ref{eq:conbevop}) that we conveniently split as:
\begin{equation}\label{eq:nthfactor}
\mathcal{M}^{(n_f)}{\bm\Gamma}^{(n_f)}={\bm \Gamma}^{(n_f)}+ T\mathbb{K}^{(n_f)}T^{-1}+\mathbb{S}^{(n_f)}\,,
\end{equation}
where we have defined
$\mathbb{S}^{(n_f)}=T\mathbb{K}^{(n_f)}
T^{-1}\left[{\bm\Gamma}^{(n_f)}-\mathbb{I}\right]$ that is given by:
% \begin{equation}
% \footnotesize
% \mathbb{S}_{ik}^{(n_f)} =
% \left\{
% \begin{array}{lll}
% K_{gg}\Gamma_{gg}+K_{gl}\Gamma_{qg} &
%                                                                       \quad i = 0 &\quad k = 0\\
% K_{gg}\Gamma_{gq}+K_{gl}\Gamma_{qq} & \quad i = 0 &\quad k =1\\
% 0 & \quad i = 0 &\quad  2\leq k \leq n_f\\
% \frac1{k(k-1)}\left[K_{gg}\Gamma_{gq}+K_{gl}\Gamma_{qq}\right] & \quad i = 0 &\quad  n_f+ 1\leq k \leq 6\\
% \\
% K_{hg}\Gamma_{gg}+K_{hl}\Gamma_{qg}+K_{ll}\Gamma_{qg} & \quad i = 1&\quad  k=0\\
% K_{hg}\Gamma_{gq}+K_{hl}\Gamma_{qq}+K_{ll}\Gamma_{qq}& \quad i = 1 &\quad  k=1\\
% 0 & \quad i = 1 &\quad  2\leq k \leq n_f\\
% \frac{1}{k(k-1)}\left[K_{hg}\Gamma_{gq}+K_{hl}\Gamma_{qq}+K_{ll}\Gamma_{qq}\right] & \quad i = 1 &\quad  n_f+1\leq k\leq 6\\
% \\
% \delta_{ik}K_{ll}\Gamma^+&\quad 2\leq i \leq n_f &\quad 0\leq k
%                                                          \leq 6\\
% \\
% -n_f \left(K_{hg}\Gamma_{gg}+K_{hl}\Gamma_{qg}\right)+K_{ll}\Gamma_{qg}&\quad i = n_f+1 &\quad k =0\\
% -n_f \left(K_{hg}\Gamma_{gq}+K_{hl}\Gamma_{qq}\right)+K_{ll}\Gamma_{qq}&\quad i = n_f+1 &\quad k =1\\
% 0 & \quad i = n_f+1 &\quad  2\leq k \leq n_f\\
%   \frac{6}{k(k-1)}\left[-n_f \left(K_{hg}\Gamma_{gq}+K_{hl}\Gamma_{qq}\right)+K_{ll}\Gamma_{qq}\right]
% &\quad i = n_f+1 &\quad n_f+1 \leq k \leq 6\\
% \\
%   \frac{6}{k(k-1)}\left[K_{hg}\Gamma_{gq}+K_{hl}\Gamma_{qq}+K_{ll}\Gamma_{qq} \right]
% & \quad n_f+2\leq i \leq 6 &\quad  n_f+
%                                                             2\leq k
%                                                             \leq 6\\
% \end{array}
% \right.
% \end{equation}
\begin{equation}\label{eq:Smatrix}
\footnotesize
\mathbb{S}_{ij}^{(n_f)} =
\left\{
\begin{array}{lll}
M_1\Gamma_{gg}+\frac{1}{1+n_f}\left(M_2-M_3\right)\Gamma_{qg} &
                                                                      \quad i = 0 &\quad j = 0\\
M_1\Gamma_{gq}+\frac{1}{1+n_f}\left(M_2-M_3\right)\Gamma_{qq} & \quad i = 0 &\quad j =1\\
0 & \quad i = 0 &\quad  2\leq j \leq n_f\\
\frac1{j(j-1)}\left[M_1\Gamma_{gq}+\frac{1}{1+n_f}\left(M_2-M_3\right)\Gamma_{qq}\right] & \quad i = 0 &\quad  n_f+ 1\leq j \leq 6\\
\\
M_4\Gamma_{gg}+\frac{1}{n_f+1}\left(M_5-M_6\right)\Gamma_{qg} & \quad i = 1&\quad  j=0\\
M_4\Gamma_{gq}+\frac{1}{n_f+1}\left(M_5-M_6\right)\Gamma_{qq}& \quad i = 1 &\quad  j=1\\
0 & \quad i = 1 &\quad  2\leq j \leq n_f\\
\frac{1}{j(j-1)}\left[M_4\Gamma_{gq}+\frac{1}{n_f+1}(M_5-M_6)\Gamma_{qq}\right] & \quad i = 1 &\quad  n_f+1\leq j\leq 6\\
\\
\delta_{ij}M_7\Gamma^+&\quad 2\leq i \leq n_f &\quad 0\leq j
                                                         \leq 6\\
\\
-n_f\left(M_4\Gamma_{gg}+\frac{1}{n_f+1}(M_5-M_6)\Gamma_{qg}\right)+(n_f+1)M_7\Gamma_{qg}&\quad i = n_f+1 &\quad j =0\\
-n_f\left(M_4\Gamma_{gq}+\frac{1}{n_f+1}(M_5-M_6)\Gamma_{qq}\right)+(n_f+1)M_7\Gamma_{qq}&\quad i = n_f+1 &\quad j =1\\
0 & \quad i = n_f+1 &\quad  2\leq j \leq n_f\\
  \frac{6}{j(j-1)}\left[-n_f\left(M_4\Gamma_{gq}+\frac{1}{n_f+1}(M_5-M_6)\Gamma_{qq}\right)+(n_f+1)M_7\Gamma_{qq}\right]
&\quad i = n_f+1 &\quad n_f+1 \leq j \leq 6\\
\\
  \frac{6}{j(j-1)}\left[M_4\Gamma_{gq}+\frac{1}{n_f+1}(M_5-M_6)\Gamma_{qq} \right]
& \quad n_f+2\leq i \leq 6 &\quad  n_f+
                                                            2\leq j
                                                            \leq 6\\
\end{array}
\right.
\end{equation}
Eq.~(\ref{eq:nthfactor}) along with Eq.~(\ref{eq:Smatrix}) allow us to
infer the general structure on the $n_f$-th factor in
Eq.~(\ref{eq:conbevop}). It should be clear that the structure is
driven by the term $T\mathbb{K}^{(n_f)}T^{-1}$ in
Eq.~(\ref{eq:algorithmEv}).

We finally comment on FFs. So far, to the best of my knowledge, none
of the $\mathcal{O}(\alpha_s^2)$ corrections to the FF matching
conditions has been computed. Therefore, we must limit to
$\mathbb{K}^{(1)(n_f)}$ that is simply given by the transpose of that
for PDFs.

\newpage
\begin{thebibliography}{alp}

\bibitem{Buza:1996wv}
  M.~Buza, Y.~Matiounine, J.~Smith and W.~L.~van Neerven,
  %``Charm electroproduction viewed in the variable-flavour number scheme
  %versus fixed-order perturbation theory,''
  Eur.\ Phys.\ J.\  C {\bf 1}, 301 (1998)
  [arXiv:hep-ph/9612398].
  %%CITATION = EPHJA,C1,301;%%

\bibitem{Vogt:2004ns}
  A.~Vogt,
  %``Efficient evolution of unpolarized and polarized parton distributions  with
  %QCD-PEGASUS,''
  Comput.\ Phys.\ Commun.\  {\bf 170} (2005) 65
  [arXiv:hep-ph/0408244].
  %%CITATION = CPHCB,170,65;%%

\end{thebibliography}





\end{document}
