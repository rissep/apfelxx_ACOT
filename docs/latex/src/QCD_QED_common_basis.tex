%% LyX 2.0.3 created this file.  For more info, see http://www.lyx.org/.
%% Do not edit unless you really know what you are doing.
\documentclass[twoside,english]{paper}
\usepackage{lmodern}
\renewcommand{\ttdefault}{lmodern}
\usepackage[T1]{fontenc}
\usepackage[latin9]{inputenc}
\usepackage[a4paper]{geometry}
\geometry{verbose,tmargin=3cm,bmargin=2.5cm,lmargin=2cm,rmargin=2cm}
\usepackage{color}
\usepackage{babel}
\usepackage{float}
\usepackage{bm}
\usepackage{amsthm}
\usepackage{amsmath}
\usepackage{amssymb}
\usepackage{graphicx}
\usepackage{esint}
\usepackage[unicode=true,pdfusetitle,
 bookmarks=true,bookmarksnumbered=false,bookmarksopen=false,
 breaklinks=false,pdfborder={0 0 0},backref=false,colorlinks=false]
 {hyperref}
\usepackage{breakurl}
\usepackage{mathrsfs}

\makeatletter

%%%%%%%%%%%%%%%%%%%%%%%%%%%%%% LyX specific LaTeX commands.
%% Because html converters don't know tabularnewline
\providecommand{\tabularnewline}{\\}

%%%%%%%%%%%%%%%%%%%%%%%%%%%%%% Textclass specific LaTeX commands.
\numberwithin{equation}{section}
\numberwithin{figure}{section}

%%%%%%%%%%%%%%%%%%%%%%%%%%%%%% User specified LaTeX commands.
\usepackage{babel}

\@ifundefined{showcaptionsetup}{}{%
 \PassOptionsToPackage{caption=false}{subfig}}
\usepackage{subfig}
\makeatother

\begin{document}

\title{A Common basis for the coupled QED$\times$QCD evolution}

\maketitle

\begin{abstract}
  In this document I will present a suitable flavour basis for the
  coupled QCD$\times$QED DGLAP evolution of PDFs.
\end{abstract}

\tableofcontents{}

\vspace{20pt}

\section{The structure of the DGLAP equation}

The DGLAP equation that governs PDF evolution has a general structure
that in QCD holds at any perturbative order. Suppose one wants to
study the coupled evolution of the gluon distribution function
$g(x,\mu)$, the $i$-th quark distribution function $q_i(x,\mu)$ and
the $j$-th anti-quark distribution function
$\overline{q}_j(x,\mu)$. In this case evolution equation would look
like this:
\begin{equation}
\mu^2\frac{\partial}{\partial \mu^2}
\begin{pmatrix}
q_i\\
g\\
\overline{q}_j
\end{pmatrix} = \sum_{k,l} 
\begin{pmatrix}
P_{q_iq_k} & P_{q_ig} & P_{q_i\overline{q}_l} \\ 
P_{gq_k} & P_{gg} & P_{g\overline{q}_l} \\
P_{\overline{q}_jq_k} & P_{\overline{q}_jg} & P_{\overline{q}_j\overline{q}_l}
\end{pmatrix}
\begin{pmatrix}
q_k\\
g\\
\overline{q}_l
\end{pmatrix}\,,
\label{GeneralDGLAP}
\end{equation}
where we are understanding the convolution and where the sum over $k$
and $l$ runs over all $n_f$ the active flavours.  Because of charge
conjugation invariance and $SU(n_f)$ flavour symmetry, one can show
that:
\begin{equation}
\begin{array}{l}
P_{q_iq_j} = P_{\overline{q}_i\overline{q}_j} = \delta_{ij} P_{qq}^V+P_{qq}^S\\
P_{\overline{q}_iq_j} = P_{q_i\overline{q}_j}  = \delta_{ij} P_{\overline{q}q}^V+P_{\overline{q}q}^S\\
P_{q_ig}  =P_{\overline{q}_ig} = P_{qg} \\
P_{gq_i}  =P_{g\overline{q}_i} = P_{gq} \,.
\end{array}
\label{decomposition}
\end{equation}
Plugging Eq.~(\ref{decomposition}) into Eq.~(\ref{GeneralDGLAP}), one finds:
\begin{equation}
\mu^2\frac{\partial}{\partial \mu^2}
\begin{pmatrix}
q_i\\
g\\
\overline{q}_j
\end{pmatrix} =  
\begin{pmatrix}
P_{qq}^V & P_{qg} & P_{q\overline{q}}^V \\ 
P_{gq} & P_{gg} & P_{gq} \\
P_{q\overline{q}}^V & P_{qg} & P_{qq}^V
\end{pmatrix}
\begin{pmatrix}
q_i\\
g\\
\overline{q}_j
\end{pmatrix}+
\begin{pmatrix}
P_{qq}^S & 0 & P_{q\overline{q}}^S \\ 
0 & 0 & 0 \\
P_{q\overline{q}}^S & 0 & P_{qq}^S
\end{pmatrix}
\begin{pmatrix}
\sum_{k}q_k\\
g\\
\sum_l\overline{q}_l
\end{pmatrix}\,.
\label{GeneralDGLAPdec}
\end{equation}
Setting $i=j$ and summing and subtracting the first
and the third row/column, we find:
\begin{equation}
\mu^2\frac{\partial}{\partial \mu^2}
\begin{pmatrix}
q_i^+\\
g\\
q_i^-
\end{pmatrix} =  
\begin{pmatrix}
(P_{qq}^V + P_{q\overline{q}}^V) & 2P_{qg} & 0 \\ 
P_{gq} & P_{gg} & 0 \\
0 & 0 & (P_{qq}^V - P_{q\overline{q}}^V)
\end{pmatrix}
\begin{pmatrix}
q_i^+\\
g\\
q_i^-
\end{pmatrix}+
\begin{pmatrix}
(P_{qq}^S + P_{q\overline{q}}^S) & 0  & 0\\ 
0 & 0 & 0 \\
0 & 0 & (P_{qq}^S - P_{q\overline{q}}^S)
\end{pmatrix}
\begin{pmatrix}
\sum_{k}q_k^+\\
g\\
\sum_{k}q_k^-
\end{pmatrix}\,,
\label{GeneralDGLAPdecSub1}
\end{equation}
where we have defined:
\begin{equation}
q^\pm_i \equiv q_i \pm \overline{q}_i\,.
\end{equation}
It is evident that in this way we have semi-diagonalised the initial
system because now the third equation is decoupled from the rest of
the system. Using the following definitions:
\begin{equation}
\begin{array}{l}
\displaystyle \Sigma \equiv \sum_{k} q_k^+ \\
\\
\displaystyle V \equiv \sum_{k} q_k^- \\
\\
\displaystyle P^\pm \equiv P_{qq}^V \pm P_{q\overline{q}}^V \\
\\
\displaystyle P_{qq} \equiv P^+ + n_f (P_{qq}^S + P_{q\overline{q}}^S)\\
\\
\displaystyle P^V \equiv P^- + n_f (P_{qq}^S - P_{q\overline{q}}^S)
\end{array}\,,
\label{defis}
\end{equation}
we have:
\begin{equation}
\left\{
\begin{array}{l}
\displaystyle \mu^2\frac{\partial}{\partial \mu^2}g =P_{gg}g + P_{gq}\Sigma \\
\\
\displaystyle \mu^2\frac{\partial}{\partial \mu^2} q_i^+=
P^+q_i^+ +\frac{1}{n_f}(P_{qq}-P^+)\Sigma + 2P_{qg}g \\ 
\\
\displaystyle \mu^2\frac{\partial}{\partial\mu^2} q_i^- = P^- q_i^-
+\frac{1}{n_f}(P^V-P^-) V
\end{array}
\right.\,.
\end{equation}
At this point we want to generalise this discussion including QED
corrections. There are two main differences. The first is obviously
the fact that we need to introduce in the DGLAP equation the parton
distribution associated to the photon $\gamma(x,\mu)$.  The second
difference is the fact that the all-order splitting functions no
longer undergo the stringent simplications of
Eq.~(\ref{decomposition}). In fact, the QED corrections introduce an
asymmetry between down-like quarks ($d$, $s$ and $b$) and up-like
quarks ($u$, $c$, $t$), due essentially to the different electric
charge, that breaks flavour symmetry for the quark splitting
functions. We then must consider the following extended evolution
system:
\begin{equation}
\mu^2\frac{\partial}{\partial \mu^2}
\begin{pmatrix}
u_j \\
d_i\\
g\\
\gamma\\
\overline{d}_k\\
\overline{u}_h
\end{pmatrix} = \sum_{e,l,m,n} 
\begin{pmatrix}
\mathcal{P}_{u_ju_e} & \mathcal{P}_{u_jd_l} & \mathcal{P}_{u_jg} & \mathcal{P}_{u_j\gamma} & \mathcal{P}_{u_j\overline{d}_m} & \mathcal{P}_{u_j\overline{u}_n} \\ 
\mathcal{P}_{d_iu_e} & \mathcal{P}_{d_id_l} & \mathcal{P}_{d_ig} & \mathcal{P}_{d_i\gamma} & \mathcal{P}_{d_i\overline{d}_m} & \mathcal{P}_{d_i\overline{u}_n} \\ 
\mathcal{P}_{gu_e}  & \mathcal{P}_{gd_l} & \mathcal{P}_{gg} & \mathcal{P}_{g\gamma} & \mathcal{P}_{g\overline{d}_m} & \mathcal{P}_{g\overline{u}_n}\\
\mathcal{P}_{\gamma u_e} & \mathcal{P}_{\gamma d_l} & \mathcal{P}_{\gamma g} & \mathcal{P}_{\gamma\gamma} & \mathcal{P}_{\gamma\overline{d}_m} & \mathcal{P}_{\gamma\overline{u}_n}\\
\mathcal{P}_{\overline{d}_ku_e} & \mathcal{P}_{\overline{d}_kd_l} & \mathcal{P}_{\overline{d}_kg} & \mathcal{P}_{\overline{d}_k\gamma} & \mathcal{P}_{\overline{d}_k\overline{d}_m} & \mathcal{P}_{\overline{d}_k\overline{u}_n}\\
\mathcal{P}_{\overline{u}_hu_e} & \mathcal{P}_{\overline{u}_hd_l} & \mathcal{P}_{\overline{u}_hg} & \mathcal{P}_{\overline{u}_h\gamma} & \mathcal{P}_{\overline{u}_h\overline{d}_m} & \mathcal{P}_{\overline{u}_h\overline{u}_n}
\end{pmatrix}
\begin{pmatrix}
u_e \\
d_l\\
g\\
\gamma\\
\overline{d}_m\\
\overline{u}_n
\end{pmatrix}
\label{EvenMoreGeneralDGLAP}
\end{equation}
where:
\begin{equation}
u_i= \{u,c,t\}\,,\qquad d_i= \{d,s,b\}\,,\qquad \overline{u}_i= \{\overline{u},\overline{c},\overline{t}\}\,,\qquad \overline{d}_i= \{\overline{d},\overline{s},\overline{b}\}\,.
\end{equation}

Each splitting function in Eq.~(\ref{EvenMoreGeneralDGLAP}) can be
split into two pieces:
\begin{equation}
  \mathcal{P}_{ab} = P_{ab} + \widetilde{P}_{ab}\,,
  \label{eq:decomposition}
\end{equation}
where $P_{ab}$ is the usual QCD splitting function which does not
contain any powers of the fine structure constant $\alpha$ and
therefore undergoes to the same simplifications discussed above. As a
further consequence if $a$ or $b$ is equal to $\gamma$, $P_{ab}$ must
vanish.  $\widetilde{P}_{ab}$ instead contains at least one power of
$\alpha$. In this way we can rearrange
Eq.~(\ref{EvenMoreGeneralDGLAP}) as follows:
\begin{equation}
\begin{array}{rcl}
\displaystyle \mu^2\frac{\partial}{\partial \mu^2}
\begin{pmatrix}
u_j \\
d_i\\
g\\
\gamma\\
\overline{d}_k\\
\overline{u}_h
\end{pmatrix} &=& \displaystyle \sum_{e,l,m,n} 
\left[\begin{pmatrix}
\delta_{je} P_{qq}^V+P_{qq}^S & P_{qq}^S & {P}_{qg} & 0 & P_{q\overline{q}}^S & \delta_{jn} P_{q\overline{q}}^V+P_{q\overline{q}}^S \\ 
P_{qq}^S & \delta_{il} P_{qq}^V+P_{qq}^S & {P}_{qg} & 0 & \delta_{im} P_{q\overline{q}}^V+P_{q\overline{q}}^S & P_{q\overline{q}}^S \\ 
{P}_{gq}  & {P}_{gq} & {P}_{gg} & 0 & {P}_{gq} & {P}_{gq}\\
0 & 0 & 0 & 0 & 0 & 0 \\
P_{q\overline{q}}^S & \delta_{kl} P_{q\overline{q}}^V+P_{q\overline{q}}^S & {P}_{qg} & 0 & \delta_{km} P_{qq}^V+P_{qq}^S & P_{qq}^S\\
\delta_{he} P_{q\overline{q}}^V+P_{q\overline{q}}^S & P_{q\overline{q}}^S & {P}_{qg} & 0 & P_{qq}^S & \delta_{hn} P_{qq}^V+P_{qq}^S
\end{pmatrix}\right.\\
\\
&+&\left.\displaystyle
\begin{pmatrix}
\widetilde{P}_{u_ju_e} & \widetilde{P}_{u_jd_l} & \widetilde{P}_{u_jg} & \widetilde{P}_{u_j\gamma} & \widetilde{P}_{u_j\overline{d}_m} & \widetilde{P}_{u_j\overline{u}_n} \\ 
\widetilde{P}_{d_iu_e} & \widetilde{P}_{d_id_l} & \widetilde{P}_{d_ig} & \widetilde{P}_{d_i\gamma} & \widetilde{P}_{d_i\overline{d}_m} & \widetilde{P}_{d_i\overline{u}_n} \\ 
\widetilde{P}_{gu_e}  & \widetilde{P}_{gd_l} & \widetilde{P}_{gg} & \widetilde{P}_{g\gamma} & \widetilde{P}_{g\overline{d}_m} & \widetilde{P}_{g\overline{u}_n}\\
\widetilde{P}_{\gamma u_e} & \widetilde{P}_{\gamma d_l} & \widetilde{P}_{\gamma g} & \widetilde{P}_{\gamma\gamma} & \widetilde{P}_{\gamma\overline{d}_m} & \widetilde{P}_{\gamma\overline{u}_n}\\
\widetilde{P}_{\overline{d}_ku_e} & \widetilde{P}_{\overline{d}_kd_l} & \widetilde{P}_{\overline{d}_kg} & \widetilde{P}_{\overline{d}_k\gamma} & \widetilde{P}_{\overline{d}_k\overline{d}_m} & \widetilde{P}_{\overline{d}_k\overline{u}_n}\\
\widetilde{P}_{\overline{u}_hu_e} & \widetilde{P}_{\overline{u}_hd_l} & \widetilde{P}_{\overline{u}_hg} & \widetilde{P}_{\overline{u}_h\gamma} & \widetilde{P}_{\overline{u}_h\overline{d}_m} & \widetilde{P}_{\overline{u}_h\overline{u}_n}
\end{pmatrix}\right]
\begin{pmatrix}
u_e \\
d_l\\
g\\
\gamma\\
\overline{d}_m\\
\overline{u}_n
\end{pmatrix}
\end{array}
\label{EvenMoreGeneralDGLAPciao}
\end{equation}

Since $\alpha$ comes always with an electric charge associated, every
$\widetilde{P}_{ab}$ can factor out at least an electric charge
$e_u^2=4/9$ or $e_d^2=1/9$. In order to see how this factorisation
takes place, we should analyse one by one the splitting functions
$\widetilde{P}_{ab}$. Defining:
\begin{equation}
e_{\Sigma}^2 = N_c(e_u^2 n_{u} + e_d^2 n_{d})\,,
\end{equation}
where $n_u$ and $n_d$ are respectively the number of up- and down-type
active quarks such that $n_u + n_d = n_f$ and $N_c = 3$ is the number
of colours, we have that:
\begin{equation}
\begin{array}{cc}
\displaystyle \widetilde{P}_{gg} \rightarrow
e_\Sigma^2\widetilde{P}_{gg}\,, & \displaystyle \widetilde{P}_{g\gamma} \rightarrow e_\Sigma^2\widetilde{P}_{g\gamma}\,,\\
\\
\displaystyle \widetilde{P}_{\gamma g} \rightarrow
e_\Sigma^2\widetilde{P}_{\gamma g}\,, & \displaystyle \widetilde{P}_{\gamma\gamma} \rightarrow e_\Sigma^2\widetilde{P}_{\gamma\gamma}\,.
\end{array}
\end{equation}
This is the consequence of the fact that, having only bosons as
external particles, the presence of any fermion in the splitting must
be summed over all flavours. This is (should be) true at any
perturbative order.

Now we consider the splitting functions involving one boson and one
quark. Here the situation is more involved because at higher orders it
may happen that the incoming/outgoing quark never couples to a photon
and thus, given that there is at least one power of $\alpha$, apart
from a term proportional to the charge of the incoming/outgoing quark,
there must also be a term proportional to the charge
$e_\Sigma^2$. However, such contributions only appear at three loops
(NNLO) and since here we are only intersted in the two-loop splitting
functions, we have:
\begin{equation}
\begin{array}{cc}
\displaystyle \widetilde{P}_{gu_i} = \widetilde{P}_{g\overline{u}_i} =
e_u^2\widetilde{P}_{gq}\,, & \displaystyle \widetilde{P}_{gd_i} = \widetilde{P}_{g\overline{d}_i} =
e_d^2\widetilde{P}_{gq}\,, \\
\\
\displaystyle \widetilde{P}_{u_i g} = \widetilde{P}_{\overline{u}_i g} =
e_u^2\widetilde{P}_{qg}\,, & \displaystyle \widetilde{P}_{d_i g} =
\widetilde{P}_{\overline{d}_i g} =
e_d^2\widetilde{P}_{qg}\,,\\
\\
\displaystyle \widetilde{P}_{\gamma u_i} = \widetilde{P}_{\gamma \overline{u}_i} =
e_u^2\widetilde{P}_{\gamma q}\,, & \displaystyle \widetilde{P}_{\gamma d_i} = \widetilde{P}_{\gamma \overline{d}_i} =
e_d^2\widetilde{P}_{\gamma q}\,, \\
\\
\displaystyle \widetilde{P}_{u_i \gamma } = \widetilde{P}_{\overline{u}_i \gamma } =
e_u^2\widetilde{P}_{q\gamma }\,, & \displaystyle \widetilde{P}_{d_i \gamma } =
\widetilde{P}_{\overline{d}_i \gamma } =
e_d^2\widetilde{P}_{q\gamma }\,.
\end{array}
\end{equation}

Finally, we consider the splitting functions involving quarks or
anti-quarks in the final and initial states. Again we will limit
ourselves to two loops and under this restriction we have:
\begin{equation}
\begin{array}{l}
\widetilde{P}_{u_iu_j} = \widetilde{P}_{\overline{u}_i\overline{u}_j} = e_u^2\delta_{ij} \widetilde{P}_{qq}^V+e_u^4\widetilde{P}_{qq}^S\\
\widetilde{P}_{d_id_j} = \widetilde{P}_{\overline{d}_i\overline{d}_j} = e_d^2\delta_{ij} \widetilde{P}_{qq}^V+e_d^4\widetilde{P}_{qq}^S\\
\widetilde{P}_{\overline{u}_iu_j} = \widetilde{P}_{u_i\overline{u}_j} = e_u^4\widetilde{P}_{qq}^S\\
\widetilde{P}_{\overline{d}_id_j} = \widetilde{P}_{d_i\overline{d}_j} = e_d^4\widetilde{P}_{qq}^S\\
\widetilde{P}_{u_id_j} = \widetilde{P}_{d_iu_j} = \widetilde{P}_{\overline{u}_id_j} = \widetilde{P}_{d_i\overline{u}_j} = \widetilde{P}_{\overline{d}_iu_j} = \widetilde{P}_{u_i\overline{d}_j} =\widetilde{P}_{\overline{u}_i\overline{d}_j} = \widetilde{P}_{\overline{d}_i\overline{u}_j} = e_u^2e_d^2\widetilde{P}_{qq}^S
\end{array}\,.
\label{decompositionQED}
\end{equation}

Using the information above, we can now write the QED correction
matrix to the splitting functions up to two loops as follows:
\begin{equation}
\begin{array}{c}
\begin{pmatrix}
e_u^2\delta_{je} \widetilde{P}_{qq}^V+e_u^4\widetilde{P}_{qq}^S & e_u^2e_d^2\widetilde{P}_{qq}^S & e_u^2\widetilde{P}_{qg} & e_u^2\widetilde{P}_{q\gamma} & e_u^2e_d^2\widetilde{P}_{qq}^S & e_u^4\widetilde{P}_{qq}^S \\ 
 e_u^2e_d^2\widetilde{P}_{qq}^S & e_d^2\delta_{il} \widetilde{P}_{qq}^V+e_d^4\widetilde{P}_{qq}^S & e_d^2\widetilde{P}_{qg} & e_d^2\widetilde{P}_{q\gamma} & e_d^4\widetilde{P}_{qq}^S &  e_u^2e_d^2\widetilde{P}_{qq}^S \\ 
e_u^2\widetilde{P}_{gq}  & e_d^2\widetilde{P}_{gq} & e_\Sigma^2\widetilde{P}_{gg} & e_\Sigma^2\widetilde{P}_{g\gamma} & e_d^2\widetilde{P}_{gq} & e_u^2\widetilde{P}_{gq}\\
e_u^2 \widetilde{P}_{\gamma q} & e_d^2\widetilde{P}_{\gamma q} &
e_\Sigma^2\widetilde{P}_{\gamma g} &
e_\Sigma^2\widetilde{P}_{\gamma\gamma} & e_d^2\widetilde{P}_{\gamma q}
& e_u^2\widetilde{P}_{\gamma q}\\
 e_u^2e_d^2\widetilde{P}_{qq}^S & e_d^4\widetilde{P}_{qq}^S & e_d^2\widetilde{P}_{qg} & e_d^2\widetilde{P}_{q\gamma} & e_d^2\delta_{km} \widetilde{P}_{qq}^V+e_d^4\widetilde{P}_{qq}^S &  e_u^2e_d^2\widetilde{P}_{qq}^S\\
e_u^4\widetilde{P}_{qq}^S & e_u^2e_d^2\widetilde{P}_{qq}^S & e_u^2 \widetilde{P}_{qg} & e_u^2\widetilde{P}_{q\gamma} &  e_u^2e_d^2\widetilde{P}_{qq}^S & e_u^2\delta_{hn} \widetilde{P}_{qq}^V+e_u^4\widetilde{P}_{qq}^S
\end{pmatrix}=\\
\\
\begin{pmatrix}
e_u^2\delta_{je}\widetilde{P}_{qq} & 0 & e_u^2\widetilde{P}_{qg} & e_u^2\widetilde{P}_{q\gamma} & 0 & 0 \\ 
0 & e_d^2\delta_{il} \widetilde{P}_{qq}^V & e_d^2\widetilde{P}_{qg} & e_d^2\widetilde{P}_{q\gamma} & 0 & 0 \\ 
e_u^2\widetilde{P}_{gq}  & e_d^2\widetilde{P}_{gq} & e_\Sigma^2\widetilde{P}_{gg} & e_\Sigma^2\widetilde{P}_{g\gamma} & e_d^2\widetilde{P}_{gq} & e_u^2\widetilde{P}_{gq}\\
e_u^2 \widetilde{P}_{\gamma q} & e_d^2\widetilde{P}_{\gamma q} &
e_\Sigma^2\widetilde{P}_{\gamma g} & e_\Sigma^2\widetilde{P}_{\gamma\gamma} & e_d^2\widetilde{P}_{\gamma q}
& e_u^2\widetilde{P}_{\gamma q}\\
0 & 0 & e_d^2\widetilde{P}_{qg} & e_d^2\widetilde{P}_{q\gamma} &
e_d^2\delta_{km} \widetilde{P}_{qq}^V & 0 \\
0 & 0 & e_u^2 \widetilde{P}_{qg} & e_u^2\widetilde{P}_{q\gamma} & 0 & e_u^2\delta_{hn} \widetilde{P}_{qq}^V
\end{pmatrix}+
\begin{pmatrix}
e_u^4\widetilde{P}_{qq}^S & e_u^2e_d^2\widetilde{P}_{qq}^S & 0 & 0 & e_u^2e_d^2\widetilde{P}_{qq}^S & e_u^4\widetilde{P}_{qq}^S \\ 
e_u^2e_d^2\widetilde{P}_{qq}^S & e_d^4\widetilde{P}_{qq}^S & 0 & 0 & e_d^4\widetilde{P}_{qq}^S &  e_u^2e_d^2\widetilde{P}_{qq}^S \\ 
0  & 0 & 0 & 0 & 0 & 0\\
0  & 0 & 0 & 0 & 0 & 0\\
e_u^2e_d^2\widetilde{P}_{qq}^S & e_d^4\widetilde{P}_{qq}^S & 0 & 0 & e_d^4\widetilde{P}_{qq}^S &  e_u^2e_d^2\widetilde{P}_{qq}^S\\
e_u^4\widetilde{P}_{qq}^S & e_u^2e_d^2\widetilde{P}_{qq}^S & 0 & 0 &  e_u^2e_d^2\widetilde{P}_{qq}^S & e_u^4\widetilde{P}_{qq}^S
\end{pmatrix}
\end{array}
\label{QEDGen}
\end{equation}

We now apply the same decomposition to the purely QCD matrix,
obtaining:
\begin{equation}
\begin{array}{c}
\begin{pmatrix}
\delta_{je} P_{qq}^V+P_{qq}^S & P_{qq}^S & {P}_{qg} & 0 & P_{q\overline{q}}^S & \delta_{jn} P_{q\overline{q}}^V+P_{q\overline{q}}^S \\ 
P_{qq}^S & \delta_{il} P_{qq}^V+P_{qq}^S & {P}_{qg} & 0 & \delta_{im} P_{q\overline{q}}^V+P_{q\overline{q}}^S & P_{q\overline{q}}^S \\ 
{P}_{gq}  & {P}_{gq} & {P}_{gg} & 0 & {P}_{gq} & {P}_{gq}\\
0 & 0 & 0 & 0 & 0 & 0 \\
P_{q\overline{q}}^S & \delta_{kl} P_{q\overline{q}}^V+P_{q\overline{q}}^S & {P}_{qg} & 0 & \delta_{km} P_{qq}^V+P_{qq}^S & P_{qq}^S\\
\delta_{he} P_{q\overline{q}}^V+P_{q\overline{q}}^S & P_{q\overline{q}}^S & {P}_{qg} & 0 & P_{qq}^S & \delta_{hn} P_{qq}^V+P_{qq}^S
\end{pmatrix}=\\
\\
\begin{pmatrix}
\delta_{je} P_{qq}^V & 0 & {P}_{qg} & 0 & 0 & \delta_{jm} P_{q\overline{q}}^V \\ 
0 & \delta_{il} P_{qq}^V & {P}_{qg} & 0 & \delta_{in} P_{q\overline{q}}^V & 0 \\ 
{P}_{gq}  & {P}_{gq} & {P}_{gg} & 0 & {P}_{gq} & {P}_{gq}\\
0 & 0 & 0 & 0 & 0 & 0 \\
0 & \delta_{kl} P_{q\overline{q}}^V & {P}_{qg} & 0 & \delta_{km} P_{qq}^V & 0\\
\delta_{he} P_{q\overline{q}}^V & 0 & {P}_{qg} & 0 & 0 & \delta_{hn} P_{qq}^V
\end{pmatrix} +
\begin{pmatrix}
P_{qq}^S & P_{qq}^S & 0 & 0 & P_{q\overline{q}}^S & P_{q\overline{q}}^S \\ 
P_{qq}^S & P_{qq}^S & 0 & 0 & P_{q\overline{q}}^S & P_{q\overline{q}}^S \\ 
0 & 0 & 0 & 0 & 0 & 0 \\
0 & 0 & 0 & 0 & 0 & 0 \\
P_{q\overline{q}}^S & P_{q\overline{q}}^S & 0 & 0 & P_{qq}^S & P_{qq}^S\\
P_{q\overline{q}}^S & P_{q\overline{q}}^S & 0 & 0 & P_{qq}^S & P_{qq}^S
\end{pmatrix}
\end{array}
\label{QCDGen}
\end{equation}

Finally, plugging Eqs.~(\ref{QEDGen}) and~(\ref{QCDGen}) into
Eq.~(\ref{EvenMoreGeneralDGLAPciao}), performing the sum over $e$,
$l$, $m$ and $n$ and identifying $k=i$ and $h=j$, we obtain:
\begin{equation}
\begin{array}{rcl}
\displaystyle \mu^2\frac{\partial}{\partial \mu^2}
\begin{pmatrix}
u_j \\
d_i\\
g\\
\gamma\\
\overline{d}_i\\
\overline{u}_j
\end{pmatrix} &=& \displaystyle
\left[\begin{pmatrix}
P_{qq}^V & 0 & {P}_{qg} & 0 & 0 & P_{q\overline{q}}^V \\ 
0 & P_{qq}^V & {P}_{qg} & 0 & P_{q\overline{q}}^V & 0 \\ 
{P}_{gq}  & {P}_{gq} & {P}_{gg} & 0 & {P}_{gq} & {P}_{gq}\\
0 & 0 & 0 & 0 & 0 & 0 \\
0 & P_{q\overline{q}}^V & {P}_{qg} & 0 & P_{qq}^V & 0\\
P_{q\overline{q}}^V & 0 & {P}_{qg} & 0 & 0 & P_{qq}^V
\end{pmatrix}+
\begin{pmatrix}
e_u^2\widetilde{P}_{qq} & 0 & e_u^2\widetilde{P}_{qg} & e_u^2\widetilde{P}_{q\gamma} & 0 & 0 \\ 
0 & e_d^2\widetilde{P}_{qq} & e_d^2\widetilde{P}_{qg} & e_d^2\widetilde{P}_{q\gamma} & 0 & 0 \\ 
e_u^2\widetilde{P}_{gq}  & e_d^2\widetilde{P}_{gq} & e_\Sigma^2\widetilde{P}_{gg} & e_\Sigma^2\widetilde{P}_{g\gamma} & e_d^2\widetilde{P}_{gq} & e_u^2\widetilde{P}_{gq}\\
e_u^2 \widetilde{P}_{\gamma q} & e_d^2\widetilde{P}_{\gamma q} &
e_\Sigma^2\widetilde{P}_{\gamma g} & e_\Sigma^2\widetilde{P}_{\gamma\gamma} & e_d^2\widetilde{P}_{\gamma q}
& e_u^2\widetilde{P}_{\gamma q}\\
0 & 0 & e_d^2\widetilde{P}_{qg} & e_d^2\widetilde{P}_{q\gamma} &
e_d^2\widetilde{P}_{qq}^V & 0 \\
0 & 0 & e_u^2 \widetilde{P}_{qg} & e_u^2\widetilde{P}_{q\gamma} & 0 & e_u^2\widetilde{P}_{qq}^V
\end{pmatrix}\right]
\begin{pmatrix}
u_j \\
d_i\\
g\\
\gamma\\
\overline{d}_i\\
\overline{u}_j
\end{pmatrix}\\
\\
&+&
\left[
\begin{pmatrix}
P_{qq}^S & P_{qq}^S & 0 & 0 & P_{q\overline{q}}^S & P_{q\overline{q}}^S \\ 
P_{qq}^S & P_{qq}^S & 0 & 0 & P_{q\overline{q}}^S & P_{q\overline{q}}^S \\ 
0 & 0 & 0 & 0 & 0 & 0 \\
0 & 0 & 0 & 0 & 0 & 0 \\
P_{q\overline{q}}^S & P_{q\overline{q}}^S & 0 & 0 & P_{qq}^S & P_{qq}^S\\
P_{q\overline{q}}^S & P_{q\overline{q}}^S & 0 & 0 & P_{qq}^S & P_{qq}^S
\end{pmatrix}+
\begin{pmatrix}
e_u^4\widetilde{P}_{qq}^S & e_u^2e_d^2\widetilde{P}_{qq}^S & 0 & 0 & e_u^2e_d^2\widetilde{P}_{qq}^S & e_u^4\widetilde{P}_{qq}^S \\ 
e_u^2e_d^2\widetilde{P}_{qq}^S & e_d^4\widetilde{P}_{qq}^S & 0 & 0 & e_d^4\widetilde{P}_{qq}^S &  e_u^2e_d^2\widetilde{P}_{qq}^S \\ 
0  & 0 & 0 & 0 & 0 & 0\\
0  & 0 & 0 & 0 & 0 & 0\\
e_u^2e_d^2\widetilde{P}_{qq}^S & e_d^4\widetilde{P}_{qq}^S & 0 & 0 & e_d^4\widetilde{P}_{qq}^S &  e_u^2e_d^2\widetilde{P}_{qq}^S\\
e_u^4\widetilde{P}_{qq}^S & e_u^2e_d^2\widetilde{P}_{qq}^S & 0 & 0 &  e_u^2e_d^2\widetilde{P}_{qq}^S & e_u^4\widetilde{P}_{qq}^S
\end{pmatrix}
\right]
\begin{pmatrix}
\sum_eu_e \\
\sum_ld_l\\
g\\
\gamma\\
\sum_m\overline{d}_m\\
\sum_n\overline{u}_n
\end{pmatrix}
\end{array}
\label{EvenMoreGeneralDGLAPciaociao}
\end{equation}

In order to have the same evolution system in terms of plus- and
minus-distributions, we apply to
Eq.~(\ref{EvenMoreGeneralDGLAPciaociao}) the following transformation:
\begin{equation}
\mathbf{T} = 
\begin{pmatrix}
1 & 0 & 0 & 0 & 0 & 1 \\
0 & 1 & 0 & 0 & 1 & 0 \\
0 & 0 & 1 & 0 & 0 & 0 \\
0 & 0 & 0 & 1 & 0 & 0 \\
0 & 1 & 0 & 0 & -1 & 0 \\
1 & 0 & 0 & 0 & 0 & -1
\end{pmatrix}\quad\Longrightarrow\quad
\mathbf{T}^{-1} = 
\frac{1}{2}\begin{pmatrix}
1 & 0 & 0 & 0 & 0 & 1 \\
0 & 1 & 0 & 0 & 1 & 0 \\
0 & 0 & 2 & 0 & 0 & 0 \\
0 & 0 & 0 & 2 & 0 & 0 \\
0 & 1 & 0 & 0 & -1 & 0 \\
1 & 0 & 0 & 0 & 0 & -1
\end{pmatrix}
\label{EvenMoreGeneralDGLAPbye}
\end{equation}
so that we get:
\begin{equation}
\begin{array}{rcl}
\displaystyle \mu^2\frac{\partial}{\partial \mu^2}
\begin{pmatrix}
u_j^+ \\
d_i^+\\
g\\
\gamma\\
d_i^-\\
u_j^-
\end{pmatrix} &=& \displaystyle
\left[\begin{pmatrix}
P^+ & 0 & 2{P}_{qg} & 0 & 0 & 0 \\ 
0 & P^+ & 2{P}_{qg} & 0 & 0 & 0 \\ 
{P}_{gq}  & {P}_{gq} & {P}_{gg} & 0 & 0 & 0 \\
0 & 0 & 0 & 0 & 0 & 0 \\
0 & 0 & 0 & 0 & P^- & 0\\
0 & 0 & 0 & 0 & 0 & P^-
\end{pmatrix}+
\begin{pmatrix}
e_u^2\widetilde{P}^+ & 0 & 2 e_u^2\widetilde{P}_{qg} & 2 e_u^2\widetilde{P}_{q\gamma} & 0 & 0 \\ 
0 & e_d^2\widetilde{P}^+ & 2 e_d^2\widetilde{P}_{qg} & 2 e_d^2\widetilde{P}_{q\gamma} & 0 & 0 \\ 
e_u^2\widetilde{P}_{gq}  & e_d^2\widetilde{P}_{gq} &
e_\Sigma^2\widetilde{P}_{gg} & e_\Sigma^2\widetilde{P}_{g\gamma} & 0 &0 \\
e_u^2 \widetilde{P}_{\gamma q} & e_d^2\widetilde{P}_{\gamma q} &
e_\Sigma^2\widetilde{P}_{\gamma g} & e_\Sigma^2\widetilde{P}_{\gamma\gamma} & 0 & 0 \\
0 & 0 & 0 & 0 & e_d^2\widetilde{P}^- & 0 \\
0 & 0 & 0 & 0 & 0 & e_u^2\widetilde{P}^-
\end{pmatrix}\right]
\begin{pmatrix}
u_j^+ \\
d_i^+\\
g\\
\gamma\\
d_i^-\\
u_j^-
\end{pmatrix}\\
\\
&+&
\displaystyle \left[
\begin{pmatrix}
P_{qq}-P^+ & P_{qq}-P^+ & 0 & 0 & 0 & 0 \\ 
P_{qq}-P^+ & P_{qq}-P^+ & 0 & 0 & 0 & 0 \\ 
0 & 0 & 0 & 0 & 0 & 0 \\
0 & 0 & 0 & 0 & 0 & 0 \\
0 & 0 & 0 & 0 & P^V-P^- & P^V-P^-\\
0 & 0 & 0 & 0 & P^V-P^- & P^V-P^-
\end{pmatrix}\right.\\
\\
&+& \displaystyle \left.
\begin{pmatrix}
e_u^4(\widetilde{P}_{qq}-\widetilde{P}^+) & e_u^2e_d^2(\widetilde{P}_{qq}-\widetilde{P}^+) & 0 & 0 & 0 & 0 \\ 
e_u^2e_d^2(\widetilde{P}_{qq}-\widetilde{P}^+) & e_d^4(\widetilde{P}_{qq}-\widetilde{P}^+) & 0 & 0 & 0 & 0 \\ 
0  & 0 & 0 & 0 & 0 & 0\\
0  & 0 & 0 & 0 & 0 & 0\\
0  & 0 & 0 & 0 & 0 & 0\\
0  & 0 & 0 & 0 & 0 & 0
\end{pmatrix}
\right]
\frac1{n_f}
\begin{pmatrix}
\sum_eu_e^+ \\
\sum_ld_l^+\\
g\\
\gamma\\
\sum_md_m^-\\
\sum_nu_n^-
\end{pmatrix}
\end{array}
\label{macheneso}
\end{equation}

Using the following definitions:
\begin{equation}
\begin{array}{ll}
\displaystyle \Sigma_u = \sum_{k=i}^{n_u} u_k^+ & \qquad \displaystyle \Sigma_d
= \sum_{k=i}^{n_d} d_k^+ \\
\\
\displaystyle V_u = \sum_{k=i}^{n_u} u_k^- & \qquad \displaystyle V_d
= \sum_{k=i}^{n_d} d_k^-\,,
\end{array}
\end{equation}
which are such that:
\begin{equation}
\Sigma = \Sigma_u + \Sigma_d \quad\mbox{and}\quad V = V_u +  V_d\,,\\
\end{equation}
we can further manipulate Eq.~(\ref{EvenMoreGeneralDGLAPbye})
obtaining the coupled system:
\begin{equation}
\left\{
\begin{array}{rcl}
\displaystyle \mu^2\frac{\partial g}{\partial \mu^2} &=& (P_{gq} + e_u^2
\widetilde{P}_{gq})\Sigma_u + (P_{gq} + e_d^2
\widetilde{P}_{gq})\Sigma_d+ (P_{gg} + e_\Sigma^2 \widetilde{P}_{gg})g
+ e_\Sigma^2 \widetilde{P}_{g\gamma} \gamma\\
\\
\displaystyle \mu^2\frac{\partial \gamma}{\partial \mu^2} &=& e_u^2
\widetilde{P}_{\gamma q}\Sigma_u + e_d^2
\widetilde{P}_{\gamma q}\Sigma_d+ e_\Sigma^2 \widetilde{P}_{\gamma g}g
+ e_\Sigma^2 \widetilde{P}_{\gamma\gamma} \gamma\\
\\
\displaystyle \mu^2\frac{\partial d_i^+}{\partial \mu^2} &=& (P^+ + e_d^2\widetilde{P}^+)d_i^++ 2(P_{qg} + e_d^2 \widetilde{P}_{qg})g
+ 2e_d^2 \widetilde{P}_{q\gamma} \gamma\\
\\
 &+&\displaystyle \frac1{n_f}[(P_{qq}-P^+)+e_u^2e_d^2(\widetilde{P}_{qq}-\widetilde{P}^+)]\Sigma_u +
\frac1{n_f}[(P_{qq}-P^+)+e_d^4(\widetilde{P}_{qq}-\widetilde{P}^+)]\Sigma_d \\
\\
\displaystyle \mu^2\frac{\partial u_j^+}{\partial \mu^2} &=& (P^+ + e_u^2\widetilde{P}^+)u_j^++ 2(P_{qg} + e_u^2 \widetilde{P}_{qg})g
+ 2e_u^2 \widetilde{P}_{q\gamma} \gamma\\
\\
 &+&\displaystyle \frac1{n_f}[(P_{qq}-P^+)+e_u^4(\widetilde{P}_{qq}-\widetilde{P}^+)]\Sigma_u +
\frac1{n_f}[(P_{qq}-P^+)+e_u^2e_d^2(\widetilde{P}_{qq}-\widetilde{P}^+)]\Sigma_d\\
\\
\displaystyle \mu^2\frac{\partial d_i^-}{\partial \mu^2} &=& \displaystyle  (P^- + e_d^2\widetilde{P}^-)d_i^-+ \frac1{n_f} (P^V-P^-)V_u + \frac1{n_f} (P^V-P^-)V_d \\
\\
\displaystyle \mu^2\frac{\partial u_j^-}{\partial \mu^2} &=& \displaystyle  (P^- +
e_u^2\widetilde{P}^-)u_j^-+ \frac1{n_f} (P^V-P^-)V_u + \frac1{n_f} (P^V-P^-)V_d
\end{array}
\right.\,.
\end{equation}


\section{Evolution basis}\label{QEDEvBasis}

In order to diagonalise as much as possible the evolution matrix in
the presence of QED corrections avoiding unnecessary couplings between
parton distributions, we propose the following evolution basis:
\begin{equation}
\begin{array}{ll}
\mbox{1) }g & \\
\mbox{2) }\gamma & \\
\mbox{3) }\displaystyle \Sigma = \Sigma_u + \Sigma_d & \quad \mbox{9) }\displaystyle V =V_u +  V_d\\
\mbox{4) }\displaystyle \Delta_\Sigma = \Sigma_u - \Sigma_d & \quad \mbox{10) }\displaystyle \Delta_V = V_u - V_d\\
\mbox{5) }T_1^u = u^+ - c^+ &\quad \mbox{11) }V_1^u = u^- - c^- \\
\mbox{6) }T_2^u = u^+ + c^+ - 2t^+ &\quad \mbox{12) }V_2^u = u^- + c^- - 2t^-\\
\mbox{7) }T_1^d = d^+ - s^+ &\quad \mbox{13) }V_1^d = d^- - s^- \\
\mbox{8) }T_2^d = d^+ + s^+ - 2b^+ &\quad \mbox{14) }V_2^d = d^- + s^- - 2b^-
\end{array}
\end{equation}

In this basis the evolution system becomes:
\begin{equation}
\begin{array}{l}
\left\{
\begin{array}{rcl}
\displaystyle \mu^2\frac{\partial g}{\partial \mu^2} &=& \displaystyle
(P_{gq} + \eta^+
\widetilde{P}_{gq})\Sigma + \eta^-
\widetilde{P}_{gq}\Delta_\Sigma+ (P_{gg} + e_\Sigma^2 \widetilde{P}_{gg})g
+ e_\Sigma^2 \widetilde{P}_{g\gamma} \gamma\\
\\
\displaystyle \mu^2\frac{\partial \gamma}{\partial \mu^2} &=& \eta^+\widetilde{P}_{\gamma q}\Sigma + \eta^-\widetilde{P}_{\gamma q}\Delta_\Sigma+ e_\Sigma^2 \widetilde{P}_{\gamma g}g
+ e_\Sigma^2 \widetilde{P}_{\gamma\gamma} \gamma\\
\\
\displaystyle \mu^2\frac{\partial \Sigma}{\partial \mu^2} &=&
\displaystyle 
\left[P_{qq} + \eta^+\widetilde{P}^+
  +\frac{\eta^+e_\Sigma^2}{n_f}(\widetilde{P}_{qq} -
  \widetilde{P}^+)\right]\Sigma+ \left[\eta^-\widetilde{P}^++\frac{\eta^-e_\Sigma^2}{n_f}(\widetilde{P}_{qq} -
  \widetilde{P}^+)\right]\Delta_\Sigma\\
\\
 &+&  2(n_fP_{qg} + e_\Sigma^2 \widetilde{P}_{qg})g
+ 2e_\Sigma^2 \widetilde{P}_{q\gamma} \gamma\\
\\
\displaystyle \mu^2\frac{\partial \Delta_\Sigma}{\partial \mu^2} &=&\displaystyle
\left[\eta^-\widetilde{P}^++\frac{n_u-n_d}{n_f}(P_{qq}-P^+)+\frac{\eta^+\delta_e^2}{n_f}(\widetilde{P}_{qq}-\widetilde{P}^+)\right]\Sigma + \left[P^+ +
\eta^+\widetilde{P}^++\frac{\eta^-\delta_e^2}{n_f}(\widetilde{P}_{qq}-\widetilde{P}^+)\right]\Delta_\Sigma\\
\\
&+& 2[(n_u-n_d)P_{qg} + \delta_e^2 \widetilde{P}_{qg}]g
+ 2\delta_e^2 \widetilde{P}_{q\gamma} \gamma\\
\end{array}
\right.
\\
\\
\begin{array}{rcl}
\\
\displaystyle \mu^2\frac{\partial T^u_{1,2}}{\partial \mu^2} &=&
\displaystyle (P^+ + e_u^2\widetilde{P}^+) T^u_{1,2}\\
\\
\displaystyle \mu^2\frac{\partial T^d_{1,2}}{\partial \mu^2} &=&
\displaystyle (P^+ + e_d^2\widetilde{P}^+) T^d_{1,2}
\end{array}
\\
\\
\left\{
\begin{array}{rcl}
\displaystyle \mu^2\frac{\partial V}{\partial \mu^2} &=& \displaystyle
(P^V+\eta^+\widetilde{P}^- ) V + \eta^- \widetilde{P}^-\Delta_V\\
\\
\displaystyle \mu^2\frac{\partial \Delta_V}{\partial \mu^2} &=&
\displaystyle  \left[\frac{n_u-n_d}{n_f}(P^V-P^-) +\eta^-\widetilde{P}^- \right]V + \left[P^-+
  \eta^+\widetilde{P}^-\right]\Delta_V
\end{array}
\right.
\\
\\
\begin{array}{rcl}
\displaystyle \mu^2\frac{\partial V^u_{1,2}}{\partial \mu^2} &=&
\displaystyle (P^- + e_u^2\widetilde{P}^-) V^u_{1,2}\\
\\
\displaystyle \mu^2\frac{\partial V^d_{1,2}}{\partial \mu^2} &=&
\displaystyle (P^- + e_d^2\widetilde{P}^-) V^d_{1,2}
\end{array}
\end{array}
\label{pippo}
\end{equation}
with the definitions:
\begin{equation}
\begin{array}{l}
\displaystyle  e_\Sigma^2 = N_c(n_u e_u^2 +n_d e_d^2)\\
\\
\displaystyle \delta_e^2 = N_c(n_u e_u^2 -n_d e_d^2)\\
\\
\displaystyle \eta^{\pm} = \frac{1}{2}(e_u^2 \pm e_d^2)
\end{array}
\end{equation}
and where we have used the curly bracket to denote the coupled
equations. The main thing to notice is that there are two coupled
sub-systems. This is in contrast with what we had in pure QCD where
there was only one coupled system.

Now let us write the Eq.~(\ref{pippo}) in a matricial form, separating
the pure QCD splitting functions (those without tilde) from the QED
contributions:
\begin{equation}
\begin{array}{rcl}
\displaystyle\mu^2\frac{\partial}{\partial \mu^2}
\begin{pmatrix}
g\\
\gamma\\
\Sigma\\
\Delta_\Sigma
\end{pmatrix} &=& \displaystyle \left[
\begin{pmatrix}
P_{gg} & 0 & P_{gq} & 0 \\
0 & 0 & 0 & 0 \\
2n_fP_{qg} & 0 & P_{qq} & 0 \\
\frac{n_u-n_d}{n_f} 2n_fP_{qg} & 0 & \frac{n_u-n_d}{n_f}(P_{qq}-P^+) & P^+
\end{pmatrix}\right.
\\
\\
&+&\left.\begin{pmatrix}
e_\Sigma^2 \widetilde{P}_{gg}          & e_\Sigma^2 \widetilde{P}_{g\gamma} & \eta^+\widetilde{P}_{gq} & \eta^-\widetilde{P}_{gq} \\
e_\Sigma^2 \widetilde{P}_{\gamma g} & e_\Sigma^2 \widetilde{P}_{\gamma\gamma} & \eta^+\widetilde{P}_{\gamma q} &\eta^-\widetilde{P}_{\gamma q} \\
2 e_\Sigma^2 \widetilde{P}_{qg} & 2 e_\Sigma^2 \widetilde{P}_{q\gamma}
& \eta^+\widetilde{P}^++\frac{\eta^+e_\Sigma^2}{n_f}(\widetilde{P}_{qq}-\widetilde{P}^+)  & \eta^-\widetilde{P}^++\frac{\eta^-e_\Sigma^2}{n_f}(\widetilde{P}_{qq}-\widetilde{P}^+)\\
2 \delta_e^2 \widetilde{P}_{qg} & 2 \delta_e^2 \widetilde{P}_{q\gamma}
&\eta^-\widetilde{P}^++\frac{\eta^+\delta_e^2}{n_f}(\widetilde{P}_{qq}-\widetilde{P}^+)
&\eta^+\widetilde{P}^++\frac{\eta^-\delta_e^2}{n_f}(\widetilde{P}_{qq}-\widetilde{P}^+)
\end{pmatrix}
\right]
\begin{pmatrix}
g\\
\gamma\\
\Sigma\\
\Delta_\Sigma
\end{pmatrix}
\end{array}
\label{SingletDGLAP}
\end{equation}

\begin{equation}
\displaystyle\mu^2\frac{\partial}{\partial \mu^2}
\begin{pmatrix}
V\\
\Delta_V
\end{pmatrix} = 
\left[
\begin{pmatrix}
P^V & 0 \\
\frac{n_u-n_d}{n_f}(P^V-P^-)  & P^-
\end{pmatrix}
+
\begin{pmatrix}
\eta^+\widetilde{P}^- & \eta^-\widetilde{P}^- \\
\eta^-\widetilde{P}^- & \eta^+\widetilde{P}^- 
\end{pmatrix}
\right]
\begin{pmatrix}
V\\
\Delta_V
\end{pmatrix}
\label{NonSingletDGLAP}
\end{equation}

It should finally be noticed that, every time one of the quark
flavours is not active, the associated non-singlet distributions
$T_{1,2}^{u,d}$ and $V_{1,2}^{u,d}$ involving that quark flavour,
start evolving as a singlet distribution according to the following
equations:
\begin{equation}
\begin{array}{l}
\displaystyle T_{1,2}^{u} = \frac{\Sigma+\Delta_\Sigma}{2}\,,\\
\\
\displaystyle T_{1,2}^{d} = \frac{\Sigma-\Delta_\Sigma}{2}\,,\\
\\
\displaystyle V_{1,2}^{u} = \frac{V+\Delta_V}{2}\,,\\
\\
\displaystyle V_{1,2}^{d} = \frac{V-\Delta_V}{2}\,.
\end{array}
\end{equation}

\section{QED corrections at LO}

If we consider only LO QED corrections to the PDF evolution equations,
there are a few simplifications that make the evolution sistem
simpler. In particular we have that:
\begin{equation}
\widetilde{P}_{gg} = \widetilde{P}_{g\gamma} = \widetilde{P}_{\gamma
  g} = \widetilde{P}_{gq} = \widetilde{P}_{qg} = 0\,.
\end{equation}
In addition:
\begin{equation}
\widetilde{P}^+ = \widetilde{P}^- = \widetilde{P}_{qq}\,.
\end{equation}

With these simplifications we can rewrite the above evolution systems
as follows:
\begin{equation}
\begin{array}{rcl}
\displaystyle\mu^2\frac{\partial}{\partial \mu^2}
\begin{pmatrix}
g\\
\gamma\\
\Sigma\\
\Delta_\Sigma
\end{pmatrix} &=& \displaystyle \left[
\begin{pmatrix}
P_{gg} & 0 & P_{gq} & 0 \\
0 & 0 & 0 & 0 \\
2n_fP_{qg} & 0 & P_{qq} & 0 \\
\frac{n_u-n_d}{n_f} 2n_fP_{qg} & 0 & \frac{n_u-n_d}{n_f}(P_{qq}-P^+) & P^+
\end{pmatrix}\right.
\\
\\
&+&\left.\begin{pmatrix}
0 & 0 & 0 & 0 \\
0 & e_\Sigma^2 \widetilde{P}_{\gamma\gamma} & \eta^+\widetilde{P}_{\gamma q} &\eta^-\widetilde{P}_{\gamma q} \\
0 & 2 e_\Sigma^2 \widetilde{P}_{q\gamma} & \eta^+\widetilde{P}_{qq} & \eta^-\widetilde{P}_{qq}\\
0 & 2 \delta_e^2 \widetilde{P}_{q\gamma}
&\eta^-\widetilde{P}_{qq}
&\eta^+\widetilde{P}_{qq}
\end{pmatrix}
\right]
\begin{pmatrix}
g\\
\gamma\\
\Sigma\\
\Delta_\Sigma
\end{pmatrix}
\end{array}
\label{APFELsys}
\end{equation}

\begin{equation}
\displaystyle\mu^2\frac{\partial}{\partial \mu^2}
\begin{pmatrix}
V\\
\Delta_V
\end{pmatrix} = 
\left[
\begin{pmatrix}
P^V & 0 \\
\frac{n_u-n_d}{n_f}(P^V-P^-)  & P^- 
\end{pmatrix}
+
\begin{pmatrix}
\eta^+\widetilde{P}_{qq} & \eta^-\widetilde{P}_{qq} \\
\eta^-\widetilde{P}_{qq} & \eta^+\widetilde{P}_{qq}
\end{pmatrix}
\right]
\begin{pmatrix}
V\\
\Delta_V
\end{pmatrix}
\end{equation}

\begin{equation}
\begin{array}{l}
\begin{array}{rcl}
\\
\displaystyle \mu^2\frac{\partial T^u_{1,2}}{\partial \mu^2} &=&
\displaystyle (P^+ + e_u^2\widetilde{P}_{qq}) T^u_{1,2}\\
\\
\displaystyle \mu^2\frac{\partial T^d_{1,2}}{\partial \mu^2} &=&
\displaystyle (P^+ + e_d^2\widetilde{P}_{qq}) T^d_{1,2}
\end{array}
\\
\\
\begin{array}{rcl}
\displaystyle \mu^2\frac{\partial V^u_{1,2}}{\partial \mu^2} &=&
\displaystyle (P^- + e_u^2\widetilde{P}_{qq}) V^u_{1,2}\\
\\
\displaystyle \mu^2\frac{\partial V^d_{1,2}}{\partial \mu^2} &=&
\displaystyle (P^- + e_d^2\widetilde{P}_{qq}) V^d_{1,2}
\end{array}
\end{array}
\end{equation}

Notice that Eq.~(\ref{APFELsys}), recognising that
$2e_\Sigma^2=\theta^-$ and $2\delta_e^2=\theta^+$, is consistent with
Eq.~(9) of the APFEL paper~\cite{Bertone:2013vaa}.

\section{Matching conditions in the QED evolution basis}

In order to couple QCD and QED evolutions in the VFNS, it is necessary
to work out the structure of the QCD matching conditions in the
evolution basis defined in Sect.~\ref{QEDEvBasis}. The matching
conditions up to $\mathcal{O}(\alpha_s^2)$ have been computed in
Ref.~\cite{Buza:1996wv} and for the transition between the $n_f$- and
the $(n_f+1)$-scheme for the gluon $g$, the light quarks $l$ and the
heavy quarks $h$ and $\overline{h}$ read:
\begin{equation}
g^{(n_f+1)} =
\left[1+\left(\frac{\alpha_s}{4\pi}\right)A_{gg,h}^{S,(1)}+\left(\frac{\alpha_s}{4\pi}\right)^2A_{gg,h}^{S,(2)}\right]g^{(n_f)}+
\left(\frac{\alpha_s}{4\pi}\right)^2A^{S,(2)}_{gq,h}\Sigma^{(n_f)}
\end{equation}
\begin{equation}
l^{(n_f+1)}=\left[1+\left(\frac{\alpha_s}{4\pi}\right)^2A_{qq,h}^{N\!S,(2)}\right]l^{(n_f)}\,. 
\end{equation}
\begin{equation}
h^{(n_f+1)}=\overline{h}^{(n_f+1)}=\frac12\left(\frac{\alpha_s}{4\pi}\right)^2\tilde{A}^{S,(2)}_{hq}\Sigma^{(n_f)}+\frac12\left[\left(\frac{\alpha_s}{4\pi}\right)\tilde{A}^{S,(1)}_{hq}+\left(\frac{\alpha_s}{4\pi}\right)^2\tilde{A}^{S,(2)}_{hg}\right]g^{(n_f)}\,.
\end{equation}

The first observation is that, given that the dynamical production of
heavy quarks at threshold always happens in pairs, it is clear that
all distributions that involve only differences between flavours and
anti-flavours match multiplicatively at the thresholds as the light
flavours do. In particular:
\begin{equation}
{V^{(n_f+1)} \choose \Delta_V^{(n_f+1)}} =
\left[1+\left(\frac{\alpha_s}{4\pi}\right)^2A_{qq,h}^{N\!S,(2)}\right] {V^{(n_f)} \choose \Delta_V^{(n_f)}} =
\end{equation}
Now we turn to consider $\Sigma$ and $\Delta_\Sigma$. The first case
is easy and we have that:
\begin{equation}
\begin{array}{rcl}
\displaystyle \Sigma^{(n_f+1)} &=&
\displaystyle
                                   \sum_{l=1}^{n_f}(l^{(n_f+1)}+\overline{l}^{(n_f+1)})+(h^{(n_f+1)}+\overline{h}^{(n_f+1)})\\
\\
&=&\displaystyle
    \left[1+\left(\frac{\alpha_s}{4\pi}\right)^2\left(\tilde{A}^{S,(2)}_{hq}+A_{qq,h}^{N\!S,(2)}\right)\right]\Sigma^{(n_f)}+\left[\left(\frac{\alpha_s}{4\pi}\right)\tilde{A}^{S,(1)}_{hg}+\left(\frac{\alpha_s}{4\pi}\right)^2\tilde{A}^{S,(2)}_{hg}\right]
    g^{(n_f)}\,.
\end{array}
\end{equation}
The second case is instead a bit trickier because:
\begin{equation}
\Delta_\Sigma ^{(n_f+1)} = \sum_{q=n_{\rm up}}
(q^{(n_f+1)}+\overline{q}^{(n_f+1)}) - \sum_{q=n_{\rm down}}
(q^{(n_f+1)}+\overline{q}^{(n_f+1)})=\Sigma_{\rm up}^{(n_f+1)}-\Sigma_{\rm down}^{(n_f+1)}
\end{equation}
and thus the way how matching conditions have to be applied depends on
whether the $(n_f+1)$-th quark is of type up or down. In particular,
every time that a threshold is crossed, only one of the components
$\Sigma_{\rm up}$ or $\Sigma_{\rm down}$ will get the contribution
from the heavy quark $h$:
\begin{equation}
\begin{array}{rcl}
\displaystyle \Delta_\Sigma ^{(n_f+1)} &=&
\displaystyle \left[1+\left(\frac{\alpha_s}{4\pi}\right)^2A_{qq,h}^{N\!S,(2)}\right]\Delta_\Sigma ^{(n_f)}\\
\\
&\pm&\displaystyle  \left(\frac{\alpha_s}{4\pi}\right)^2\tilde{A}^{S,(2)}_{hq}\Sigma^{(n_f)}\pm \left[\left(\frac{\alpha_s}{4\pi}\right)\tilde{A}^{S,(1)}_{hg}+\left(\frac{\alpha_s}{4\pi}\right)^2\tilde{A}^{S,(2)}_{hg}\right]g^{(n_f)}
\end{array}
\end{equation}
with $+$ if $h$ is an up-type quark and $-$ if it is a down-type
quark. The matching conditions for $g$, $\Sigma$, and $\Delta_\Sigma$
can be written in a matricial form as:
\begin{equation}
\begin{pmatrix}
g ^{(n_f+1)}\\
\Sigma ^{(n_f+1)}\\
\Delta_\Sigma ^{(n_f+1)}
\end{pmatrix} = 
\left[\begin{pmatrix}
1 & 0 & 0 \\
0 & 1 & 0\\
0 & 0 & 1\\
\end{pmatrix}
+
\left(\frac{\alpha_s}{4\pi}\right)
\begin{pmatrix}
A_{gg,h}^{S,(1)} & 0 & 0 \\
\tilde{A}^{S,(1)}_{hg} & 0 & 0\\
\pm\tilde{A}^{S,(1)}_{hg} & 0 & 0\\
\end{pmatrix}+
\left(\frac{\alpha_s}{4\pi}\right)^2
\begin{pmatrix}
A_{gg,h}^{S,(2)} & A^{S,(2)}_{gq,h} & 0 \\
\tilde{A}^{S,(2)}_{hg} & \tilde{A}^{S,(2)}_{hq}+A_{qq,h}^{N\!S,(2)} & 0\\
\pm\tilde{A}^{S,(2)}_{hg} & \pm \tilde{A}^{S,(2)}_{hq} & A_{qq,h}^{N\!S,(2)}\\
\end{pmatrix}
\right]
\begin{pmatrix}
g ^{(n_f)}\\
\Sigma ^{(n_f)}\\
\Delta_\Sigma ^{(n_f)}
\end{pmatrix}
\end{equation}
If we define:
\begin{equation}
\begin{array}{rcl}
A^{N\!S} &=& \displaystyle 1+\left(\frac{\alpha_s}{4\pi}\right)^2A_{qq,h}^{N\!S,(2)} \\
\\
A_{gg} &=& \displaystyle 1+\left(\frac{\alpha_s}{4\pi}\right)A_{gg,h}^{S,(1)}+\left(\frac{\alpha_s}{4\pi}\right)^2A_{gg,h}^{S,(2)}\\
\\
A_{gq} &=& \displaystyle \left(\frac{\alpha_s}{4\pi}\right)^2A^{S,(2)}_{gq,h} \\
\\
A_{qg} &=& \displaystyle \left(\frac{\alpha_s}{4\pi}\right)\tilde{A}^{S,(1)}_{hg}+\left(\frac{\alpha_s}{4\pi}\right)^2\tilde{A}^{S,(2)}_{hg}\\
\\
A_{qq} &=& \displaystyle 1+\left(\frac{\alpha_s}{4\pi}\right)^2\left(\tilde{A}^{S,(2)}_{hq}+A_{qq,h}^{N\!S,(2)}\right)
\end{array}
\end{equation}
we can shrink the system of equations above into:
\begin{equation}
\begin{pmatrix}
g^{(n_f+1)}\\
\gamma^{(n_f+1)}\\
\Sigma^{(n_f+1)}\\
\Delta_\Sigma^{(n_f+1)}
\end{pmatrix} = 
\begin{pmatrix}
A_{gg} & 0 & A_{gq} & 0 \\
0 & 1 & 0 & 0 \\
A_{qg} & 0 & A_{qq} & 0 \\
\pm A_{qg} & 0 & \pm(A_{qq}-A^{N\!S}) & A^{N\!S}
\end{pmatrix}
\begin{pmatrix}
g^{(n_f)}\\
\gamma^{(n_f)}\\
\Sigma^{(n_f)}\\
\Delta_\Sigma^{(n_f)}
\end{pmatrix}
\end{equation}
and in addition:
\begin{equation} {V^{(n_f+1)} \choose \Delta_V^{(n_f+1)}} = A^{N\!S}
  {V^{(n_f)} \choose \Delta_V^{(n_f)}}
\end{equation}
and:
\begin{equation}
h^{(n_f+1)}+\overline{h}^{(n_f+1)}=(A_{qq}-A^{N\!S})\Sigma^{(n_f)}+A_{qg}g^{(n_f)}\,.
\end{equation}

We finally turn to consider the matching of $T_1^u$, $T_2^u$, $T_1^d$,
and $T_2^d$. Due to the way these distributions are constructed, they
might behave in different ways below and above a given
threshold. Therefore we need to consider what happens at each of them.
\begin{itemize}
\item charm threshold ($n_f=3$):
\iffalse
\begin{equation}
\begin{array}{rcl}
\displaystyle T_1^{u,(n_f+1)} &=&\displaystyle \left(A_{qg},\quad 0,\quad -A_{qq}+\frac32A^{N\!S},\quad \frac12A^{N\!S}\right)
\begin{pmatrix}
g^{(n_f)}\\
\gamma^{(n_f)}\\
\Sigma^{(n_f)}\\
\Delta_\Sigma^{(n_f)}
\end{pmatrix}\\
\\
\displaystyle T_2^{u,(n_f+1)} &=&\displaystyle
\left(A_{qg},\quad 0,\quad A_{qq}-\frac12A^{N\!S},\quad\frac12A^{N\!S}\right)
\begin{pmatrix}
g^{(n_f)}\\
\gamma^{(n_f)}\\
\Sigma^{(n_f)}\\
\Delta_\Sigma^{(n_f)}
\end{pmatrix}\\
\\
\displaystyle T_1^{d,(n_f+1)} &=&\displaystyle A^{N\!S} T_1^{d,(n_f)} \\
\\
\displaystyle T_2^{d,(n_f+1)} &=& \displaystyle \left(0,\quad 0,\quad \frac12A^{N\!S},\quad -\frac12A^{N\!S}\right)
\begin{pmatrix}
g^{(n_f)}\\
\gamma^{(n_f)}\\
\Sigma^{(n_f)}\\
\Delta_\Sigma^{(n_f)}
\end{pmatrix}\\
\\
\displaystyle V_1^{u,(n_f+1)} &=&\displaystyle \left(\frac12A^{N\!S},\quad \frac12A^{N\!S}\right)
\begin{pmatrix}
V^{(n_f)}\\
\Delta_V^{(n_f)}
\end{pmatrix}\\
\\
\displaystyle V_2^{u,(n_f+1)} &=&\displaystyle \left(\frac12A^{N\!S},\quad \frac12A^{N\!S}\right)
\begin{pmatrix}
V^{(n_f)}\\
\Delta_V^{(n_f)}
\end{pmatrix}\\
\\
\displaystyle V_1^{d,(n_f+1)} &=&\displaystyle A^{N\!S} V_1^{d,(n_f)} \\
\\
\displaystyle V_2^{d,(n_f+1)} &=& \displaystyle \left(\frac12A^{N\!S},\quad -\frac12A^{N\!S}\right)
\begin{pmatrix}
V^{(n_f)}\\
\Delta_V^{(n_f)}
\end{pmatrix}
\end{array}
\end{equation}
\fi

%==============================================================================================
%==============================================================================================
%==============================================================================================
\begin{equation}
\tiny
\begin{pmatrix}
g^{(4)}\\
\gamma^{(4)}\\
\Sigma^{(4)}\\
\Delta_\Sigma^{(4)}\\
T_1^{u,(4)}\\
T_2^{u,(4)}\\
T_1^{d,(4)}\\
T_2^{d,(4)}\\
V^{(4)}\\
\Delta_V^{(4)}\\
V_1^{u,(4)}\\
V_2^{u,(4)}\\
V_1^{d,(4)}\\
V_2^{d,(4)}
\end{pmatrix}=
\left(
\begin{array}{c|cccc|cccc|cc|cccc}
& 0 & 1 & 2 & 3 & 4 & 5 & 6 & 7 & 8 & 9 & 10 & 11 & 12 & 13 \\
\hline
0 & A_{gg} & 0 & A_{gq} & 0 & 0 & 0 & 0 & 0 & 0 & 0 & 0 & 0 & 0 & 0 \\
1 & 0 & 1 & 0 & 0 & 0 & 0 & 0 & 0 & 0 & 0 & 0 & 0 & 0 & 0 \\
2 & A_{qg} & 0 & A_{qq} & 0 & 0 & 0 & 0 & 0 & 0 & 0 & 0 & 0 & 0 & 0 \\
3 & A_{qg} & 0 & (A_{qq}-A^{N\!S}) & A^{N\!S} & 0 & 0 & 0 & 0 & 0 & 0 & 0 & 0 & 0 & 0\\
\hline
4 & A_{qg} & 0 & -A_{qq}+\frac32A^{N\!S} & \frac12A^{N\!S} & 0 & 0 & 0 & 0 & 0 & 0 & 0 & 0 & 0 & 0\\
5 & A_{qg} & 0 & A_{qq}-\frac12A^{N\!S} & \frac12A^{N\!S} & 0 & 0 & 0 & 0 & 0 & 0 & 0 & 0 & 0 & 0\\
6 & 0 & 0 & 0 & 0 & 0 & 0 & A^{N\!S} & 0 & 0 & 0 & 0 & 0 & 0 & 0\\
7 & 0 & 0 & \frac12A^{N\!S} & -\frac12A^{N\!S} & 0 & 0 & 0 & 0 & 0 & 0 & 0 & 0 & 0 & 0\\
\hline
8 & 0 & 0 & 0 & 0 & 0 & 0 & 0 & 0 & A^{N\!S} & 0 & 0 & 0 & 0 & 0\\
9 & 0 & 0 & 0 & 0 & 0 & 0 & 0 & 0 & 0 & A^{N\!S} & 0 & 0 & 0 & 0\\
\hline
10 & 0 & 0 & 0 & 0 & 0 & 0 & 0 & 0 & \frac12A^{N\!S} & \frac12A^{N\!S} & 0 & 0 & 0 & 0\\
11 & 0 & 0 & 0 & 0 & 0 & 0 & 0 & 0 & \frac12A^{N\!S} & \frac12A^{N\!S} & 0 & 0 & 0 & 0\\
12 & 0 & 0 & 0 & 0 & 0 & 0 & 0 & 0 & 0 & 0 & 0 & 0 & A^{N\!S} & 0\\
13 & 0 & 0 & 0 & 0 & 0 & 0 & 0 & 0 & \frac12A^{N\!S} & -\frac12A^{N\!S} & 0 & 0 & 0 & 0
\end{array}
\right)
\begin{pmatrix}
g^{(3)}\\
\gamma^{(3)}\\
\Sigma^{(3)}\\
\Delta_\Sigma^{(3)}\\
T_1^{u,(3)}\\
T_2^{u,(3)}\\
T_1^{d,(3)}\\
T_2^{d,(3)}\\
V^{(3)}\\
\Delta_V^{(3)}\\
V_1^{u,(3)}\\
V_2^{u,(3)}\\
V_1^{d,(3)}\\
V_2^{d,(3)}
\end{pmatrix}
\end{equation}
%==============================================================================================
%==============================================================================================
%==============================================================================================

\item bottom threshold ($n_f=4$):
\iffalse
\begin{equation}
\begin{array}{rcl}
\displaystyle T_1^{u,(n_f+1)} &=&\displaystyle A^{N\!S} T_1^{u,(n_f)} \\
\\
\displaystyle T_2^{u,(n_f+1)} &=&\displaystyle \left(0,\quad 0,\quad \frac12A^{N\!S},\quad \frac12A^{N\!S}\right)
\begin{pmatrix}
g^{(n_f)}\\
\gamma^{(n_f)}\\
\Sigma^{(n_f)}\\
\Delta_\Sigma^{(n_f)}
\end{pmatrix}\\
\displaystyle T_1^{d,(n_f+1)} &=&\displaystyle A^{N\!S}T_1^{d,(n_f)} \\
\\
\displaystyle T_2^{d,(n_f+1)} &=& \displaystyle
                                  \left( -2A_{qg},\quad 0,\quad -2A_{qq}+\frac52A^{N\!S},\quad -\frac12A^{N\!S} \right)
\begin{pmatrix}
g^{(n_f)}\\
\gamma^{(n_f)}\\
\Sigma^{(n_f)}\\
\Delta_\Sigma^{(n_f)}
\end{pmatrix}\\
\\
\displaystyle V_1^{u,(n_f+1)} &=&\displaystyle A^{N\!S}V_1^{u,(n_f)}\\
\\
\displaystyle V_2^{u,(n_f+1)} &=&\displaystyle \left(\frac12A^{N\!S},\quad \frac12A^{N\!S}\right)
\begin{pmatrix}
V^{(n_f)}\\
\Delta_V^{(n_f)}
\end{pmatrix}\\
\\
\displaystyle V_1^{d,(n_f+1)} &=&\displaystyle A^{N\!S} V_1^{d,(n_f)} \\
\\
\displaystyle V_2^{d,(n_f+1)} &=& \displaystyle \left(\frac12A^{N\!S},\quad -\frac12A^{N\!S}\right)
\begin{pmatrix}
V^{(n_f)}\\
\Delta_V^{(n_f)}
\end{pmatrix}
\end{array}
\end{equation}
\fi

%==============================================================================================
%==============================================================================================
%==============================================================================================
\begin{equation}
\tiny
\begin{pmatrix}
g^{(5)}\\
\gamma^{(5)}\\
\Sigma^{(5)}\\
\Delta_\Sigma^{(5)}\\
T_1^{u,(5)}\\
T_2^{u,(5)}\\
T_1^{d,(5)}\\
T_2^{d,(5)}\\
V^{(5)}\\
\Delta_V^{(5)}\\
V_1^{u,(5)}\\
V_2^{u,(5)}\\
V_1^{d,(5)}\\
V_2^{d,(5)}
\end{pmatrix}=
\left(
\begin{array}{c|cccc|cccc|cc|cccc}
& 0 & 1 & 2 & 3 & 4 & 5 & 6 & 7 & 8 & 9 & 10 & 11 & 12 & 13 \\
\hline
0 & A_{gg} & 0 & A_{gq} & 0 & 0 & 0 & 0 & 0 & 0 & 0 & 0 & 0 & 0 & 0 \\
1 & 0 & 1 & 0 & 0 & 0 & 0 & 0 & 0 & 0 & 0 & 0 & 0 & 0 & 0 \\
2 & A_{qg} & 0 & A_{qq} & 0 & 0 & 0 & 0 & 0 & 0 & 0 & 0 & 0 & 0 & 0 \\
3 & -A_{qg} & 0 & -(A_{qq}-A^{N\!S}) & A^{N\!S} & 0 & 0 & 0 & 0 & 0 & 0 & 0 & 0 & 0 & 0\\
\hline
4 & 0 & 0 & 0 & 0 & A^{N\!S} & 0 & 0 & 0 & 0 & 0 & 0 & 0 & 0 & 0\\
5 & 0 & 0 & \frac12A^{N\!S} & \frac12A^{N\!S} & 0 & 0 & 0 & 0 & 0 & 0 & 0 & 0 & 0 & 0\\
6 & 0 & 0 & 0 & 0 & 0 & 0 & A^{N\!S} & 0 & 0 & 0 & 0 & 0 & 0 & 0\\
7 & -2A_{qg} & 0 & -2A_{qq}+\frac52A^{N\!S} & -\frac12A^{N\!S} & 0 & 0 & 0 & 0 & 0 & 0 & 0 & 0 & 0 & 0\\
\hline
8 & 0 & 0 & 0 & 0 & 0 & 0 & 0 & 0 & A^{N\!S} & 0 & 0 & 0 & 0 & 0\\
9 & 0 & 0 & 0 & 0 & 0 & 0 & 0 & 0 & 0 & A^{N\!S} & 0 & 0 & 0 & 0\\
\hline
10 & 0 & 0 & 0 & 0 & 0 & 0 & 0 & 0 & 0 & 0 &  A^{N\!S} & 0 & 0 & 0\\
11 & 0 & 0 & 0 & 0 & 0 & 0 & 0 & 0 & \frac12A^{N\!S} & \frac12A^{N\!S} & 0 & 0 & 0 & 0\\
12 & 0 & 0 & 0 & 0 & 0 & 0 & 0 & 0 & 0 & 0 & 0 & 0 & A^{N\!S} & 0\\
13 & 0 & 0 & 0 & 0 & 0 & 0 & 0 & 0 & \frac12A^{N\!S} & -\frac12A^{N\!S} & 0 & 0 & 0 & 0
\end{array}
\right)
\begin{pmatrix}
g^{(4)}\\
\gamma^{(4)}\\
\Sigma^{(4)}\\
\Delta_\Sigma^{(4)}\\
T_1^{u,(4)}\\
T_2^{u,(4)}\\
T_1^{d,(4)}\\
T_2^{d,(4)}\\
V^{(4)}\\
\Delta_V^{(4)}\\
V_1^{u,(4)}\\
V_2^{u,(4)}\\
V_1^{d,(4)}\\
V_2^{d,(4)}
\end{pmatrix}
\end{equation}
%==============================================================================================
%==============================================================================================
%==============================================================================================

\item top threshold ($n_f=5$):
\iffalse
\begin{equation}
\begin{array}{rcl}
\displaystyle T_1^{u,(n_f+1)} &=&\displaystyle A^{N\!S}T_1^{u,(n_f)} \\
\\
\displaystyle T_2^{u,(n_f+1)} &=&\displaystyle
\left( -2A_{qg},\quad 0,\quad -2A_{qq}+\frac52A^{N\!S},\quad \frac12A^{N\!S} \right)
\begin{pmatrix}
g^{(n_f)}\\
\gamma^{(n_f)}\\
\Sigma^{(n_f)}\\
\Delta_\Sigma^{(n_f)}
\end{pmatrix}\\
\\
\displaystyle T_1^{d,(n_f+1)} &=&\displaystyle A^{N\!S}T_1^{d,(n_f)} \\
\\
\displaystyle T_2^{d,(n_f+1)} &=& \displaystyle A^{N\!S}T_2^{d,(n_f)}\\
\\
\displaystyle V_1^{u,(n_f+1)} &=&\displaystyle A^{N\!S}V_1^{u,(n_f)}\\
\\
\displaystyle V_2^{u,(n_f+1)} &=&\displaystyle \left(\frac12A^{N\!S},\quad \frac12A^{N\!S}\right)
\begin{pmatrix}
V^{(n_f)}\\
\Delta_V^{(n_f)}
\end{pmatrix}\\
\\
\displaystyle V_1^{d,(n_f+1)} &=&\displaystyle A^{N\!S} V_1^{d,(n_f)} \\
\\
\displaystyle V_2^{d,(n_f+1)} &=& \displaystyle A^{N\!S}V_1^{d,(n_f)}
\end{array}
\end{equation}
\fi

%==============================================================================================
%==============================================================================================
%==============================================================================================
\begin{equation}
\tiny
\begin{pmatrix}
g^{(6)}\\
\gamma^{(6)}\\
\Sigma^{(6)}\\
\Delta_\Sigma^{(6)}\\
T_1^{u,(6)}\\
T_2^{u,(6)}\\
T_1^{d,(6)}\\
T_2^{d,(6)}\\
V^{(6)}\\
\Delta_V^{(6)}\\
V_1^{u,(6)}\\
V_2^{u,(6)}\\
V_1^{d,(6)}\\
V_2^{d,(6)}
\end{pmatrix}=
\left(\begin{array}{c|cccc|cccc|cc|cccc}
& 0 & 1 & 2 & 3 & 4 & 5 & 6 & 7 & 8 & 9 & 10 & 11 & 12 & 13 \\
\hline
0 & A_{gg} & 0 & A_{gq} & 0 & 0 & 0 & 0 & 0 & 0 & 0 & 0 & 0 & 0 & 0 \\
1 & 0 & 1 & 0 & 0 & 0 & 0 & 0 & 0 & 0 & 0 & 0 & 0 & 0 & 0 \\
2 & A_{qg} & 0 & A_{qq} & 0 & 0 & 0 & 0 & 0 & 0 & 0 & 0 & 0 & 0 & 0 \\
3 & A_{qg} & 0 & (A_{qq}-A^{N\!S}) & A^{N\!S} & 0 & 0 & 0 & 0 & 0 & 0 & 0 & 0 & 0 & 0\\
\hline
4 & 0 & 0 & 0 & 0 & A^{N\!S} & 0 & 0 & 0 & 0 & 0 & 0 & 0 & 0 & 0\\
5 & -2A_{qg} & 0 & -2A_{qq}+\frac52A^{N\!S} & \frac12A^{N\!S} & 0 & 0 & 0 & 0 & 0 & 0 & 0 & 0 & 0 & 0\\
6 & 0 & 0 & 0 & 0 & 0 & 0 & A^{N\!S} & 0 & 0 & 0 & 0 & 0 & 0 & 0\\
7 & 0 & 0 & 0 & 0 & 0 & 0 & 0 & A^{N\!S} & 0 & 0 & 0 & 0 & 0 & 0\\
\hline
8 & 0 & 0 & 0 & 0 & 0 & 0 & 0 & 0 & A^{N\!S} & 0 & 0 & 0 & 0 & 0\\
9 & 0 & 0 & 0 & 0 & 0 & 0 & 0 & 0 & 0 & A^{N\!S} & 0 & 0 & 0 & 0\\
\hline
10 & 0 & 0 & 0 & 0 & 0 & 0 & 0 & 0 & 0 & 0 &  A^{N\!S} & 0 & 0 & 0\\
11 & 0 & 0 & 0 & 0 & 0 & 0 & 0 & 0 & \frac12A^{N\!S} & \frac12A^{N\!S} & 0 & 0 & 0 & 0\\
12 & 0 & 0 & 0 & 0 & 0 & 0 & 0 & 0 & 0 & 0 & 0 & 0 & A^{N\!S} & 0\\
13 & 0 & 0 & 0 & 0 & 0 & 0 & 0 & 0 & 0 & 0 & 0 & 0 & 0 & A^{N\!S}
\end{array}
\right)
\begin{pmatrix}
g^{(5)}\\
\gamma^{(5)}\\
\Sigma^{(5)}\\
\Delta_\Sigma^{(5)}\\
T_1^{u,(5)}\\
T_2^{u,(5)}\\
T_1^{d,(5)}\\
T_2^{d,(5)}\\
V^{(5)}\\
\Delta_V^{(5)}\\
V_1^{u,(5)}\\
V_2^{u,(5)}\\
V_1^{d,(5)}\\
V_2^{d,(5)}
\end{pmatrix}
\end{equation}
%==============================================================================================
%==============================================================================================
%==============================================================================================
\end{itemize}

\section{Including the lepton PDFs}

In order to include the lepton PDFs in the coupled QCD$\times$QED
DGLAP evolution, we first need to make some preliminary
considerations. The first thing to notice is that, considering only
QED corrections and not electroweak corrections, we do not need to
introduce neutrino PDFs as neutrinos do not couple neither to the
gluon nor to the photon. Therefore the only PDFs that need to be
introduced are those of the charged leptons $e^\pm$, $\mu^\pm$ and
$\tau^\pm$. The second point to consider is that the absolute value of
the charge of all leptons is always equal to one and, since charges
enter the DGLAP evolution as squares, this allows us to maintain, at
least in the leptonic sector, the isospin symmetry
$l^+\leftrightarrow l^-$. As a final remark, we notice that, while the
muon and the electron mass, $\simeq 0.5$ MeV and $\simeq 105$ MeV
respectively, are below the $\Lambda_{\rm QCD}$ and thus they do not
introduce any threshold in the DGLAP evolution, the tauon mass, whose
mass is $m_\tau = 1.777$ GeV, is well above $\Lambda_{\rm QCD}$ and
above the initial scale at which PDFs are usually parameterised
($Q_0 = 1 - 1.4$ GeV). As a consequence, the presence of tauons in the
evolution implies the introduction of a new threshold between $m_c$
and $m_b$ at which the $\tau$ PDFs are dynamically generated from the
photon. On the contrary, $e$ and $\mu$ PDFs cannot be dynamically
generated by evolution and need to be parameterised at the initial
scale. We will see later how the functional form of the $e$ and $\mu$
PDFs at the initial scale can be guessed by assuming a dynamical
generation by photon splitting at their respective mass thresholds.

In order to write the full DGLAP equations in the presence of quarks,
leptons, gluon and photon, we start considering
Eq.~(\ref{EvenMoreGeneralDGLAP}) where we add the leptons
$\ell_\alpha$ and $\overline{\ell}_\beta$:
\begin{equation}
\mu^2\frac{\partial}{\partial \mu^2}
\begin{pmatrix}
\ell_\alpha\\
u_j \\
d_i\\
g\\
\gamma\\
\overline{d}_k\\
\overline{u}_h\\
\overline{\ell}_\beta
\end{pmatrix} = \sum_{e,l,m,n,\gamma,\delta} 
\begin{pmatrix}
\mathcal{P}_{\ell_\alpha \ell_\gamma} & \mathcal{P}_{\ell_\alpha u_e} & \mathcal{P}_{\ell_\alpha d_l} & \mathcal{P}_{\ell_\alpha g} & \mathcal{P}_{\ell_\alpha \gamma} & \mathcal{P}_{\ell_\alpha \overline{d}_m} & \mathcal{P}_{\ell_\alpha \overline{u}_n} & \mathcal{P}_{\ell_\alpha \overline{\ell}_\delta} \\ 
\mathcal{P}_{u_j\ell_\gamma} & \mathcal{P}_{u_ju_e} & \mathcal{P}_{u_jd_l} & \mathcal{P}_{u_jg} & \mathcal{P}_{u_j\gamma} & \mathcal{P}_{u_j\overline{d}_m} & \mathcal{P}_{u_j\overline{u}_n} & \mathcal{P}_{u_j\overline{\ell}_\delta} \\ 
\mathcal{P}_{d_i\ell_\gamma} & \mathcal{P}_{d_iu_e} & \mathcal{P}_{d_id_l} & \mathcal{P}_{d_ig} & \mathcal{P}_{d_i\gamma} & \mathcal{P}_{d_i\overline{d}_m} & \mathcal{P}_{d_i\overline{u}_n} & \mathcal{P}_{d_i\overline{\ell}_\delta} \\ 
\mathcal{P}_{g\ell_\gamma}  & \mathcal{P}_{gu_e}  & \mathcal{P}_{gd_l} & \mathcal{P}_{gg} & \mathcal{P}_{g\gamma} & \mathcal{P}_{g\overline{d}_m} & \mathcal{P}_{g\overline{u}_n} & \mathcal{P}_{g\overline{\ell}_\delta}\\
\mathcal{P}_{\gamma\ell_\gamma} & \mathcal{P}_{\gamma u_e} & \mathcal{P}_{\gamma d_l} & \mathcal{P}_{\gamma g} & \mathcal{P}_{\gamma\gamma} & \mathcal{P}_{\gamma\overline{d}_m} & \mathcal{P}_{\gamma\overline{u}_n} & \mathcal{P}_{\gamma\overline{\ell}_\delta}\\
\mathcal{P}_{\overline{d}_k\ell_\gamma} & \mathcal{P}_{\overline{d}_ku_e} & \mathcal{P}_{\overline{d}_kd_l} & \mathcal{P}_{\overline{d}_kg} & \mathcal{P}_{\overline{d}_k\gamma} & \mathcal{P}_{\overline{d}_k\overline{d}_m} & \mathcal{P}_{\overline{d}_k\overline{u}_n} & \mathcal{P}_{\overline{d}_k\overline{\ell}_\delta}\\
\mathcal{P}_{\overline{u}_h\ell_\gamma} & \mathcal{P}_{\overline{u}_hu_e} & \mathcal{P}_{\overline{u}_hd_l} & \mathcal{P}_{\overline{u}_hg} & \mathcal{P}_{\overline{u}_h\gamma} & \mathcal{P}_{\overline{u}_h\overline{d}_m} & \mathcal{P}_{\overline{u}_h\overline{u}_n} & \mathcal{P}_{\overline{u}_h\overline{\ell}_\delta}\\
\mathcal{P}_{\overline{\ell}_\beta\ell_\gamma} & \mathcal{P}_{\overline{\ell}_\beta u_e} & \mathcal{P}_{\overline{\ell}_\beta d_l} & \mathcal{P}_{\overline{\ell}_\beta g} & \mathcal{P}_{\overline{\ell}_\beta \gamma} & \mathcal{P}_{\overline{\ell}_\beta \overline{d}_m} & \mathcal{P}_{\overline{\ell}_\beta \overline{u}_n} & \mathcal{P}_{\overline{\ell}_\beta \overline{\ell}_\delta}
\end{pmatrix}
\begin{pmatrix}
\ell_\gamma\\
u_e \\
d_l\\
g\\
\gamma\\
\overline{d}_m\\
\overline{u}_n\\
\overline{\ell}_\delta
\end{pmatrix}\,,
\end{equation}
where $\ell_\alpha,\ell_\gamma \in \{e^-,\mu^-,\tau^-\}$ and
$\overline{\ell}_\beta,\overline{\ell}_\delta
\in\{e^+,\mu^+,\tau^+\}$. It should be clear that the introduction of
leptons automatically carries one or more powers of $\alpha$. As a
consequence Eq.~(\ref{EvenMoreGeneralDGLAPciao}) is untouched by the
inclusion of the leptons in the evolution and only the QED-correction
matrix gets contributions. In other words only the
$\widetilde{P}_{ab}$ component of the splitting-function matrix (see
Eq.~(\ref{eq:decomposition})) is modified when considering also
leptons.

Now we can start analising the splitting functions involving leptons.
First of all we observe that up to two loops leptons and gluon do not
couple so that one has:
\begin{equation}
\mathcal{P}_{\ell_\alpha g} = \mathcal{P}_{\overline{\ell}_\beta g} =
\mathcal{P}_{g\ell_\gamma} = \mathcal{P}_{g\overline{\ell}_\delta} =
0\,.
\end{equation}

% In addition, all splitting functions connecting quarks and leptons
% like $\mathcal{P}_{\ell_\alpha u_e}$ start at $\mathcal{O}(\alpha^2)$,
% that is at two loops and thus do not have any pure-QCD
% contribution. Finally, the splitting functions connecting leptons with
% leptons and the photon with leptons start at one loop but they are
% $\mathcal{O}(\alpha)$ and thus do not have any pure-QCD
% contribution. In addition, it is easy to realise that such splitting
% functions do not get QCD corrections up to three loops. In conclusion,
% the pure-QCD matrix in the r.h.s. of
% Eq.~(\ref{EvenMoreGeneralDGLAPciao}) is untouched by the inclusion of
% the leptons in the evolution and only the QED-correction matrix gets
% contributions.

For the splitting functions involving leptons and photon, we can use
charge conjugation invariance and flavour symmetry, up to two loops,
to obtain:
\begin{equation}
\begin{array}{l}
\mathcal{P}_{\ell_\alpha\ell_\beta} = \mathcal{P}_{\overline{\ell}_\alpha\overline{\ell}_\beta} = \delta_{ij} \widetilde{P}_{\ell\ell}^V+\widetilde{P}_{\ell\ell}^S\\
\mathcal{P}_{\overline{\ell}_\alpha\ell_\beta} =
\mathcal{P}_{\ell_\alpha\overline{\ell}_\beta}  = \delta_{ij} \widetilde{P}_{\overline{\ell}\ell}^V+\widetilde{P}_{\ell\ell}^S\\
\mathcal{P}_{\ell_\alpha\gamma}  =\mathcal{P}_{\overline{\ell}_\alpha\gamma} = \widetilde{P}_{\ell\gamma} \\
\mathcal{P}_{\gamma\ell_\alpha}  =\mathcal{P}_{\gamma\overline{\ell}_\alpha} = \widetilde{P}_{\gamma\ell} \,.
\end{array}
\label{decompositionLept}
\end{equation}
In addition, since to this perturbative order they are pure QED
splitting functions, they can be derived starting from the pure QCD
splitting functions just by adjusting the colour factors. Therefore,
apart from replacing the strong coupling $\alpha_s$ with the
fine-structure running constant $\alpha$ and assuming that the lepton
charge is one, we can simply write:
\begin{equation}
\begin{array}{l}
\widetilde{P}_{\ell\ell}^{V,S} = P_{qq}^{V,S}(T_R=1,C_F=1,C_A=0)\,,\\
\\
\widetilde{P}_{\overline{\ell}\ell}^V=P_{\overline{q}q}^{V}
(T_R=1,C_F=1,C_A=0) \,,\\
\\
\widetilde{P}_{\ell\gamma} = P_{qg}(T_R=1,C_F=1,C_A=0) \,,\\
\\
\widetilde{P}_{\gamma\ell} = P_{gq}(T_R=1,C_F=1,C_A=0) \,.
\end{array}
\label{decompositionLept}
\end{equation}

The last category of splitting functions that we need to consider is
that connecting quarks and leptons of the kind
$\mathcal{P}_{\ell_\alpha q_i}$ or
$\mathcal{P}_{q_i\ell_\alpha}$. They start at two loops, \textit{i.e.}
$\mathcal{O}(\alpha^2)$, and at this order they can be written as:
\begin{equation}
\mathcal{P}_{\ell_\alpha q_i} = \mathcal{P}_{q_i \ell_\alpha} =
\mathcal{P}_{\overline{\ell}_\alpha q_i} = \mathcal{P}_{q_i
  \overline{\ell}_\alpha} =
\mathcal{P}_{\ell_\alpha \overline{q}_i} = \mathcal{P}_{\overline{q}_i
  \ell_\alpha} = \mathcal{P}_{\overline{\ell}_\alpha \overline{q}_i} =
\mathcal{P}_{\overline{q}_i \overline{\ell}_\alpha} = e_{q_i}^2\widetilde{P}_{q\ell}^S\,.
\end{equation}

%My guess is that at $\mathcal{O}(\alpha\alpha_s)$ the
%$\widetilde{P}_{q\ell}^S$ can be obtained from the
%$\mathcal{O}(\alpha_s^2)$ contribution of the $P_{qq}$ splitting
%functions just by adjusting the colour factors. However, we will leave
%these consideration for a later study when the full two-loop splitting
%functions will be explicitly considered.

We can now generalise Eq.~(\ref{EvenMoreGeneralDGLAPciaociao}) to
include the presence of leptons in the DGLAP evolution into the
system:
\begin{equation}
\begin{array}{rcl}
\displaystyle \mu^2\frac{\partial}{\partial \mu^2}
\begin{pmatrix}
\ell_\alpha\\
u_j \\
d_i\\
g\\
\gamma\\
\overline{d}_i\\
\overline{u}_j\\
\overline{\ell}_\alpha
\end{pmatrix} &=& \displaystyle
\left[
\mbox{QCD}_{\rm NS}+\mbox{QED}_{\rm NS}+
\begin{pmatrix}
0 & 0 & 0 & 0 &0 & 0 & 0 & 0\\
0 & e_u^2\widetilde{P}_{qq} & 0 & e_u^2\widetilde{P}_{qg} & e_u^2\widetilde{P}_{q\gamma} & 0 & 0 & 0\\ 
0 & 0 & e_d^2\widetilde{P}_{qq} & e_d^2\widetilde{P}_{qg} & e_d^2\widetilde{P}_{q\gamma} & 0 & 0 & 0\\ 
0 & e_u^2\widetilde{P}_{gq}  & e_d^2\widetilde{P}_{gq} &
e_\Sigma^2\widetilde{P}_{gg} & e_\Sigma^2\widetilde{P}_{g\gamma} & e_d^2\widetilde{P}_{gq} & e_u^2\widetilde{P}_{gq} & 0\\
0 & e_u^2 \widetilde{P}_{\gamma q} & e_d^2\widetilde{P}_{\gamma q} & e_\Sigma^2\widetilde{P}_{\gamma g} & e_\Sigma^2\widetilde{P}_{\gamma\gamma} & e_d^2\widetilde{P}_{\gamma q}
& e_u^2\widetilde{P}_{\gamma q} & 0\\
0 & 0 & 0 & e_d^2\widetilde{P}_{qg} & e_d^2\widetilde{P}_{q\gamma} &
e_d^2\widetilde{P}_{qq}^V & 0 & 0\\
0 & 0 & 0 & e_u^2 \widetilde{P}_{qg} & e_u^2\widetilde{P}_{q\gamma} & 0 &
e_u^2\widetilde{P}_{qq}^V & 0\\
0 & 0 & 0 & 0 & 0 & 0 & 0 & 0\\
\end{pmatrix}\right]
\begin{pmatrix}
\ell_\alpha\\
u_j \\
d_i\\
g\\
\gamma\\
\overline{d}_i\\
\overline{u}_j\\
\overline{\ell}_\alpha
\end{pmatrix}\\
\\
&+&
\left[
\mbox{QCD}_{\rm SG}+\mbox{QED}_{\rm SG}+
\begin{pmatrix}
0 & 0  & 0 & 0 & 0 & 0 & 0 & 0\\
0 & e_u^4\widetilde{P}_{qq}^S & e_u^2e_d^2\widetilde{P}_{qq}^S & 0 & 0 & e_u^2e_d^2\widetilde{P}_{qq}^S & e_u^4\widetilde{P}_{qq}^S & 0\\ 
0 & e_u^2e_d^2\widetilde{P}_{qq}^S & e_d^4\widetilde{P}_{qq}^S & 0 & 0 & e_d^4\widetilde{P}_{qq}^S &  e_u^2e_d^2\widetilde{P}_{qq}^S & 0\\ 
0 & 0  & 0 & 0 & 0 & 0 & 0 & 0\\
0 & 0  & 0 & 0 & 0 & 0 & 0 & 0\\
0 & e_u^2e_d^2\widetilde{P}_{qq}^S & e_d^4\widetilde{P}_{qq}^S & 0 & 0 & e_d^4\widetilde{P}_{qq}^S &  e_u^2e_d^2\widetilde{P}_{qq}^S & 0\\
0 & e_u^4\widetilde{P}_{qq}^S & e_u^2e_d^2\widetilde{P}_{qq}^S & 0 & 0 &  e_u^2e_d^2\widetilde{P}_{qq}^S & e_u^4\widetilde{P}_{qq}^S & 0 \\
0 & 0  & 0 & 0 & 0 & 0 & 0 & 0
\end{pmatrix}
\right]
\begin{pmatrix}
\sum_\gamma \ell_\gamma\\
\sum_eu_e \\
\sum_ld_l\\
g\\
\gamma\\
\sum_m\overline{d}_m\\
\sum_n\overline{u}_n\\
\sum_\delta \overline{\ell}_\delta
\end{pmatrix}
\end{array}
\end{equation}
where:
\begin{equation}
\begin{array}{rcl}
\mbox{QED}_{\rm NS}=
\begin{pmatrix}
\widetilde{P}_{\ell\ell}^V & 0 & 0 & 0 & \widetilde{P}_{\ell\gamma} & 0 & 0 & \widetilde{P}_{\overline{\ell}\ell}^V\\
0 & 0 & 0 & 0 & 0 & 0 & 0 & 0\\ 
0 & 0 & 0 & 0 & 0 & 0 & 0 & 0\\ 
0 & 0 & 0 & 0 & 0 & 0 & 0 & 0\\
\widetilde{P}_{\gamma\ell} & 0 & 0 & 0 & 0 & 0 & 0 & \widetilde{P}_{\gamma\ell}\\
0 & 0 & 0 & 0 & 0 & 0 & 0 & 0\\
0 & 0 & 0 & 0 & 0 & 0 & 0 & 0\\
\widetilde{P}_{\overline{\ell}\ell}^V & 0 & 0 & 0 & \widetilde{P}_{\ell\gamma} & 0 & 0 & \widetilde{P}_{\ell\ell}^V\\
\end{pmatrix}
\end{array}
\end{equation}
and:
\begin{equation}
\begin{array}{rcl}
\mbox{QED}_{\rm SG}=
\begin{pmatrix}
\widetilde{P}_{\ell\ell}^S & e_u^2\widetilde{P}_{q\ell}^S & e_d^2\widetilde{P}_{q\ell}^S & 0 & 0 & e_d^2\widetilde{P}_{q\ell}^S & e_u^2\widetilde{P}_{q\ell}^S & \widetilde{P}_{\ell\ell}^S \\
e_u^2\widetilde{P}_{q\ell}^S & 0  & 0 & 0 & 0 & 0 & 0 & e_u^2\widetilde{P}_{q\ell}^S\\
e_d^2\widetilde{P}_{q\ell}^S & 0  & 0 & 0 & 0 & 0 & 0 & e_d^2\widetilde{P}_{q\ell}^S\\
0 & 0  & 0 & 0 & 0 & 0 & 0 & 0\\
0 & 0  & 0 & 0 & 0 & 0 & 0 & 0\\
e_d^2\widetilde{P}_{q\ell}^S & 0  & 0 & 0 & 0 & 0 & 0 & e_d^2\widetilde{P}_{q\ell}^S\\
e_u^2\widetilde{P}_{q\ell}^S & 0  & 0 & 0 & 0 & 0 & 0 & e_u^2\widetilde{P}_{q\ell}^S\\
\widetilde{P}_{\ell\ell}^S & e_u^2\widetilde{P}_{q\ell}^S & e_d^2\widetilde{P}_{q\ell}^S & 0 & 0 & e_d^2\widetilde{P}_{q\ell}^S & e_u^2\widetilde{P}_{q\ell}^S & \widetilde{P}_{\ell\ell}^S 
\end{pmatrix}
\end{array}
\end{equation}

Now we use the tranformation:
\begin{equation}
\mathbf{T} = 
\begin{pmatrix}
1 & 0 & 0 & 0 & 0 & 0 & 0 & 1 \\
0 & 1 & 0 & 0 & 0 & 0 & 1  & 0\\
0 & 0 & 1 & 0 & 0 & 1 & 0  & 0\\
0 & 0 & 0 & 1 & 0 & 0 & 0  & 0\\
0 & 0 & 0 & 0 & 1 & 0 & 0  & 0\\
0 & 0 & 1 & 0 & 0 & -1 & 0 & 0\\
0 & 1 & 0 & 0 & 0 & 0 & -1 & 0\\
1 & 0 & 0 & 0 & 0 & 0 & 0 & -1
\end{pmatrix}\quad\Longrightarrow\quad
\mathbf{T}^{-1} = 
\frac{1}{2}\begin{pmatrix}
1 & 0 & 0 & 0 & 0 & 0 & 0 & 1 \\
0 & 1 & 0 & 0 & 0 & 0 & 1 & 0\\
0 & 0 & 1 & 0 & 0 & 1 & 0 & 0\\
0 & 0 & 0 & 2 & 0 & 0 & 0 & 0\\
0 & 0 & 0 & 0 & 2 & 0 & 0 & 0\\
0 & 0 & 1 & 0 & 0 & -1 & 0 & 0\\
0 & 1 & 0 & 0 & 0 & 0 & -1 & 0\\
1 & 0 & 0 & 0 & 0 & 0 & 0 & -1
\end{pmatrix}
\label{EvenMoreGeneralDGLAPbye}
\end{equation}
For the leptonic parts, we get again the system in
Eq.~(\ref{macheneso}), defining:
\begin{equation}
\begin{array}{l}
\displaystyle \widetilde{P}_\ell^\pm \equiv \widetilde{P}_{\ell\ell}^V \pm \mathcal{P}_{\ell\overline{\ell}}^V \\
\\
\displaystyle \widetilde{P}_{\ell\ell} \equiv \widetilde{P}^+ + 2n_\ell \widetilde{P}_{\ell\ell}^S\\
\\
\displaystyle \widetilde{P}^V \equiv \widetilde{P}^-
\end{array}\,,
\end{equation}
being $n_\ell$ the number of active leptons, we have that:
\begin{equation}
\begin{array}{rcl}
\mathbf{T}\mbox{QED}_{\rm NS}\mathbf{T}^{-1}=
\begin{pmatrix}
\widetilde{P}^+ & 0 & 0 & 0 & 2\widetilde{P}_{\ell\gamma} & 0 & 0 & 0\\
0 & 0 & 0 & 0 & 0 & 0 & 0 & 0\\ 
0 & 0 & 0 & 0 & 0 & 0 & 0 & 0\\ 
0 & 0 & 0 & 0 & 0 & 0 & 0 & 0\\
\widetilde{P}_{\gamma\ell} & 0 & 0 & 0 & 0 & 0 & 0 & 0\\
0 & 0 & 0 & 0 & 0 & 0 & 0 & 0\\
0 & 0 & 0 & 0 & 0 & 0 & 0 & 0\\
0 & 0 & 0 & 0 & 0 & 0 & 0 & \widetilde{P}^-\\
\end{pmatrix}
\end{array}
\end{equation}
and:
\begin{equation}
\begin{array}{rcl}
\displaystyle \mathbf{T}\mbox{QED}_{\rm SG}\mathbf{T}^-1=\frac1{n_\ell}
\begin{pmatrix}
(\widetilde{P}_{\ell\ell}-\widetilde{P}^+) & 2 n_\ell e_u^2\widetilde{P}_{q\ell}^S & 2 n_\ell e_d^2\widetilde{P}_{q\ell}^S & 0 & 0 & 0 & 0 & 0 \\
2 n_\ell e_u^2\widetilde{P}_{q\ell}^S & 0 & 0 & 0 & 0 & 0 & 0 & 0\\
2 n_\ell e_d^2\widetilde{P}_{q\ell}^S & 0 & 0 & 0 & 0 & 0 & 0 & 0\\
0 & 0 & 0 & 0 & 0 & 0 & 0 & 0\\
0 & 0 & 0 & 0 & 0 & 0 & 0 & 0\\
0 & 0 & 0 & 0 & 0 & 0 & 0 & 0\\
0 & 0 & 0 & 0 & 0 & 0 & 0 & 0\\
0 & 0 & 0 & 0 & 0 & 0 & 0 & 0
\end{pmatrix}
\end{array}
\end{equation}
In conclusion, the single equations are:
\begin{equation}
\begin{array}{l}
\displaystyle \mu^2\frac{\partial \ell_\alpha^+}{\partial \mu^2} = \widetilde{P}^+\ell_\alpha^+
+\frac1{n_\ell}(\widetilde{P}_{\ell\ell}-\widetilde{P}^+)\Sigma_\ell+
2 \widetilde{P}_{q\ell}^S (\eta^+\Sigma+\eta^-\Delta_\Sigma)
+2\widetilde{P}_{\ell\gamma}\gamma\\
\\
\displaystyle \mu^2\frac{\partial \ell_\alpha^-}{\partial \mu^2} =
\widetilde{P}^-\ell_\alpha^-\\
\\
\displaystyle \mu^2\frac{\partial u_j^+}{\partial \mu^2} =\dots + 2
e_u^2\widetilde{P}_{q\ell}^S\Sigma_\ell\\
\\
\displaystyle \mu^2\frac{\partial d_i^+}{\partial \mu^2} =\dots + 2
e_d^2\widetilde{P}_{q\ell}^S\Sigma_\ell\\
\\
\displaystyle \mu^2\frac{\partial \gamma}{\partial \mu^2} =\dots + 
\widetilde{P}_{\gamma\ell}\Sigma_\ell
\end{array}
\end{equation}
where, as usual, we have defined:
\begin{equation}
\ell_\alpha^{\pm} = \ell_\alpha \pm \overline{\ell}_\alpha\,.
\end{equation}

Now, let us consider the following combinations:
\begin{equation}
\begin{array}{l}
\displaystyle \Sigma_\ell = \sum_{\alpha=e,\mu,\tau} \ell_{\alpha}^+\\
\\
\displaystyle  V_\ell = \sum_{\alpha=e,\mu,\tau} \ell_{\alpha}^-\\
\\
\displaystyle  T_{1}^{\ell} = \ell_e^+ - \ell_\mu^+\\
\\
\displaystyle  T_{2}^{\ell} = \ell_e^+ + \ell_\mu^+-2\ell_\tau^+\\
\\
\displaystyle  V_{1}^{\ell} = \ell_e^- - \ell_\mu^-\\
\\
\displaystyle  V_{2}^{\ell} = \ell_e^- + \ell_\mu^--2\ell_\tau^-
\end{array}
\end{equation}
It is easy to see that above the $\tau$ mass threshold, $i.e.$ where
$n_\ell = 3$, they
evolve according to the following equations:
\begin{equation}
\begin{array}{l}
\displaystyle \mu^2\frac{\partial \Sigma_\ell}{\partial \mu^2} =
2n_\ell\widetilde{P}_{\ell\gamma}\gamma +
2n_\ell \widetilde{P}_{q\ell}^S (\eta^+\Sigma+\eta^-\Delta_\Sigma) +\widetilde{P}_{\ell\ell}\Sigma_\ell\\
\\
\displaystyle  \mu^2\frac{\partial V_\ell}{\partial \mu^2} =
\widetilde{P}^-  V_\ell = \widetilde{P}^V  V_\ell\\
\\
\displaystyle  \mu^2\frac{\partial T_{1,2}^{\ell}}{\partial \mu^2} =
\widetilde{P}^+  T_{1,2}^{\ell}\\
\\
\displaystyle  \mu^2\frac{\partial V_{1,2}^{\ell}}{\partial \mu^2} =
\widetilde{P}^-  V_{1,2}^{\ell}
\end{array}
\end{equation}
In addition, the photon and the QCD singlet distributions $\Sigma$ and
$\Delta_\Sigma$ acquire the following terms:
\begin{equation}
\begin{array}{l}
\displaystyle \mu^2\frac{\partial \Sigma}{\partial \mu^2} =\dots + 2
e_\Sigma^2\widetilde{P}_{q\ell}^S\Sigma_\ell\\
\\

\displaystyle \mu^2\frac{\partial \Delta_\Sigma}{\partial \mu^2} =\dots + 2
\delta_e^2\widetilde{P}_{q\ell}^S\Sigma_\ell\\
\\
\displaystyle \mu^2\frac{\partial \gamma}{\partial \mu^2} =\dots + 
\widetilde{P}_{\gamma\ell}\Sigma_\ell
\end{array}
\end{equation}

In conclusion, the full system of equations in the evolution
basis including leptons is the following:
\begin{equation}
\begin{array}{rcl}
\displaystyle\mu^2\frac{\partial}{\partial \mu^2}
\begin{pmatrix}
g\\
\gamma\\
\Sigma\\
\Delta_\Sigma\\
\Sigma_\ell
\end{pmatrix} &=& \displaystyle \left[
\begin{pmatrix}
P_{gg} & 0 & P_{gq} & 0 & 0\\
0 & 0 & 0 & 0  & 0\\
2n_fP_{qg} & 0 & P_{qq} & 0  & 0\\
\frac{n_u-n_d}{n_f} 2n_fP_{qg} & 0 & \frac{n_u-n_d}{n_f}(P_{qq}-P^+) &
P^+ & 0\\
0 & 0 & 0 & 0  & 0
\end{pmatrix}\right.
\\
\\
&+&\left.\begin{pmatrix}
e_\Sigma^2 \widetilde{P}_{gg}          & e_\Sigma^2 \widetilde{P}_{g\gamma} & \eta^+\widetilde{P}_{gq} & \eta^-\widetilde{P}_{gq}  & 0\\
e_\Sigma^2 \widetilde{P}_{\gamma g} & e_\Sigma^2 \widetilde{P}_{\gamma\gamma} & \eta^+\widetilde{P}_{\gamma q} &\eta^-\widetilde{P}_{\gamma q}  & 0\\
2 e_\Sigma^2 \widetilde{P}_{qg} & 2 e_\Sigma^2 \widetilde{P}_{q\gamma}
& \eta^+\widetilde{P}^++\frac{\eta^+e_\Sigma^2}{n_f}(\widetilde{P}_{qq}-\widetilde{P}^+)  & \eta^-\widetilde{P}^++\frac{\eta^-e_\Sigma^2}{n_f}(\widetilde{P}_{qq}-\widetilde{P}^+)  & 0\\
2 \delta_e^2 \widetilde{P}_{qg} & 2 \delta_e^2 \widetilde{P}_{q\gamma}
&\eta^-\widetilde{P}^++\frac{\eta^+\delta_e^2}{n_f}(\widetilde{P}_{qq}-\widetilde{P}^+)
&\eta^+\widetilde{P}^++\frac{\eta^-\delta_e^2}{n_f}(\widetilde{P}_{qq}-\widetilde{P}^+)  & 0\\
0 & 0 & 0 & 0  & 0
\end{pmatrix}\right.
\\
\\
&+&\left.\begin{pmatrix}
0 & 0 & 0 & 0  & 0\\
0 & 0 & 0 & 0  & \widetilde{P}_{\gamma\ell}\\
0 & 0 & 0 & 0  & 2e_\Sigma^2\widetilde{P}_{q\ell}^S\\
0 & 0 & 0 & 0  & 2\delta_e^2\widetilde{P}_{q\ell}^S\Sigma_\ell\\
0 & 2n_\ell\widetilde{P}_{\ell\gamma} & 2n_\ell \eta^+ \widetilde{P}_{q\ell}^S & 2n_\ell \eta^- \widetilde{P}_{q\ell}^S  & \widetilde{P}_{\ell\ell}
\end{pmatrix}\right]
\begin{pmatrix}
g\\
\gamma\\
\Sigma\\
\Delta_\Sigma\\
\Sigma_\ell
\end{pmatrix}
\end{array}
\label{SingletDGLAPlept}
\end{equation}

\begin{equation}
\displaystyle\mu^2\frac{\partial}{\partial \mu^2}
\begin{pmatrix}
V\\
\Delta_V
\end{pmatrix} = 
\left[
\begin{pmatrix}
P^V & 0 \\
\frac{n_u-n_d}{n_f}(P^V-P^-)  & P^-
\end{pmatrix}
+
\begin{pmatrix}
\eta^+\widetilde{P}^- & \eta^-\widetilde{P}^- \\
\eta^-\widetilde{P}^- & \eta^+\widetilde{P}^- 
\end{pmatrix}
\right]
\begin{pmatrix}
V\\
\Delta_V
\end{pmatrix}
\end{equation}

\begin{equation}
\begin{array}{l}
\displaystyle  \mu^2\frac{\partial V_\ell}{\partial \mu^2} =
\widetilde{P}^-  V_\ell = \widetilde{P}^V  V_\ell
\end{array}
\end{equation}

\begin{equation}
\begin{array}{rcl}
\\
\displaystyle \mu^2\frac{\partial T^u_{1,2}}{\partial \mu^2} &=&
\displaystyle (P^+ + e_u^2\widetilde{P}^+) T^u_{1,2}\\
\\
\displaystyle \mu^2\frac{\partial T^d_{1,2}}{\partial \mu^2} &=&
\displaystyle (P^+ + e_d^2\widetilde{P}^+) T^d_{1,2}
\end{array}
\end{equation}

\begin{equation}
\begin{array}{rcl}
\displaystyle \mu^2\frac{\partial V^u_{1,2}}{\partial \mu^2} &=&
\displaystyle (P^- + e_u^2\widetilde{P}^-) V^u_{1,2}\\
\\
\displaystyle \mu^2\frac{\partial V^d_{1,2}}{\partial \mu^2} &=&
\displaystyle (P^- + e_d^2\widetilde{P}^-) V^d_{1,2}
\end{array}
\end{equation}

\begin{equation}
\begin{array}{l}
\displaystyle  \mu^2\frac{\partial T_{1,2}^{\ell}}{\partial \mu^2} =
\widetilde{P}^+  T_{1,2}^{\ell}\\
\\
\displaystyle  \mu^2\frac{\partial V_{1,2}^{\ell}}{\partial \mu^2} =
\widetilde{P}^-  V_{1,2}^{\ell}
\end{array}
\end{equation}


\subsection{Evolution equations at LO}

At LO in QED, considering that $\widetilde{P}_{q\ell}^S = 0$ and that $\mathcal{P}_{ij}=\widetilde{P}_{ij}$, the
evolution equations above reduce to:
\begin{equation}
\begin{array}{rcl}
\displaystyle\mu^2\frac{\partial}{\partial \mu^2}
\begin{pmatrix}
g\\
\gamma\\
\Sigma\\
\Delta_\Sigma\\
\Sigma_\ell
\end{pmatrix} &=& \displaystyle \left[
\begin{pmatrix}
P_{gg} & 0 & P_{gq} & 0 & 0\\
0 & 0 & 0 & 0 & 0\\
2n_fP_{qg} & 0 & P_{qq} & 0 & 0\\
\frac{n_u-n_d}{n_f} 2n_fP_{qg} & 0 & \frac{n_u-n_d}{n_f}(P_{qq}-P^+) &
P^+ & 0\\
0 & 0 & 0 & 0 & 0\\
\end{pmatrix}\right.
\\
\\
&+&\left.\begin{pmatrix}
0 & 0 & 0 & 0 & 0\\
0 & e_\Sigma^2 \widetilde{P}_{\gamma\gamma} & \eta^+\widetilde{P}_{\gamma q} &\eta^-\widetilde{P}_{\gamma q} & \widetilde{P}_{\gamma q}\\
0 & 2 e_\Sigma^2 \widetilde{P}_{q\gamma} & \eta^+\widetilde{P}_{qq} &
\eta^-\widetilde{P}_{qq} & 0\\
0 & 2 \delta_e^2 \widetilde{P}_{q\gamma}
&\eta^-\widetilde{P}_{qq}
&\eta^+\widetilde{P}_{qq} & 0\\
0 & 2n_\ell\widetilde{P}_{q\gamma} & 0 & 0 & \widetilde{P}_{qq}\\
\end{pmatrix}\right]
\begin{pmatrix}
g\\
\gamma\\
\Sigma\\
\Delta_\Sigma\\
\Sigma_\ell
\end{pmatrix}
\end{array}
\label{APFELsysLept}
\end{equation}

\begin{equation}
\displaystyle\mu^2\frac{\partial}{\partial \mu^2}
\begin{pmatrix}
V\\
\Delta_V
\end{pmatrix} = 
\left[
\begin{pmatrix}
P^V & 0 \\
\frac{n_u-n_d}{n_f}(P^V-P^-)  & P^- 
\end{pmatrix}
+
\begin{pmatrix}
\eta^+\widetilde{P}_{qq} & \eta^-\widetilde{P}_{qq} \\
\eta^-\widetilde{P}_{qq} & \eta^+\widetilde{P}_{qq}
\end{pmatrix}
\right]
\begin{pmatrix}
V\\
\Delta_V
\end{pmatrix}
\end{equation}

\begin{equation}
\begin{array}{l}
\displaystyle  \mu^2\frac{\partial V_\ell}{\partial \mu^2} =
\widetilde{P}_{qq}  V_\ell
\end{array}
\end{equation}

\begin{equation}
\begin{array}{l}
\begin{array}{rcl}
\\
\displaystyle \mu^2\frac{\partial T^u_{1,2}}{\partial \mu^2} &=&
\displaystyle (P^+ + e_u^2\widetilde{P}_{qq}) T^u_{1,2}\\
\\
\displaystyle \mu^2\frac{\partial T^d_{1,2}}{\partial \mu^2} &=&
\displaystyle (P^+ + e_d^2\widetilde{P}_{qq}) T^d_{1,2}
\end{array}
\\
\\
\begin{array}{rcl}
\displaystyle \mu^2\frac{\partial V^u_{1,2}}{\partial \mu^2} &=&
\displaystyle (P^- + e_u^2\widetilde{P}_{qq}) V^u_{1,2}\\
\\
\displaystyle \mu^2\frac{\partial V^d_{1,2}}{\partial \mu^2} &=&
\displaystyle (P^- + e_d^2\widetilde{P}_{qq}) V^d_{1,2}
\end{array}
\end{array}
\end{equation}

\begin{equation}
\begin{array}{l}
\displaystyle  \mu^2\frac{\partial T_{1,2}^{\ell}}{\partial \mu^2} =
\widetilde{P}_{qq}  T_{1,2}^{\ell}\\
\\
\displaystyle  \mu^2\frac{\partial V_{1,2}^{\ell}}{\partial \mu^2} =
\widetilde{P}_{qq}  V_{1,2}^{\ell}
\end{array}
\end{equation}

There is one last detail to be discussed. Contrary to electrons and
muons, whose masses are well below $\Lambda_{\rm QCD}$, the $\tau$ has
a mass equal to $m_\tau = 1.777$ GeV which is well above
$\Lambda_{\rm QCD}$ and even above the typical initial scale
$Q_0 \simeq 1$ GeV from which PDFs are usually evolved. As a
consequence, we need to account for the possibility to cross the
$\tau$ mass threshold. To do so, we just need to realise that below
the $\tau$ threshold, where $n_\ell = 2$, the $T_8^\ell$ and
$V_8^\ell$ reduce to the lepton singlet $\Sigma_\ell$ and total
valence $V_\ell$ distributions and thus evolve as such.

From the implementation point of view, the main problem is the fact
that we need to introduce a new threshold between the charm and the
bottom thresholds and this will complicate the structure of the code.

\section{QED corrections at NLO}

In this section we discuss the details of the implementation of the
NLO QED corrections.  While the inclusion of the LO corrections
presents many simplifications, $e.g.$ QED and QCD corrections do not
mix and thus the DGLAP equations as well as the $\alpha_s$ and
$\alpha$ evolution equations are decoupled, when including NLO
corrections QED and QCD corrections mix both in the DGLAP and in the
coupling evolution equations. In addition, as far as the DIS structure
functions are concerned, such corrections induce photon initiated
diagrams that have to be included in order to have the full set of
corrections. In the following, we will first discuss how to generalise
the coupling evolution equations, we will the consider the DIS
structure functions, and finally we will turn to the DGLAP.

\subsection{Evolution equations for the couplings}

As already mentioned, NLO QED corrections induce a mixing with QCD.
At the level of the couplings, this essentially means that the QCD
$\beta$-function will get corrections proportional to $\alpha$ and
vice-versa, the QED $\beta$-function will get corrections proportional
to $\alpha_s$, that is:
\begin{equation}
\begin{array}{rcl}
\displaystyle \mu^2\frac{\partial \alpha_s}{\partial \mu^2} &=& \displaystyle
                                                \beta^{\rm QCD}(\alpha_s,\alpha)\,.\\
\\
\displaystyle \mu^2\frac{\partial \alpha}{\partial \mu^2} &=& \displaystyle \beta^{\rm QED}(\alpha_s,\alpha)\,.
\end{array}
\label{CoupledEq}
\end{equation}
As a consequence, these evolution equations form a set of coupled
differential equations.

Before discussing the numerical solution of these equations, we first
need to know explicitly the new contributions to the
$\beta$-functions. In particular, up to NLO one has: 
\begin{equation}
\beta^{\rm QCD}(\alpha_s,\alpha) = -\alpha_s\left[\beta_0^{(\alpha_s)}\left(\frac{\alpha_s}{4\pi}\right)+\beta_1^{(\alpha_s\alpha)}\left(\frac{\alpha_s}{4\pi}\right) \left(\frac{\alpha}{4\pi}\right)+\beta_1^{(\alpha_s^2)}\left(\frac{\alpha_s}{4\pi}\right)^2+\dots\right]\,,
\end{equation}
and:
\begin{equation}
\beta^{\rm QED}(\alpha_s,\alpha) = -\alpha\left[\beta_0^{(\alpha)}\left(\frac{\alpha}{4\pi}\right)+\beta_1^{(\alpha\alpha_s)}\left(\frac{\alpha}{4\pi}\right) \left(\frac{\alpha_s}{4\pi}\right)+\beta_1^{(\alpha^2)}\left(\frac{\alpha}{4\pi}\right)^2+\dots\right]\,,
\end{equation}
where the new terms $\beta_1^{(\alpha_s\alpha)}$ and
$\beta_1^{(\alpha\alpha_s)}$ can be read from
Ref.~\cite{Surguladze:1996hx}. Taking into account the additional
factor four in the definition of the expansion parameter and writing
explicitly the colour factors, one finds:
\begin{equation}
\beta_1^{(\alpha_s\alpha)} = -2\sum_{i=1}^{n_f}
q_i^2\quad\mbox{and}\quad \beta_1^{(\alpha\alpha_s)} = -\frac{16}{3}N_c\sum_{i=1}^{n_f} q_i^2\,,
\end{equation}
where $n_f$ is the number of active quark flavours and $N_c=3$ is the
number of colours.  We also need the term $\beta_1^{(\alpha^2)}$ which
can again be taken from the same paper:
\begin{equation}
\beta_1^{(\alpha^2)} = -4\left(n_l+N_c\sum_{i=1}^{n_f} q_i^2\right)\,,
\end{equation}
where $n_l$ is the number of active lepton flavours.

Eq.~(\ref{CoupledEq}) can be written in the vectorial form:
\begin{equation}
\frac{\partial {\bm \alpha}}{\partial t} = {\bm \beta}\left({\bm
    \alpha}(t)\right)\,,
\label{CoupledEqVect}
\end{equation}
with $t = \ln\mu^2$ and:
\begin{equation}
 {\bm \alpha} = {\alpha_s \choose \alpha}\qquad\mbox{and}\qquad  {\bm \beta} = {\beta^{\rm QCD} \choose \beta^{\rm QED}}\,.
\end{equation}
Eq.~(\ref{CoupledEqVect}) is an ordinary differential equation that
can be numerically solved using the Runge-Kutta method.

The vector form of the fourth order Runge-Kutta algorithm for a step $h$
is:
\begin{equation}
\begin{array}{rcl}
\displaystyle{\bm k}_1 &=& \displaystyle h{\bm
                           \beta}\left({\bm \alpha}(t)\right)\\
\\
\displaystyle{\bm k}_2 &=& \displaystyle  h{\bm
                           \beta}\left({\bm \alpha}(t)+\frac{{\bm k}_1}{2}
                           \right)\\
\\
\displaystyle{\bm k}_3 &=& \displaystyle  h{\bm
                           \beta}\left({\bm \alpha}(t)+\frac{{\bm
                           k}_2}{2}\right)\\
\\
\displaystyle{\bm k}_4 &=& \displaystyle  h{\bm
                           \beta}\left({\bm \alpha}(t)+{\bm k}_3\right)\\
\\
\displaystyle{\bm \alpha}(t+h) &=& \displaystyle {\bm \alpha}(t)+\frac{{\bm k}_1+2 {\bm k}_2+2 {\bm k}_3+{\bm k}_4}{6} +\mathcal{O}(h^5).
\end{array}
\label{RK4th}
\end{equation}
This formulation assumes that $\alpha_s$ and $\alpha$ evolve from the
same scale $t$ to $t+h$. In practice, this means that the reference
scale at which the two coupling are given must be the same. This is
usually not the case because one might want to define $\alpha_s$, say,
at the $Z$ mass scale $M_Z$ and $\alpha$, say, at the $\tau$ mass
scale $m_\tau$. This is no longer possible when introducing mixed
$\mathcal{O}(\alpha_s\alpha)$ corrections because the subtraction of
the ultraviolet divergences must happen at the same scale $\mu$.

\subsection{NLO QED corrections to DIS structure functions}

The first photon-initiated corrections to the DIS structure functions
represent the NLO QED corrections. Such corrections provide a direct
handle on the photon PDF from DIS data. In fact, at LO in QED the
photon PDF does not contribute directly to structure functions and it
is only indirectly constrained from data through its coupling to the
singlet PDF in the DGLAP evolution.

When considering NLO QED corrections to DIS structure functions, one
has to include into the hard cross sections all the
$\mathcal{O}(\alpha)$ diagrams where one single photon is either in
the initial state or emitted from an incoming quark (or possibly an
incoming lepton). Such diagrams are purely of QED origin and no QCD
contributions are present. As a consequence, the corresponding
coefficient functions can be easily derived from the QCD expressions
just by properly adjusting the colour factors. In addition, this
correspondence holds regardless of whether mass effects are included
or not and thus, starting from the pure NLO QCD expressions, one can
obtain the NLO QED corrections to be added to any general-mass scheme.

The main complication arises from the flavour structure. In fact, due
to the fact that the coupling of the photon is proportional to the
squared charge of the parton it couples to (a quark or a lepton), in
the case of the quarks the isospin symmetry is broken.

Let us start with the ZM coefficient functions. We concentrate on the
$\mathcal{O}(\alpha)$ contribution to the generic NC structure
function $F$. This correction can easily be derived from the structure
of the $\mathcal{O}(\alpha_s)$ correction. The algorithm is very
simple, for the coefficient functions one has:
\begin{equation}
\begin{array}{rcl}
\displaystyle C_+^{(\alpha)} &=& \displaystyle \frac{C_+^{(\alpha_s)}}{C_F}\\
\\
\displaystyle C_g^{(\alpha)} &=& \displaystyle \frac{C_g^{(\alpha_s)}}{T_R}
\end{array}
\end{equation}
as there is no pure-singlet contribution at this order where $C_F=4/3$
and $T_R=1/2$ are the usual QCD colour factors. For constructing the
structure functions, given the simple structure of the diagrams
involved, one just has to do the following replacements in the pure
QCD contributions:
 \begin{equation}
\begin{array}{rcl}
B_q(Q) &\rightarrow& B_q(Q)e_q^2\quad\mbox{for}\quad F_2,F_L \\
\\
D_q(Q) &\rightarrow& D_q(Q)e_q^2\quad\mbox{for}\quad F_3 \\
\end{array}
\end{equation}
where $B_q(Q)$ and $D_q(Q)$ are the NC couplings. With this algorithm
at hand one can write the $\mathcal{O}(\alpha)$ contributions to the
light and heavy quark structure functions as:
\begin{equation}
\begin{array}{rcl}
\displaystyle F^{(\alpha),l} &=& \displaystyle \langle B_le_l^2 \rangle \left[\frac{C_g^{(\alpha_s)}}{T_R} \gamma +\frac1{n_f}\frac{C_+^{(\alpha_s)}}{C_F}
\Sigma\right]+\frac{1}{2}(B_ue_u^2-B_de_d^2)
  \frac{C_+^{(\alpha_s)}}{C_F}
  T_3\\
\\
&+&\displaystyle \frac{1}{6}(B_ue_u^2+B_de_d^2-2B_se_s^2)
  \frac{C_+^{(\alpha_s)}}{C_F} T_8+\langle B_le_l^2 \rangle
  \frac{C_+^{(\alpha_s)}}{C_F}\sum_{j=4}^{n_f} \frac{1}{j(j-1)} f_j\,.
\end{array}
\label{LightSF}
\end{equation}
The heavy-quark components instead are:
\begin{equation}
\begin{array}{rcl}
F^{(\alpha),c} &=& \displaystyle \theta(Q^2-m_c^2)
                   B_ce_c^2\left\{\left[\frac{C_g^{(\alpha_s)}}{T_R}  \gamma + \frac1{n_f}\frac{C_+^{(\alpha_s)}}{C_F}
                   \Sigma\right]-\frac{1}{4}\frac{C_+^{(\alpha_s)}}{C_F} T_{15}+\frac{C_+^{(\alpha_s)}}{C_F}\sum_{j=5}^{n_f}
            \frac{1}{j(j-1)} f_j
            \right\}\,,\\
\\
F^{(\alpha),b} &=& \displaystyle \theta(Q^2-m_b^2) B_be_b^2\left\{\left[\frac{C_g^{(\alpha_s)}}{T_R}  \gamma +\frac1{n_f}\frac{C_+^{(\alpha_s)}}{C_F}
\Sigma\right]-\frac{1}{5}\frac{C_+^{(\alpha_s)}}{C_F} T_{24}+\frac{C_+^{(\alpha_s)}}{C_F}\sum_{j=6}^{n_f}
            \frac{1}{j(j-1)} f_j
            \right\}\,,\\
\\
F^{(\alpha),t} &=& \displaystyle \theta(Q^2-m_t^2) B_te_t^2\left\{\left[\frac{C_g^{(\alpha_s)}}{T_R}  \gamma +\frac1{n_f}\frac{C_+^{(\alpha_s)}}{C_F}
                   \Sigma\right]-\frac{1}{6}\frac{C_+^{(\alpha_s)}}{C_F} T_{35}\right\}\,.
\end{array}
\label{HeavySF}
\end{equation}

The CC structure functions are more complicated to treat because their
flavour structure is more complex. As a first step, we write the
$\mathcal{O}(\alpha_s)$ contribution to $F=F_2,F_L$ (we will consider
$F_3$ later) in a convenient way as:
\begin{equation}
F^{\nu,(\alpha_s)} =
\sum_{U=u,c,t}\sum_{D=d,s,b}|V_{UD}|^2\left[C_\pm^{(\alpha_s)}\left(D
    +\overline{U}\right) +2 C_g^{(\alpha_s)}g\right]
\label{compactNu}
\end{equation}
and:
\begin{equation}
F^{\overline{\nu},(\alpha_s)} =
\sum_{U=u,c,t}\sum_{D=d,s,b}|V_{UD}|^2\left[C_\pm^{(\alpha_s)}\left(\overline{D}
    +U\right) +2 C_g^{(\alpha_s)}g\right] 
\label{compactNub}
\end{equation}
where we have omitted the convolution symbol and overall factor
$2x$. At this order we do not have to worry about whether $C_+$ or
$C_-$ has to be used because they coincide. However, in the following
it will appear naturally which one has be used where so we keep them
distinguished. One can combine the expressions above conveniently as:
\begin{equation}
F^{\nu,(\alpha_s)} + F^{\overline{\nu},(\alpha_s)}=
\sum_{U=u,c,t}\sum_{D=d,s,b}|V_{UD}|^2\left[C_+^{(\alpha_s)}\left(D^+
    +U^+\right) + 4 C_g^{(\alpha_s)}g\right]
\end{equation}
and:
\begin{equation}
F^{\nu,(\alpha_s)} - F^{\overline{\nu},(\alpha_s)} =
\sum_{U=u,c,t}\sum_{D=d,s,b}|V_{UD}|^2\left[C_-^{(\alpha_s)}\left(D^-
    -U^-\right)\right]
\end{equation}

Using the usual algorithm one can write down the $\mathcal{O}(\alpha)$
contribution to the CC structure functions as:
\begin{equation}
F^{\nu,(\alpha)} + F^{\overline{\nu},(\alpha)}=
\sum_{U=u,c,t}\sum_{D=d,s,b}|V_{UD}|^2\left[\frac{C_+^{(\alpha_s)}}{C_F}\left(e_D^2D^+
    +e_U^2U^+\right) + 2(e_D^2+e_U^2) \frac{C_g^{(\alpha_s)}}{T_R}\gamma\right]
\end{equation}
and:
\begin{equation}
F^{\nu,(\alpha)} - F^{\overline{\nu},(\alpha)} =
\sum_{U=u,c,t}\sum_{D=d,s,b}|V_{UD}|^2\left[\frac{C_-^{(\alpha_s)}}{C_F}\left(e_D^2 D^-
    -e_U^2 U^-\right)\right]
\end{equation}
Now the question is expressing these combinations in terms of PDFs in
the evolution basis. The starting point are the relations:
\begin{equation}
q_i^\pm = \sum_{j=1}^6M_{ij}d^\pm_j\,,
\label{TranformationBella}
\end{equation}
where $d^\pm_j$ belong to the QCD evolution basis, that is:
$d^+_1=\Sigma$, $d^+_2=-T_3$, $d^+_3=T_8$, $d^+_4=T_{15}$,
$d^+_5=T_{24}$, and $d^+_6=T_{35}$ and $d^-_1=V$, $d^-_2=-V_3$,
$d^-_3=V_8$, $d^-_4=V_{15}$, $d^-_5=V_{24}$, and $d^-_6=V_{35}$.  Note
that here we are using the more ``natural'' ordering for the
distributions $q_i=\{d,u,s,c,b,t\}$ rather than that where $u$ comes
before $d$; this is the reason of the minus sign in front of $T_3$ and
$V_3$. The trasformation matrix $M_{ij}$ can be written as:
\begin{equation}
\begin{array}{l}
\displaystyle M_{ij}=\theta_{ji}\frac{1-\delta_{ij}j}{j(j-1)}\quad j\geq 2\,,\\
\\
\displaystyle M_{i1} = \frac{1}{6}\,,
\end{array}
\label{TransDef}
\end{equation}
with $\theta_{ji}=1$ for $j\geq i$ and zero otherwise. In addition,
one can show$M_{ij}$ is such that:
\begin{equation}
\sum_{j=1}^6M_{ij} = 0\,,\quad\mbox{and}\quad \sum_{i=1}^6M_{ij} = \delta_{1j}\,.
\end{equation}

Using Eq.~(\ref{TranformationBella}) we can make the following
identifications:
\begin{equation}
D^{\pm} = q_{2j-1}^\pm\quad\mbox{and}\quad U^{\pm} =
q_{2j}^\pm\,,\quad j=1,2,3\,.
\end{equation}
In addition $e_U^2$ and $e_D^2$ do not depend on the particular
``value'' of $U$ and $D$. So that we can write:
\begin{equation}
F^{\nu,(\alpha)} + F^{\overline{\nu},(\alpha)}=
\sum_{i=1}^3\sum_{j=1}^3|V_{2i,(2j-1)}|^2\left[\frac{C_+^{(\alpha_s)}}{C_F}\left(e_D^2q_{2j-1}^+
    +e_U^2q_{2i}^+\right) + 2(e_D^2+e_U^2) \frac{C_g^{(\alpha_s)}}{T_R}\gamma\right]
\end{equation}
and:
\begin{equation}
F^{\nu,(\alpha)} - F^{\overline{\nu},(\alpha)}=
\sum_{i=1}^3\sum_{j=1}^3|V_{2i,(2j-1)}|^2\left[\frac{C_-^{(\alpha_s)}}{C_F}\left(e_D^2q_{2j-1}^-
    -e_U^2q_{2i}^-\right)\right]
\end{equation}
Now, let us concentrate on the combinations:
\begin{equation}
e_D^2q_{2j-1}^\pm \pm e_U^2q_{2i}^\pm = e_D^2
\sum_{k=1}^6M_{(2j-1),k}d^\pm_k \pm e_U^2 \sum_{k=1}^6M_{2i,k}d^\pm_k
= \sum_{k=1}^6 \left(e_D^2 M_{(2j-1),k} \pm e_U^2 M_{2i,k} \right) d^\pm_k
\end{equation}
Now we use the explicit for of $M_{ij}$ given in Eq.~(\ref{TransDef})
to make some simplifications:
\begin{equation}
\begin{array}{rcl}
e_D^2q_{2j-1}^\pm \pm e_U^2q_{2i}^\pm &=&\displaystyle \frac{1}{6}(e_D^2\pm e_U^2)
d^\pm_1 + \sum_{k=2}^6 \left(e_D^2
  \theta_{k,(2j-1)}\frac{1-\delta_{(2j-1),k}k}{k(k-1)}  \pm e_U^2 
\theta_{k,2i}\frac{1-\delta_{2i,k}k}{k(k-1)}\right) d^\pm_k
\end{array}
\label{complicated}
\end{equation}

Since things are getting quite complicated, we make the approximation
of diagonal CKM matrix for the quark contributions. The error one does
in taking this approximation is proportional to $\alpha$ times the off
diagonal CMK matrix elements which is clearly quite small. In
practice, this means setting:
\begin{equation}
|V_{2i,(2j-1)}|^2 = \delta_{ij}
\end{equation}
so that:
\begin{equation}
\begin{array}{c}
\displaystyle F_\pm^{(\alpha)}\equiv F^{\nu,(\alpha)} \pm F^{\overline{\nu},(\alpha)}=
\sum_{i=1}^3\left[\frac{C_\pm^{(\alpha_s)}}{C_F}\left(e_D^2q_{2i-1}^\pm
    \pm e_U^2q_{2i}^\pm\right) + P_\pm2(e_D^2\pm e_U^2)
  \frac{C_g^{(\alpha_s)}}{T_R}\gamma\right]=\\
  \\
  \displaystyle\sum_{i=1}^3 \left\{\frac{C_\pm^{(\alpha_s)}}{C_F}\left[\frac{1}{6}(e_D^2\pm e_U^2)
  d^\pm_1 + \sum_{k=2}^6 \left(e_D^2
  \theta_{k,(2i-1)}\frac{1-\delta_{(2i-1),k}k}{k(k-1)}  \pm e_U^2 
  \theta_{k,2i}\frac{1-\delta_{2i,k}k}{k(k-1)}\right) d^\pm_k\right] + P_\pm2(e_D^2\pm e_U^2)
  \frac{C_g^{(\alpha_s)}}{T_R}\gamma\right\}
\end{array}
\end{equation}
with:
\begin{equation}
P_\pm=\frac{1\pm 1}{2}
\end{equation}

According to our definition of light, charm, bottom, and top
components we have that $i=1$ belongs to the light structure function,
$i=2$ to the charm structure function, and $i=2$ to the top structure
function. The bottom structure function does not get any contribution
because, according to our definition, it is proportional to
off-diagonal elements of the CKM matrix, so that:
\begin{equation}
F_\pm^{(\alpha)} = F_\pm^{(\alpha),l} + F_\pm^{(\alpha),c} + F_\pm^{(\alpha),t}
\end{equation}
with:
\begin{equation}
F_\pm^{(\alpha),l} = \frac{C_\pm^{(\alpha_s)}}{C_F}\left[\frac{1}{6}(e_D^2\pm e_U^2)
  d^\pm_1 + \frac{e_D^2\mp e_U^2}{2}d^\pm_2 +(e_D^2\pm e_U^2)\sum_{k=3}^6 
  \frac{1}{k(k-1)} d^\pm_k\right] + P_\pm2(e_D^2\pm e_U^2)
  \frac{C_g^{(\alpha_s)}}{T_R}\gamma
\end{equation}

\begin{equation}
F_\pm^{(\alpha),c} = \frac{C_\pm^{(\alpha_s)}}{C_F}\left[\frac{1}{6}(e_D^2\pm e_U^2)
  d^\pm_1 - \frac{e_D^2}{3}d^\pm_3+ \frac{e_D^2\mp 3e_U^2}{12}d^\pm_4
  +(e_D^2\pm e_U^2)\sum_{k=5}^6 \frac{1}{k(k-1)} d^\pm_k\right] + P_\pm2(e_D^2\pm e_U^2)
  \frac{C_g^{(\alpha_s)}}{T_R}\gamma
\end{equation}

\begin{equation}
F_\pm^{(\alpha),t}=\frac{C_\pm^{(\alpha_s)}}{C_F}\left[\frac{1}{6}(e_D^2\pm e_U^2)
  d^\pm_1 - \frac{e_D^2}{5}d^\pm_5 +\frac{e_D^2\mp 5e_U^2}{30}d^\pm_6\right] + P_\pm2(e_D^2\pm e_U^2)
  \frac{C_g^{(\alpha_s)}}{T_R}\gamma
\end{equation}
Let us define:
\begin{equation}
\epsilon_\pm = \frac{e_D^2\pm e_U^2}2
\end{equation}
and recall that:
\begin{equation}
F^{\nu,(\alpha)} = \frac{F_+^{(\alpha)}+F_-^{(\alpha),t}}2
\end{equation}
and:
\begin{equation}
F^{\overline{\nu},(\alpha)} = \frac{F_+^{(\alpha)}-F_-^{(\alpha),t}}2
\end{equation}
and use the fact that:
\begin{equation}
C_+^{(\alpha_s)} = C_-^{(\alpha_s)} = C_q^{(\alpha_s)}
\end{equation}
so that:
\begin{equation}
  F^{\nu,(\alpha),l} = \frac{C_q^{(\alpha_s)}}{C_F}\left[\frac{\epsilon_+}{6}
    d^+_1+\frac{\epsilon_-}{6}
    d^-_1 + \frac{\epsilon_-}{2}d^+_2 + \frac{\epsilon_+}{2}d^-_2 +\sum_{k=3}^6 
    \frac{\epsilon_+d^+_k +\epsilon_-d^-_k}{k(k-1)} \right] + 2\epsilon_+
  \frac{C_g^{(\alpha_s)}}{T_R}\gamma
\end{equation}

\begin{equation}
  F^{\overline{\nu},(\alpha),l} = \frac{C_q^{(\alpha_s)}}{C_F}\left[\frac{\epsilon_+}{6}
    d^+_1-\frac{\epsilon_-}{6}
    d^-_1 + \frac{\epsilon_-}{2}d^+_2 - \frac{\epsilon_+}{2}d^-_2 +\sum_{k=3}^6 
    \frac{\epsilon_+d^+_k -\epsilon_-d^-_k}{k(k-1)} \right] + 2\epsilon_+
  \frac{C_g^{(\alpha_s)}}{T_R}\gamma
\end{equation}

\begin{equation}
F^{\nu(\alpha),c} =
\frac{C_q^{(\alpha_s)}}{C_F}\left[\frac{\epsilon_+}{6} d^+_1 + \frac{\epsilon_+}{6} d^-_1
  - \frac{e_D^2}{6}(d^+_3+d^-_3)
  + \frac{e_D^2+ 3e_U^2}{24}d^+_4 + \frac{e_D^2- 3e_U^2}{24}d^-_4
  +\sum_{k=5}^6
    \frac{\epsilon_+d^+_k +\epsilon_-d^-_k}{k(k-1)} \right] + 2\epsilon_+
  \frac{C_g^{(\alpha_s)}}{T_R}\gamma
\end{equation}

\begin{equation}
F^{\overline{\nu}(\alpha),c} =
\frac{C_q^{(\alpha_s)}}{C_F}\left[\frac{\epsilon_+}{6} d^+_1 - \frac{\epsilon_+}{6} d^-_1
  - \frac{e_D^2}{6}(d^+_3-d^-_3)
  + \frac{e_D^2+ 3e_U^2}{24}d^+_4 - \frac{e_D^2- 3e_U^2}{24}d^-_4
 + \sum_{k=5}^6
    \frac{\epsilon_+d^+_k -\epsilon_-d^-_k}{k(k-1)} \right] + 2\epsilon_+
  \frac{C_g^{(\alpha_s)}}{T_R}\gamma
\end{equation}

\begin{equation}
F^{\nu,(\alpha),t}=\frac{C_q^{(\alpha_s)}}{C_F}\left[\frac{\epsilon_+}{6}d^+_1+\frac{\epsilon_-}{6}d^-_1 - \frac{e_D^2}{10}(d^+_5+d^-_5) +\frac{e_D^2+ 5e_U^2}{60}d^+_6 +\frac{e_D^2- 5e_U^2}{60}d^-_6\right] + 2\epsilon_+
  \frac{C_g^{(\alpha_s)}}{T_R}\gamma
\end{equation}

\begin{equation}
F^{\overline{\nu},(\alpha),t}=\frac{C_q^{(\alpha_s)}}{C_F}\left[\frac{\epsilon_+}{6}d^+_1-\frac{\epsilon_-}{6}d^-_1 - \frac{e_D^2}{10}(d^+_5-d^-_5) +\frac{e_D^2+ 5e_U^2}{60}d^+_6 -\frac{e_D^2- 5e_U^2}{60}d^-_6\right] + 2\epsilon_+
  \frac{C_g^{(\alpha_s)}}{T_R}\gamma
\end{equation}

The relations above hold for $F_2$ and $F_L$. For $F_3$ the analogous
of Eqs.~(\ref{compactNu}) and~(\ref{compactNub}) are:
\begin{equation}
F_3^{\nu,(\alpha_s)} =
\sum_{U=u,c,t}\sum_{D=d,s,b}|V_{UD}|^2\left[C_{3,q}^{(\alpha_s)}\left(D
    -\overline{U}\right) +2 C_{3,g}^{(\alpha_s)}g\right]
\label{compactNu}
\end{equation}
and:
\begin{equation}
F_3^{\overline{\nu},(\alpha_s)} =
\sum_{U=u,c,t}\sum_{D=d,s,b}|V_{UD}|^2\left[C_{3,q}^{(\alpha_s)}\left(-\overline{D}
    +U\right) +2 C_{3,g}^{(\alpha_s)}g\right] 
\label{compactNub}
\end{equation}
In practice, as compared to $F_2$ and $F_L$, the structure of the
observables is the same with the only difference that the anti-quark
get a minus sign. In terms of the distributions in the evolution basis
$d_k^\pm$ this is equivalent to exchange the plus with the minus
distributions, $i.e.$ $d_k^+\leftrightarrow d_k^-$. Starting from the
relations above we can directly write the results for $F_3$:
\begin{equation}
  F_3^{\nu,(\alpha),l} = \frac{C_{3,q}^{(\alpha_s)}}{C_F}\left[\frac{\epsilon_+}{6}
    d^{-}_1+\frac{\epsilon_-}{6}
    d^{+}_1 + \frac{\epsilon_-}{2}d^{-}_2 + \frac{\epsilon_+}{2}d^{+}_2 +\sum_{k=3}^6 
    \frac{\epsilon_+d^{-}_k +\epsilon_-d^{+}_k}{k(k-1)} \right] + 2\epsilon_+
  \frac{C_{3,g}^{(\alpha_s)}}{T_R}\gamma
\end{equation}

\begin{equation}
  F_3^{\overline{\nu},(\alpha),l} = \frac{C_{3,q}^{(\alpha_s)}}{C_F}\left[\frac{\epsilon_+}{6}
    d^{-}_1-\frac{\epsilon_-}{6}
    d^{+}_1 + \frac{\epsilon_-}{2}d^{-}_2 - \frac{\epsilon_+}{2}d^{+}_2 +\sum_{k=3}^6 
    \frac{\epsilon_+d^{-}_k -\epsilon_-d^{+}_k}{k(k-1)} \right] + 2\epsilon_+
  \frac{C_{3,g}^{(\alpha_s)}}{T_R}\gamma
\end{equation}

\begin{equation}
F_3^{\nu(\alpha),c} =
\frac{C_{3,q}^{(\alpha_s)}}{C_F}\left[\frac{\epsilon_+}{6} d^{-}_1 + \frac{\epsilon_+}{6} d^{+}_1
  - \frac{e_D^2}{6}(d^{-}_3+d^{+}_3)
  + \frac{e_D^2+ 3e_U^2}{24}d^{-}_4 + \frac{e_D^2- 3e_U^2}{24}d^{+}_4
  +\sum_{k=5}^6
    \frac{\epsilon_+d^{-}_k +\epsilon_-d^{+}_k}{k(k-1)} \right] + 2\epsilon_+
  \frac{C_{3,g}^{(\alpha_s)}}{T_R}\gamma
\end{equation}

\begin{equation}
F_3^{\overline{\nu}(\alpha),c} =
\frac{C_{3,q}^{(\alpha_s)}}{C_F}\left[\frac{\epsilon_+}{6} d^{-}_1 - \frac{\epsilon_+}{6} d^{+}_1
  - \frac{e_D^2}{6}(d^{-}_3-d^{+}_3)
  + \frac{e_D^2+ 3e_U^2}{24}d^{-}_4 - \frac{e_D^2- 3e_U^2}{24}d^{+}_4
 + \sum_{k=5}^6
    \frac{\epsilon_+d^{-}_k -\epsilon_-d^{+}_k}{k(k-1)} \right] + 2\epsilon_+
  \frac{C_{3,g}^{(\alpha_s)}}{T_R}\gamma
\end{equation}

\begin{equation}
F_3^{\nu,(\alpha),t}=\frac{C_{3,q}^{(\alpha_s)}}{C_F}\left[\frac{\epsilon_+}{6}d^{-}_1+\frac{\epsilon_-}{6}d^{+}_1 - \frac{e_D^2}{10}(d^{-}_5+d^{+}_5) +\frac{e_D^2+ 5e_U^2}{60}d^{-}_6 +\frac{e_D^2- 5e_U^2}{60}d^{+}_6\right] + 2\epsilon_+
  \frac{C_{3,g}^{(\alpha_s)}}{T_R}\gamma
\end{equation}

\begin{equation}
F_3^{\overline{\nu},(\alpha),t}=\frac{C_{3,q}^{(\alpha_s)}}{C_F}\left[\frac{\epsilon_+}{6}d^{-}_1-\frac{\epsilon_-}{6}d^{+}_1 - \frac{e_D^2}{10}(d^{-}_5-d^{+}_5) +\frac{e_D^2+ 5e_U^2}{60}d^{-}_6 -\frac{e_D^2- 5e_U^2}{60}d^{+}_6\right] + 2\epsilon_+
  \frac{C_{3,g}^{(\alpha_s)}}{T_R}\gamma
\end{equation}

\subsection{The DGLAP equations}

In this section we address the question of implementing the full NLO
QCD$\otimes$QED corrections to the DGLAP equation. For the moment we
limit ourselves to considering only the presence of the photon PDF and
only later we will include also the lepton PDFs. The starting point is
Eqs.~(\ref{SingletDGLAP})-(\ref{NonSingletDGLAP}) and what we have to
do is identifying the form of the $\widetilde{P}$ splitting
functions. By definition they are all proportional to at least one
power of $\alpha$. In particular, when including only LO QED
corrections:
\begin{equation}
\widetilde{P} = \alpha \mathcal{P}^{(0,1)} + \dots
\end{equation}
where we are using the notation of
Refs.~\cite{deFlorian:2015ujt,deFlorian:2016gvk} to indicate the power
of $\alpha_s$ and $\alpha$ that a given splitting function
multiplies. The inclusion of the full NLO QCD$\otimes$QED corrections
implies including more terms to $\widetilde{P}$. In particular, we
have to include all terms whose sum of the power of $\alpha_s$ and
$\alpha$ is equal to two excluding the terms which are proportional to
$\alpha_s$ only. As a consequence, we have that:
\begin{equation}
\widetilde{P} = \alpha \mathcal{P}^{(0,1)} + \alpha_s\alpha \mathcal{P}^{(1,1)}+\alpha^2 \mathcal{P}^{(0,2)}\dots
\end{equation}
and the corrections $\mathcal{P}^{(1,1)}$ and $\mathcal{P}^{(0,2)}$
have been recently computed and published in
Refs.~\cite{deFlorian:2015ujt,deFlorian:2016gvk}.

It is now necessary to analyse the structure of the two additional
terms $\mathcal{P}^{(1,1)}$ and $\mathcal{P}^{(0,2)}$ in order to
construct the corresponding splitting matrices to be included in the
DGLAP equations. Let us start with the $\mathcal{O}(\alpha_s\alpha)$
correction. We first identify the vanishing terms and from Eq.~(32) of
Ref.~\cite{deFlorian:2015ujt} we read that:
\begin{equation}
\mathcal{P}_{qq}^{S(1,1)}=\mathcal{P}_{q\overline{q}}^{S(1,1)}=0
\end{equation}
while from Eq.~(\ref{defis}) we deduce that:
\begin{equation}
\begin{array}{l}
\mathcal{P}_{qq}^{(1,1)} = \mathcal{P}^{+(1,1)}\\
\\
\mathcal{P}^{V(1,1)} = \mathcal{P}^{-(1,1)}
\end{array}
\end{equation}

In the following we list the remaining terms taking them from
Ref.~\cite{deFlorian:2015ujt} but stripping them of the electric
charges which are already accounted for in our formulation. An
additional factor 4 is also included due to the difference in the
definition of the expansion parameters:
%\begin{equation}
%\begin{array}{rcl}
%  \mathcal{P}_{q\gamma}^{(1,1)} &=& \displaystyle 2C_F
%                                    \left\{ 4 -9x-(1-4x) \ln x - (1-2x)
%                                    \ln^2x +4 \ln(1-x) \right.\\
%  \\
%                                &+& \displaystyle
%                                    \left. p_{qg}(x) \left[
%                                    2\ln^2\left(\frac{1-x}{x}\right) -
%                                    4\ln\left(\frac{1-x}{x}\right)
%                                    -\frac{2\pi^2}{3}+10
%                                    \right]\right\}\,,\\
%  \\
%  \mathcal{P}_{g\gamma}^{(1,1)} &=& \displaystyle 4C_F \left\{ -16
%                                    +8 x +\frac{20}{3}x^2+\frac{4}{3x}
%                                    - \vphantom{\frac{20}{3}x^2}
%                                    (6+10x) \ln{x} -2(1+x)\ln^2{x}
%                                    \right\} \,,\\
%  \\
%  \mathcal{P}_{\gamma\gamma}^{(1,1)} &=& \displaystyle -4C_F
%                                         \delta(1-x)\,,\\
%  \\
%  \mathcal{P}_{qg}^{(1,1)} &=& \displaystyle  2T_R\left\{ 4
%                               -9x-(1-4x) \ln{x} - (1-2x)
%                               \ln^2{x} +4 \ln(1-x) \right.\\
%  \\
%                                &+& \displaystyle \left. 
%                                    p_{qg}(x) \left[  2\ln^2\left(\frac{1-x}{x}\right) -
%                                    4\ln\left(\frac{1-x}{x}\right)  -\frac{2\pi^2}{3}+10
%                                    \right]\right\}=\frac{T_R}{C_F} \mathcal{P}_{q\gamma}^{(1,1)}\,,\\
%  \\ 
%  \mathcal{P}_{\gamma g}^{(1,1)} &=& \displaystyle 4T_R \left\{ -16 +8
%                                     x +\frac{20}{3}x^2+\frac{4}{3x} -
%                                     \vphantom{\frac{20}{3}x^2}
%                                     (6+10x) \ln{x} -2(1+x)\ln^2{x}
%                                     \right\}=\frac{T_R}{C_F} \mathcal{P}_{g\gamma}^{(1,1)}\,,\\ 
%  \\
%  \mathcal{P}_{gg}^{(1,1)} &=& \displaystyle -4T_R \delta(1-x)=\frac{T_R}{C_F} \mathcal{P}_{\gamma\gamma}^{(1,1)}\,,\\
%  \\
%  \mathcal{P}_{qq}^{V(1,1)} &=& \displaystyle 8 C_F\left[-\left(2 \ln{x}
%                                \ln(1-x)+\frac{3}{2}\ln{x}\right)
%                                p_{qq}(x) - \frac{3+7x}{2}\ln{x} -
%                                \frac{1+x}{2}{\ln^2{x}}  \right.\\
%  \\
%                                &-& \displaystyle \left. 5(1-x) -
%                                    \left(
%                                    \frac{\pi^2}{2}-\frac{3}{8}-6
%                                    \zeta_3 \right) \delta(1-x)
%                                    \right]\,,\\
%  \\
%  \mathcal{P}_{q\bar{q}}^{V(1,1)} &=& \displaystyle
%                                      8C_F\left[4(1-x)+2(1+x)\ln{x} +
%                                      2p_{qq}(-x)S_2(x)\right]\,,\\
%  \\
%  \mathcal{P}^{+(1,1)} =
%  \mathcal{P}_{qq}^{V(1,1)}+\mathcal{P}_{q\bar{q}}^{V(1,1)}&=&
%  \displaystyle 8 C_F\left[-\left(2 \ln{x}
%    \ln(1-x)+\frac{3}{2}\ln{x}\right)
%                                p_{qq}(x) + \frac{1-3x}{2}\ln{x} -
%                                \frac{1+x}{2}{\ln^2{x}}  \right.\\
%  \\
%                                &-& \displaystyle \left. (1-x) +
%                                    2p_{qq}(-x)S_2(x) -
%                                    \left(
%                                    \frac{\pi^2}{2}-\frac{3}{8}-6
%                                    \zeta_3 \right) \delta(1-x)
%                                    \right]\,,\\
%  \\
%  \mathcal{P}^{-(1,1)} =
%  \mathcal{P}_{qq}^{V(1,1)}-\mathcal{P}_{q\bar{q}}^{V(1,1)}&=&
%  \displaystyle 8C_F\left[-\left(2 \ln{x}
%                                \ln(1-x)+\frac{3}{2}\ln{x}\right)
%                                p_{qq}(x) - \frac{7+11x}{2}\ln{x} -
%                                \frac{1+x}{2}{\ln^2{x}}  \right.\\
%  \\
%                                &-& \displaystyle \left. 9(1-x) -2p_{qq}(-x)S_2(x)-
%                                    \left(
%                                    \frac{\pi^2}{2}-\frac{3}{8}-6
%                                    \zeta_3 \right) \delta(1-x)
%                                    \right]\,,\\
%  \\
%  \mathcal{P}_{gq}^{(1,1)} &=& \displaystyle
%                               4C_F\left[-(3\ln(1-x)+\ln^2(1-x))p_{gq}(x)
%                               \vphantom{\left(2+\frac{7}{2}x\right)}
%                               +
%                               \left(2+\frac{7}{2}x\right)\ln{x}\right.\\
%  \\
%                                &-& \displaystyle
%                                    \left. \left(1-\frac{x}{2}\right)\ln^2{x}
%                                    -
%                                    2x\ln(1-x)-\frac{7}{2}x-\frac{5}{2}\right]\,,\\
%  \\
%  \mathcal{P}_{\gamma q}^{(1,1)} &=& \displaystyle \mathcal{P}_{gq}^{(1,1)}\,,
%\end{array}
%\end{equation}

\begin{equation}
\begin{array}{rcl}
  \mathcal{P}_{q\gamma}^{(1,1)} &=& \displaystyle 2C_F
                                    \left\{ 4 -9x-(1-4x) \ln x - (1-2x)
                                    \ln^2x +4 \ln(1-x) \right.\\
  \\
                                &+& \displaystyle
                                    \left. p_{qg}(x) \left[
                                    2\ln^2\left(\frac{1-x}{x}\right) -
                                    4\ln\left(\frac{1-x}{x}\right)
                                    -\frac{2\pi^2}{3}+10
                                    \right]\right\}\,,\\
  \\
  \mathcal{P}_{g\gamma}^{(1,1)} &=& \displaystyle 4C_F \left\{ -16
                                    +8 x +\frac{20}{3}x^2+\frac{4}{3x}
                                    - \vphantom{\frac{20}{3}x^2}
                                    (6+10x) \ln{x} -2(1+x)\ln^2{x}
                                    \right\} \,,\\
  \\
  \mathcal{P}_{\gamma\gamma}^{(1,1)} &=& \displaystyle -4C_F
                                         \delta(1-x)\,,\\
  \\
  \mathcal{P}_{qg}^{(1,1)} &=& \displaystyle \frac{T_R}{C_F} \mathcal{P}_{q\gamma}^{(1,1)}\,,\\
  \\ 
  \mathcal{P}_{\gamma g}^{(1,1)} &=& \displaystyle \frac{T_R}{C_F} \mathcal{P}_{g\gamma}^{(1,1)}\,,\\ 
  \\
  \mathcal{P}_{gg}^{(1,1)} &=& \displaystyle \frac{T_R}{C_F} \mathcal{P}_{\gamma\gamma}^{(1,1)}\,,\\
  \\
  \mathcal{P}_{qq}^{V(1,1)} &=& \displaystyle 8 C_F\left[-\left(2 \ln{x}
                                \ln(1-x)+\frac{3}{2}\ln{x}\right)
                                p_{qq}(x) - \frac{3+7x}{2}\ln{x} -
                                \frac{1+x}{2}{\ln^2{x}}  \right.\\
  \\
                                &-& \displaystyle \left. 5(1-x) -
                                    \left(
                                    \frac{\pi^2}{2}-\frac{3}{8}-6
                                    \zeta_3 \right) \delta(1-x)
                                    \right]\,,\\
  \\
  \mathcal{P}_{q\bar{q}}^{V(1,1)} &=& \displaystyle
                                      8C_F\left[4(1-x)+2(1+x)\ln{x} +
                                      2p_{qq}(-x)S_2(x)\right]\,,\\
  \\
  \mathcal{P}^{+(1,1)} =
  \mathcal{P}_{qq}^{V(1,1)}+\mathcal{P}_{q\bar{q}}^{V(1,1)}&=&
  \displaystyle 8 C_F\left[-\left(2 \ln{x}
    \ln(1-x)+\frac{3}{2}\ln{x}\right)
                                p_{qq}(x) + \frac{1-3x}{2}\ln{x} -
                                \frac{1+x}{2}{\ln^2{x}}  \right.\\
  \\
                                &-& \displaystyle \left. (1-x) +
                                    2p_{qq}(-x)S_2(x) -
                                    \left(
                                    \frac{\pi^2}{2}-\frac{3}{8}-6
                                    \zeta_3 \right) \delta(1-x)
                                    \right]\,,\\
  \\
  \mathcal{P}^{-(1,1)} =
  \mathcal{P}_{qq}^{V(1,1)}-\mathcal{P}_{q\bar{q}}^{V(1,1)}&=&
  \displaystyle 8C_F\left[-\left(2 \ln{x}
                                \ln(1-x)+\frac{3}{2}\ln{x}\right)
                                p_{qq}(x) - \frac{7+11x}{2}\ln{x} -
                                \frac{1+x}{2}{\ln^2{x}}  \right.\\
  \\
                                &-& \displaystyle \left. 9(1-x) -2p_{qq}(-x)S_2(x)-
                                    \left(
                                    \frac{\pi^2}{2}-\frac{3}{8}-6
                                    \zeta_3 \right) \delta(1-x)
                                    \right]\,,\\
  \\
  \mathcal{P}_{\gamma q}^{(1,1)} &=& \displaystyle
                               4C_F\left[-(3\ln(1-x)+\ln^2(1-x))p_{gq}(x)
                               \vphantom{\left(2+\frac{7}{2}x\right)}
                               +
                               \left(2+\frac{7}{2}x\right)\ln{x}\right.\\
  \\
                                &-& \displaystyle
                                    \left. \left(1-\frac{x}{2}\right)\ln^2{x}
                                    -
                                    2x\ln(1-x)-\frac{7}{2}x-\frac{5}{2}\right]\,,\\
  \\
  \mathcal{P}_{gq}^{(1,1)} &=& \displaystyle \mathcal{P}_{\gamma q}^{(1,1)}\,,
\end{array}
\end{equation}

The function $S_2(x)$ is given by:
\begin{equation}
  S_2(x) = \int_{\frac{x}{1+x}}^{\frac{1}{1+x}} \frac{dz}{z}
  \ln\left(\frac{1-z}{z}\right) = \mbox{Li}_2\left(-\frac{1}{x}\right)
  - \mbox{Li}_2(-x) +
  \ln^2\left(\frac{x}{1+x}\right)-\ln^2\left(\frac{1}{1+x}\right)\,.
\label{eq:S2definicion}
\end{equation}
and:
\begin{equation}
\begin{array}{rcl}
p_{qq}(x)&=&\displaystyle \frac{1+x^2}{(1-x)_+}\\
\\
p_{qg}(x)&=&\displaystyle x^2+(1-x)^2\\
\\
p_{gq}(x)&=&\displaystyle \frac{1+(1-x)^2}{x}\\
\end{array}
\end{equation}
It should be noticed that, excluding the colour factors and some
additional pre-factors, all functions appearing above appear also in
the $\mathcal{O}(\alpha_s^2)$ expressions of the splitting
functions. As a consequence, there is no additional work to be done to
reduce these expressions to the form needed by {\tt APFEL}, that is to
separate regular, singular, and local contributions.

The resulting evolution equations at $\mathcal{O}(\alpha_s\alpha)$
are:
\begin{equation}
\begin{array}{rcl}
\displaystyle\left.\mu^2\frac{\partial}{\partial \mu^2}
\begin{pmatrix}
g\\
\gamma\\
\Sigma\\
\Delta_\Sigma
\end{pmatrix}\right|_{\mathcal{O}(\alpha_s \alpha)} &=& \displaystyle \begin{pmatrix}
e_\Sigma^2 \mathcal{P}^{(1,1)}_{gg}      & e_\Sigma^2 \mathcal{P}^{(1,1)}_{g\gamma} & \eta^+\mathcal{P}^{(1,1)}_{gq} & \eta^-\mathcal{P}^{(1,1)}_{gq} \\
e_\Sigma^2 \mathcal{P}^{(1,1)}_{\gamma g} & e_\Sigma^2 \mathcal{P}^{(1,1)}_{\gamma\gamma} & \eta^+\mathcal{P}^{(1,1)}_{\gamma q} &\eta^-\mathcal{P}^{(1,1)}_{\gamma q} \\
2 e_\Sigma^2 \mathcal{P}^{(1,1)}_{qg}    & 2 e_\Sigma^2 \mathcal{P}^{(1,1)}_{q\gamma} & \eta^+\mathcal{P}^{+(1,1)}  & \eta^-\mathcal{P}^{+(1,1)}\\
2 \delta_e^2 \mathcal{P}^{(1,1)}_{qg} & 2 \delta_e^2 \mathcal{P}^{(1,1)}_{q\gamma} &\eta^-\mathcal{P}^{+(1,1)} &\eta^+\mathcal{P}^{+(1,1)}
\end{pmatrix}
\begin{pmatrix}
g\\
\gamma\\
\Sigma\\
\Delta_\Sigma
\end{pmatrix}
\end{array}
\end{equation}


\begin{equation}
\displaystyle\left.\mu^2\frac{\partial}{\partial \mu^2}
\begin{pmatrix}
V\\
\Delta_V
\end{pmatrix} \right|_{\mathcal{O}(\alpha_s \alpha)}= 
\begin{pmatrix}
\eta^+\mathcal{P}^{-(1,1)} & \eta^-\mathcal{P}^{-(1,1)} \\
\eta^-\mathcal{P}^{-(1,1)} & \eta^+\mathcal{P}^{-(1,1)} 
\end{pmatrix}
\begin{pmatrix}
V\\
\Delta_V
\end{pmatrix}
\end{equation}


\begin{equation}
\begin{array}{l}
\begin{array}{rcl}
\\
\displaystyle \left.\mu^2\frac{\partial T^u_{1,2}}{\partial \mu^2}\right|_{\mathcal{O}(\alpha_s \alpha)} &=&
\displaystyle e_u^2\mathcal{P}^{+(1,1)} T^u_{1,2}\\
\\
\displaystyle \left.\mu^2\frac{\partial T^d_{1,2}}{\partial \mu^2}\right|_{\mathcal{O}(\alpha_s \alpha)} &=&
\displaystyle e_d^2\mathcal{P}^{+(1,1)} T^d_{1,2}
\end{array}
\\
\\
\begin{array}{rcl}
\displaystyle \left.\mu^2\frac{\partial V^u_{1,2}}{\partial \mu^2}\right|_{\mathcal{O}(\alpha_s \alpha)} &=&
\displaystyle e_u^2\mathcal{P}^{-(1,1)} V^u_{1,2}\\
\\
\displaystyle \left.\mu^2\frac{\partial V^d_{1,2}}{\partial \mu^2}\right|_{\mathcal{O}(\alpha_s \alpha)} &=&
\displaystyle e_d^2\mathcal{P}^{-(1,1)} V^d_{1,2}
\end{array}
\end{array}
\end{equation}

Now we turn to $\mathcal{P}^{(0,2)}$ and, referring to Eq.~(58) of
Ref.~\cite{deFlorian:2016gvk}, we first observe that:
\begin{equation}
\mathcal{P}_{qq}^{S(0,2)}=\mathcal{P}_{q\overline{q}}^{S(0,2)} \neq 0
\end{equation}
so that:
\begin{equation}
\begin{array}{l}
\mathcal{P}_{qq}^{(0,2)} \neq \mathcal{P}^{+(0,2)}\\
\\
\mathcal{P}^{V(0,2)} = \mathcal{P}^{-(0,2)}
\end{array}
\end{equation}
The fact that $\mathcal{P}^{V} = \mathcal{P}^{-}$ was actually one of
our assumptions which is verified. Of course if we were to include QED
corrections beyond NLO this assumption would no longer hold. Also,
from Eq.~(53) of the same reference we read:
\begin{equation}
\begin{array}{l}
\mathcal{P}_{qg}^{(0,2)} = \mathcal{P}_{gq}^{(0,2)} = 0\\
\\
\mathcal{P}_{gg}^{(0,2)} = \mathcal{P}_{g\gamma}^{(0,2)} =
  \mathcal{P}_{\gamma g}^{(0,2)} = 0
\end{array}
\end{equation}
which provide a big simplification. The non-vanishing
$\mathcal{O}(\alpha^2)$ splitting functions can mostly be written in
terms of the $\mathcal{O}(\alpha_s\alpha)$ ones and are:
%\begin{equation}
%\begin{array}{rcl}
%  \mathcal{P}_{\gamma \gamma}^{(0,2)} &=& \displaystyle 4\left[ -16 +8
%    x +\frac{20}{3}x^2+\frac{4}{3x} -(6+10x) \ln{x} -
%    2(1+x)\ln^2{x} -\delta(1-x)\right]\,,\\
%  \\
%  \mathcal{P}_{q\gamma}^{(0,2)}
%  &=&\displaystyle 2\left\{ 4-9x-(1-4x) \ln{x} -(1-2x)\ln^2{x} +4
%  \ln(1-x) \right.\\
%  \\
%                               &+&\displaystyle \left. p_{qg}(x)
%                                   \left[
%                                   2\ln^2\left(\frac{1-x}{x}\right)
%                                   -4\ln\left(\frac{1-x}{x}\right)
%                                   -\frac{2\pi^2}{3}+10 \right]
%                                   \right\}\,,\\
%  \\
%  \mathcal{P}_{\gamma q}^{(0,2)} &=&\displaystyle 4
%                                     \left[-\left(3\ln(1-x)+\ln^2(1-x)\right)p_{gq}(x)+\left(2+\frac{7}{2}x\right)\ln{x}-\left(1-\frac{x}{2}\right)\ln^2{x}
%                                     \right.\\
%  \\
%                               &-&\displaystyle
%                                   \left. 2x\ln(1-x)-\frac{7}{2}x-\frac{5}{2}\right]
%                                   - 4\left(\frac1{e_q^2}e_\Sigma^2
%                                   \right) \left[  \frac{4}{3}x +
%                                   p_{gq}(x) \left(
%                                   \frac{20}{9}+\frac{4}{3}\ln(1-x)
%                                   \right)\right]\,,\\
%  \\
%  \mathcal{P}_{qq}^{V(0,2)} &=&\displaystyle  4 \left[-\left(2
%                                \ln{x}\ln(1-x)+\frac{3}{2}\ln{x}\right)
%                                p_{qq}(x) - \frac{3+7x}{2}\ln{x}
%                                \right.\\
%  \\
%                               &-&\displaystyle \left.
%                                   \frac{1+x}{2}\ln^2{x}-5(1-x)
%                                   -\left(\frac{\pi^2}{2}-\frac{3}{8}-6
%                                   \zeta_3 \right) \delta(1-x)
%                                   \right]\\
%  \\
%                               &-&\displaystyle 4
%                                   \left(\frac1{e_q^2}e_\Sigma^2\right) \left[
%                                   \frac{4}{3}(1-x) +  p_{qq}(x)
%                                   \left( \frac{2}{3}
%                                   \ln{x}+\frac{10}{9}  \right)
%                                   +\left(\frac{2
%                                   \pi^2}{9}+\frac{1}{6}\right)
%                                   \delta(1-x)\right]\,,\\
%  \\
%  \mathcal{P}_{q\bar{q}}^{V(0,2)} &=&\displaystyle 4
%                                      \left[4(1-x)+2(1+x)\ln{x}+2p_{qq}(-x)S_2(x)\right]\,,\\
%  \\
%  \mathcal{P}_{qq}^{+(0,2)}=\mathcal{P}_{qq}^{V(0,2)}+\mathcal{P}_{q\overline{q}}^{V(0,2)} &=&\displaystyle  4 \left[-\left(2
%                                \ln{x}\ln(1-x)+\frac{3}{2}\ln{x}\right)
%                                p_{qq}(x) + \frac{1-3x}{2}\ln{x}
%                                \right.\\
%  \\
%                               &-&\displaystyle \left.
%                                   \frac{1+x}{2}\ln^2{x}-(1-x)+2p_{qq}(-x)S_2(x)
%                                   -\left(\frac{\pi^2}{2}-\frac{3}{8}-6
%                                   \zeta_3 \right) \delta(1-x)
%                                   \right]\\
%  \\
%                               &-&\displaystyle 4
%                                   \left(\frac1{e_q^2}e_\Sigma^2\right) \left[
%                                   \frac{4}{3}(1-x) +  p_{qq}(x)
%                                   \left( \frac{2}{3}
%                                   \ln{x}+\frac{10}{9}  \right)
%                                   +\left(\frac{2
%                                   \pi^2}{9}+\frac{1}{6}\right)
%                                   \delta(1-x)\right]\,,\\
%  \\
%  \mathcal{P}_{qq}^{-(0,2)}=\mathcal{P}_{qq}^{V(0,2)}-\mathcal{P}_{q\overline{q}}^{V(0,2)}  &=&\displaystyle  4 \left[-\left(2
%                                \ln{x}\ln(1-x)+\frac{3}{2}\ln{x}\right)
%                                p_{qq}(x) - \frac{7+11x}{2}\ln{x}
%                                \right.\\
%  \\
%                               &-&\displaystyle \left.
%                                   \frac{1+x}{2}\ln^2{x}-9(1-x)-2p_{qq}(-x)S_2(x)
%                                   -\left(\frac{\pi^2}{2}-\frac{3}{8}-6
%                                   \zeta_3 \right) \delta(1-x)
%                                   \right]\\
%  \\
%                               &-&\displaystyle 4
%                                   \left(\frac1{e_q^2}e_\Sigma^2\right) \left[
%                                   \frac{4}{3}(1-x) +  p_{qq}(x)
%                                   \left( \frac{2}{3}
%                                   \ln{x}+\frac{10}{9}  \right)
%                                   +\left(\frac{2
%                                   \pi^2}{9}+\frac{1}{6}\right)
%                                   \delta(1-x)\right]\,,\\
%  \\
%  \mathcal{P}_{qq}^{S(0,2)} &=&\displaystyle 4\left(\frac{20}{9x}
%                                -2+6x-\frac{56}{9}x^2+\left(1+5x+\frac{8}{3}x^2\right)\ln{x}-(1+x)\ln^2{x}\right)\,,
%\end{array}
%\end{equation}

\begin{equation}
\begin{array}{rcl}
  \mathcal{P}_{\gamma \gamma}^{(0,2)} &=& \displaystyle \frac1{C_F}\mathcal{P}_{g \gamma}^{(1,1)} -4\delta(1-x)\\
  \\
  \mathcal{P}_{q\gamma}^{(0,2)}      &=&\displaystyle \frac1{C_F}\mathcal{P}_{q \gamma}^{(1,1)}\\
  \\
  \mathcal{P}_{\gamma q}^{(0,2)} &=&\displaystyle \frac1{C_F}\mathcal{P}_{\gamma q}^{(1,1)} + \left(\frac{e_\Sigma^2}{e_q^2}
                                   \right) 4\left[ - \frac{4}{3}x -
                                   p_{gq}(x) \left(
                                   \frac{20}{9}+\frac{4}{3}\ln(1-x)
                                   \right)\right]\,,\\
  \\
  \mathcal{P}_{qq}^{\pm(0,2)}     &=&\displaystyle  \frac1{2C_F}\mathcal{P}^{\pm(1,1)} 
                                   +\left(\frac{e_\Sigma^2}{e_q^2}\right) 4\left[
                                   -\frac{4}{3}(1-x) - p_{qq}(x)
                                   \left( \frac{2}{3}
                                   \ln{x}+\frac{10}{9} \right)
                                   -\left(\frac{2
                                   \pi^2}{9}+\frac{1}{6}\right)
                                   \delta(1-x)\right]\,,\\
  \\
  \mathcal{P}_{qq}^{S(0,2)} &=&\displaystyle 4\left[\frac{20}{9x}
                                -2+6x-\frac{56}{9}x^2+\left(1+5x+\frac{8}{3}x^2\right)\ln{x}-(1+x)\ln^2{x}\right]\,,
\end{array}
\end{equation}

It is interesting to notice that the expressions above for the
$\mathcal{O}(\alpha^2)$ splitting functions coincide with the
$\mathcal{O}(\alpha_s^2)$ where $C_F$ and $T_R$ are set to one and
$C_A$ set to zero.

Contrary to all cases we have treated so far in which the electric
charges appeared to the second power at most, here they appear to the
fourth power and thus we need to adjust the couplings accordingly.
Although, we could adjust the couplings relying on basic
considerations, we adopt the brute force approach and re-derive the
DGLAP equations in the evolution basis defined above but concentrating
only on the $\mathcal{O}(\alpha^2)$ contributions. This will provide a
more solid result. In order to simplify the notation, we will get rid
of all unneeded indices under the assumption that we are dealing only
with the $\mathcal{O}(\alpha^2)$ contributions to the splitting
functions.  We now define:
\begin{equation}
e_{\Sigma}^4 = N_c(e_u^4 n_{u} + e_d^4 n_{d})\,,
\end{equation}
so that:
\begin{equation}
{P}_{\gamma\gamma} \rightarrow e_\Sigma^4\mathcal{P}_{\gamma\gamma}\,.
\end{equation}
In addition, for the splitting functions involving one photon and one
quark we have:
\begin{equation}
\begin{array}{cc}
\displaystyle {P}_{\gamma u_i} = {P}_{\gamma \overline{u}_i} \rightarrow
e_u^4\mathcal{P}_{\gamma u}\,, & \displaystyle {P}_{\gamma d_i} = {P}_{\gamma \overline{d}_i} \rightarrow
e_d^4\mathcal{P}_{\gamma d}\,, \\
\\
\displaystyle {P}_{u_i \gamma } = {P}_{\overline{u}_i \gamma } \rightarrow
e_u^4{P}_{q\gamma }\,, & \displaystyle {P}_{d_i \gamma } =
{P}_{\overline{d}_i \gamma } \rightarrow
e_d^4{P}_{q\gamma }\,.
\end{array}
\end{equation}
Finally, we consider the splitting functions involving quarks or
anti-quarks in the final and initial states:
\begin{equation}
\begin{array}{l}
{P}_{u_iu_j} = {P}_{\overline{u}_i\overline{u}_j} \rightarrow e_u^4\delta_{ij} \mathcal{P}_{uu}^V+e_u^4\mathcal{P}_{qq}^S\\
{P}_{d_id_j} = {P}_{\overline{d}_i\overline{d}_j} \rightarrow e_d^4\delta_{ij} \mathcal{P}_{dd}^V+e_d^4\mathcal{P}_{qq}^S\\
{P}_{\overline{u}_iu_j} = {P}_{u_i\overline{u}_j} \rightarrow e_u^4\delta_{ij} \mathcal{P}_{q\overline{q}}^V+e_u^4\mathcal{P}_{qq}^S\\
{P}_{\overline{d}_id_j} = {P}_{d_i\overline{d}_j} \rightarrow e_d^4\delta_{ij} \mathcal{P}_{q\overline{q}}^V+e_d^4\mathcal{P}_{qq}^S\\
{P}_{u_id_j} = {P}_{d_iu_j} = {P}_{\overline{u}_id_j} = {P}_{d_i\overline{u}_j} = {P}_{\overline{d}_iu_j} = {P}_{u_i\overline{d}_j} = {P}_{\overline{u}_i\overline{d}_j} = {P}_{\overline{d}_i\overline{u}_j} \rightarrow e_u^2e_d^2\mathcal{P}_{qq}^S
\end{array}\,.
\label{decompositionQED}
\end{equation}
so that we finally get:
\begin{equation}
\begin{array}{rcl}
\displaystyle \mu^2\frac{\partial}{\partial \mu^2}
\left.\begin{pmatrix}
u_j^+ \\
d_i^+\\
g\\
\gamma\\
d_i^-\\
u_j^-
\end{pmatrix}\right|_{\mathcal{O}(\alpha^2)} &=& \displaystyle
\begin{pmatrix}
e_u^4\mathcal{P}_{uu}^+ & 0 & 0 & 2 e_u^4\mathcal{P}_{q\gamma} & 0 & 0 \\ 
0 & e_d^4\mathcal{P}_{dd}^+ & 0 & 2 e_d^4\mathcal{P}_{q\gamma} & 0 & 0 \\ 
0 & 0 & 0 & 0 & 0 & 0 \\
e_u^4 \mathcal{P}_{\gamma u} & e_d^4\mathcal{P}_{\gamma d} & 0 & e_\Sigma^4\mathcal{P}_{\gamma\gamma} & 0 & 0 \\
0 & 0 & 0 & 0 & e_d^4\mathcal{P}_{dd}^- & 0 \\
0 & 0 & 0 & 0 & 0 & e_u^4\mathcal{P}_{uu}^-
\end{pmatrix}
\begin{pmatrix}
u_j^+ \\
d_i^+\\
g\\
\gamma\\
d_i^-\\
u_j^-
\end{pmatrix}\\
\\
&+&
\displaystyle 
\begin{pmatrix}
e_u^4\mathcal{P}_{qq}^{S} & e_u^2e_d^2\mathcal{P}_{qq}^{S} & 0 & 0 & 0 & 0 \\ 
e_u^2e_d^2\mathcal{P}_{qq}^{S} & e_d^4\mathcal{P}_{qq}^{S} & 0 & 0 & 0 & 0 \\ 
0  & 0 & 0 & 0 & 0 & 0\\
0  & 0 & 0 & 0 & 0 & 0\\
0  & 0 & 0 & 0 & 0 & 0\\
0  & 0 & 0 & 0 & 0 & 0
\end{pmatrix}
\begin{pmatrix}
\Sigma_u \\
\Sigma_d\\
g\\
\gamma\\
V_d\\
V_u
\end{pmatrix}
\end{array}
\end{equation}
that can also be written as:
\begin{equation}
\left\{
\begin{array}{rcl}
\displaystyle \left.\mu^2\frac{\partial g}{\partial \mu^2}\right|_{\mathcal{O}(\alpha^2)} &=& 0\\
\\
\displaystyle \left.\mu^2\frac{\partial \gamma}{\partial \mu^2}\right|_{\mathcal{O}(\alpha^2)} &=& e_u^4
\mathcal{P}_{\gamma u}\Sigma_u + e_d^4
\mathcal{P}_{\gamma d}\Sigma_d+ e_\Sigma^4 \mathcal{P}_{\gamma\gamma} \gamma\\
\\
\displaystyle \left.\mu^2\frac{\partial \Sigma_d}{\partial \mu^2}\right|_{\mathcal{O}(\alpha^2)} &=& \displaystyle e_d^4\left(\mathcal{P}_{dd}^+ + n_d\mathcal{P}^{S}_{qq}\right)\Sigma_d + n_de_d^2 e_u^2\mathcal{P}_{qq}^S\Sigma_u + 2n_de_d^4 \mathcal{P}_{q\gamma} \gamma\\
\\
\displaystyle \left.\mu^2\frac{\partial \Sigma_u}{\partial \mu^2}\right|_{\mathcal{O}(\alpha^2)} &=& \displaystyle e_u^4\left(\mathcal{P}_{uu}^+ +n_u\mathcal{P}^{S}_{qq}\right)\Sigma_u + n_u e_u^2 e_d^2\mathcal{P}^{S}_{qq}\Sigma_d  + 2n_ue_u^4 \mathcal{P}_{q\gamma} \gamma\\
\\
\displaystyle \left.\mu^2\frac{\partial V_d}{\partial \mu^2}\right|_{\mathcal{O}(\alpha^2)} &=& \displaystyle e_d^4\mathcal{P}_{dd}^-V_d\\
\\
\displaystyle \left.\mu^2\frac{\partial V_u}{\partial \mu^2}\right|_{\mathcal{O}(\alpha^2)} &=& \displaystyle e_u^4\mathcal{P}_{uu}^-V_u
\end{array}
\right.\,.
\end{equation}
Now, using the fact that:
\begin{equation}
\begin{array}{l}
\Sigma = \Sigma_u +\Sigma_d\\
\Delta_\Sigma = \Sigma_u -\Sigma_d\\
V = V_u + V_d\\
\Delta_V = V_u - V_d
\end{array}
\end{equation}
we can write:
\begin{equation}
\left\{
\begin{array}{rcl}
\displaystyle \left.\mu^2\frac{\partial g}{\partial \mu^2}\right|_{\mathcal{O}(\alpha^2)} &=& 0\\
\\
\displaystyle \left.\mu^2\frac{\partial \gamma}{\partial \mu^2}\right|_{\mathcal{O}(\alpha^2)} &=& \displaystyle
\frac12\left(e_u^4 \mathcal{P}_{\gamma u} + e_d^4 \mathcal{P}_{\gamma d}\right)\Sigma+
\frac12\left(e_u^4 \mathcal{P}_{\gamma u} - e_d^4 \mathcal{P}_{\gamma d}\right)\Delta_\Sigma
+ e_\Sigma^4 \mathcal{P}_{\gamma\gamma} \gamma\\
\\
\displaystyle \left.\mu^2\frac{\partial \Sigma}{\partial \mu^2}\right|_{\mathcal{O}(\alpha^2)} &=& \displaystyle \frac12\left(e_u^4\mathcal{P}_{uu}^+ +e_d^4\mathcal{P}_{dd}^++2\eta^+e_\Sigma^2\mathcal{P}^{S}_{qq}\right)\Sigma + \frac12\left(e_u^4\mathcal{P}_{uu}^+-e_d^4\mathcal{P}_{dd}^+ + 2\eta^-e_\Sigma^2\mathcal{P}^{S}_{qq}\right)\Delta_\Sigma+ 2e_
\Sigma^4 \mathcal{P}_{q\gamma} \gamma\\
\\
\displaystyle \left.\mu^2\frac{\partial \Delta_\Sigma}{\partial \mu^2}\right|_{\mathcal{O}(\alpha^2)} &=& \displaystyle \frac12\left(e_u^4\mathcal{P}_{uu}^+ -e_d^4\mathcal{P}_{dd}^++2\eta^-\delta_e^2 \mathcal{P}^{S}_{qq}\right)\Sigma + \frac12\left(e_u^4\mathcal{P}_{uu}^++e_d^4\mathcal{P}_{dd}^+ + 2\eta^+\delta_e^2 \mathcal{P}^{S}_{qq}\right)\Delta_\Sigma + 2\delta_e^4 \mathcal{P}_{q\gamma} \gamma\\
\\
\displaystyle \left.\mu^2\frac{\partial V}{\partial \mu^2}\right|_{\mathcal{O}(\alpha^2)} &=& \displaystyle \frac12\left(e_u^4\mathcal{P}_{uu}^-+e_d^4\mathcal{P}_{dd}^-\right)V+\frac12\left(e_u^4\mathcal{P}_{uu}^--e_d^4\mathcal{P}_{dd}^-\right)\Delta_V\\
\\
\displaystyle \left.\mu^2\frac{\partial \Delta_V}{\partial \mu^2}\right|_{\mathcal{O}(\alpha^2)} &=& \displaystyle \frac12\left(e_u^4\mathcal{P}_{uu}^--e_d^4\mathcal{P}_{dd}^-\right)V+\frac12\left(e_u^4\mathcal{P}_{uu}^-+e_d^4\mathcal{P}_{dd}^-\right)\Delta_V
\end{array}
\right.\,.
\end{equation}
And the remaining distributions evolving multiplicatively as:
\begin{equation}
\begin{array}{rcl}
\displaystyle \left.\mu^2\frac{\partial T^u_{1,2}}{\partial \mu^2}\right|_{\mathcal{O}(\alpha^2)}  &=& \displaystyle e_u^4\mathcal{P}_{uu}^+ T^u_{1,2}\\
\\
\displaystyle \left.\mu^2\frac{\partial T^d_{1,2}}{\partial \mu^2}\right|_{\mathcal{O}(\alpha^2)}  &=& \displaystyle e_d^4\mathcal{P}_{dd}^+ T^d_{1,2}\\
\\
\displaystyle \left.\mu^2\frac{\partial V^u_{1,2}}{\partial \mu^2}\right|_{\mathcal{O}(\alpha^2)}  &=& \displaystyle e_u^4\mathcal{P}_{uu}^- V^u_{1,2}\\
\\
\displaystyle \left.\mu^2\frac{\partial V^d_{1,2}}{\partial \mu^2}\right|_{\mathcal{O}(\alpha^2)}  &=& \displaystyle e_d^4\mathcal{P}_{dd}^- V^d_{1,2}
\end{array}
\end{equation}

\newpage
\bibliographystyle{ieeetr}
\bibliography{bibliography}

\end{document}

