%% LyX 2.0.3 created this file.  For more info, see http://www.lyx.org/.
%% Do not edit unless you really know what you are doing.
\documentclass[twoside,english]{paper}
\usepackage{lmodern}
\renewcommand{\ttdefault}{lmodern}
\usepackage[T1]{fontenc}
\usepackage[latin9]{inputenc}
\usepackage[a4paper]{geometry}
\geometry{verbose,tmargin=3cm,bmargin=2.5cm,lmargin=2cm,rmargin=2cm}
\usepackage{color}
\usepackage{babel}
\usepackage{float}
\usepackage{bm}
\usepackage{amsthm}
\usepackage{amsmath}
\usepackage{amssymb}
\usepackage{graphicx}
\usepackage{esint}
\usepackage[unicode=true,pdfusetitle,
 bookmarks=true,bookmarksnumbered=false,bookmarksopen=false,
 breaklinks=false,pdfborder={0 0 0},backref=false,colorlinks=false]
 {hyperref}
\usepackage{breakurl}
\usepackage{mathrsfs}

\makeatletter

%%%%%%%%%%%%%%%%%%%%%%%%%%%%%% LyX specific LaTeX commands.
%% Because html converters don't know tabularnewline
\providecommand{\tabularnewline}{\\}

%%%%%%%%%%%%%%%%%%%%%%%%%%%%%% Textclass specific LaTeX commands.
\numberwithin{equation}{section}
\numberwithin{figure}{section}

%%%%%%%%%%%%%%%%%%%%%%%%%%%%%% User specified LaTeX commands.
\usepackage{babel}

\@ifundefined{showcaptionsetup}{}{%
 \PassOptionsToPackage{caption=false}{subfig}}
\usepackage{subfig}
\makeatother

\begin{document}

\title{Computation of the charged current structure functions}

\maketitle

\section{The structure of the observables}

The structure in terms of PDFs of the charged current (CC) structure
functions is complicated by the mixing between down- and up-type
quarks provided by the CKM matrix. As a first step, we write the
$\mathcal{O}(\alpha_s)$ contribution to $F=F_2,F_L$ (we will consider
$F_3$ later) in a convenient way as:
\begin{equation}
F^{\nu} = \sum_{U=u,c,t}\sum_{D=d,s,b}|V_{UD}|^2\left[C_\pm\left(D
    +\overline{U}\right) +2 C_gg\right]
\label{compactNu}
\end{equation}
and:
\begin{equation}
F^{\overline{\nu}} =
\sum_{U=u,c,t}\sum_{D=d,s,b}|V_{UD}|^2\left[C_\pm\left(\overline{D}
    +U\right) +2 C_gg\right] 
\label{compactNub}
\end{equation}
where we have omitted the convolution symbol and an overall factor
$2x$. At this order we don't have to worry about whether $C_+$ or
$C_-$ has to be used because they coincide. However, in the following
it will appear naturally which one has be used and where. One can
combine the expressions above conveniently as:
\begin{equation}
F^{\pm} \equiv \frac{F^{\nu} \pm F^{\overline{\nu}}}2=\frac12
\sum_{U=u,c,t}\sum_{D=d,s,b}|V_{UD}|^2\left[C_\pm\left(D^\pm
    \pm U^\pm\right) + P^{\pm}4C_gg\right]
\end{equation}
where we have used the usual definition $q^{\pm} = q\pm\overline{q}$
and defined the projector:
\begin{equation}
P^\pm=\frac{1\pm1}2\,.
\end{equation}
It should be noted that the subscript $\pm$ to the quark coefficient
function $C_\pm$ because is now associated to each of $F^\pm$.

Now we need to express these observables in terms of PDFs in the
evolution basis. The starting point is the relation:
\begin{equation}
q_i^\pm = \sum_{j=1}^6M_{ij}d^\pm_j\,,
\label{TranformationBella}
\end{equation}
where $d^\pm_j$ belong to the QCD evolution basis, that is:
$d^+_1=\Sigma$, $d^+_2=-T_3$, $d^+_3=T_8$, $d^+_4=T_{15}$, $d^+_5=T_{24}$, and
$d^+_6=T_{35}$ and $d^-_1=V$, $d^-_2=-V_3$, $d^-_3=V_8$, $d^-_4=V_{15}$,
$d^-_5=V_{24}$, and $d^-_6=V_{35}$.  Note that here we are using the more
``natural'' ordering for the distibutions $q_i=\{d,u,s,c,b,t\}$ rather
than that where $u$ comes before $d$; this is the reason of the minus
sign in front of $T_3$ and $V_3$. The trasformation matrix $M_{ij}$
can be written as:
\begin{equation}
\begin{array}{l}
\displaystyle M_{ij}=\theta_{ji}\frac{1-\delta_{ij}j}{j(j-1)}\quad j\geq 2\,,\\
\\
\displaystyle M_{i1} = \frac{1}{6}\,,
\end{array}
\label{TransDef}
\end{equation}
with $\theta_{ji}=1$ for $j\geq i$ and zero otherwise. In addition,
one can show that $M_{ij}$ is such that:
\begin{equation}
\sum_{j=1}^6M_{ij} = 0\,,\quad\mbox{and}\quad \sum_{i=1}^6M_{ij} =
\delta_{1j}\,.
\label{eq:properties}
\end{equation}

Using eq.~(\ref{TranformationBella}) we can make the following
identifications:
\begin{equation}
D^{\pm} = q_{2j-1}^\pm\quad\mbox{and}\quad U^{\pm} =
q_{2j}^\pm\,,\quad j=1,2,3\,,
\end{equation}
so that we can write:
\begin{equation}
F^\pm=
\frac12\sum_{i=1}^3\sum_{j=1}^3|V_{2i,(2j-1)}|^2\left[C_\pm\left(q_{2j-1}^\pm
    \pm q_{2i}^\pm\right) + 4P^{\pm} C_g g\right]\,.
\end{equation}
Using the definition of $M_{ij}$ in eq.~(\ref{TransDef}), we can
rewrite $F^{\pm}$ in terms of PDFs in the evolution basis as:
\begin{equation}
F^\pm=
\sum_{i=1}^3\sum_{j=1}^3|V_{2i,(2j-1)}|^2 F_{ij}^\pm\,,
\label{eq:decompF2L}
\end{equation}
with:
\begin{equation}
F_{ij}^\pm=
C_g 2P^\pm g
+
C_\pm^{\rm S} P^\pm \frac16 d_1^\pm
+ C_\pm\sum_{k=2}^6\frac{\theta_{k,2j-1}(1-\delta_{2j-1,k}k)\pm
  \theta_{k,2i}(1-\delta_{2i,k}k) }{2k(k-1)}d_k^\pm\,.
\label{F2Ldef}
\end{equation}

Eq.~(\ref{F2Ldef}) is valid only for $F_2$ and $F_L$. In order to
obtain a similar equation also for $F_3$, one needs to change sign to
the antiquark distributions, $i.e.$
$\overline{q}_i\rightarrow - \overline{q}_i$. In the QCD evolution
basis, this has the consequence of exchanging the $T$-like
distributions with the $V$-like ones, that is to say
$d_k^+\leftrightarrow d_k^-$. It is the easy to see that:
\begin{equation}
F_3^\pm=
\sum_{i=1}^3\sum_{j=1}^3|V_{2i,(2j-1)}|^2 F_{3,ij}^\pm\,,
\end{equation}
with:
\begin{equation}
F_{3,ij}^\pm=
C_g 2P^\pm g
+
C_\pm^{\rm S} P^\mp \frac16 d_1^\pm
+ C_\pm\sum_{k=2}^6\frac{\theta_{k,2j-1}(1-\delta_{2j-1,k}k)\mp
  \theta_{k,2i}(1-\delta_{2i,k}k) }{2k(k-1)}d_k^\pm\,.
\label{F3def}
\end{equation}

It is now useful to consider the inclusive structure functions and
exploit the unitarity of the CKM matrix elements $V_{UD}$:
\begin{equation}
\sum_{i=1}^3|V_{2i,(2j-1)}|^2 = \sum_{j=1}^3|V_{2i,(2j-1)}|^2 = 1\quad\Rightarrow\quad \sum_{i=1}^3\sum_{j=1}^3|V_{2i,(2j-1)}|^2 = 3\,.
\end{equation}
Summing over $i$ and $j$ in eq.~(\ref{eq:decompF2L}) and using
eq.~(\ref{F2Ldef}), one obtains:
\begin{equation}
F^\pm=
C_g 6P^\pm g
+
C_\pm^{\rm S} P^\pm \frac12 d_1^\pm
+ \frac12 C_\pm\sum_{k=2}^6 d_k^\pm\sum_{l=1}^6(\pm 1)^{l+1}M_{lk}\,.
\end{equation}
Considering separately $F^+$ and $F^-$ and using
eq.~(\ref{eq:properties}), one finds:
\begin{equation}
F^+= C_g 6g + C_+^{\rm S} \frac12 d_1^+
\end{equation}
and:
\begin{equation}
F^-= \frac12 C_-\sum_{k=2}^6\left[\frac{P^+}{k-1}-\frac{P^-}{k}\right]d_k^-\,,
\end{equation}
with the even/odd projectors defined as:
\begin{equation}
P_k^{\pm} = \frac{1\pm(-1)^k}{2}\,.
\end{equation}

It should be pointed out that such simple expressions (independent of
the CMK matrix elements) is achievable only if it is possible to
factorize the non-singlet coefficient functions as implicitly done in
eqs.~(\ref{F2Ldef}) and~(\ref{F3def}). In fact, this is possible only
in the ZM case in which the coefficient functions of each PDF
combination is the same.

For $F_3$ we find:
\begin{equation}
F_3^+= C_g 6g + C_-^{\rm S} \frac12 d_1^-
\end{equation}
and:
\begin{equation}
F_3^-= \frac12 C_+\sum_{k=2}^6\left[\frac{P_k^+}{k-1}-\frac{P_k^-}{k}\right]d_k^+\,.
\end{equation}

\end{document}
