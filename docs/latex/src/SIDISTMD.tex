\documentclass[10pt,a4paper]{article}
\usepackage{amsmath,amssymb,bm,makeidx,subfigure}
\usepackage[italian,english]{babel}
\usepackage[center,small]{caption}[2007/01/07]
\usepackage{fancyhdr}
\usepackage{color}
\usepackage{graphicx}

\definecolor{blu}{rgb}{0,0,1}
\definecolor{verde}{rgb}{0,1,0}
\definecolor{rosso}{rgb}{1,0,0}
\definecolor{viola}{rgb}{1,0,1}
\definecolor{arancio}{rgb}{1,0.5,0}
\definecolor{celeste}{rgb}{0,1,1}
\definecolor{rosa}{rgb}{1,0.3,0.5}

\oddsidemargin = 12pt
\topmargin = 0pt
\textwidth = 440pt
\textheight = 650pt

\makeindex

\begin{document}

\title{SIDIS cross section in TMD factorisation}

\author{Valerio Bertone}

%\institution{$^{a}$PH Department, TH Unit, CERN, CH-1211 Geneva 23, Switzerland}
\maketitle

%\begin{abstract}
%In this document 
%\end{abstract}
\tableofcontents{}

\section{Structure of the observable}

In this document we report the relevant formulas for the computation
of semi-inclusive deep-inelastic scattering (SIDIS) multiplicities
under the assumption that the (negative) virtuality of the $Q^2$ of
the exchanged vector boson is much smaller than the $Z$ mass. This
allows us to neglect weak contributions and write the cross section
in TMD factorisation as:
\begin{equation}\label{eq:sidisxsec}
  \frac{d\sigma}{dxdQdz d q_T} = \frac{4\pi \alpha^2q_T}{z x Q^3}Y_+ H(Q,\mu) \sum_q e_q^2
  \int_0^\infty db \,b J_0\left(bq_T\right)\overline{F}_q(x,b;\mu,\zeta_1) \overline{D}_{q}(z,b;\mu,\zeta_2)\,,
\end{equation}
with $\zeta_1\zeta_2=Q^4$ and:
\begin{equation}
  Y_+=1+(1-y)^2=1+\left(1-\frac{Q^2}{xs}\right)^2\,,
\end{equation}
where $s$ is the squared center of mass energy. The single TMDs are
evolved and matched onto the respective collinear functions as usual:
\begin{equation}
\overline{F}_i(x,b;\mu,\zeta) =xF_i(x,b;\mu,\zeta) = R_q(\mu_0,\zeta_0\rightarrow \mu,\zeta;b) \sum_{j}\int_x^1dy\,\mathcal{C}_{ij}(y;\mu_0,\zeta_0)\left[\frac{x}{y}f_j\left(\frac{x}{y},\mu_0\right)\right]\,,
\end{equation}
and:
\begin{equation}
\overline{D}_{i}(z,b;\mu,\zeta) =z^3D_{i}(z,b;\mu,\zeta) = R_q(\mu_0,\zeta_0\rightarrow \mu,\zeta;b) \sum_{j}\int_z^1dy\,\left[y^2\mathbb{C}_{ij}(y;\mu_0,\zeta_0)\right]\left[\frac{z}{y}d_j\left(\frac{z}{y},\mu_0\right)\right]\,.
\end{equation}
Notice that here we limit to the case $Q\ll M_Z$ such that we can
neglect the contribution of the $Z$ boson and thus the electroweak
couplings are given by the squared electric charges.

As usual, low-$q_T$ non-perturbative corrections are taken into
account by introducing the monotonic function $b_*(b)$ that behaves
as:
\begin{equation}
  \lim_{b\rightarrow 0}
  b_*(b) = b_{\rm min}\quad\mbox{and}\quad\lim_{b\rightarrow \infty}
  b_*(b) = b_{\rm max}\,.
\end{equation}
This allows us to replace the TMDs in Eq.~(\ref{eq:sidisxsec}) with
their ``regularised'' version:
\begin{equation}\label{eq:fandDNP}
\begin{array}{rcl}
  \overline{F}_i(x,b;\mu,\zeta) &\rightarrow&
  \overline{F}_i(x,b_*(b);\mu,\zeta) f_{\rm NP}(x,b,\zeta)\,,\\
\\
  \overline{D}_i(z,b;\mu,\zeta) &\rightarrow&
  \overline{D}_i(z,b_*(b);\mu,\zeta) D_{\rm NP}(z,b,\zeta)\,,
\end{array}
\end{equation}
where we have introduced the non-perturbative functions $f_{\rm NP}$
and $D_{\rm NP}$. It is important to stress that these functions
further factorise as follows:
\begin{equation}\label{NPfuncts}
\begin{array}{rcl}
f_{\rm NP}(x,b,\zeta)&=&\displaystyle\widetilde{f}_{\rm NP}(x,b)\exp\left[g_K(b)\ln\left(\frac{\zeta}{Q_0^2}\right)\right]\,,\\
\\
D_{\rm NP}(z,b,\zeta) &=&\displaystyle\widetilde{D}_{\rm NP}(x,b)\exp\left[g_K(b)\ln\left(\frac{\zeta}{Q_0^2}\right)\right]\,.
\end{array}
\end{equation}
The common exponential function represents the non-perturbative
corrections to TMD evolution and the specific functional form is
driven by the solution of the Collins-Soper equation where $Q_0$ is
some initial scale. Finally the set of non-perturbative functions to
be determined from fits to data are $\widetilde{f}_{\rm NP}$,
$\widetilde{D}_{\rm NP}$, and $g_K(b)$. It is worth noticing that by definition
\begin{equation}
 f_{\rm NP}(x,b,\zeta) = \frac{\overline{F}_i(x,b;\mu,\zeta)}{\overline{F}_i(x,b_*(b);\mu,\zeta)} \,,\\
\end{equation}
and similarly for $D_{\rm NP}$. Therefore, one has a partial handle on
the $b$-dependence of these functions from the region in which $b$ is
small enough to make both numerator and denominator perturbatively
computable. Making use of Eq.~(\ref{NPfuncts}) and setting
$\zeta_1=\zeta_2=Q^2$ allows us to rewrite Eq.~(\ref{eq:sidisxsec})
as:
\begin{equation}\label{eq:sidisxsec3}
\begin{array}{rcl}
\displaystyle   \frac{d\sigma}{dxdQdz d q_{T}} &=&\displaystyle
\frac{4\pi \alpha^2q_{T}}{xzQ^3}Y_+ H(Q,\mu) \sum_q e_q^2\\
\\
&\times&\displaystyle 
  \int_0^\infty db\, J_0\left(bq_T\right) \,b\overline{F}_i(x,b_*(b);\mu,Q^2) \overline{D}_i(z,b_*(b);\mu,Q^2) f_{\rm NP}(x,b,Q^2) D_{\rm NP}(z,b,Q^2)\,.
\end{array}
\end{equation}
The integral in the r.h.s. can be numerically computed using the Ogata
quadrature of zero-th degree (because $J_0$ enters the integral):
\begin{equation}
\begin{array}{rcl}
\displaystyle   \frac{d\sigma}{dxdQdz d q_{T}} &\simeq&\displaystyle
\frac{4\pi \alpha^2}{xzQ^3}Y_+ H(Q,\mu) \sum_q e_q^2\\
\\
&\times&\displaystyle
  \sum_{n=1}^{N}w_n^{(0)}\,\frac{\xi_n^{(0)}}{q_{T}}\overline{F}_i\left(x,b_*\left(\frac{\xi_n^{(0)}}{q_{T}}\right);\mu,Q^2\right)
         \overline{D}_i\left(z,b_*\left(\frac{\xi_n^{(0)}}{q_{T}}\right);\mu,Q^2\right)\\
\\
&\times& \displaystyle f_{\rm NP}\left(x, \frac{\xi_n^{(0)}}{q_{T}},Q^2\right) D_{\rm NP}\left(z, \frac{\xi_n^{(0)}}{q_{T}},Q^2\right)\,,
\end{array}
\end{equation}
where $w_n^{(0)}$ and $\xi_n^{(0)}$ are the Ogata weights and
coordinates, respectively, and the sum over $n$ is truncated to the
$N$-th term that should be chosen in such a way to guarantee a given
target accuracy. The equation above can be conveniently recasted as
follows:
\begin{equation}
\displaystyle \frac{d\sigma}{dxdQdz d q_{T}} \simeq
  \sum_{n=1}^{N}w_n^{(0)}\,\frac{\xi_n^{(0)}}{q_{T}}S\left(x,z,\frac{\xi_n^{(0)}}{q_{T}};\mu,Q^2\right) f_{\rm NP}\left(x, \frac{\xi_n^{(0)}}{q_{T}},Q^2\right) D_{\rm NP}\left(z, \frac{\xi_n^{(0)}}{q_{T}},Q^2\right)\,,
\end{equation}
where:
\begin{equation}\label{eq:Sall}
\displaystyle S\left(x,z,b;\mu,Q^2\right)\simeq
\frac{4\pi \alpha^2}{xzQ^3}Y_+ H(Q,\mu) \sum_q e_q^2 \left[\overline{F}_i\left(x,b_*(b);\mu,Q^2\right)\right]\left[\overline{D}_i\left(z,b_*(b);\mu,Q^2\right)\right]\,.
\end{equation}

\section{Integrating over the final-state kinematic variables}

Experimental measurements of differential distributions for SIDIS
production are often delivered as integrated over finite regions of
the final-state kinematic phase space. 

More specifically, the cross section is not integrated of the
transverse momentum of the vector boson, $q_T$, but over the
transverse momentum of the outgoing hadron, $p_{Th}$, that is
connected to the former through:
\begin{equation}
p_{Th} = zq_T\,.
\end{equation}
The integrated cross section then reads:
\begin{equation}\label{eq:IntcrosssectionSIDIS}
  \widetilde{\sigma}=\int_{Q_{\rm min}}^{Q_{\rm max}}dQ \int_{x_{\rm min}}^{x_{\rm max}}dx
  \int_{z_{\rm
      min}}^{z_{\rm max}}dz \int_{p_{Th,\rm min}/z}^{p_{Th,\rm max}/z}dq_{T}\left[\frac{d\sigma}{dxdQ
      dz dq_{T}} \right]\,.
\end{equation}
One can exploit a property of the Bessel functions to compute the
indefinite integral in $q_{T}$ of the cross section in
Eq.~(\ref{eq:approxint}). Specifically, we now compute:
\begin{equation}
  K(x,z,Q,q_{T}) = \int dq_{T}\left[\frac{d\sigma}{dxdQdz dq_{T}}\right]\,.
\end{equation}
This is easily done by using the following property of the Bessel
functions:
\begin{equation}
\int dx\,x J_0(x) = xJ_1(x)\,,
\end{equation}
that is equivalent to:
\begin{equation}
  \int dq_{T}\,q_{T} J_0\left(bq_T\right) = \frac{q_T}{b}J_1\left(bq_T\right)\,.
\end{equation}
Therefore:
\begin{equation}
\begin{array}{rcl}
\displaystyle   K(x,z,Q,q_{T}) &=&\displaystyle
\frac{4\pi \alpha^2q_{T}}{xzQ^3}Y_+ H(Q,\mu) \sum_q e_q^2\\
\\
&\times&\displaystyle 
  \int_0^\infty db\, J_1\left(bq_T\right)
         \,\overline{F}_i(x,b_*(b);\mu,Q^2)
         \overline{D}_i(z,b_*(b);\mu,Q^2) f_{\rm NP}(x,b,Q^2) D_{\rm NP}(z,b,Q^2)\,.
\end{array}
\end{equation}
The integral can again be computed using the Ogata quadrature as:
\begin{equation}\label{eq:primitivepTh}
\displaystyle   K(x,z,Q,q_{T}) \simeq \displaystyle
  \sum_{n=1}^N w_n^{(1)}
         \,S\left(x,z,\frac{\xi_n^{(1)}}{q_{T}};\mu,Q^2\right) f_{\rm NP}\left(x, \frac{\xi_n^{(1)}}{q_{T}},Q^2\right) D_{\rm NP}\left(z, \frac{\xi_n^{(1)}}{q_{T}},Q^2\right)\,,
\end{equation}
with $S$ given in Eq.~(\ref{eq:Sall}). Once $K$ is known, the integral
of the cross section over the bin
$q_{T}\in [p_{Th,\rm min}/z: p_{Th,\rm max}/z]$ is computed as:
\begin{equation}
  \int_{p_{Th,\rm min}/z}^{p_{Th,\rm max}/z} dq_{T}\left[\frac{d\sigma}{dxdQdz d
      q_{T}}\right]=K(x,z,Q, p_{Th,\rm max}/z)-K(x,z,Q, p_{Th,\rm min}/z)\,.
\end{equation}
This allows one to compute analytically one of the integrals that are
often required to compare predictions to data. 

\subsection{Integrating over $x$, $z$, and $Q$}

We now move to considering the integral of the cross section over $x$,
$z$, and $Q$. Since these integrals usually come together with an
integration in $q_{T}$, in the following we will consider the
primitive function $K$ in Eq.~(\ref{eq:primitivepTh}) rather than the
cross section itself, that is:
\begin{equation}\label{eq:IntcrosssectionSIDISPrim}
  \widetilde{K}(p_{Th})=\int_{Q_{\rm min}}^{Q_{\rm max}}dQ \int_{z_{\rm
      min}}^{z_{\rm max}}dz \int_{x_{\rm min}}^{x_{\rm max}}dx\,
  K(x,z,Q,p_{Th}/z)\,,
\end{equation}
so that:
\begin{equation}
  \widetilde{\sigma}=\widetilde{K}(p_{Th,\rm max})-\widetilde{K}(p_{Th,\rm min})\,.
\end{equation}

The amount of numerical computation required to carry out the
integration of a single bin is very large. Indicatively, it amounts to
computing a three-dimensional integral for each of the terms of the
Ogata quadrature that usually range from a few tens to
hundreds. Therefore, in order to be able to do the integrations in a
reasonable amount of time and yet obtain accurate results, it is
necessary to put in place an efficient integration strategy. This goal
can be achieved by exploiting a numerical integration based on
interpolation techniques to precompute the relevant quantities. To
this purpose, we first define one grid in $x$, $\{x_\alpha\}$ with
$\alpha=0,\dots,N_x$, one grid in $z$, $\{z_\beta\}$ with
$\beta=0,\dots,N_z$, and one grid in $Q$, $\{Q_\tau\}$ with
$\tau=0,\dots,N_Q$, each of which with a set of interpolating
functions $\mathcal{I}$ associated. The grids should be such to span
the full kinematic range covered by given data set. Then the value of
$K$ in Eq.~(\ref{eq:primitivepTh}) for any kinematics can be obtained
through interpolation as:
\begin{equation}
\begin{array}{rcl}
\displaystyle   K(x,z,Q,q_T) &\simeq& \displaystyle 
  \sum_{n=1}^N w_n^{(1)} S\left(x,z,\frac{\xi_n^{(1)}}{q_T};\mu,Q^2\right)
         \sum_{\alpha=1}^{N_x}\sum_{\beta=1}^{N_z}\sum_{\tau=1}^{N_Q}
                                         
                                         \mathcal{I}_\alpha(x)
                                         \mathcal{I}_\beta(z)
                                         \mathcal{I}_\tau(Q) \\
\\
&\times&\displaystyle f_{\rm NP}\left(x_\alpha, \frac{\xi_n^{(1)}}{ q_{T}},Q_\tau^2\right) D_{\rm NP}\left(z_\beta, \frac{\xi_n^{(1)}}{q_{T}},Q_\tau^2\right)\,.
\end{array}
\end{equation}
Once we have $K$ in this form, the integration over $x$, $z$, and $Q$
in Eq.~(\ref{eq:IntcrosssectionSIDISPrim}) does not involve the
non-perturbative functions $f_{\rm NP}$ and $D_{\rm NP}$ and can be
written as:
\begin{equation}\label{eq:numintDIS}
  \widetilde{K}(p_{Th}) = \sum_{n=1}^N 
                                                 \sum_{\alpha=1}^{N_x}\sum_{\beta=1}^{N_z}\sum_{\tau=1}^{N_Q}
                                                 W_{n\alpha\beta\tau}(p_{Th}) f_{\rm NP}\left(x_\alpha, \frac{z_\beta\xi_n^{(1)}}{p_{Th}},Q_\tau^2\right) D_{\rm NP}\left(z_\beta, \frac{z_\beta\xi_n^{(1)}}{p_{Th}},Q_\tau^2\right)\,,
\end{equation}
with:
\begin{equation}\label{eq:WDIS}
  W_{n\alpha\beta\tau}(p_{Th}) = 
  w_n^{(1)} \int_{Q_{\rm min}}^{Q_{\rm
      max}}dQ\,\mathcal{I}_\tau(Q) \int_{z_{\rm min}}^{z_{\rm max}}dz\,
  \mathcal{I}_\beta(z) \int_{x_{\rm min}}^{x_{\rm
      max}}dx\, \mathcal{I}_\alpha(x) S\left(x,z,\frac{z\xi_n^{(1)}}{p_{Th}};\mu,Q^2\right)\,.
\end{equation}
Since the aim is to fit the functions $f_{\rm NP}$ and $D_{\rm NP}$ to
data, one can precompute and store the coefficients $W$ defined in
Eq.~(\ref{eq:WDIS}) and compute the cross sections in a fast way
making use of Eq.~(\ref{eq:numintDIS}).

It is often the case that the integrated cross section,
Eq.~(\ref{eq:IntcrosssectionSIDIS}), is given within a certain
acceptance region which is typically defined as:
\begin{equation}
W=\sqrt{\frac{(1-x)Q^2}{x}}\geq W_{\rm min}\,,\quad y_{\rm min}\leq y
\left(=\frac{Q^2}{sx}\right) \leq y_{\rm max}\,.
\end{equation}
These constraints can be expressed as constraints on the variable $x$
for a fixed value of $Q$:
\begin{equation}
  x\leq \frac{Q^2}{W_{\rm min}^2+Q^2}\,,\quad x\geq\frac{Q^2}{s y_{\rm
      max}}\,,\quad x\leq 
  \frac{Q^2}{sy_{\rm min}}\,.
\end{equation}
Therefore, in order to implement the acceptance cuts in the
computation of the integrated cross sections, it is enough to replace
the integration bounds of the integral in $x$ in
Eq.~(\ref{eq:IntcrosssectionSIDIS}) as follows:
\begin{equation}
  x_{\rm min}\rightarrow \overline{x}_{\rm min}(Q)=\mbox{max}\left[x_{\rm min},\frac{Q^2}{s y_{\rm
        max}}\right]\,,\quad   x_{\rm max}\rightarrow \overline{x}_{\rm max}(Q)=\mbox{min}\left[x_{\rm max},\frac{Q^2}{s y_{\rm
        min}}, \frac{Q^2}{W_{\rm min}^2+Q^2}\right]\,.
\end{equation}


\end{document}
