%% LyX 2.0.3 created this file.  For more info, see http://www.lyx.org/.
%% Do not edit unless you really know what you are doing.
\documentclass[twoside,english]{paper}
\usepackage{lmodern}
\renewcommand{\ttdefault}{lmodern}
\usepackage[T1]{fontenc}
\usepackage[latin9]{inputenc}
\usepackage[a4paper]{geometry}
\geometry{verbose,tmargin=3cm,bmargin=2.5cm,lmargin=2cm,rmargin=2cm}
\usepackage{color}
\usepackage{babel}
\usepackage{float}
\usepackage{bm}
\usepackage{amsthm}
\usepackage{amsmath}
\usepackage{amssymb}
\usepackage{graphicx}
\usepackage{esint}
\usepackage[unicode=true,pdfusetitle,
 bookmarks=true,bookmarksnumbered=false,bookmarksopen=false,
 breaklinks=false,pdfborder={0 0 0},backref=false,colorlinks=false]
 {hyperref}
\usepackage{breakurl}
\usepackage{mathrsfs}

\makeatletter

%%%%%%%%%%%%%%%%%%%%%%%%%%%%%% LyX specific LaTeX commands.
%% Because html converters don't know tabularnewline
\providecommand{\tabularnewline}{\\}

%%%%%%%%%%%%%%%%%%%%%%%%%%%%%% Textclass specific LaTeX commands.
\numberwithin{equation}{section}
\numberwithin{figure}{section}

%%%%%%%%%%%%%%%%%%%%%%%%%%%%%% User specified LaTeX commands.
\usepackage{babel}

\@ifundefined{showcaptionsetup}{}{%
 \PassOptionsToPackage{caption=false}{subfig}}
\usepackage{subfig}
\makeatother

\begin{document}

\title{Jet TMD}

\maketitle

\section{Definition of the jet TMD}

Following the notes written by Lorenzo and Yannis, the
impact-parameter-space inclusive jet-production cross section in DIS
in TMD factorisation takes the standard form:
\begin{equation}
d\sigma \sim H_{ij}(Q;\mu)D_{i\rightarrow {\rm jet}}(b,Q,t_{\mathcal{R}};\mu,\zeta_1) F_{j\leftarrow P}(x,b,Q;\mu,\zeta_2)\,,
\end{equation}
where $H_{ij}$ is the DIS hard function, $F_{j\leftarrow P}$ is the
usual TMD PDF of the quark flavour $j$ inside the proton, and
$D_{i\rightarrow {\rm jet}}$ is the TMD of the jet generated by the
quark flavour $i$. $x$ and $Q$ are the usual DIS variable
corresponding to the Bjorken variable and the negative virtuality of
the vector boson while $b$ is the Fourier conjugate variable of the
partonic transverse momentum $k_T$. The scale $\mu$ is the resummation
scale and must be $\mu=C_f Q$, with $C_f$ order one, while $\zeta_1$
and $\zeta_2$ are the rapidity scale that in this case must obey the
momentum-space equality $\zeta_1\zeta_2=Q^2t_{\mathcal{R}}k_T^2 $,
with $k_T$ being the partonic transverse momentum. In impact parameter
space this equality naturally turns into
$\zeta_1\zeta_2=Q^2 t_{\mathcal{R}}^2b_0^2/b^2$ with
$b_0=2e^{-\gamma_{\rm E}}$. Without loss of generality one can choose
$\zeta_2= Q^2$ such that $\zeta_1=t_{\mathcal{R}}^2 b_0^2
/b^2$. Finally, the variable $t_{\mathcal{R}}=\tan(\mathcal{R}/2)$
depends on the jet opening $\mathcal{R}$ and for TMD factorisation to
be valid one needs $\mathcal{R}\sim 1$.

Using the definition devised by Yannis and Lorenzo, both jet TMD and
TMD PDFs evolve multiplicatively through the standard Sudakov form
factor $R$ as:
\begin{equation}
  G(\mu,\zeta) =
  R[(\mu,\zeta)\leftarrow(\mu_0,\zeta_0)]G(\mu_0,\zeta_0)\quad G =
  D_{i\rightarrow {\rm jet}}, F_{j\leftarrow P}\,,
\end{equation}
where we simplified the notation by dropping the unnecessary
variables. In the following we take:
\begin{equation}
\mu_0=\sqrt{\zeta_0}=C_i\mu_b\,,\quad\mbox{with}\quad
\mu_b=\frac{b_0}{b}\,,
\end{equation}
where $C_i$ is a constant of order one. As is well know,
$F_{j\leftarrow P}(\mu_0,\zeta_0)$ can be matched onto collinear PDFs
for small values of $b$ while a non-perturbative component needs to be
accounted for larger values of $b$. This is the standard procedure and
will not be discussed any further here. Let us now turn to the
initial-scale jet TMD that can be further factorised as:
\begin{equation}
D_{i\rightarrow {\rm jet}} (\mu_0,\zeta_0) = D_{i\rightarrow {\rm jet}}
(\mu_0,\zeta_0;\mu_0) = U_J[\mu_0\leftarrow \mu_J]D_{i\rightarrow {\rm jet}}
(\mu_0,\zeta_0;\mu_J)\,,
\end{equation}
with $\mu_J=C_JQt_{\mathcal{R}}$, $C_J\sim 1$. The evolution factor $U_J$ takes the
following explicit form:
\begin{equation}
U_J[\mu_0\leftarrow \mu_J]=\exp\left[-\int_{\mu_0}^{\mu_J}\frac{d\mu'}{\mu'}\gamma_J(\mu')\right]\,.
\end{equation}
The anomalous dimension valid up to NLL reads:
\begin{equation}
\gamma_J(\mu') = \left(\frac{\alpha_s(\mu')}{4\pi}\right) \gamma_F^{(0)}-\left[\left(\frac{\alpha_s(\mu')}{4\pi}\right)\gamma_K^{(0)}+\left(\frac{\alpha_s(\mu')}{4\pi}\right)^2\gamma_K^{(1)}\right]\ln\frac{\mu_J}{\mu'}\,,
\end{equation}
where $\gamma_F^{(i)}$ and $\gamma_K^{(i)}$ are the coefficients of
the perturbative expansion of the non-cusp and cusp anomalous
dimensions, respectively. Finally, the initial-scale jet TMD reads:
\begin{equation}
D_{i\rightarrow {\rm jet}}
(\mu_0,\zeta_0;\mu_J) = 1 +
\frac{\alpha_s(\mu_J)}{4\pi}\left[\frac12\gamma_K^{(0)}\ln^2C_J+\gamma_F^{(0)}\ln
  C_J+d_J^{q,\rm alg}\right]\,,
\end{equation}
where the coefficient $d_J^{q,\rm alg}$ depends on the jet
algorithm. For cone a $k_T$ algorithms respectively reads:
\begin{equation}
d_J^{q,\rm cone} = 7 + 6\ln 2-\frac{5\pi^2}{6}\,,\quad\mbox{and}\quad
d_J^{q,k_T} = 13 - \frac{3\pi^2}{2}\,.
\end{equation}

\end{document}
