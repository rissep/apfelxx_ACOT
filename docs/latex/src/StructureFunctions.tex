%% LyX 2.0.3 created this file.  For more info, see http://www.lyx.org/.
%% Do not edit unless you really know what you are doing.
\documentclass[twoside,english]{paper}
\usepackage{lmodern}
\renewcommand{\ttdefault}{lmodern}
\usepackage[T1]{fontenc}
\usepackage[latin9]{inputenc}
\usepackage[a4paper]{geometry}
\geometry{verbose,tmargin=3cm,bmargin=2.5cm,lmargin=2cm,rmargin=2cm}
\usepackage{color}
\usepackage{babel}
\usepackage{float}
\usepackage{bm}
\usepackage{amsthm}
\usepackage{amsmath}
\usepackage{amssymb}
\usepackage{graphicx}
\usepackage{esint}
\usepackage[unicode=true,pdfusetitle,
 bookmarks=true,bookmarksnumbered=false,bookmarksopen=false,
 breaklinks=false,pdfborder={0 0 0},backref=false,colorlinks=false]
 {hyperref}
\usepackage{breakurl}
\usepackage{makeidx}

\makeatletter

%%%%%%%%%%%%%%%%%%%%%%%%%%%%%% LyX specific LaTeX commands.
%% Because html converters don't know tabularnewline
\providecommand{\tabularnewline}{\\}

%%%%%%%%%%%%%%%%%%%%%%%%%%%%%% Textclass specific LaTeX commands.
\numberwithin{equation}{section}
\numberwithin{figure}{section}

%%%%%%%%%%%%%%%%%%%%%%%%%%%%%% User specified LaTeX commands.
\usepackage{babel}

\@ifundefined{showcaptionsetup}{}{%
 \PassOptionsToPackage{caption=false}{subfig}}
\usepackage{subfig}
\makeatother

\usepackage{listings}

\begin{document}

\title{Structure functions}

\maketitle

\tableofcontents{}

\vspace{20pt}

In this part the technology developed to numerically compute Mellin
convolutions exploiting interpolation functions will be applied to the
computation of totally inclusive structure functions. On top of the
numerical aspects of the computation, more ``theoretical'' aspects
will be touched such as the computation of scale variations and of the
target-mass corrections.

\section{Zero-mass structure functions}

A generic structure function in the zero-mass (ZM) scheme, in which
the finitness of the quark mass is neglected, computed at the value
$x$ of the Bjorken variable and at the (absolute) value $Q$ of the
vector-boson virtuality is given by the following Mellin convolution:
\begin{equation}
  F(x,Q) = \sum_{i=g,q}x\int_x^1\frac{dy}y
  C_i\left(y,\alpha_s(Q)\right)q_i\left(\frac{x}{y},Q\right)=\sum_{i=g,q}\int_x^1dy\,
  C_i\left(y,\alpha_s(Q)\right)\frac{x}{y}q_i\left(\frac{x}{y},Q\right)\,.
\label{eq:structfunc}
\end{equation}
The functions $C_i$, often dubbed coefficient functions, are
perturbatively computable and thus, to $N$-th order, admit the
expansion:
\begin{equation}
C_i\left(y,\alpha_s(Q)\right) = \sum_{n=0}^{N}\left(\frac{\alpha_s(Q)}{4\pi}\right)^nC_i^{(n)}(x)\,.
\end{equation}
The integral in Eq.~(\ref{eq:structfunc}) can be computed using the
numerical techniques developed in the part of this documentation
devoted to the structure of the Mellin-convolution
integrals. Specifically, introducing a grid $\{x_\alpha\}$ having a
set of interpolating functions of degree $k$, $\{w_\alpha^{(k)}(y)\}$, 
associated, the structure function on the grid point is computed as:
\begin{equation}
  F(x_\beta,Q) =
  \sum_{i=g,q}\sum^{N_{x}}_{\alpha=0}\sum_{n=0}^{N}\left(\frac{\alpha_s(Q)}{4\pi}\right)^n\underbrace{\int_{{\rm
          max}(x_\beta,x_\beta/x_{\alpha+1})}^{{\rm
          min}(1,x_\beta/x_{\alpha-k})}dy\,
      C_i^{(n)}\left(y\right)
      w_{\alpha}^{(k)}\left(\frac{x_\beta}{y}\right)}_{\Gamma_{i,\beta\alpha}^{(n)}}\overline{q}_{\alpha}\,.
\label{eq:stffast}
\end{equation}
with $\overline{q}_\alpha=x_\alpha q(x_\alpha,Q)$. The value of the
structure function for any generic value of $x$ can then be obtained
through interpolation. The integrals $\Gamma_{i,\beta\alpha}^{(n)}$
are efficiently computed according to the procedure discussed in the
part on the integration. Notice, that this integrals are totally scale
independent in that the $Q$ dependence of the structure functions is
totally driven by $\alpha_s$ and the PDFs $q$. This allows for a
pre-computation of these integrals that can be used to compute
structure functions at any scale $Q$.

Before moving to considering the massive structure functions, it is
useful to point out that in the ZM scheme there is no need to
distinguish between charged- and neutral-current coefficient
functions. In other words, when considering the exchange of a neutral
vector boson $Z/\gamma^*$ or of a charged vector boson $W^\pm$, the
difference in functional form of the coefficient functions $C_i$ only
amounts to an overall factor associated to the electroweak charge.

% It is opportune at this point to mention that, when considering
% charged-current observables at NLO in the massive scheme, there is a
% further contribution to be added to Eq.~(\ref{CFstructure}) that has
% the form:
% \begin{equation} C_{i}^{SL}(t)\frac{d}{dx}\delta(1-x)\,.
% \end{equation} Starting from the relation:
% \begin{equation} x\frac{d}{dx}\delta(x) = -\delta(x)\,,
% \end{equation} one can easily show that:
% \begin{equation}
% \left[\frac{\delta(1-x)}{1-x}\right]_+\,.
%\label{KeyIdentity} \frac{d}{dx}\delta(1-x) =
% \end{equation} To make sure that this identity is correct, we try to
% convolute both the r.h.s. and the l.h.s. of Eq.~(\ref{KeyIdentity})
% with the test function $f(x)$, such that $f(1) = 0$, to see what is
% the result and whether the results are equal. Using the l.h.s. we
% have:
% \begin{equation} \int_x^1dy\,f(y)\,\frac{d}{dx}\delta(1-y) =
% \underbrace{f(y)\delta(1-y)\Big{|}_{x}^1}_{=0} -
% \int_x^1dy\frac{df(y)}{dy}\delta(1-y) =
% -\frac{df(y)}{dy}\bigg{|}_{y=1}\,,
% \end{equation} while using the r.h.s.(\footnote{Since the delta
% function selects the point $y=1$ in the following integral, the
% ``incomplete'' integral of plus-prescribed function does not give
% rise to any residual logarithm of the form $\ln(1-x)$.}):
% \begin{equation}
% \begin{array}{c}
%   \displaystyle
%   \int_x^1dy\,f(y)\,\left[\frac{\delta(1-y)}{1-y}\right]_+ =
%   \int_x^1dy\frac{f(y)-f(1)}{1-y}\delta(1-y) = \lim_{\epsilon\rightarrow
%   0^+} \int_x^1dy\frac{f(y)-f(1)}{1-y}\delta(1-\epsilon-y) =\\ \\
%   \displaystyle - \lim_{\epsilon\rightarrow 0^+}
%   \frac{f(1)-f(1-\epsilon)}{\epsilon} =
%   -\frac{df(y)}{dy}\bigg{|}_{y=1}\,.
% \end{array}
% \end{equation} 
% So the results are equal and the distributions in
% Eq.~(\ref{KeyIdentity}) when convoluted with a test function extract
% its derivative in $y=1$, up to a minus sign.

\section{Massive structure functions}

When computing structure functions retaining the mass of the quarks,
the convolution integrals become more complicated. In addition,
contrary to the massless case, the single precomputed intergrals
$\Gamma_{i,\beta\alpha}^{(n)}$, introduced in Eq.~(\ref{eq:stffast}),
will depend on the scale $Q$. This prevents a scale independent
pre-computation of the coefficient functions on the $x$-space grid.  A
possible solution to this problem relies on
interpolation. Specifically, the integrals
$\Gamma_{i,\beta\alpha}^{(n)}$ are tabulated over a grid in the
variable:
\begin{equation}
  \xi = \frac{Q^2}{m_H^2}\,,
\label{eq:scalingvar}
\end{equation}
where $m_H$ is the mass of the heavy quark under consideration, and
subsequently interpolated to obtain the structure function for any
value of $Q$.

\subsection{Neutral Current Coefficient Functions}

As far as the neutral-current coefficient functions are concerned,
explicit expressions beyond $\mathcal{O}(\alpha_s)$ are complicated
and thus not suitable for fast numerical computations. Fortunately,
the authors of Ref.~\cite{Riemersma:1994hv} have tabulated these
functions and published the corresponding interpolating routines. It
is these routines that {\tt APFEL++} uses for the
$\mathcal{O}(\alpha_s^2)$ corrections to the massive neutral current
coefficient functions. The only exception is the pure-singlet
$\mathcal{O}(\alpha_s^2)$ coefficient functions (sometimes called
gluon-radiation terms) in which case the analytical expressions given
in Appendix A of Ref.~\cite{Buza:1995ie} are used.  At
$\mathcal{O}(\alpha_s)$ the analytic expressions are instead
used.\footnote{Actually, if neglecting intrinsic heavy-quark
  components, at $\mathcal{O}(\alpha_s)$ there is one single
  coefficient function for each structure function, available for
  example in available in Ref.~\cite{Forte:2010ta}, that is convoluted
  with the gluon distribution.}

In the implementation of a so-called General-Mass (GM) scheme, it is
usually required to know the massless limit of the massive coefficient
functions analytically. To be more precise, this limit prescribes to
set to zero all mass-suppressed contributions and retain only the
mass-independent and the logarithmically enhanced contributions. In
this case, exact expressions up to $\mathcal{O}(\alpha_s^2)$ have been
computed in Ref.~\cite{Buza:1995ie} and reported in Appendix D of this
paper. These expressions are implemented in {\tt APFEL++}.

As a final remark, the massive coefficient functions for the neutral
current structure functions are presently known only for $F_2$ and
$F_L$. For the parity-violating structure function $F_3$ the massless
coefficient functions are instead used. This is usually acceptable
because the neutral-current $F_3$ in the typical kinematics covered by
modern experiments is sizeable at large $Q$, \textit{i.e.} where mass
effects are negligible.

\subsection{Charged Current Coefficient Functions}

We can now consider the charged-current sector in which massive
coefficient functions are know up to
$\mathcal{O}(\alpha_s^2)$. However, the $\mathcal{O}(\alpha_s^2)$
corrections, recently computed in Ref.~\cite{Gao:2017kkx}, are not
publicly available. Therefore a computation of charged-current
structure functions at NNLO in {\tt APFEL++} is currently
impossible.

The structure functions associated to the heavy quark $H$ in the
approximation of diagonal CKM matrix\footnote{This approximation will
  be released later.} are given in terms of the following
convolutions:
\begin{equation}
  F_1^H(x,Q,m_H)=\frac12\int_{\chi}^{1}\frac{dy}{y}\left[C_{1,q}(y,Q)s\left(\frac{\chi}{y},Q\right)+C_{1,g}(y,Q)g\left(\frac{\chi}{y},Q\right)\right]
\label{F1}
\end{equation}
\begin{equation}
  F_2^H(x,Q,m_H)=\chi\int_{\chi}^{1}\frac{dy}{y}\left[C_{2,q}(y,Q)s\left(\frac{\chi}{y},Q\right)+C_{2,g}(y,Q)g\left(\frac{\chi}{y},Q\right)\right]
\label{F2}
\end{equation}
\begin{equation}
  F_3^H(x,Q,m_H)=\int_{\chi}^{1}\frac{dy}{y}\left[C_{3,q}(y,Q)s\left(\frac{\chi}{y},Q\right)+C_{3,g}(y,Q)g\left(\frac{\chi}{y},Q\right)\right]
\label{F3}
\end{equation}
with:
\begin{equation}
  \chi = x\left(1+\frac{m_H^2}{Q^2}\right) =
  \frac{x}{\lambda}\,,
\end{equation}
where:
\begin{equation}
  \lambda = \frac{Q^2}{Q^2+m_H^2} =
  \frac{\xi}{1+\xi}\,,
\end{equation}
with $\xi$ given in Eq.~(\ref{eq:scalingvar}). Defining:
\begin{equation}
  F_L^H(x,Q,m_H) = F_2^H(x,Q,m_H) - 2xF_1^H(x,Q,m_H)\,,
\end{equation}
one has that:
\begin{equation}
  F_L^H(x,Q,m_H)=\chi\int_{\chi}^{1}\frac{dy}{y}\left[C_{L,q}(y,Q)s\left(\frac{\chi}{y},Q\right)+C_{L,g}(y,Q)g\left(\frac{\chi}{y},Q\right)\right]\,,
\label{FL}
\end{equation}
with:
\begin{equation}
  C_{L,q(g)}(y,Q) =
  C_{2,q(g)}(y,Q)-\lambda C_{1,q(g)}(y,Q)
\label{clll}
\end{equation}

As usual, the coefficient functions admit a perturbative expansion
that at N$^N$LO reads:
\begin{equation}
  C_{k,q(g)}(y,Q) = \sum_{n=0}^N \left(\frac{\alpha_s(Q)}{4\pi}\right)^n
  C_{k,q(g)}^{(n)}(y,\xi)\,,\quad k = 1,2,3,L\,.
\end{equation}
In the following we will truncate the expansion at $N=1$,
\textit{i.e.} at NLO.

At LO the coefficient functions read:
\begin{equation}
\begin{array}{l}
  \displaystyle C^{(0)}_{1,q}(x,\xi) = \delta(1-x)\,,\\
  \\ \displaystyle C^{(0)}_{2,q}(x,\xi) = \delta(1-x) \,,\\ \\
  \displaystyle C^{(0)}_{3,q}(x,\xi) = \delta(1-x) \,,\\ \\
  \displaystyle C^{(0)}_{L,q}(x,\xi) = (1-\lambda)\delta(1-x) \,,\\ \\
  \displaystyle  C_{k,g}^{(0)}(y,\xi)=0\,, \quad k=1,2,3,L\,.
\end{array}
\end{equation}

The $\mathcal{O}(\alpha_s)$ (NLO) charged-current massive coefficient
have been computed and reported in Appendix A of
Ref.~\cite{Gluck:1996ve}. However, their implementation in
{\tt APFEL++} requires some manipulations. We start by defining:
\begin{equation}
  K_A=\frac{1}{\lambda}(1-\lambda)\ln(1-\lambda)\quad \mbox{and}\quad K_F =\frac{Q}{\mu_F}\,.
\end{equation}

The explicit expressions of the NLO quark coefficient functions then
read:
\begin{equation}
\begin{array}{rcl}
  C^{(1)}_{1,q}&=&\displaystyle 2C_F \bigg\{
                   \bigg(-4-\frac{1}{2\lambda}-2\zeta_2-\frac{1+3\lambda}{2\lambda}K_A+\frac32
                   \ln\frac{K_F^2}{\lambda}\bigg)\delta(1-z)\\ \\ &-&\displaystyle
                                                                      \frac{(1+z^2)\ln z}{1-z} +
                                                                      \left(-\ln\frac{K_F^2}{\lambda}-2\ln(1-z)+\ln(1-\lambda
                                                                      z)\right)(1+z)+(3-z)+\frac{1}{\lambda^2}+\frac{z-1}{\lambda}\\ \\
               &+&\displaystyle 2 \left[\frac{2\ln(1-z)-\ln(1-\lambda
                   z)}{1-z}\right]_++
                   2\left(-1+\ln\frac{K_F^2}{\lambda}\right)\left[\frac{1}{1-z}\right]_+\\
  \\ &+& \displaystyle
         \frac{\lambda-1}{\lambda^2}\left[\frac{1}{1-\lambda z}\right]_+
         +\frac{1}{2}\left[\frac{1-z}{(1-\lambda z)^2}\right]_+\bigg\}\,,
\end{array}
\end{equation}

\begin{equation}
\begin{array}{rcl}
  C^{(1)}_{2,q}&=&\displaystyle 2C_F \bigg\{
                   \bigg(-4-\frac{1}{2\lambda}-2\zeta_2-\frac{1+\lambda}{2\lambda}K_A+\frac32
                   \ln\frac{K_F^2}{\lambda}\bigg)\delta(1-z)\\ \\ &-&\displaystyle
                                                                      \frac{(1+z^2)\ln z}{1-z} +
                                                                      \left(2-\ln\frac{K_F^2}{\lambda}-2\ln(1-z)+\ln(1-\lambda
                                                                      z)\right)(1+z)+\frac{1}{\lambda}\\ \\ &+&\displaystyle 2
                                                                                                                \left[\frac{2\ln(1-z)-\ln(1-\lambda z)}{1-z}\right]_++
                                                                                                                2\left(-1+\ln\frac{K_F^2}{\lambda}\right)\left[\frac{1}{1-z}\right]_+\\
  \\ &+& \displaystyle
         \frac{2\lambda^2-\lambda-1}{\lambda}\left[\frac{1}{1-\lambda
         z}\right]_+ +\frac{1}{2}\left[\frac{1-z}{(1-\lambda
         z)^2}\right]_+\bigg\}\,,
\end{array}
\end{equation}

\begin{equation}
\begin{array}{rcl}
  C^{(1)}_{3,q}&=&\displaystyle 2C_F \bigg\{
                   \bigg(-4-\frac{1}{2\lambda}-2\zeta_2-\frac{1+3\lambda}{2\lambda}K_A+\frac32
                   \ln\frac{K_F^2}{\lambda}\bigg)\delta(1-z)\\ \\ &-&\displaystyle
                                                                      \frac{(1+z^2)\ln z}{1-z} +
                                                                      \left(1-\ln\frac{K_F^2}{\lambda}-2\ln(1-z)+\ln(1-\lambda
                                                                      z)\right)(1+z)+\frac{1}{\lambda}\\ \\ &+&\displaystyle 2
                                                                                                                \left[\frac{2\ln(1-z)-\ln(1-\lambda z)}{1-z}\right]_++
                                                                                                                2\left(-1+\ln\frac{K_F^2}{\lambda}\right)\left[\frac{1}{1-z}\right]_+\\
  \\ &+& \displaystyle \frac{\lambda-1}{\lambda}\left[\frac{1}{1-\lambda
         z}\right]_+ +\frac{1}{2}\left[\frac{1-z}{(1-\lambda
         z)^2}\right]_+\bigg\}\,,
\end{array}
\end{equation}

\begin{equation}
\begin{array}{rcl}
  C^{(1)}_{L,q} &=&\displaystyle 2C_F
                    (1-\lambda)\bigg\{
                    \bigg(-4-\frac{1}{2\lambda}-2\zeta_2-\frac{1+\lambda}{2\lambda}K_A+\frac32
                    \ln\frac{K_F^2}{\lambda}\bigg)\delta(1-z)\\ \\ &-&\displaystyle
                                                                       \frac{(1+z^2)\ln z}{1-z} +
                                                                       \left(-\ln\frac{K_F^2}{\lambda}-2\ln(1-z)+\ln(1-\lambda
                                                                       z)\right)(1+z)+3\\ \\ &+&\displaystyle 2
                                                                                                 \left[\frac{2\ln(1-z)-\ln(1-\lambda z)}{1-z}\right]_++ 2
                                                                                                 \left(-1+\ln\frac{K_F^2}{\lambda}\right)\left[\frac{1}{1-z}\right]_+\\
  \\ &-& \displaystyle 2\left[\frac{1}{1-\lambda z}\right]_+
         +\frac{1}{2}\left[\frac{1-z}{(1-\lambda z)^2}\right]_+\bigg\} + 2C_F
         \left[\lambda K_A \delta(1-z) + (1+\lambda)z\right]\,.
\end{array}
\end{equation}
In order to proceed, it is useful to work out the effect of the
$+$-prescription in the presence of an incomplete integration:
\begin{equation}
  \displaystyle \int_x^1 dz\left[f(z)\right]_+g(z) =
  \int_x^1
  dz\,f(z)\left[g(z)-g(1)\right]-g(1)\underbrace{\int_0^xdz\,f(z)}_{-R_f(x)}=\displaystyle \int_x^1
  dz\left\{\left[f(z)\right]_{x}+R_f(x)\delta(1-z)\right\}g(z)\,.
\end{equation} 
where the $x$-prescription in the r.h.s. of the equation above should
be understood in as a usual $+$-prescription regardless of the
integration bounds.

Despite most of the times the residual function $R_f(x)$ can be
evaluated analytically, sometimes it needs to be evaluated
numerically. The $+$-prescribed functions present in the expressions
above give rise to the following residual functions that can be
computed analytically:
\begin{equation}
  -\int_0^x\frac{dz}{1-z} = \ln(1-x)\,,
\end{equation}
\begin{equation}
  -\int_0^xdz\frac{\ln(1-z)}{1-z} = \frac12
  \ln^2(1-x)\,,
\end{equation}
\begin{equation}
  -\int_0^x\frac{dz}{1-\lambda z} =
  \frac{1}{\lambda}\ln(1-\lambda x)\,,
\end{equation}
\begin{equation}
  -\int_0^xdz\frac{1-z}{(1-\lambda z)^2} =
  \frac{1}{\lambda^2}\ln(1-\lambda
  x)+\frac{1-\lambda}{\lambda}\frac{x}{1-\lambda x}\,,
\end{equation}
\begin{equation}
 -\int_0^xdz\frac{\ln(1-\lambda z)}{1-z}= R(x)=-\text{Li}_2\left(\frac{1}{1-\lambda }\right)+\text{Li}_2\left(\frac{1-x \lambda}{1-\lambda }\right)+\ln \left(\frac{\lambda  (1-x)}{1-\lambda }\right) \ln (1-\lambda  x)\,.
\end{equation}
The advantage of the $x$-prescription is
that, when convoluting the coefficient functions above with PDFs at
the point $x$, one can treat the $+$-prescribed functions using the
standard definition at the price of adding to the local terms the
following functions:
\begin{equation}
\begin{array}{rcl}
  C^{(1)}_{1,q}&\rightarrow& \displaystyle
                             C^{(1)}_{1,q} +
                             2C_F\bigg[2\ln^2(1-x)-2R(x)+2\left(-1+\ln\frac{K_F^2}{\lambda}\right)\ln(1-x)\\
  \\ &&\displaystyle +\frac{\lambda-1}{\lambda^3}\ln(1-\lambda
        x)+\frac{1}{2\lambda^2}\ln(1-\lambda
        x)+\frac{1-\lambda}{2\lambda}\frac{x}{1-\lambda x}\bigg]\delta(1-z) \,,
\end{array}
\end{equation}
\begin{equation}
\begin{array}{rcl}
  C^{(1)}_{2,q}&\rightarrow& \displaystyle
                             C^{(1)}_{2,q} +
                             2C_F\bigg[2\ln^2(1-x)-2R(x)+2\left(-1+\ln\frac{K_F^2}{\lambda}\right)\ln(1-x)\\
  \\ &&\displaystyle
        +\frac{2\lambda^2-\lambda-1}{\lambda^2}\ln(1-\lambda
        x)+\frac{1}{2\lambda^2}\ln(1-\lambda
        x)+\frac{1-\lambda}{2\lambda}\frac{x}{1-\lambda x}\bigg]\delta(1-z)
\end{array}\,,
\end{equation}
\begin{equation}
\begin{array}{rcl}
  C^{(1)}_{3,q}&\rightarrow& \displaystyle
                             C^{(1)}_{3,q} +
                             2C_F\bigg[2\ln^2(1-x)-2R(x)+2\left(-1+\ln\frac{K_F^2}{\lambda}\right)\ln(1-x)\\
  \\ &&\displaystyle +\frac{\lambda-1}{\lambda^2}\ln(1-\lambda
        x)+\frac{1}{2\lambda^2}\ln(1-\lambda
        x)+\frac{1-\lambda}{2\lambda}\frac{x}{1-\lambda x}\bigg]\delta(1-z) \,,
\end{array}
\end{equation}
\begin{equation}
\begin{array}{rcl}
  C^{(1)}_{L,q}&\rightarrow& \displaystyle
                             C^{(1)}_{L,q} +
                             2C_F(1-\lambda)\bigg[2\ln^2(1-x)-2R(x)+2\left(-1+\ln\frac{K_F^2}{\lambda}\right)\ln(1-x)\\
  \\ &&\displaystyle -\frac{2}{\lambda}\ln(1-\lambda
        x)+\frac{1}{2\lambda^2}\ln(1-\lambda
        x)+\frac{1-\lambda}{2\lambda}\frac{x}{1-\lambda x}\bigg]\delta(1-z) \,.
\end{array}
\end{equation}

Now let us consider the gluon coefficient functions. They read:
\begin{equation}
\begin{array}{rcl}
  C^{(1)}_{1,g}&=&\displaystyle
                   2T_R\bigg\{[z^2+(1-z)^2]\left[\ln\left(\frac{1-z}{z}\right)
                   -\frac12\ln(1-\lambda) +\frac12\ln\frac{K_F^2}{\lambda} \right]+\\ \\
               &&\displaystyle 4z(1-z) - 1+\\ \\ &&\displaystyle
                                                    (1-\lambda)\left[-4z(1-z) + \frac{z}{1-\lambda z} +2z(1-2\lambda
                                                    z)\ln\frac{1-\lambda z}{(1-\lambda)z}\right]\bigg\}\,,
\end{array}
\end{equation}
\begin{equation}
\begin{array}{rcl}
  C^{(1)}_{2,g}&=&\displaystyle 2T_R
                   \bigg\{[z^2+(1-z)^2]\left[\ln\left(\frac{1-z}{z}\right)
                   -\frac12\ln(1-\lambda) +\frac12\ln\frac{K_F^2}{\lambda} \right]+\\ \\
               &&\displaystyle 8z(1-z)- 1+\\ \\ &&\displaystyle
                                                   (1-\lambda)\left[-6(1+2\lambda) z(1-z)+\frac{1}{1-\lambda z} +
                                                   6\lambda z(1-2\lambda z)\ln\frac{1-\lambda
                                                   z}{(1-\lambda)z}\right]\bigg\}\,,
\end{array}
\end{equation}
\begin{equation}
\begin{array}{rcl}
  C^{(1)}_{3,g}&=&\displaystyle 2T_R
                   \bigg\{[z^2+(1-z)^2]\left[ 2\ln\left(\frac{1-z}{1-\lambda
                   z}\right)+\frac12\ln(1-\lambda)
                   +\frac12\ln\frac{K_F^2}{\lambda}\right]+\\ \\ &&\displaystyle
                                                                    (1-\lambda)\left[2 z(1-z) - 2z[1-(1+\lambda )z]\ln\frac{1-\lambda
                                                                    z}{(1-\lambda)z}\right]\bigg\}\,,
\end{array}
\end{equation}
\begin{equation}
\begin{array}{rcl}
  C^{(1)}_{L,g}&=&\displaystyle 2T_R
                   \bigg\{(1-\lambda)[z^2+(1-z)^2]\left[\ln\left(\frac{1-z}{z}\right)
                   -\frac12\ln(1-\lambda) +\frac12\ln\frac{K_F^2}{\lambda} \right]+\\ \\
               &&\displaystyle 4(2-\lambda)z(1-z)+\\ \\ &&\displaystyle
                                                           (1-\lambda)\left[-2(3+4\lambda) z(1-z)+ 4\lambda z(1-2\lambda
                                                           z)\ln\frac{1-\lambda z}{(1-\lambda)z}\right]\bigg\}\,.
\end{array}
\end{equation}
Since these functions do not contain any $+$-prescribed functions,
they can be implemented as they are.

We now consider the massless limit of the above massive coefficient
functions. In the limit $m_H\rightarrow 0$, one has:
\begin{equation}
\begin{array}{l} 
  \lambda \rightarrow 1\quad\mbox{and}\quad K_A \rightarrow 0
\end{array}\,.
\end{equation}
For the quark coefficient functions this gives:
\begin{equation}
\begin{array}{rcl}
  C^{(1)}_{1,q}
  \displaystyle\mathop{\longrightarrow}_{m_H\rightarrow 0}
  C^{0,(1)}_{1,q} &=&\displaystyle 2C_F \bigg\{
                      -\left(\frac{9}{2}+2\zeta_2-\frac32 \ln K_F^2\right)\delta(1-z)\\ \\
                  &-&\displaystyle \frac{(1+z^2)\ln z}{1-z} -\left(\ln(1-z)+\ln
                      K_F^2\right)(1+z)+3\\ \\ &+&\displaystyle 2
                                                   \left[\frac{\ln(1-z)}{1-z}\right]_{x} -\left(\frac{3}{2}-2\ln
                                                   K_F^2\right)\left[\frac{1}{1-z}\right]_{x}\bigg\}\,,
\end{array}
\end{equation}
\begin{equation}
\begin{array}{rcl}
  C^{(1)}_{2,q}\displaystyle\mathop{\longrightarrow}_{m_H\rightarrow
  0}C^{0,(1)}_{2,q}&=&\displaystyle 2C_F \bigg\{
                       -\left(\frac{9}{2}+2\zeta_2-\frac32 \ln K_F^2\right)\delta(1-z)\\ \\
                   &-&\displaystyle \frac{(1+z^2)\ln z}{1-z} -\left(\ln(1-z)+\ln
                       K_F^2\right)(1+z)+2z+3\\ \\ &+&\displaystyle 2
                                                       \left[\frac{\ln(1-z)}{1-z}\right]_{x} -\left(\frac{3}{2}-2\ln
                                                       K_F^2\right)\left[\frac{1}{1-z}\right]_{x}\bigg\}\,,
\end{array}
\end{equation}
\begin{equation}
\begin{array}{rcl}
  C^{(1)}_{3,q}\displaystyle\mathop{\longrightarrow}_{m_H\rightarrow 0}
  C^{0,(1)}_{3,q}&=&\displaystyle 2C_F \bigg\{
                      -\left(\frac{9}{2}+2\zeta_2-\frac32 \ln K_F^2\right)\delta(1-z)\\ \\
                  &-&\displaystyle \frac{(1+z^2)\ln z}{1-z} -\left(\ln(1-z)+\ln
                      K_F^2\right)(1+z)+z+2\\ \\ &+&\displaystyle 2
                                                     \left[\frac{\ln(1-z)}{1-z}\right]_{x} -\left(\frac{3}{2}-2\ln
                                                     K_F^2\right)\left[\frac{1}{1-z}\right]_{x}\bigg\}\,,
\end{array}
\end{equation}
\begin{equation} C^{(1)}_{L,q}\mathop{\longrightarrow}_{m_H\rightarrow
0} C^{0,(1)}_{L,q} = 4C_F z \,.
\end{equation}

Considering that:
\begin{equation}
  R(x) \mathop{\longrightarrow}_{m_H\rightarrow 0}
  \frac12\ln(1-x)^2\,,
\end{equation}
the local terms to be added to the quark coefficient functions are:
\begin{equation}
  C^{0,(1)}_{1,q}\rightarrow C^{0, (1)}_{1,q} +
  2C_F\left[\ln^2(1-x)-\left(\frac32-2\ln
      K_F^2\right)\ln(1-x)\right]\delta(1-z)\,,
\end{equation}
\begin{equation}
  C^{0, (1)}_{2,q}\rightarrow C^{0, (1)}_{2,q} +
  2C_F\left[\ln^2(1-x)-\left(\frac32-2\ln
      K_F^2\right)\ln(1-x)\right]\delta(1-z)\,,
\end{equation}
\begin{equation}
  C^{0, (1)}_{3,q}\rightarrow C^{0, (1)}_{3,q} +
  2C_F\left[\ln^2(1-x)-\left(\frac32-2\ln
      K_F^2\right)\ln(1-x)\right]\delta(1-z)\,,
\end{equation}
while no local term needs to be added to $C^{0, (1)}_{L,q}$.

Now we turn to consider the zero-mass limit of the gluon coefficient
functions for which we need to know that:
\begin{equation} 
  \ln(1-\lambda)
  \mathop{\longrightarrow}_{m_H\rightarrow 0}
  -\ln\left(\frac{Q^2}{m_H^2}\right)\,.
\end{equation}
This term is clearly divergent in the zero-mass limit and embeds a
collinear divergence typical of any massless calculations. However, we
retain all such terms so that:
\begin{equation}
  C^{(1)}_{1,g}\mathop{\longrightarrow}_{m_H\rightarrow
    0} C^{0, (1)}_{1,g} =
  2T_R\left\{[z^2+(1-z)^2]\left[\ln\left(\frac{1-z}{z}\right) +\frac12
      \ln\left(\frac{Q^2}{m_H^2}\right) +\frac12\ln K_F^2\right]+ 4z(1-z) -
    1\right\}\,,
\end{equation}
\begin{equation}
  C^{(1)}_{2,g}\mathop{\longrightarrow}_{m_H\rightarrow
    0} C^{0, (1)}_{2,g} =
  2T_R\left\{[z^2+(1-z)^2]\left[\ln\left(\frac{1-z}{z}\right) +\frac12
      \ln\left(\frac{Q^2}{m_H^2}\right) +\frac12\ln K_F^2\right]+ 8z(1-z) -
    1\right\}\,,
\end{equation}
\begin{equation}
  C^{(1)}_{3,g}\mathop{\longrightarrow}_{m_H\rightarrow
    0} C^{0, (1)}_{3,g} = 2T_R[z^2+(1-z)^2]\left[-\frac12
    \ln\left(\frac{Q^2}{m_H^2}\right) +\frac12\ln K_F^2\right]\,,
\end{equation}
\begin{equation}
  C^{(1)}_{L,g}\mathop{\longrightarrow}_{m_H\rightarrow
    0} C^{0,(1)}_{L,g} = 2T_R\left[4z(1-z)\right]\,.
\end{equation}
We also note that in the limit $m_H\rightarrow 0$, the convolution
integrals in Eqs.~(\ref{F1}), (\ref{F2}), (\ref{F3}) and~(\ref{FL})
will extend from $x$ to 1 rather than from $\chi$ to 1.

The massive structure functions will eventually need to be combined to
the massless and the massive-zero ones to compute the GM structure
functions.  Since the latter are computed through Mellin convolutions
whose integral extends from $x$ to 1, it is convenient to rewrite
Eqs.~(\ref{F1}), (\ref{F2}), (\ref{F3}) and~(\ref{FL}) in such a way
that the lower integration bound is also $x$ rather than $\chi$. To
this end, let us consider the integral:
\begin{equation}
  I=\int_\chi^1\frac{dy}{y}
  C(y)f\left(\frac{\chi}{y}\right)\,,
\end{equation}
where $\chi=x/\lambda$. By changing of integration variable
$z = \lambda y$, integral above becomes:
\begin{equation}
  I=\int_x^\lambda\frac{dz}{z}
  C\left(\frac{z}{\lambda}\right)f\left(\frac{x}{y}\right) =
  \int_x^1\frac{dz}{z}
  \widetilde{C}(z,\lambda)f\left(\frac{x}{y}\right)\,,
\end{equation}
where:
\begin{equation}
  \widetilde{C}(z,\lambda)=\theta(\lambda-z)C\left(\frac{z}{\lambda}\right)\,.
\end{equation}
In this way we have achieved the goal of expressing the ``reduced''
convolution in Eqs.~(\ref{F1}), (\ref{F2}), (\ref{F3}) and~(\ref{FL})
as a ``standard'' convolution between $x$ and 1. As already mentioned
above, this does not need to be done in the massive-zero case as the
convolution already extends between $x$ and $1$.

\section{Target Mass Corrections}

Kinematic corrections due to the finite mass of the hadron $M_p$ which
recoils against the vector boson might be relevant in the small-$Q$
region. The leading contributions to these corrections have been
computed long time ago in Ref.~\cite{Georgi:1976ve} and, denoting the
target-mass corrected structure functions with the symbol
$\widetilde{\quad}$, they take the form:
\begin{equation}
\begin{array}{rcl}
  \displaystyle \widetilde{F}_2(x,Q) &=&
                                         \displaystyle \frac{x^2}{\xi^2 \tau^{3/2}} F_2(\xi,Q) + \frac{6\rho
                                         x^3}{\tau^2} I_2(\xi,Q)\,,\\ \\ \displaystyle \widetilde{F}_L(x,Q) &=&
                                                                                                                \displaystyle F_L(\xi,Q)+\frac{x^2(1-\tau)}{\xi^2 \tau^{3/2}}
                                                                                                                F_2(\xi,Q) + \frac{\rho x^3(6-2\tau)}{\tau^2} I_2(\xi,Q)\,,\\ \\
  \displaystyle x\widetilde{F}_3(x,Q) &=& \displaystyle \frac{x^2}{\xi^2
                                          \tau} \xi F_3(\xi,Q) + \frac{2\rho x^3}{\tau^{3/2}} I_3(\xi,Q)\,,
\end{array}
\end{equation}
where:
\begin{equation}
  \rho = \frac{M_p^2}{Q^2}\,,\qquad \tau = 1 + 4\rho
  x^2\,,\qquad \xi = \frac{2x}{1+\sqrt{\tau}}\,,
\end{equation}
and:
\begin{equation}
  I_2(\xi,Q) = \int_\xi^1dy \frac{F_2(y,Q)}{y^2}\,,
  \qquad I_3(\xi,Q) = \int_\xi^1dy \frac{yF_3(y,Q)}{y^2}\,.
\end{equation}
These integrals can be efficiently computed relying on the procedure
based on the interpolating functions discussed in the part of the
documentation devoted to the interpolation. From the equations above,
it is clear that in the limit $M_p \rightarrow 0$, that implies
$\rho \rightarrow 0$, $\tau \rightarrow 1$, and $\xi \rightarrow x$,
all structure functions reduce to the usual formulas.

\section{Renormalisation and factorisation scale variations}

In the previous sections, when discussing the implementation of the
structure functions in {\tt APFEL++}, we implicitly assumed that the
renormalisation scale $\mu_R$ and the factorisation scale $\mu_F$ were
identified with the scale $Q$. The purpose of this section is to relax
this assumption. To do so, it is necessary to consider the expansion
of the DGLAP and of the renormalisation-group (RG) equation for
$\alpha_s$ up to $\mathcal{O}(\alpha_s^2)$ that is the maximum order
considered in the structure functions discussed above. These equations
read:
\begin{equation}
  \frac{\partial f_{i}}{\partial\ln\mu_F^2} =
  \frac{\alpha_s(\mu_F)}{4\pi}\left[ P_{ij}^{(0)}(x) +
    \frac{\alpha_s(\mu_F)}{4\pi} P_{ij}^{(1)}(x) + \dots\right]\otimes
  f_j(x,\mu_F)\,,
\end{equation}
where a summation over repeated indices is understood, and:
\begin{equation}
  \frac{\partial
  }{\partial\ln\mu_R^2}\left(\frac{\alpha_s}{4\pi}\right) =
  -\left(\frac{\alpha_s(\mu_R)}{4\pi}\right)^2\left[ \beta_0 +
    \frac{\alpha_s(\mu_R)}{4\pi}\beta_1 + \dots\right]\,.
\end{equation}
Defining:
\begin{equation}
  \xi_R\equiv\frac{\mu_R}{Q}\,,\quad\xi_F\equiv\frac{\mu_F}{Q}\quad\mbox{and}\quad
  a_s=\frac{\alpha_s}{4\pi}
\end{equation}
and:
\begin{equation}
  t_R \equiv \ln\xi_R^2\quad\mbox{and}\quad t_F \equiv
  \ln\xi_F^2\,,
\end{equation}
they can be written more compactly as:
\begin{equation}
  \frac{\partial f_{i}}{\partial t_F}
  = a_s(t_F)\left[ P_{ij}^{(0)} + a_s(t_F) P_{ij}^{(1)} +
    \dots\right]\otimes f_j(t_F)\,,
\label{DGLAPsimp}
\end{equation}
and:
\begin{equation}
  \frac{\partial a_s}{\partial t_R} =
  -a_s^2(t_R)\left[ \beta_0 + a_s(t_R)\beta_1 + \dots\right]\,.
\label{BETAsimp}
\end{equation}
Now, the Taylor expansion of $f_i(t)$ around $t=t_F$, up to second
order is:
\begin{equation}
  f_i(t) = f_i(t_F)+\frac{\partial
    f_{i}}{\partial t}\bigg|_{t=t_F} (t-t_F) + \frac12 \frac{\partial^2
    f_{i}}{\partial t^2}\bigg|_{t=t_F} (t-t_F)^2 + \dots
\label{expDGLAP}
\end{equation}
Using Eqs.~(\ref{DGLAPsimp}) and~(\ref{BETAsimp}), we have that:
\begin{equation}
\begin{array}{l} 
  \displaystyle \frac{\partial f_{i}}{\partial
  t}\bigg|_{t=t_F} = \left[ a_s(t_F) P_{ij}^{(0)} + a_s^2(t_F)
  P_{ij}^{(1)}\right]\otimes f_j(t_F) + \mathcal{O}(a_s^3)\\ \\
  \displaystyle \frac{\partial^2 f_{i}}{\partial t^2}\bigg|_{t=t_F} =
  a_s^2(t_F)\left[ P_{il}^{(0)}\otimes P_{lj}^{(0)} - \beta_0
  P_{ij}^{(0)} \right]\otimes f_j(t_F) + \mathcal{O}(a_s^3)
\end{array}
\end{equation}
Choosing $t=0$ in Eq.~(\ref{expDGLAP}), that is equivalent to setting
$\mu_F=Q$, gives:
\begin{equation}
\begin{array}{rcl}
  f_i(0) &=&\displaystyle  \Bigg\{1-a_s(t_F) t_F
    P_{ij}^{(0)} + a_s^2(t_F)\left[-t_F P_{ij}^{(1)}+ t_F^2 \frac12\left(
        P_{il}^{(0)}\otimes P_{lj}^{(0)} - \beta_0 P_{ij}^{(0)}
      \right)\right]\\
\\
&+&\displaystyle a_s^3(t_F)\Bigg[ \frac12 t_F^2 \left(P_{il}
    ^{(0)}\otimes P_{lj} ^{(1)}+P_{il} ^{(1)} \otimes P_{lj}
    ^{(0)}\right)-\frac{1}{2} t_F^2 \beta _1 P_{ij}
    ^{(0)}+\frac{1}{2} t_F^3\beta _0 P_{il} ^{(0)}\otimes P_{lj}
    ^{(0)}\\
\\
&-&\displaystyle \frac{1}{3} t_F^3\beta _0^2 P_{ij}^{(0)}-\frac{1}{6}t_F^3 P_{il} ^{(0)}\otimes P_{lk} ^{(0)} \otimes
    P_{kj} ^{(0)}-t_F^2\beta _0 P_{ij} ^{(1)}- t_F P_{ij} ^{(2)}\Bigg]\Bigg\}\otimes f_j(t_F) + \mathcal{O}(a_s^4)\,.
\end{array}
\label{expDGLAP1}
\end{equation}
In addition, using Eq.~(\ref{BETAsimp}), one easily finds:
\begin{equation}
  a_s(t_F)=
  a_s(t_R)\left[1+a_s(t_R)\beta_{0}(t_R-t_F)+a_s^2(t_R)\left(\beta_{1} (t_R-t_F)+\beta_{0}^2 (t_R-t_F)^2\right)+\mathcal{O}(a_s^3)\right]\,,
\label{BETAexp} 
\end{equation}
which can be plugged into Eq.~(\ref{expDGLAP1}) to give:
\begin{equation}
\begin{array}{rcl}
  f_i(0) &=&\displaystyle \Bigg\{1-a_s(t_R) t_F
    P_{ij}^{(0)} + a_s^2(t_R)\left[-t_F P_{ij}^{(1)}+ t_F^2 \frac12\left(
        P_{il}^{(0)}\otimes P_{lj}^{(0)} + \beta_0 P_{ij}^{(0)} \right)-
      t_Ft_R\beta_{0}P_{ij}^{(0)}\right]\\
\\
&+&\displaystyle a_s^3(t_R)\Bigg[\frac{1}{2}  t_F^2 \left(P_{il}
    ^{(0)} P_{lj} ^{(1)}+P_{il} ^{(1)} P_{lj}
    ^{(0)}\right)+\frac{1}{2} t_F^2 \beta _1 P_{ij} ^{(0)}-t_F t_R
    \beta _1 P_{ij} ^{(0)}-\frac{1}{2} t_F^3\beta _0 P_{il}
    ^{(0)}\otimes P_{lj} ^{(0)}\\
\\
&-&\displaystyle \frac{1}{3} t_F^3\beta _0^2 P_{ij} ^{(0)}-\frac{1}{6}
    t_F^3 P_{il} ^{(0)}\otimes P_{lk} ^{(0)}\otimes P_{kj} ^{(0)}+
    t_F^2 t_R\beta _0 P_{ik} ^{(0)}\otimes P_{kj} ^{(0)}+ t_F^2 t_R
    \beta _0^2 P_{ij} ^{(0)}- t_F t_R^2 \beta _0^2 P_{ij} ^{(0)}\\
\\
&+&\displaystyle  t_F^2\beta _0 P_{ij} ^{(1)}-2 t_F t_R \beta _0 P_{ij} ^{(1)}- t_F P_{ij} ^{(2)}\Bigg]\Bigg\}\otimes f_j(t_F) +
  \mathcal{O}(a_s^4)\,.
\end{array}
\label{expDGLAP2}
\end{equation}
This equality can be conveniently written as:
\begin{equation}
f_i(0) =\sum_{k=0}^3a_s^k(t_R)f_i^{[k]}(t_R,t_F)+
\mathcal{O}(a_s^4)\,,
\label{expDGLAP3}
\end{equation}
where the coefficients $f_i^{[k]}(t_R,t_F)$ can be read off from
Eq.~(\ref{expDGLAP2}).

It is also useful to consider $t_R=0$ in Eq.~(\ref{BETAexp}), that is
equivalent to set $\mu_R=Q$, which gives:
\begin{equation}
  a_s(0)=
  a_s(t_R)\left[1+a_s(t_R)\beta_{0}t_R+a_s^2(t_R)\left(\beta_{1} t_R+\beta_{0}^2t_R^2\right) \right]+\mathcal{O}(a_s^4)\,.
\label{BETAexp1} 
\end{equation}
We are now ready to use these equations to derive the scale variation
terms to be included in ZM structure functions. Truncating the
perturbative series to $\mathcal{O}(\alpha_s^2)$, they are written in
terms of PDFs and coefficient functions as:
\begin{equation}
  F(t_R,t_F) / x = \left[\sum_{k=0}^{3} a_s^k(t_R)
    \widetilde{\mathcal{C}}_i^{(k)}(t_R,t_F)\right]\otimes f_i(t_F) + \mathcal{O}(a_s^4)\,,
\label{NonZeroScales}
\end{equation}
where the symbol $\otimes$ represents the convolution. Since structure
functions are physically observable, they must be renormalisation and
factorisation scale invariant, that is:
\begin{equation}
  F(t_R,t_F) = F(0,0)\,,
\label{invariance}
\end{equation} 
order by order in perturbation theory. Since:
\begin{equation}
  F(0,0) / x = \left[\sum_{k=0}^{3}
    a_s^k(0) \widetilde{C}_i^{(k)}\right]\otimes
  f_i(0) + \mathcal{O}(a_s^4)\,,
\label{ZeroScales}
\end{equation} 
where $\widetilde{C}_i^{(k)}$ are the usual perturbative contributions
to the ZM coefficient functions, one can plug Eqs.~(\ref{expDGLAP2})
and~(\ref{BETAexp1}) into Eq.~(\ref{ZeroScales}) and impose the
identity in Eq.~(\ref{invariance}). By doing so, one finds the
explicit expression of the ``generalised'' coefficient functions
$\widetilde{\mathcal{C}}_i^{(k)}(t_R,t_F)$ that include also the scale
variation terms. It is convenient to first compute renormalisation
scale variations while leaving $t_F=0$:
\begin{equation}
\begin{array}{rcl}
  \displaystyle
  \widetilde{\mathcal{C}}_j^{(0)}(t_R,0) &=& \displaystyle
                                               \widetilde{C}_j^{(0)} \\ \\ \displaystyle
  \widetilde{\mathcal{C}}_j^{(1)}(t_R,0) &=& \displaystyle
                                               \widetilde{C}_j^{(1)}\\
  \\
\displaystyle \widetilde{\mathcal{C}}_j^{(2)}(t_R,0) &=&
                                                                   \displaystyle
                                                              \widetilde{C}_j^{(2)}
                                                              +
                                                              t_R\beta_0
                                                              \widetilde{C}_j^{(1)}\,,\\
\\
\displaystyle \widetilde{\mathcal{C}}_j^{(3)}(t_R,0) &=&
                                                                   \displaystyle
                                                              \widetilde{C}_j^{(3)}+2 t_R\beta _0 \widetilde{C}_j^{(2)} 
                                                              +t_R \left(\beta_1 +\beta _0^2 t_R\right) \widetilde{C}_j^{(1)}\,,
\end{array}
\label{CFtR}
\end{equation}
so that:
\begin{equation}
  F(t_R,0) / x = \left[\sum_{k=0}^{3} a_s^k(t_R)
    \widetilde{\mathcal{C}}_i^{(k)}(t_R,0)\right]\otimes f_i(0) + \mathcal{O}(a_s^4)\,.
\label{NonZeroScalestR}
\end{equation}
We can now use Eq.~(\ref{expDGLAP3}) to express the PDF $f(0)$ in
terms of $f(t_F)$ finally obtaining:
\begin{equation}
  F(t_R,0) / x = \sum_{k=0}^{3} a_s^k(t_R)
    \sum_{j=0}^{k}\widetilde{\mathcal{C}}_i^{(k-j)}(t_R,0)\otimes f_i^{[j]}(t_R,t_F) + \mathcal{O}(a_s^4)\,.
\label{NonZeroScalestRtF}
\end{equation}
In the absence of factorisation scale variations, only the term with $j=0$
contributes to the inner sum.


% \newpage
% The result is:
% \begin{equation}
%   \begin{array}{rcl}
%     F(0,0)/x &=&\bigg\{ \widetilde{C}_j^{(0)} \\ \\
%              &+&\displaystyle a_s(t_R)\left[\widetilde{C}_j^{(1)}- t_F
%                  \widetilde{C}_i^{(0)} \otimes P_{ij}^{(0)}\right]\\ \\
%              &+&\displaystyle a_s^2(t_R)\bigg[\widetilde{C}_j^{(2)} + t_R\beta_0
%                  \widetilde{C}_j^{(1)} -t_F \left(\widetilde{C}_i^{(0)} \otimes
%                  P_{ij}^{(1)}+\widetilde{C}_i^{(1)} \otimes P_{ij}^{(0)}\right)\\ \\
%              &+&\displaystyle \frac{t_F^2}2 \widetilde{C}_i^{(0)} \otimes \left(
%                  P_{il}^{(0)}\otimes P_{lj}^{(0)} + \beta_0 P_{ij}^{(0)} \right)-
%                  t_Ft_R\beta_{0}\widetilde{C}_i^{(0)} \otimes
%                  P_{ij}^{(0)}\bigg]\\
% \\
% &+&\displaystyle a_s^3(t_R)\bigg[\widetilde{C}_j^{(3)} \bigg]\bigg\}\otimes f_j(t_F)+\mathcal{O}(a_s^4)\,.
% \end{array}
% \end{equation}
% Finally, using the identity in Eq.~(\ref{invariance}), it is easy to
% find that:
% \begin{equation}
% \begin{array}{rcl}
%   \displaystyle
%   \widetilde{\mathcal{C}}_j^{(0)}(t_R,t_F) &=& \displaystyle
%                                                \widetilde{C}_j^{(0)} \\ \\ \displaystyle
%   \widetilde{\mathcal{C}}_j^{(1)}(t_R,t_F) &=& \displaystyle
%                                                \widetilde{C}_j^{(1)}-t_F \widetilde{C}_i^{(0)} \otimes P_{ij}^{(0)}
%   \\ \\ \displaystyle \widetilde{\mathcal{C}}_j^{(2)}(t_R,t_F) &=&
%                                                                    \displaystyle \widetilde{C}_j^{(2)} + t_R\beta_0 \widetilde{C}_j^{(1)}
%                                                                    -t_F \left(\widetilde{C}_i^{(0)} \otimes
%                                                                    P_{ij}^{(1)}+\widetilde{C}_i^{(1)} \otimes P_{ij}^{(0)}\right)\\ \\
%                                            &+&\displaystyle\frac{t_F^2}2 \widetilde{C}_i^{(0)} \otimes \left(
%                                                P_{il}^{(0)}\otimes P_{lj}^{(0)} + \beta_0 P_{ij}^{(0)} \right)-
%                                                t_Ft_R\beta_{0}\widetilde{C}_i^{(0)} \otimes P_{ij}^{(0)}\,.
% \end{array}
% \label{generalizedCF}
% \end{equation}
% Unsurprisingly, setting $\mu_F=\mu_R=Q$, that results in $t_F=t_R=0$,
% one finds
% $\widetilde{\mathcal{C}}_j^{(k)}(0,0) =\widetilde{C}_j^{(k)}$ as
% required by construction.

In order to provide an operative formulation of scale variations, it
is necessary to specify the basis in which PDFs are expressed. The
preferred choice in {\tt APFEL++} is the so-called QCD evolution
basis:
\begin{equation}
\{g,\Sigma,V,T_{3},V_{3},T_{8},V_{8},T_{15},V_{15},T_{24},V_{24},T_{35},V_{35}\}\,. 
\end{equation}
The distributions in the QCD evolution basis can be written in terms
of distributions in the more familiar ``physical'' basis,
\textit{i.e.}
$\{\overline{t},\overline{b},\overline{c},\overline{s},\overline{u},\overline{d},d,u,s,c,b,t\}$,
as follows:
\begin{equation}
\begin{array}{rcl}
\Sigma&=& \sum_{q}q^+\,,\\
V &=& \sum_{q}q^-\,,\\
T_3&=& u^+-d^+\,,\\
V_3&=& u^--d^-\,,\\
T_8&=& u^++d^+-2s^+\,, \\
V_8&=& u^-+d^- -2s^-\,, \\
T_{15}&=& u^++d^++s^+-3c^{+}\,, \\
V_{15}&=& u^-+d^- +s^--3c^{-}\,, \\
T_{24}&=& u^++d^++s^++c^{+}-4b^+\,, \\
V_{24}&=& u^-+d^- +s^-+c^{-}-4b^-\,, \\
T_{35}&=& u^++d^++s^++c^{+}+b^+-5t^{+}\,, \\
V_{35}&=& u^-+d^- +s^-+c^{-}+b^--5t^{-}\,.
\end{array}
\end{equation}
where the notation $q^{\pm}\equiv q\pm \overline{q}$ is used. As the
name suggests, the QCD evolution basis is particularly useful when
evolving PDFs because in this basis the DGLAP evolution equations take
a maximally diagonalised form. Adopting the QCD evolution basis
implies that the indices $i$, $j$, and $l$ in
Eq.~(\ref{generalizedCF}) run between 0 and 12 over this basis with 0
corresponding to the gluon component.

In the following, we will consider a ZM neutral current structure
function for $Q\ll M_Z$ in such a way that only the photon
contributes. We will extend the treatment to the general
neutral-current case and to the charge-current one below. Omitting for
simplicity the convolution sign and an overall factor $x$, the
starting point is the definition of the structure function in terms
of the PDFs in the physical basis that reads:
\begin{equation}
F=\langle e_q^2 \rangle \left\{C_gg +\sum_{i=u}^{t}\underbrace{\theta(Q^2-m_i^2)\left[C_{\rm PS}+\frac{e_i^2}{\langle e_q^2
      \rangle}C_+\right]}_{\hat{C}_i}q_i^+\right\}\,,
\label{StructFuncDef}
\end{equation}
where $C_+$ and $C_{\rm PS}$ correspond to the non-singlet and
pure-singlet coefficient functions, that are usually the quantities
computed in perturbation theory, and where:
\begin{equation}
\langle e_q^2 \rangle = \sum_{i=u}^t e_i^2\theta(Q^2-m_i^2)\,.
\end{equation}
with $m_i$ is the mass of the $i$-th quark flavour. Now, in order to
express the structure function in Eq.~(\ref{StructFuncDef}) in the
evolution basis, we need to find the tranformation $T$ such that:
\begin{equation}
q_i^+ = \sum_{j=1}^6T_{ij}f_j\,,
\label{Rotation}
\end{equation}
where $f_j$ belongs to the evolution basis, that is: $f_1=\Sigma$,
$f_2=T_3$, $f_3=T_8$ and so on. One can show that the trasformation
matrix $T$ takes the form:
\begin{equation}
\begin{array}{l}
\displaystyle T_{ij}=\theta_{ji}\frac{1-\delta_{ij}j}{j(j-1)}\quad j\geq 2\,,\\
\\
\displaystyle T_{i1} = \frac{1}{6}\,,
\end{array}
\label{TransDef}
\end{equation}
with $\theta_{ji}=1$ for $j\geq i$ and zero otherwise. In addition,
one can show that:
\begin{equation}
\sum_{j=1}^6T_{ij} = 0\,,\quad\mbox{and}\quad \sum_{i=1}^6T_{ij} = \delta_{1j}\,.
\end{equation}
Now, we can plug Eq.~(\ref{Rotation}) into Eq.~(\ref{StructFuncDef})
and, using Eq.~(\ref{TransDef}), we get:
\begin{equation}
F=\langle e_q^2 \rangle \left\{C_gg +\frac16\left(C_++n_f C_{\rm
      PS}\right)\Sigma+\sum_{j=2}^{6}\frac{1}{j(j-1)}\left[\sum_{i=1}^j\hat{C}_i-j\hat{C}_j\right]f_j\right\}\,,
\label{StructFuncDefEvol}
\end{equation}
where we have transmuted the sum over $u$, $d$ and so on into a sum
between 1 and 6 and we have defined the number of active flavours
$n_f$ as:
\begin{equation}
n_f = \sum_{i=1}^6\theta(Q^2-m_i^2)\,.
\end{equation}
We can now express the term in square brackets in terms of $C_+$ and
$C_{\rm PS}$. In particular:
\begin{equation}
\sum_{i=1}^j\hat{C}_i-j\hat{C}_j = \sum_{i=1}^j \theta(Q^2-m_i^2)\left(C_{\rm PS}+\frac{e_i^2}{\langle e_q^2
      \rangle}C_+\right) - j \theta(Q^2-m_j^2)\left(C_{\rm PS}+\frac{e_j^2}{\langle e_q^2
      \rangle}C_+\right)\,.
\end{equation}
We can distinguish two cases. The first is $Q^2<m_j^2$ and under this
assumption we have:
\begin{equation}
\sum_{i=1}^j\hat{C}_i-j\hat{C}_j = C_++n_fC_{\rm PS}\,.
\end{equation}
If instead $Q^2 \geq m_j^2$, then:
\begin{equation}
\sum_{i=1}^j\hat{C}_i-j\hat{C}_j = K_j C_+\,,
\end{equation}
with:
\begin{equation}
K_j=\frac{1}{\langle e_q^2\rangle}\left(\sum_{i=1}^{j}e_i^2-je_j^2\right)=\frac{1}{\langle e_q^2\rangle}\left(\sum_{i=1}^{j-1}e_i^2-(j-1)e_j^2\right)\,.
\end{equation}
Both cases can be gathered in one single formula as:
\begin{equation}
\sum_{i=1}^j\hat{C}_i-j\hat{C}_j = \theta(m_j^2-Q^2-\epsilon)\left[ C_++n_f C_{\rm PS}\right]
+\theta(Q^2-m_j^2)\left[K_jC_+\right]\,.
\label{SumCoef}
\end{equation}
where $\epsilon$ is an infinitesimal positive number that ensures that
the case $Q^2=m_j^2$ is included in the second term of the r.h.s. of
Eq.~(\ref{SumCoef}). Eq.~(\ref{SumCoef}) can be rewritten as:
\begin{equation}
\sum_{i=1}^j\hat{C}_i-j\hat{C}_j = \theta_{n_fj}\left[K_jC_+\right]+\theta_{j,n_f-1}\left[ C_++n_f C_{\rm PS}\right]\,.
\end{equation}
In addition, neglecting intrinsic heavy-quark contributions, one can
easily see that:
\begin{equation}
f_j = \theta_{n_fj}f_j+\theta_{j,n_f+1}\Sigma\,,
\label{SingletReduction}
\end{equation}
and thus:
\begin{equation}
\sum_{j=2}^{6}\frac{1}{j(j-1)}\left[\sum_{i=1}^j\hat{C}_i-j\hat{C}_j\right]f_j
= C_+\left[\sum_{j=2}^{n_f}\frac{K_j}{j(j-1)}f_j\right]+\left[\sum_{j=n_f+1}^{6}\frac{1}{j(j-1)}\right] \left[ C_++n_f C_{\rm PS}\right]\Sigma\,.
\end{equation}
But:
\begin{equation}
  \sum_{j=n_f+1}^{6}\frac{1}{j(j-1)}=\frac{1}{n_f}-\frac{1}{6}\,,
\end{equation}
and moreover:
\begin{equation}
\frac{K_j}{j(j-1)} = \frac{1}{\langle e_q^2\rangle}\frac{1}{j(j-1)}\left(\sum_{i=1}^je_i^2-je_j^2\right)=\frac{1}{\langle e_q^2\rangle}\underbrace{\frac{1}{j(j-1)}\sum_{i=1}^6e_i^2\left[\theta_{ji}-j\delta_{ij}\right]}_{d_j}\,.
\end{equation}
Finally, putting all pieces together, gives:
\begin{equation}
\begin{array}{rcl}
F&=&\displaystyle \langle e_q^2 \rangle \left[C_g g +\left(C_{\rm PS} +\frac1{n_f}C_+
  \right) \Sigma\right]+C_+\sum_{j=2}^{n_f} d_j f_j\\
\\
&=&\displaystyle \langle e_q^2
\rangle \left[C_g g +\frac{1}{6}\left(C_+ + 6C_{\rm PS}
  \right) \Sigma\right]+C_+\sum_{j=2}^{6} d_j f_j\,.
\end{array}
\label{StructureFunctionEvol}
\end{equation}
The second version of the equation above is particularly useful for
the implementation because the flavours structure does not have any
explicit dependence on $n_f$ and does not rely on the absence of
intrinsic heavy-quark contributions (Eq.~(\ref{SingletReduction})). It
is useful to separate the contributions deriving from the coupling of
the vector boson to the different quark flavours. To do so, one just
needs to select the contributions proportional to, say, the $k$-th
charge $e_k^2$. This is easily done with the following replacement:
\begin{equation}
e_i^2\rightarrow \delta_{ik}e_i^2\,.
\end{equation}
This gives:
\begin{equation}
\langle e_q^2 \rangle \rightarrow \theta(Q^2-m_k^2) e_k^2\,,
\end{equation}
and:
\begin{equation}
d_j \rightarrow \frac{e_k^2\left[\theta_{jk}-j\delta_{kj}\right]}{j(j-1)}=\theta(Q^2-m_k^2) e_k^2\frac{\left[\theta_{jk}-j\delta_{kj}\right]}{j(j-1)}\,,
\end{equation}
so that the component of the structure function $F$ associated to the
$k$-th quark flavour is:
\begin{equation}
\begin{array}{rcl}
F^{(k)} &=&\displaystyle  \theta(Q^2-m_k^2) e_k^2\left\{\left[C_g g +\frac{1}{6}\left(C_++6 C_{\rm PS} \right) \Sigma\right]+C_+\sum_{j=2}^{6}
            \frac{\left[\theta_{jk}-j\delta_{kj}\right]}{j(j-1)} f_j
            \right\}\\
\\
&=& \displaystyle \theta(Q^2-m_k^2) e_k^2\left\{\left[C_g g +\frac{1}{6}\left(C_++6 C_{\rm PS} \right) \Sigma\right]-\frac{1}{k}C_+f_k+C_+\sum_{j=k+1}^{6}
            \frac{1}{j(j-1)} f_j
            \right\}
\end{array}
\label{eq:StructureFunctionSingle}
\end{equation}
and it is such that:
\begin{equation}
F = \sum_{k=1}^6 F^{(k)}\,.
\label{eq:StructureFunctionSingleSum}
\end{equation}

Phenomenologically relevant combinations are the light component, that
includes the contribution of down, up, and strange, and the three
heavy quark components (even though the top component is typically not
relevant). The light component is defined as:
\begin{equation}
F^l = \sum_{k=1}^3 F^{(k)}= \langle e_l^2 \rangle \left[C_g g +\frac{1}{6}\left(C_++6 C_{\rm PS} \right) \Sigma\right]+C_+\sum_{j=2}^{6} d_j^{(l)} f_j\,.
\end{equation}
where:
\begin{equation}
\langle e_l^2\rangle = \sum_{i=1}^3e_i^2\,,
\end{equation}
and:
\begin{equation}
d_j^{(l)}=
\frac{1}{j(j-1)}\sum_{i=1}^3e_i^2\left[\theta_{ji}-j\delta_{ij}\right]=
\left\{
\begin{array}{ll}
\frac{1}{2}(e_u^2-e_d^2)\,,\quad& j= 2 \\
\\
\frac{1}{6}(e_u^2+e_d^2-2e_s^2)\,,\quad& j= 3 \\
\\
\frac{\langle e_l^2\rangle}{j(j-1)}\,,\quad & j\geq 4
\end{array}
\right.\,,
\end{equation}
no need of the $\theta$-functions as the scale $Q$ will always be
above the strange threshold. Therefore, the explicit form of $F^l$ is:
\begin{equation}
F^l = \langle e_l^2 \rangle \left[C_g g +\frac{1}{6}\left(C_++6 C_{\rm
      PS} \right) \Sigma\right]+\frac{1}{2}(e_u^2-e_d^2)C_+T_3+\frac{1}{6}(e_u^2+e_d^2-2e_s^2)C_+T_8+\langle
e_l^2 \rangle C_+\sum_{j=4}^{6} \frac{1}{j(j-1)} f_j\,.
\label{LightSF}
\end{equation}

The heavy-quark components are instead defined as:
\begin{equation}
\begin{array}{rcl}
F^c &=& \displaystyle \theta(Q^2-m_c^2) e_c^2\left\{\left[C_g g +\frac{1}{6}\left(C_++6 C_{\rm PS} \right) \Sigma\right]-\frac{1}{4}C_+T_{15}+C_+\sum_{j=5}^{6}
            \frac{1}{j(j-1)} f_j
            \right\}\,,\\
\\
F^b &=& \displaystyle \theta(Q^2-m_b^2) e_b^2\left\{\left[C_g g +\frac{1}{6}\left(C_++6 C_{\rm PS} \right) \Sigma\right]-\frac{1}{5}C_+T_{24}+C_+\sum_{j=6}^{6}
            \frac{1}{j(j-1)} f_j
            \right\}\,,\\
\\
F^t &=& \displaystyle \theta(Q^2-m_t^2) e_t^2\left\{\left[C_g g +\frac{1}{6}\left(C_++6 C_{\rm PS} \right) \Sigma\right]-\frac{1}{6}C_+T_{35}\right\}\,.
\end{array}
\label{HeavySF}
\end{equation}

To conclude the treatment of all the structure functions, it should be
mentioned that Eq.~(\ref{StructureFunctionEvol}) is valid only for
$F_2$ and $F_L$. However, $F_3$, that appears when weak contributions
are included, can be easily derived following the same steps with the
only differences being that:
\begin{itemize}
\item the distributions $\{\Sigma,T_3,T_8,T_{15},T_{24},T_{35}\}$ must
  be replaced with $\{V,V_3,V_8,V_{15},V_{24},V_{35}\}$,
\item $C_+$ must be replaced with $C_-$,
\item the pure-singlet coefficient function $C_{\rm PS}$ is to be
  replaced with the total-valence coefficient function $C_V$,
\item the gluon coefficient function is identically zero,
\item the squared electric charges must be replaced with the
  appropriate electroweak charges $c_i$.
\end{itemize}
Following this recipe, one finds:

\begin{equation}
F_3= \langle c_q^2
\rangle \frac{1}{6}\left(C_- + 6C_V
  \right) V + C_-\sum_{j=2}^{6} d_j g_j\,,
\label{StructureFunctionEvolF3}
\end{equation}
where $g_j$ runs over $\{V,V_3,V_8,V_{15},V_{24},V_{35}\}$.

Despite Eq.~(\ref{eq:StructureFunctionSingle}) has been derived in the
ZM scheme, it can be generalised to the massive scheme. The
complication is that, due to the explicit dependence of the
coefficient functions on the $k$-th quark mass,\footnote{In general,
  the $k$-th structure function will not only depend on the $k$-th
  mass but also on the other masses. However, in the neutral-current
  case and neglecting intrinsic heavy-quark contributions, the
  dependence on masses other than the $k$-th one in the coefficient
  functions only enters at $\mathcal{O}(\alpha_s^3)$ and therefore
  will be neglected here.} the $k$-th structure function will read:
\begin{equation}
F^{{\rm M},(k)} =  e_k^2\left\{\left[C_g^{(k)} g
    +\frac{1}{6}\left(C_+^{(k)} + 6C_{\rm PS}^{(k)}
  \right) \Sigma\right]+C_+^{(k)}\sum_{j=2}^{6}
            \frac{\left[\theta_{jk}-j\delta_{kj}\right]}{j(j-1)} f_j
            \right\}\,,
\label{eq:StructureFunctionSingleHeavy}
\end{equation}
where an explicit dependence of the index $k$ has been introduced in
the coefficient functions. This dependence invalidates the equality in
Eq.~(\ref{eq:StructureFunctionSingleSum}) where $F$ is given in
Eq.~(\ref{StructureFunctionEvol}). Therefore, in order to compute
inclusive structure functions in the massive case a different
combination is to be taken. Specifically one defines:
\begin{equation}
F^{{\rm M}} = \sum_{k=1}^6F^{{\rm M},(k)} =  \left[\langle C_g\rangle g +\frac{1}{6}\langle C_{q}\rangle \Sigma\right]+\sum_{j=2}^{6}
            \frac{\langle C_{\rm +}\rangle_j}{j(j-1)} f_j\,,
\label{eq:StructureFunctionHeavy}
\end{equation}
with the  definitions:
\begin{equation}
\begin{array}{rcl}
  \langle C_g\rangle &=&\displaystyle \sum_{k=1}^6e_k^2 C_g^{(k)}\,,\\
  \\
  \langle C_{q}\rangle &=&\displaystyle \sum_{k=1}^6e_k^2
                               \left(C_+^{(k)} +6C_{\rm PS}^{(k)} \right)\,,\\
  \\
  \langle C_{\rm +}\rangle_j &=&\displaystyle \sum_{k=1}^6e_k^2\left[\theta_{jk}-j\delta_{kj}\right]C_+^{(k)}=\sum_{k=1}^{j-1}e_k^2C_+^{(k)}+e_j^2(1-j)C_+^{(j)}\,.
\end{array}
\end{equation}

Eq.~(\ref{StructureFunctionEvol}) is the result that allows us to
implement the scale variation formulae given in
Eq.~(\ref{generalizedCF}) in {\tt APFEL++}. An important aspect of
Eq.~(\ref{StructureFunctionEvol}) is that it is written in terms of
the fundamental coefficient functions $C_g$, $C_+$ and $C_{\rm PS}$
and PDFs appear in the evolution basis in which the splitting-function
matrix diagonalises. In particular, up to $\mathcal{O}(\alpha_s^2)$,
one has that:
\begin{equation}
\begin{array}{ll}
  P_{ij}^{(k)} \rightarrow P_{ij}^{(k)} &\quad i,j=g,q(\Sigma)\\
  P_{ij}^{(k)} \rightarrow \delta_{ij}P_+^{(k)} &\quad i,j=T_{3},T_{8},V_{15},T_{24},T_{35}\\
  P_{ij}^{(k)} \rightarrow \delta_{ij}P_-^{(k)} &\quad i,j=V,V_{3},V_{8},V_{15},V_{24},V_{35}
\end{array}
\end{equation}
Also, defining:
\begin{equation}
C_q=C_{\rm PS} +\frac{1}{n_f}C_+\,,
\end{equation}
one can connect Eq.~(\ref{ZeroScales}) and
Eq.~(\ref{StructureFunctionEvol}) by observing that:
\begin{equation}
\begin{array}{ll} 
  \widetilde{C}_j^{(k)} \rightarrow \langle e_q^2\rangle C_j^{(k)}
  &\quad j=g,q(\Sigma)\\
  \widetilde{C}_j^{(k)}
  \rightarrow d_jC_+^{(k)} &\quad j=T_{3},T_{8},T_{15},T_{24},T_{35}\\
  \widetilde{C}_j^{(k)}\rightarrow d_jC_-^{(k)} &\quad j=V_{3},V_{8},V_{15},V_{24},V_{35}
\end{array}
\end{equation}
where we have also considered the ``minus'' distributions that appear
in the $F_3$ structure function. Of course, the same relations must
hold also for Eq.~(\ref{NonZeroScales}):
\begin{equation}
\begin{array}{ll}
  \widetilde{\mathcal{C}}_j^{(k)} \rightarrow
  \langle e_q^2\rangle \mathcal{C}_j^{(k)} &\quad j=g,q(\Sigma)\\
  \widetilde{\mathcal{C}}_j^{(k)} \rightarrow d_j\mathcal{C}_+^{(k)}
                         &\quad j=T_{3},T_{8},T_{15},T_{24},T_{35}\\
  \widetilde{\mathcal{C}}_j^{(k)} \rightarrow d_j\mathcal{C}_-^{(k)}
                         &\quad j=V_{3},V_{8},V_{15},V_{24},V_{35}
\end{array}
\end{equation}
with
\begin{equation}
\mathcal{C}_q=\mathcal{C}_{\rm PS} +\frac{1}{n_f}\mathcal{C}_+\,.
\end{equation}
In addition, in the following we will make use of the following
identity:
\begin{equation}
  P_-^{(0)}=P_+^{(0)}=P_{qq}^{(0)}\,.
\end{equation}
Now, considering that $C_{\rm PS}$ starts at
$\mathcal{O}(\alpha_s^2)$, we can write:
\begin{equation}
\begin{array}{l}
  \displaystyle C_-^{(0)}(x) = C_+^{(0)}(x) =
  \Delta_{\rm SF}\delta(1-x)\\
  \displaystyle C_j^{(0)}(x) = \left(\Delta_{\rm
  SF}/n_f\right)\delta_{qj}\delta(1-x) \quad \mbox{for } j=q,g
\end{array}
\end{equation}
where $\Delta_{\rm SF}=1$ for $F_2$ and $F_3$ and
$\Delta_{\rm SF} = 0$ for $F_L$.  From Eq.~(\ref{generalizedCF}) it
follows that:
\begin{equation}
\begin{array}{rcl}
  \displaystyle \mathcal{C}_\pm^{(0)}(t_R,t_F) &=&
                                                   \displaystyle \Delta_{\rm SF}\delta(1-x) \\ \\ \displaystyle
  \mathcal{C}_\pm^{(1)}(t_R,t_F) &=& \displaystyle
                                     C_\pm^{(1)}-\Delta_{\rm SF} t_F P_{qq}^{(0)} \\ \\ \displaystyle
  \mathcal{C}_\pm^{(2)}(t_R,t_F) &=& \displaystyle C_\pm^{(2)} +
                                     t_R\beta_0 C_\pm^{(1)} -t_F \left(\Delta_{\rm SF}
                                     P_\pm^{(1)}+C_\pm^{(1)} \otimes P_{qq}^{(0)}\right)\\ \\
                                               &+&\displaystyle\Delta_{\rm SF} \frac{t_F^2}2 \left(
                                                   P_{qq}^{(0)}\otimes P_{qq}^{(0)} + \beta_0 P_{qq}^{(0)} \right)-
                                                   \Delta_{\rm SF} t_Ft_R\beta_{0}P_{qq}^{(0)}\,,
\end{array}
\label{NonSingletCF}
\end{equation}
that can be rearranged as:
\begin{equation}
\begin{array}{rcl}
  \displaystyle \mathcal{C}_\pm^{(0)}(t_R,t_F) &=&
                                                   \displaystyle \Delta_{\rm SF}\delta(1-x) \\ \\ \displaystyle
  \mathcal{C}_\pm^{(1)}(t_R,t_F) &=& \displaystyle
                                     C_\pm^{(1)}-\Delta_{\rm SF} t_F P_{qq}^{(0)} \\ \\ \displaystyle
  \mathcal{C}_\pm^{(2)}(t_R,t_F) &=& \displaystyle C_\pm^{(2)} +
                                     t_R\beta_0 C_\pm^{(1)} -t_F C_\pm^{(1)} \otimes P_{qq}^{(0)}\\ \\
                                               &+&\displaystyle\Delta_{\rm SF}\frac{t_F^2}2 \left(
                                                   P_{qq}^{(0)}\otimes P_{qq}^{(0)} - \beta_0 P_{qq}^{(0)} \right)-
                                                   \Delta_{\rm SF} t_F\left[P_\pm^{(1)} - (t_F-t_R)\beta_{0}
                                                   P_{qq}^{(0)}\right]\,.
\end{array}
\label{NonSingletCF1}
\end{equation}
The term in square brackets in the r.h.s. of the third line
corresponds to what we define
$\widetilde{\mathcal{P}}_\pm^{(1)}(t_R,t_F)$, that is the NLO
contribution to the non-singlet splitting functions in the presence of
scale variations ($\mu_R\neq\mu_F$).

Let us now consider the singlet sector that becomes:
\begin{equation}
\begin{array}{rcl}
  \displaystyle {\mathcal{C}}_j^{(0)}(t_R,t_F) &=&
                                                   \displaystyle \frac{\Delta_{\rm SF}}{n_f}\delta_{qj}\delta(1-x) \\ \\
  \displaystyle {\mathcal{C}}_j^{(1)}(t_R,t_F) &=& \displaystyle
                                                   {C}_j^{(1)}- \frac{\Delta_{\rm SF}}{n_f} t_F P_{qj}^{(0)} \\ \\ \displaystyle
  {\mathcal{C}}_j^{(2)}(t_R,t_F) &=& \displaystyle {C}_j^{(2)} +
                                     t_R\beta_0 {C}_j^{(1)} -t_F {C}_i^{(1)} \otimes P_{ij}^{(0)}\\ \\
                                               &+&\displaystyle\frac{\Delta_{\rm SF}}{n_f} \frac{t_F^2}2 \left(
                                                   P_{qi}^{(0)}\otimes P_{ij}^{(0)} - \beta_0 P_{qj}^{(0)} \right)-
                                                   \frac{\Delta_{\rm SF}}{n_f} t_F\widetilde{P}_{qj}^{(1)}\,,
\end{array}
\label{SingletCF}
\end{equation}
for $j=g,q$. Taking into account Eq.~(\ref{NonSingletCF1}) and
considering also that $C_{\rm PS}^{(0)}=C_{\rm PS}^{(1)}=0$, it is
easy to see that:
\begin{equation}
\begin{array}{rcl}
  \displaystyle {\mathcal{C}}_{\rm PS}^{(0)}(t_R,t_F)
  &=& \displaystyle 0 \\ \\ \displaystyle {\mathcal{C}}_{\rm
  PS}^{(1)}(t_R,t_F) &=& \displaystyle 0 \\ \\ \displaystyle
  {\mathcal{C}}_{\rm PS}^{(2)}(t_R,t_F) &=& \displaystyle {C}_{\rm
                                            PS}^{(2)}-t_FC_g^{(1)}\otimes
  P_{gq}^{(0)}+\frac{\Delta_{\rm SF}}{n_f}\frac{t_F^2}{2}P_{qg}^{(0)}\otimes P_{gq}^{(0)}-\frac{\Delta_{\rm SF}}{n_f}t_F\left[\widetilde{P}_{qq}^{(1)}-\widetilde{P}_{+}^{(1)}\right]\,.
\end{array}
\label{PureSingletCF}
\end{equation}

Now we consider the massive case. In the neutral-current sector the
leading-order coefficient functions $C_i^{(0)}$ are identically zero
which substantially simplifies the structure of the coefficient
functions in the presence of scale variations:
\begin{equation}
\begin{array}{rcl}
  \displaystyle \mathcal{C}_j^{(0)}(t_R,t_F) &=&
                                                 \displaystyle 0 \\ \\ \displaystyle \mathcal{C}_j^{(1)}(t_R,t_F) &=&
                                                                                                                      \displaystyle C_j^{(1)} \\ \\ \displaystyle
  \mathcal{C}_j^{(2)}(t_R,t_F) &=& \displaystyle C_j^{(2)} + t_R\beta_0
                                   C_j^{(1)} -t_F C_i^{(1)} \otimes P_{ij}^{(0)}\,.
\end{array}
\end{equation}
In addition, the factorisation scale variation terms are already
present in the implementation of the massive coefficient functions in
{\tt APFEL++}. As a consequence, only the renormalisation variation
terms need to be implemented.

As far as the massive charged-current sector is concerned, no
$\mathcal{O}(\alpha_s^2)$ are presently available and thus only the
first two lines of Eq.~(\ref{generalizedCF}) are actually
required. Also in this case the factorisation scale variation terms
are already present in the implementation of the coefficient functions
and again this avoids the pre-computation of additional terms.

% Now let us discuss how to implement in {\tt APFEL++} the additional
% terms needed to perform scale variations. The only terms that are a
% bit more complicated to implement are those that require a convolution
% between two splitting functions of between a plitting functions and a
% coefficient functions. More in particular, we only need to compute the
% terms: $P_{ij}^{(0)}(x)\otimes P_{jk}^{(0)}(x)$ and
% $C_{i}^{(1)}(x)\otimes P_{ij}^{(0)}(x)$. In pricinple, these terms
% could be evaluated analitically by computing the explicit convolution
% between the know expressions that are involved. However, it seems
% easier in {\tt APFEL++} to compute these terms numerically using the
% ingredients that have already been evaluated in the initialization
% stage. To show how to reduce these terms to known quantity, let us
% cosider the following convolution:
% \begin{equation}
%   F(x_\alpha)=x_\alpha C(x_\alpha)\otimes Q(x_\alpha) =
%   x_\alpha\int_{x_\alpha}^1\frac{dy}{y}C(y)Q\left(\frac{x_\alpha}{y}\right)
%   =\int_{x_\alpha}^1\frac{dy}{y}yC(y)\frac{x_\alpha}{y}Q\left(\frac{x_\alpha}{y}\right)
%   =
%   \int_{x_\alpha}^1\frac{dy}{y}\widetilde{C}(y)\widetilde{Q}\left(\frac{x_\alpha}{y}\right)\,,
% \end{equation}
% where $x_\alpha$ is node of the $x$-space grid of {\tt APFEL++} and
% $\widetilde{C}(y)=yC(y)$ and $\widetilde{Q}(y)=yQ(y)$. Now, using the
% well-known interpolation formula we can write:
% \begin{equation}
%   \int_{x_\alpha}^1\frac{dy}{y}\widetilde{C}(y)\widetilde{Q}\left(\frac{x_\alpha}{y}\right)
%   =
%   \sum_{\beta}\underbrace{\left[\int_{x_\alpha}^1\frac{dy}{y}\widetilde{C}(y)w_{\beta}^{(k)}\left(\frac{x_\alpha}{y}\right)\right]}_{\Gamma_{\alpha\beta}}\widetilde{Q}(x_\beta)\,,
% \end{equation}
% where $w_{\beta}^{(k)}$ are the usual interpolation functions of
% degree $k$. Now suppose that in turn:
% \begin{equation} 
%   \widetilde{Q}(x_\beta)=x_\beta P(x_\beta)\otimes
%   f(x_\beta) =
%   \int_{x_\beta}^1\frac{dz}{z}\widetilde{P}(z)\widetilde{f}\left(\frac{x_\beta}{z}\right)=\sum_{\gamma}\underbrace{\left[\int_{x_\beta}^1\frac{dz}{z}\widetilde{P}(z)w_{\gamma}^{(k)}\left(\frac{x_\beta}{z}\right)\right]}_{\Pi_{\beta\gamma}}\widetilde{f}(x_\gamma)\,,
% \end{equation}
% it follows that:
% \begin{equation}
%   F(x_\alpha)=\widetilde{C}(x_\alpha)\otimes
%   \widetilde{P}(x_\alpha)\otimes \widetilde{f}(x_\alpha) =
%   \sum_{\beta,\gamma}
%   \Gamma_{\alpha\beta}\Pi_{\beta\gamma}\widetilde{f}(x_\gamma)\,.
% \end{equation} 
% The formula above clearly shows that the missing pieces can be easily
% obtained by properly multiplying the precomputed splitting function
% matrices $\Pi_{ij,\alpha\beta}$ and the coefficient function matrices
% $\Gamma_{i,\alpha\beta}$ according to the scale variation formulas
% derived above.

% As an alternative to the numerical convolution of the new pieces
% arising when including renormalisation- and factorisation-scale
% variations, one can try to compute the analytically the convolutions
% above. In fact, all the terms involved in the new convolutions are
% usually simple enough to make the analytic computation possible using,
% for instance, {\tt Mathematica}. This is advantageous because it
% avoids any inaccuracy of numerical origin coming from the numerical
% convolution of the operators involved. In order to do so, we only need
% to know how to treat some particular term that appear in the
% combinations. In particular, we need to be able to treat terms in
% which Dirac $\delta$-functions and $+$-prescribed functions are
% present at the same time. The most trivial convolutions are those
% involving one or two $\delta$-functions, that is:
% \begin{equation}
% \begin{array}{l}
% \displaystyle \delta(1-x)\otimes\delta(1-x) = \delta(1-x)\,,\\
% \\
% \displaystyle \left(\frac{\ln^n(1-x)}{1-x}\right)_+\otimes\delta(1-x)
%   = \left(\frac{\ln^n(1-x)}{1-x}\right)_+\quad n\geq0\,,
% \end{array}
% \label{ConvolutionDelta}
% \end{equation}
% that can be easily proven in Mellin space where the convolution
% $\otimes$ becomes a simple product and the $\delta$-function
% corresponds to the unity. The Mellin-space method can be used also in
% the cases where two $+$-prescribed functions are involved. Up to
% $\mathcal{O}(\alpha_s^2)$ there are only two possible combinations,
% that are:
% \begin{equation}
% \begin{array}{l}
% \displaystyle \left(\frac{1}{1-x}\right)_+\otimes
%   \left(\frac{1}{1-x}\right)_+= 2
%   \left(\frac{\ln(1-x)}{1-x}\right)_+-\frac{\ln(x)}{1-x}-\zeta(2)\delta(1-x)\,,\\
% \\
% \displaystyle \left(\frac{1}{1-x}\right)_+\otimes
%   \left(\frac{\ln(1-x)}{1-x}\right)_+= \frac32\left(\frac{\ln^2(1-x)}{1-x}\right)_+-\zeta(2) \left(\frac{1}{1-x}\right)_+-\frac{\ln(x)\ln(1-x)}{1-x}+\zeta(3)\delta(1-x)\,.
% \end{array}
% \label{ConvolutionPlus}
% \end{equation}
% The relations in Eq.~(\ref{ConvolutionPlus}) can be obtained
% rearranging, in Mellin space, the terms is such a way to reconstruct
% the Mellin-transform of well-known terms.

% Now, given the LO splitting functions (with expansion parameter
% $\alpha_s/4\pi$ and such that they can be used to evolve the singlet
% combination $\{q^+,g\}$):
% \begin{equation}
% \begin{array}{l}
% \displaystyle P_{qq}^{(0)}(x) = 2 C_F \left[2
%   \left(\frac{1}{1-x}\right)_+ - (1 + x) +\frac{3}{2} \delta(1 -
%   x)\right]\,,\\
% \\
% \displaystyle P_{qg}^{(0)}(x) = 4 n_f T_R \left[x^2 + (1 - x)^2\right]\,,\\
% \\
% \displaystyle P_{gq}^{(0)}(x) = 2 C_F \left[\frac{1+ (1 -
%   x)^2}{x}\right]\,,\\
% \\
% \displaystyle P_{gg}^{(0)}(x) = 4 C_A \left[\left(\frac{1}{1-x}\right)_+
%   - 2 + x - x^2 + \frac{1}{x}\right] + \frac{11 C_A - 4 n_f T_R}{3}
%   \delta(1 - x)\,,
% \end{array}
% \end{equation}
% we can compute the additional terms involving only combinations of
% splitting functions. In particular, we need to compute:
% \begin{equation}
% P_{qq}^{(0)}(x) \otimes P_{qq}^{(0)}(x)\,,
% \label{NSP0P0}
% \end{equation}
% involved in the $\mathcal{O}(\alpha_s^2)$ non-singlet coefficient
% functions, and:
% \begin{equation}
% \begin{array}{l}
% P_{qg}^{(0)}(x) \otimes P_{gq}^{(0)}(x)\,,\\
% \\
% \displaystyle P_{qi}^{(0)}(x) \otimes P_{ig}^{(0)}(x)=P_{qq}^{(0)}(x) \otimes
%   P_{qg}^{(0)}(x)+P_{qg}^{(0)}(x) \otimes P_{gg}^{(0)}(x)\,.
% \end{array}
% \label{SGP0P0}
% \end{equation}
% present in the pure-singlet and in the gluon coefficient functions,
% respectively.

% The convolution in Eq.~(\ref{NSP0P0}) can be easily computed by hand
% using Eqs.~(\ref{ConvolutionDelta}) and~(\ref{ConvolutionPlus}) and
% the result is:
% \begin{equation}
% \begin{array}{rcl}
% P_{qq}^{(0)}(x) \otimes P_{qq}^{(0)}(x) &=&\displaystyle 4
%                                             C_F^2\bigg[8\left(\frac{\ln(1-x)}{1-x}\right)_++6\left(\frac{1}{1-x}\right)_+-4\frac{\ln(x)}{1-x}-4(1+x)\ln(1-x)\\
% \\
% &+&\displaystyle 3(1+x)\ln(x)-(x+5)+\left(\frac{9}{4}-4\zeta(2)\right)\delta(1-x)\bigg]\,.
% \end{array}
% \end{equation}
% As for Eq.~(\ref{SGP0P0}), where no convolutions of the kinds given in
% Eqs.~(\ref{ConvolutionDelta}) and~(\ref{ConvolutionPlus}) are present,
% we can safely use {\tt Mathematica}, obtaining:
% \begin{equation}
% \begin{array}{rcl}
% P_{qg}^{(0)}(x) \otimes P_{gq}^{(0)}(x)&=&\displaystyle C_F n_f
%                                            T_R\left[\frac{8}{3} \left(-4 x^2-3 x+\frac{4}{x}+3\right)+16 (x+1) \ln(x)\right]\,,\\
% \\
% P_{qi}^{(0)}(x) \otimes P_{ig}^{(0)}(x)&=&\displaystyle n_f T_R C_A
%                                            \left[16 \left(2 x^2-2
%                                            x+1\right) \ln
%                                            (1-x)+16(4 x+1) \ln
%                                            (x)+\frac{4}{3} \left(-40
%                                            x^2+26 x+17+\frac{8}{x}\right)
%                                            \right]\\
% \\
% &+&\displaystyle n_f T_R C_F \left[16 \left(2 x^2-2 x+1\right) \ln
%     (1-x)-8 \left(4 x^2-2 x+1\right) \ln (x)+4\left(4 x-1\right)\right]\\
% \\
% &+&\displaystyle n_f^2 T_R^2 \left[-\frac{16}{3} (2 x^2-2 x+1)\right]
% \end{array}
% \end{equation}

% Now we need to consider the additional terms involving combinations of
% splitting functions and coefficient functions. Let us start
% considering $F_L$ and it is the easiest case. Here we have:
% \begin{equation}
% \begin{array}{rcl}
% \displaystyle C_{L,\pm}^{(1)}(x) &=& \displaystyle 4 C_F x\,,\\
% \\
% \displaystyle C_{L,q}^{(1)}(x) &=& \displaystyle  \frac1{n_f} C_{L,\pm}^{(1)}(x)\,,\\
% \\
% \displaystyle C_{L,g}^{(1)}(x) &=& \displaystyle  4 T_Rx(1-x)\,,
% \end{array}
% \end{equation}
% and for the non-singlet case we need to compute:
% \begin{equation}
% C_{L,\pm}^{(1)}(x)\otimes P_{qq}^{(0)}(x) = 4C_F^2 \left[ (x+2)+4 x \ln (1-x)-2 x \ln(x)\right]\,.
% \end{equation}
% For the pure-singlet and the gluon coefficient functions, instead, we
% need to compute:
% \begin{equation}
% \begin{array}{rcl}
% C_{L,g}^{(1)}(x)\otimes P_{gq}^{(0)}(x) &=& \displaystyle C_F T_R \left[\frac{32}{3} \left(2 x^2-3+\frac{1}{x}\right)-32 x \ln (x)\right]\,,\\
% \\
% C_{L,i}^{(1)}(x)\otimes P_{ig}^{(0)}(x) &=& \displaystyle  C_A T_R
%                                             \left[64 x (1-x) \ln
%                                             (1-x)-128 x \ln
%                                             (x)+\frac{16}{3} \left(23
%                                             x^2-19 x-6 + \frac{2}{x}\right)\right]\\
% \\
% &+&\displaystyle C_F T_R \left[\frac{16}{3} x \ln
%                                             (x)-\frac{8}{3} \left(2
%                                             x^2-x-1\right)\right]\\
% \\
% &+&\displaystyle n_fT_R^2\left[-\frac{64}{3}x(1-x)\right]\,.
% \end{array}
% \end{equation}

% Now we consider $F_2$, for which we have:
% \begin{equation}
% \begin{array}{rcl}
% \displaystyle C_{2,\pm}^{(1)}(x) &=& \displaystyle 
%                                      2C_F \bigg[2 \left(\frac{\ln(1-x)}{1-x}\right)_+-\frac{3}{2}\left(\frac{1}{1-x}\right)_+
%                                      -2\frac{\ln
%                                      (x)}{1-x}-(x+1) \left[\ln (1-x)-\ln
%                                      (x)\right] \\
% \\
% &+&\displaystyle 2 x+3 - \left(2 \zeta(2)+\frac{9}{2}\right) \delta(1-x)\bigg] \,,\\
% \\
% \displaystyle C_{2,q}^{(1)}(x) &=& \displaystyle  \frac1{n_f} C_{\pm,2}^{(1)}(x) \,,\\
% \\
% \displaystyle C_{2,g}^{(1)}(x) &=& \displaystyle  4 T_R \left[\left(x^2+(1-x)^2\right) [\ln
%                                    (1-x)-\ln (x)]-8x^2+8 x-1\right] \,.
% \end{array}
% \end{equation}
% Also in this case we need to compute $C_{2,\pm}^{(1)}(x)\otimes
% P_{qq}^{(0)}(x)$ and $C_{2,i}^{(1)}(x)\otimes P_{ig}^{(0)}(x)$.

\section{Single-inclusive $e^+e^-$ annihilation structure functions}

The implementation of the Single-Inclusive $e^+e^-$ Annihilation (SIA)
structure functions in {\tt APFEL++} is not very complicated. The
reason is that SIA is structurally identical to the case of
deep-inelastic scattering (DIS) that was (implicitly) discussed
above. As a matter fact, one can regard SIA as the time-like
counterpart of DIS and the differences are only at the level of
coefficient functions and splitting functions. Presently, the
coefficient functions for SIA are known up to
$\mathcal{O}(\alpha_s^2)$ (NNLO) in the zero-mass scheme and they have
been computed in Ref.~\cite{Mitov:2006wy} and the $x$-space
expressions reported in Appendix C of that paper.

The way in which the SIA expressions are reported is slightly
different w.r.t. DIS.It is then useful to reduce the SIA expressions
to the same form of DIS in such a way to use the DIS-based structure
of {\tt APFEL++} also for SIA. In particular, the SIA cross section in
Ref.~\cite{Mitov:2006wy} is expressed in terms of the three structure
functions: $F_T$, $F_L$ and $F_A$. However, comparing the SIA cross
section with the DIS one it is easy to realise that defining:
\begin{equation}
\begin{array}{l}
F_2(x,Q) = F_T(x,Q) + F_L(x,Q)\,,\\
 F_L(x,Q) =
F_L(x,Q)\,,\\
F_A(x,Q) = xF_3(x,Q)\,,
\end{array}
\label{SIAtoDIS}
\end{equation}
the SIA cross section reduces to the same structure of DIS. Upon this
identification, the usual factorised form for the structure functions
applies:\footnote{Notice that, to uniform the notation, we understood
  the factor $x$ in front of $F_3$.}
\begin{equation}
  F_k(x,Q) = \sum_{j=q,g} x\int_x^1\frac{dy}{y}
  c_{k,j}(\alpha_s(Q),x)\mathcal{D}_j\left(\frac{x}{y},Q\right)\,,\quad\mbox{with}\quad
  k = 2,L,3\,,
\end{equation} 
where $\mathcal{D}_j$ is the fragmentation function of the flavour $j$
and the coefficient functions $c_{k,j}$ admit the perturbative
expansion:
\begin{equation}
  c_{k,j}(\alpha_s(Q),x) = \sum_{n=0}^N \left(\frac{\alpha_s(Q)}{4\pi}\right)^n
  c_{k,j}^{(n)}(x)\,.
\end{equation}
The leading-order coefficient functions are trivially:
\begin{equation}
\begin{array}{l}
  c_{k,g}^{(0)}(x) = 0 \,,\quad k = 2,L,3\,,\\ \\
  c_{L,q}^{(0)}(x) = 0\,, \\ \\ c_{2,q}^{(0)}(x) = c_{3,q}^{(0)}(x) =
  \delta(1-x)\,.
\end{array}
\end{equation}

Now we consider the NLO coefficient functions. Their explicit
expressions are give in Eqs.~(C.13)-(C.17) of Ref.~\cite{Mitov:2006wy}
but, in order to write them in a form suitable for the implementation
in {\tt APFEL++}, we need to isolate regular, singular, and local
terms and finally combine them according to Eq.~(\ref{SIAtoDIS}).
\begin{equation}
\begin{array}{lcl}
  \displaystyle c_{L,q}^{(1)}(x) = 2C_F\,, & &\\ \\
  \displaystyle c_{L,g}^{(1)}(x) = 2C_F\frac{4(1-x)}{x}\,, & &\\ \\
  \displaystyle c_{2,q}^{(1)}(x) = c_{T,q}^{(1)}(x) + c_{L,q}^{(1)}(x)
                                        &=& \displaystyle
                                            2C_F\bigg[2\left(\frac{\ln(1-x)}{1-x}\right)_+-\frac{3}{2}\left(\frac{1}{1-x}\right)_+
                                            - (1+x)\ln(1-x)\\ \\ & &\displaystyle +2\frac{1+x^2}{1-x}\ln x
                                                                     +\frac{5}{2} -\frac{3}{2} x
                                                                     +\left(4\zeta_2-\frac{9}{2}\right)\delta(1-x)\bigg]\,,\\ \\ \displaystyle
  c_{2,g}^{(1)}(x) = c_{T,g}^{(1)}(x) + c_{L,g}^{(1)}(x) &=&
                                                             \displaystyle 4C_F\frac{1+(1-x)^2}{x} \ln[x^2(1-x)]\,,\\ \\ \displaystyle
  c_{3,q}^{(1)}(x) &=& \displaystyle
                       2C_F\bigg[2\left(\frac{\ln(1-x)}{1-x}\right)_+-\frac{3}{2}\left(\frac{1}{1-x}\right)_+
                       - (1+x)\ln(1-x)\\ \\ & &\displaystyle +2\frac{1+x^2}{1-x}\ln x
                                                +\frac{1}{2} -\frac{1}{2} x
                                                +\left(4\zeta_2-\frac{9}{2}\right)\delta(1-x)\bigg]\,,\\ \\ \displaystyle
  c_{3,g}^{(1)}(x) = 0\,.
\end{array}
\end{equation}
The NNLO coefficient functions, despite implemented in {\tt APFEL++},
are not reported here because too lengthy.

\section{Longitudinally polarised structure functions}

Let us now consider the differential cross sections for unpolarised
and polarised DIS (see {\it e.g.} Eq.~(19.16) of Sec.~19 in
Ref.~\cite{Agashe:2014kda}):
\begin{equation}
\begin{array}{lcl}
\displaystyle \frac{d^2\sigma^i}{dxdy}
& = &
\displaystyle \frac{2\pi\alpha^2}{xyQ^2}\eta^i
\left[
+Y_+ F_2^i \mp Y_- x F_3^i - y^2 F_L^i
\right]
\\ \\
\displaystyle \frac{d^2\Delta\sigma^i}{dxdy}
& = &
\displaystyle \frac{2\pi\alpha^2}{xyQ^2}\eta^i
\left[
-Y_+ g_4^i \mp Y_- 2x g_1^i + y^2 g_L^i
\right]\,,
\end{array}
\end{equation}
where $i={\rm NC, CC}$, $Y_{\pm}=1 \pm (1-y)^2$, $\eta^{\rm NC}=1$,
$\eta^{\rm CC}=(1\pm \lambda)^2\eta_W$ (with $\lambda=\pm 1$ is the 
helicity of the incoming lepton and $\eta_W=\frac{1}{2}
\left(\frac{G_FM_W}{4\pi\alpha}\frac{Q^2}{Q^2+M_W^2} \right)^2$), and 
\begin{equation}
\begin{array}{lcl}
\displaystyle F_L^i & = & \displaystyle F_2^i - 2xF_1^i\\ \\
\displaystyle F_L^i & = & \displaystyle g_4^i - 2xg_5^i\,.
\end{array}
\end{equation}
Because the same tensor structure occurs in the spin-dependent and 
spin-independent parts of the DIS hadronic tensor (in the limit $M^2/Q^2\to 0$),
the polarised cross section can be obtained from the unpolarised cross section
with the following replacement
\begin{equation}
\displaystyle F_2^i \rightarrow -2g_4^i
\ \ \ \ \ \ \ \ \ 
\displaystyle F_3^i \rightarrow +4g_1^i
\ \ \ \ \ \ \ \ \
\displaystyle F_L^i \rightarrow -2g_L^i \,.
\end{equation}
Note that the extra factor two is due to the fact that the total cross section 
is an average over initial-state polarisations.

The {\it polarised} structure functions $g_4$, $g_1$ and $g_L$ 
are expressed as a convolution of coefficient functions, $\Delta c_{k,j}$,
and polarised PDFs, $\Delta f_j$, (summed over all flavors $j$)
\begin{equation}
g_k(x,Q) 
=
\sum_{j=q,g}x \int_x^1 \frac{dy}{y} \Delta c_{k,j}(\alpha_s(Q),x)
\Delta f_j\left(\frac{x}{y},Q\right)\,,
\ \ \ \ \ \ \ \ \ \ \
{\rm with} 
\ \ k=4,1,L\,.
\end{equation} 
The coefficient functions $\Delta c_{k,j}$ allow for the usual perturbative
expansion
\begin{equation}
\Delta c_{k,j}(\alpha_s(Q),x) 
=
\sum_{n=0}^N\left(\frac{\alpha_s(Q)}{4\pi}\right)^n\Delta c_{k,j}^{(n)}(x)
\,,
\end{equation}
where the coefficients $\Delta c_{k,j}^{(n)}(x)$ are known up to NLO,
{\it i.e.} $n=1$ (see {\it e.g.}~\cite{deFlorian:2012wk} and references 
therein). At LO they are 
\begin{equation}
\begin{array}{lcl}
\Delta c_{4,q}^{(0)}(x) = \Delta c_{1,q}^{(0)}(x) &=& \delta(1-x)
\\ \\
\Delta c_{L,q}^{(0)}(x) &=& 0
\,,
\\ \\
\Delta c_{k,g}^{(0)}(x) &=& 0
\ \ \ \ \ \ \ \ \ \
{\rm with} \ k=4,1,L
\,\mbox{.}
\end{array}
\end{equation}
At NLO they read 
\begin{equation}
\begin{array}{lcl}
  %C4q
  \displaystyle \Delta c_{4,q}^{(1)}(x) 
  &=& 
  \displaystyle 2C_F\,
  \bigg\{
  2\left[\frac{\ln(1-x)}{1-x}\right]_+ 
  - \frac{3}{2}\left[\frac{1}{1-x}\right]_+
  - (1+x)\ln(1-x)
  \\ \\ 
  & &\displaystyle - \frac{1+x^2}{1-x}\ln x
  + 3 + 2x
  -\left(\frac{9}{2} + 2\zeta_2\right)\delta(1-x)\bigg\}
  \,\mbox{,}
  \\ \\ 
  %C4g
  \displaystyle \Delta c_{4,g}^{(1)}(x) 
  &=& 
  0
  \,\mbox{,} 
  \\ \\
  %C1q
  \displaystyle \Delta c_{1,q}^{(1)}(x) 
  &=& 
  \displaystyle 2C_F\,
  \bigg\{
  2\left[\frac{\ln(1-x)}{1-x}\right]_+ 
  - \frac{3}{2}\left[\frac{1}{1-x}\right]_+
  - (1+x)\ln(1-x)
  \\ \\ 
  & &\displaystyle - \frac{1+x^2}{1-x}\ln x
  + 2 + x
  -\left(\frac{9}{2} + 2\zeta_2\right)\delta(1-x)\bigg\}
  \,\mbox{,}
  \\ \\ 
  %C1g
  \displaystyle \Delta c_{1,g}^{(1)}(x) 
  &=&
  \displaystyle 4T_R
  \bigg\{(2x - 1)\ln \frac{1-x}{x} - 4x +3 \bigg\} 
  \,\mbox{,}
  \\ \\ 
  %CLq
  \displaystyle \Delta c_{L,q}^{(1)}(x) 
  &=& 
  2C_F\, 2x 
  \,\mbox{,}
  \\ \\
  %CLg
  \displaystyle \Delta c_{L,g}^{(1)}(x) 
  &=& 
  0 
  \,\mbox{.}
\end{array}
\end{equation}

In the NC case the couplings can be written as:
\begin{equation}
\begin{array}{l}
\displaystyle B_q(Q^2) = -e_qA_q(V_e\pm \lambda A_e)P_Z+V_qA_q(V_e^2+A_e^2\pm2\lambda V_eA_e)P_Z^2 \,\mbox{,}\\
\\
\displaystyle D_q(Q^2) = \pm\frac12 \lambda e_q^2 - e_qV_q(A_e\pm\lambda V_e)P_Z +\frac12(V_q^2+A_q^2)\left[2V_eA_e\pm\lambda (V_e^2+A_e^2)\right]P_Z^2\,\mbox{.}
\end{array}
\end{equation}
where $\lambda$ corresponds to the polarisation of the incoming
lepton. It should be stressed that $B_q$ multiplies $g_4$ and $g_L$
while $D_q$ multiplies $g_1$.

% \section{The $\chi$ Prescription in FONLL}

% As is well known, the original formulation of the FONLL matched scheme
% gives rise to discontinuities in correspondence of the heavy quark
% thresholds arising from uncontrolled subleading terms. Such subleading
% terms can however be numerically important especially arond the charm
% threshold where the numerical value of the strong coupling $\alpha_s$
% is large. In order to remedy this unwanted feature different
% prescriptions have been introduced and traditionally the FONLL schem
% DIS has been implemented using the so-called damping factor which
% directly suppresses the unwanted subleading terms by means of a
% function that goes smoothly to zero at the threshold and below and
% tends to one for energies much larger than the threshold itself.

% As an alternative to the damping factor, one can damp the subleading
% terms close to the threshold by mimicing in the subtraction terms the
% phase-space suppression given by the presence of one or more heavy
% quarks in the final state. This is easily done juct by introducing the
% so-called slow-rescaling variable $\chi$, that in the NC case is:
% \begin{equation}
% \chi=x\left(1+\frac{4m_H^2}{Q^2}\right)=\frac{x}{\eta}\,,
% \end{equation}
% $m_H$ being the mass of the heavy quark, in the convolution between
% coefficient functions and PDFs in the zero-mass and in the
% massless-limit bits of the FONLL structure function. In other words,
% the usual zero-mass Mellin convolution becomes:
% \begin{equation}
% x\int_x^1\frac{dy}{y}C\left(\frac{x}{y}\right)f(y)\rightarrow x\int_\chi^1\frac{dy}{y}C\left(\frac{\chi}{y}\right)f(y)=x\int_\chi^1\frac{dy}{y}C(y)f\left(\frac{\chi}{y}\right)\,.
% \end{equation}
% The question is how to treat the new integral on a discreet $x$-space
% grid. What we have done so far for the massive integrals like that in
% the r.h.s. of the equation above is re-express it in terms of the
% physical Bjorken $x$ as:
% \begin{equation}
% x\int_\chi^1\frac{dy}{y}C(y)f\left(\frac{\chi}{y}\right)
% \end{equation}

\newpage
\bibliographystyle{ieeetr}
\bibliography{bibliography}

\end{document}
