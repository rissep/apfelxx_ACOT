%% LyX 2.0.3 created this file.  For more info, see http://www.lyx.org/.
%% Do not edit unless you really know what you are doing.
\documentclass[twoside,english]{paper}
\usepackage{lmodern}
\renewcommand{\ttdefault}{lmodern}
\usepackage[T1]{fontenc}
\usepackage[latin9]{inputenc}
\usepackage[a4paper]{geometry}
\geometry{verbose,tmargin=3cm,bmargin=2.5cm,lmargin=2cm,rmargin=2cm}
\usepackage{color}
\usepackage{babel}
\usepackage{float}
\usepackage{bm}
\usepackage{amsthm}
\usepackage{amsmath}
\usepackage{amssymb}
\usepackage{graphicx}
\usepackage{esint}
\usepackage[unicode=true,pdfusetitle,
 bookmarks=true,bookmarksnumbered=false,bookmarksopen=false,
 breaklinks=false,pdfborder={0 0 0},backref=false,colorlinks=false]
 {hyperref}
\usepackage{breakurl}

\makeatletter

%%%%%%%%%%%%%%%%%%%%%%%%%%%%%% LyX specific LaTeX commands.
%% Because html converters don't know tabularnewline
\providecommand{\tabularnewline}{\\}

%%%%%%%%%%%%%%%%%%%%%%%%%%%%%% Textclass specific LaTeX commands.
\numberwithin{equation}{section}
\numberwithin{figure}{section}

%%%%%%%%%%%%%%%%%%%%%%%%%%%%%% User specified LaTeX commands.
\usepackage{babel}

\@ifundefined{showcaptionsetup}{}{%
 \PassOptionsToPackage{caption=false}{subfig}}
\usepackage{subfig}
\makeatother


\usepackage{listings}

\begin{document}

\title{The APFEL DIS Module}

\author{Valerio Bertone$^{a,b}$}

\institution{$^{a}$PH Department, TH Unit, CERN, CH-1211 Geneva 23, Switzerland\\$^{b}$Rudolf Peierls Centre for Theoretical Physics, 1 Keble Road, University of Oxford,
OX1 3NP Oxford, UK}
\maketitle

\begin{abstract} In this document I will descrive the old and the new
DIS module embedded in APFEL.
\end{abstract}
\tableofcontents{}

\newpage{}

\section{Computing DIS Structure Functions on a Grid: the New DIS Module}

In order to speed up and optimize the compution of the DIS structure
functions in {\tt APFEL} we decided to use the same technology used
for the PDF evolution. In fact, up to version 2.0.0, the computation
of such observables in {\tt APFEL} was perfomed by directly
convoluting PDFs with the coefficient functions by mean of a numerical
integration.

Now the aim is that of precomputing on a grid the convolution of the
coefficient functions with a set of interpolating polynomials. This
way, the time consuming task of precomputing the coefficient functions
of the grid needs to be done only once and the numerical convolution
with any PDF set is instead very fast. In addition, as we will see
below, this approach provides a very natural framework to combine
precomputed coefficient functions and evolution operators, so that any
prediction of structure functions at any scale $Q$ can be obtained
very quickly by convolution with PDFs at some initial scale $Q_0$.
Ultimately, this is particularly suitable for PDF fits.

\subsection{Zero Mass Structure Functions}

A structure function in the Zero-Mass (ZM) scheme is given by the
following convolution:
\begin{equation}
  F(x,Q) = \sum_{i=g,q}x\int_x^1\frac{dy}y
  C_i\left(\frac{x}{y},\alpha_s(Q)\right)q_i(y,Q)\,,
\end{equation}
Now, defining $t = \ln (Q^2)$,
$\widetilde{C}_i^{(n)}(y,t) = yC_i^{(n)}(y,\alpha_s(Q))$ and
$\widetilde{q}_i(y,t) = y q_i(y,Q)$, the integral above can be written
as:
\begin{equation}\label{exact}
  F(x,t) = \sum_{i=g,q}\int_x^1\frac{dy}y
  \widetilde{C}_i\left(\frac{x}{y},t\right)\widetilde{q}_i(y,t)\,.
\end{equation}
But, using a suitable interpolation basis, we can write:
\begin{equation}
  \tilde{q}_i(y,t)=\sum^{N_{x}}_{\alpha=0}w_{\alpha}^{(k)}(y)\tilde{q}_i(x_{\alpha},t)\,,
\end{equation}
so that eq. (\ref{exact}) becomes:
\begin{equation}
  F(x,t) =
  \sum_{i=g,q}\sum^{N_{x}}_{\alpha=0}\left[\int_x^1\frac{dy}y
    \widetilde{C}_i\left(\frac{x}{y},t\right)
    w_{\alpha}^{(k)}(y)\right]\tilde{q}_i(x_{\alpha},t)\,.
\end{equation}
Now let's assume that $x$ is on the grid, so that $x = x_\beta$. This
way we have:
\begin{equation}
  F(x_\beta,t) =
  \sum_{i=g,q}\sum^{N_{x}}_{\alpha=0}\underbrace{\left[\int_{x_\beta}^1\frac{dy}y
      \widetilde{C}_i\left(\frac{x_\beta}{y},t\right)
      w_{\alpha}^{(k)}(y)\right]}_{\Gamma_{i,\beta\alpha}(t)}\tilde{q}(x_{\alpha},t)\,.
\end{equation}

Using the same arguments presented in the evolution code notes, we
have that:
\begin{equation} 
  \Gamma_{i,\beta\alpha}(t) \neq 0
  \quad\mbox{for}\quad\beta\leq\alpha\,,
\end{equation}
and:
\begin{equation}
  \Gamma_{i,\beta\alpha}(t) = \int_{c}^d\frac{dy}y
  \widetilde{C}_i\left(y,t\right)
  w_{\alpha}^{(k)}\left(\frac{x_\beta}{y}\right)
\end{equation}
with:
\begin{equation}\label{bounds2}
  c =
  \mbox{max}(x_\beta,x_\beta/x_{\alpha+1}) \quad\mbox{and}\quad d =
  \mbox{min}(1,x_\beta/x_{\alpha-k}) \,.
\end{equation}
The same symmetries holding for the splitting function case hold also
here.

\subsubsection{Coefficient Functions Treatment}

The structure of the DIS coefficient functions is very similar to
that of splitting functions with only one small complication, that is
the presence of a more divergent singular term. In practice the
structure of the DIS coefficient functions is the following:
\begin{equation}\label{CFstructure}
  \widetilde{C}_{i}(x,t) =
  xC_{i}^{R}(x,t) + xC_{i}^{S1}(t)\left[\frac{1}{1-x}\right]_+ +
  xC_{i}^{S2}(t)\left[\frac{\ln(1-x)}{1-x}\right]_+ +
  xC_{i}^{L}(t)\delta(1-x)\,.
\end{equation} 
The term proportional to $C_{i}^{S2}$ can be treated, considering
that:
\begin{equation}
  \begin{array}{rcl} \displaystyle
    \int_c^ddy\left[\frac{\ln(1-y)}{1-y}\right]_+f(y) &=&\displaystyle
                                                          \int_c^ddy\frac{\ln(1-y)}{1-y}\left[f(y)-f(1)\theta(d-1)\right]\\ \\
                                                      &+&\displaystyle \frac12 f(1) \ln^2(1-c)\theta(d-1)
\end{array}
\end{equation}
On the same line of splitting functions, we know that the coefficient
functions ha the following perturbative expansion:
\begin{equation}
  C_{i}^{J}(x,t) =
  \sum_{n=0}^{N}a_s^{n}(t)C_{i}^{J,(n)}(x)\qquad\mbox{with}\qquad
  J=R,S1,S2,L
\end{equation}

Therefore one has that:
\begin{equation}\label{Kernels}
\begin{array}{c}
  \displaystyle \Gamma_{j,\beta\alpha}(t) = \\ \\
  \displaystyle \sum_{n=0}^{N} a_s^{n}(t)
  \bigg\{\int^{d}_{c}dy\left[{C}_{i}^{R,(n)}(y)w_{\alpha}\left(\frac{x_\beta}{y}\right)+\frac{{C}_{i}^{S1,(n)}+{C}_{i}^{S2,(n)}\ln(1-y)}{1-y}\left(w_{\alpha}\left(\frac{x_\beta}{y}\right)-\delta_{\beta\alpha}\theta(d-1)\right)\right]\\
  \\ \displaystyle
  +\left[{C}_{i}^{S1,(n)}\ln(1-c)\theta(d-1)+\frac12{C}_{i}^{S2,(n)}\ln^2(1-c)\theta(d-1)+{C}_{i}^{L,(n)}\right]\delta_{\beta\alpha}\bigg\}\,.
\end{array}
\end{equation}

Calling:
\begin{equation}
\begin{array}{c} 
  \displaystyle \Gamma_{i,\beta\alpha}^{(n)}(t) = \\ \\
  \displaystyle
  \int^{d}_{c}dy\left[{C}_{i}^{R,(n)}(y)w_{\alpha}\left(\frac{x_\beta}{y}\right)+\frac{{C}_{i}^{S1,(n)}+{C}_{i}^{S2,(n)}\ln(1-y)}{1-y}\left(w_{\alpha}\left(\frac{x_\beta}{y}\right)-\delta_{\beta\alpha}\theta(d-1)\right)\right]\\
  \\ \displaystyle
  +\left[{C}_{i}^{S1,(n)}\ln(1-c)\theta(d-1)+\frac12{C}_{i}^{S2,(n)}\ln^2(1-c)\theta(d-1)+{C}_{i}^{L,(n)}\right]\delta_{\beta\alpha}\,,
\end{array}
\end{equation}
 we have that:
\begin{equation}\label{splittingexp} 
  \Gamma_{i,\beta\alpha}(t) =
  \sum_{n=0}^{N} a_s^{n}(t) \Gamma_{i,\beta\alpha}^{(n)}\,,
\end{equation} 
and the integrals $\Gamma_{i,\beta\alpha}^{(n)}$ do not depend on the
energy therefore, once the grid (and the number of active flavours)
has been fixed, they can be evaluate once and for all at the beginning
and used for the convolution at any scale.

Now, assuming to have computed the evolution operator
$M_{ij,\alpha\beta}(t,t_0)$ between the scales $t=\ln(Q^2)$ and
$t_0=\ln(Q_0^2)$ on the same grid where we have computed the operator
$\Gamma_{i,\beta\alpha}(t)$, one can esily combine the two obtaining
the prediction for the structure function $F$ on the grid in terms of
PDFs at the initial scale $Q_0$ just by performing the following
convolution:
\begin{equation}\label{CFtimesEvolution}
  F(x_\alpha,t) =
  \Gamma_{i,\alpha\beta}(t) M_{ij,\beta\gamma}(t,t_0)
  \tilde{q}_j(x_{\gamma},t_0)
\end{equation} 
where a sum of the repeated indeces is understood.

Before proceeding to treatment of the massive coefficient functions,
we stress that in the massless scheme, for obviuos kinematical
reasons, there is no need to distinguish between charged- and
neutral-current coefficient functions. The difference between the two
cases appears only at the level of structure functions where the
coefficient functions are comvoluted with different combinations of
PDFs and combined according to the structure of the couplings to
quarks of the $Z/\gamma^*$ vector bosons in the neutral-current case
and the $W^\pm$ in the charged-current case.

It is opportune at this point to mention that, when considering
charged-current observables at NLO in the massive scheme, there is a
further contribution to be added to eq.~(\ref{CFstructure}) that has
the form:
\begin{equation} C_{i}^{SL}(t)\frac{d}{dx}\delta(1-x)\,.
\end{equation} Starting from the relation:
\begin{equation} x\frac{d}{dx}\delta(x) = -\delta(x)\,,
\end{equation} one can easily show that:
\begin{equation}\label{KeyIdentity} \frac{d}{dx}\delta(1-x) =
\left[\frac{\delta(1-x)}{1-x}\right]_+\,.
\end{equation} To make sure that this identity is correct, we try to
convolute both the r.h.s. and the l.h.s. of eq.~(\ref{KeyIdentity})
with the test function $f(x)$, such that $f(1) = 0$, to see what is
the result and whether the results are equal. Using the l.h.s. we
have:
\begin{equation} \int_x^1dy\,f(y)\,\frac{d}{dx}\delta(1-y) =
\underbrace{f(y)\delta(1-y)\Big{|}_{x}^1}_{=0} -
\int_x^1dy\frac{df(y)}{dy}\delta(1-y) =
-\frac{df(y)}{dy}\bigg{|}_{y=1}\,,
\end{equation} while using the r.h.s.(\footnote{Since the delta
function selects the point $y=1$ in the following integral, the
``incomplete'' integral of plus-prescripted function does not give
rise to any residual logarithm of the form $\ln(1-x)$.}):
\begin{equation}
\begin{array}{c}
  \displaystyle
  \int_x^1dy\,f(y)\,\left[\frac{\delta(1-y)}{1-y}\right]_+ =
  \int_x^1dy\frac{f(y)-f(1)}{1-y}\delta(1-y) = \lim_{\epsilon\rightarrow
  0^+} \int_x^1dy\frac{f(y)-f(1)}{1-y}\delta(1-\epsilon-y) =\\ \\
  \displaystyle - \lim_{\epsilon\rightarrow 0^+}
  \frac{f(1)-f(1-\epsilon)}{\epsilon} =
  -\frac{df(y)}{dy}\bigg{|}_{y=1}\,.
\end{array}
\end{equation} 
So the results are equal and the distributions in
eq.~(\ref{KeyIdentity}) when convoluted with a test function extract
its derivative in $y=1$, up to a minus sign.

At the end of the day one has to include inside the curly brackets of
eq.~(\ref{Kernels}) the term:
\begin{equation}
  - C_{i}^{SL,(n)}\frac{dw_\alpha^{(k)}(x_\beta)}{dx}
\end{equation}

In addition, when using a Lagrange interpolation, one can show that
the first derivative of the Lagrange polynomials have the form:
\begin{equation}\label{LagrangFirstDerivative}
  \frac{d
    w_\alpha^{(k)}(x_\rho)}{dx} = \left\{
    \begin{array}{ll} \displaystyle \sum_{\sigma=0\atop
      \sigma\neq\alpha}^k\frac{1}{x_\alpha-x_\sigma} & \quad \rho = \alpha
      \\ \\ \displaystyle \frac{1}{x_\alpha-x_\rho}
      \prod_{\sigma=0\atop\sigma\neq\alpha,\rho}^k\frac{x_\rho -
      x_\sigma}{x_\alpha-x_\sigma} & \quad \rho \neq \alpha
    \end{array} \right.
\end{equation}
The relation in eq.~(\ref{LagrangFirstDerivative}) is proved in the
``Lagrange\_derivative.pdf'' notes.

\subsection{Massive Structure Functions}

Now we can proceed considering the massive structure functions. When
computing structure functions in the massive scheme, there is a
further complication that complicates a fast precomputation of the
cefficient functions on the $x$-space grid and it is the fact that the
coefficients of the perturbative expansion of the massive coefficient
functions carry an intrinsic dependence on the scale of the process.
This prevents a scale independent pre-tabulation of the coefficient
functions on on $x$-space grid.

One possible way out is to pre-tabulate the coefficient functions, not
only on an $x$-space grid, but also on a $Q$-space grid, where $Q$ is
the scale at which the structure functions are evaluated. Actually the
most efficient way of precomputing the massive coefficient functions
is on $\xi$-space grid, where $\xi$ is defined as:
\begin{equation}
  \xi = \frac{Q^2}{m_H^2}\,.
\end{equation}
where $m_H$ is the mass of the heavy quark under consideration.  In
fact, for dimentional reasons, massive coefficient functions depend on
the scale $Q$ through $\xi$. Neglecting for the moment the dependence
on the renormalization and on the factorization scales, the massive
coefficient functions allow for the following expansion:
\begin{equation}
  C_i(x,Q,m_H) = \sum_n a_s^n(t)C_i^{(n)}(x,\xi)\,.
\end{equation}
Given this relation, the massive analogous of the
eq.~(\ref{splittingexp}) is:
\begin{equation}\label{splittingexp}
  \Gamma_{i,\beta\alpha}(Q,m_H) =
  \sum_n a_s^{n}(t) \Gamma_{i,\beta\alpha}^{(n)}(\xi)\,.
\end{equation}
In order not to recompute the operator
$\Gamma_{i,\beta\alpha}^{(n)}(\xi)$ any time that $\xi$ changes, we
can tabulate the on a grid in $\xi$,
$\{\xi_1,\dots,\xi_\tau,\dots,\xi_{N_\xi}\}$, defining:
\begin{equation}
  \Gamma_{i,\beta\alpha,\tau}^{(n)} =
  \Gamma_{i,\beta\alpha}^{(n)}(\xi_\tau),.
\end{equation}
and then interpolate to obtain the operator for a generic value of
$\xi$. We have chosen to use a linear interpolation so that:
\begin{equation}
  \Gamma_{i,\beta\alpha}^{(n)}(\xi) =
  c^{(0)}(\xi)\Gamma_{i,\beta\alpha,\tau}^{(n)} +
  c^{(1)}(\xi)\Gamma_{i,\beta\alpha,\tau+1}^{(n)}\,,
\end{equation}
with:
\begin{equation}
  c^{(0)}(\xi) = \frac{\ln\xi_{\tau+1} -
    \ln\xi}{\ln\xi_{\tau+1} - \ln\xi_\tau}\quad\mbox{and}\quad
  c^{(1)}(\xi) = \frac{\ln\xi - \ln\xi_\tau}{\ln\xi_{\tau+1} -
    \ln\xi_\tau} \,,
\end{equation}
provided that $\xi_\tau \leq \xi < \xi_{\tau+1}$.

To conclude, once the operators $\Gamma_{i,\beta\alpha,\tau}^{(n)}$
have been precomputed, the operator for a generic value of $\xi$ can
be quickly computed by interpolation.

\subsubsection{Neutral Current Coefficient Functions}

As far as the neutral current coefficient functions are concerned,
beyond LO(\footnote{We remind that, in the neutral current case, the
LO is order $\alpha_s$.}), a close analytical form is not available
and only a semi-analitical form~\cite{Laenen:1992xs} which is not
suitable for a fast numerical implementation. The authors of
Ref.~\cite{Alekhin:2003ev} have used a simple parametrization to fit
the exact coefficient functions. Such parametrization is actually
meant to be used in Mellin space, however it can equally be used in
$x$ space providing a fast and accurate enough alternative to the
original implemetation. The parametrization of
Ref.~\cite{Alekhin:2003ev} has the form:
\begin{equation}
  C(x,\xi) =
  \theta(\rho-x)(\rho-x)^{-\kappa}\sum_{k=0}^K
  a_k(\rho)x^k\quad\mbox{with}\quad\rho = \frac{\xi}{\xi+4}\,,
\end{equation}
and the authors provide the numerical values of $\kappa$, $K$ and
$a_k(\rho)$ for all the relevant coefficient functions at LO
($\mathcal{O}(\alpha_s)$) and NLO ($\mathcal{O}(\alpha_s^2)$)
tabulated on a $\xi$-space grid for large enough range in $\xi$.  Note
the presence of the $\theta$-function that has the scope of reducing
the phase scace available for the process due the production of two
heavy quarks in the final state.

In {\tt APFEL} we make use of the parametrization above only for the
NLO coefficient functions as the exact form of the LO ones is
available in Ref.~\cite{Forte:2010ta} and compact enough for an
efficient implementation. In addition, also for the pure singlet NLO
coefficient functions (sometimes called gluon-radiation terms) we
employ the analytical expressions given in Appendix A of
Ref.~\cite{Buza:1995ie}.

We finally remark that massive coeffient functions for the neutral
current structure functions are presently known only for $F_2$ and
$F_L$. For the parity-violating structure function $F_3$ we thus use
the massless coefficient functions.

As far as the massless limit of the massive (massive-zero) coefficient
functions is concerned, exact expressions up to
$\mathcal{O}(\alpha_s^2)$ have been evaluated in
Ref.~\cite{Buza:1995ie} and reported in Appendix D. Such expressions
are implemented in {\tt APFEL}

As in the massive case, massive-zero coefficient functions are know
only for $F_2$ and $F_L$ and again for the $F_3$ structure function we
use the massless coefficient functions.

\subsubsection{Charged Current Coefficient Functions}

We can now consider the charged-current sector. In this case, massive
coeffincient functions are know only up to $\mathcal{O}(\alpha_s)$
(NLO), therefore a proper computation of charged-current structure
functions the NNLO version of the FONLL scheme (called FONLL-C) is
impossible. However, the best we can do when computing charged-current
structure functions in the FONLL-C scheme is to set the NNLO
contributions to zero in the massive sectors but keeping those in the
massless sector, as well as using NNLO evolution for PDFs and
$\alpha_s$.

The charged-current massive structure functions, like the other
structure functions, are given by the convolution of PDFs with
coefficient functions. Considering the heavy-quark $H$ structure
functions in the approximation of diagonal CKM matrix, the definitions
are:
\begin{equation}\label{F1}
  F_1^H(x,Q,m_H)=\frac12\int_{\chi}^{1}\frac{dy}{y}\left[C_{1,q}(y,\xi)s\left(\frac{\chi}{y},Q\right)+C_{1,g}(y,\xi)g\left(\frac{\chi}{y},Q\right)\right]
\end{equation}
\begin{equation}\label{F2}
  F_2^H(x,Q,m_H)=\chi\int_{\chi}^{1}\frac{dy}{y}\left[C_{2,q}(y,\xi)s\left(\frac{\chi}{y},Q\right)+C_{2,g}(y,\xi)g\left(\frac{\chi}{y},Q\right)\right]
\end{equation}
\begin{equation}\label{F3}
  F_3^H(x,Q,m_H)=\int_{\chi}^{1}\frac{dy}{y}\left[C_{3,q}(y,\xi)s\left(\frac{\chi}{y},Q\right)+C_{3,g}(y,\xi)g\left(\frac{\chi}{y},Q\right)\right]
\end{equation}
with:
\begin{equation}
  \chi = x\left(1+\frac{m_H^2}{Q^2}\right) =
  \frac{x}{\lambda}\,,
\end{equation}
where:
\begin{equation}
  \lambda = \frac{Q^2}{Q^2+m_H^2} =
  \frac{\xi}{1+\xi}\,.
\end{equation}
Now, defining:
\begin{equation}
  F_L^H(x,Q,m_H) = F_2^H(x,Q,m_H) - 2xF_1^H(x,Q,m_H)\,,
\end{equation}
we have that:
\begin{equation}\label{FL}
  F_L^H(x,Q,m_H)=\chi\int_{\chi}^{1}\frac{dy}{y}\left[C_{L,q}(y,\xi)s\left(\frac{\chi}{y},Q\right)+C_{L,g}(y,\xi)g\left(\frac{\chi}{y},Q\right)\right]\,,
\end{equation}
where we have defined:
\begin{equation}\label{clll}
  C_{L,q(g)}(y,\xi) =
  C_{2,q(g)}(y,\xi)-\lambda C_{1,q(g)}(y,\xi)
\end{equation}

All the coefficient functions entering the structure functions above
admit a perturbative expansion that at N$^N$LO reads:
\begin{equation}
  C_{k,q(g)}(y,\xi) = \sum_{n=0}^N a_s^n(Q)
  C_{k,q(g)}^{(n)}(y,\xi)\,,\quad k = 1,2,3,L\,.
\end{equation}
In the following we will truncate the expansion at NLO.

After the definitions above we can write down, first the LO
coefficient functions. While at LO the gluon coefficient functions are
all zero ($C_{k,g}^{(0)}(y,\xi)=0$), the quark coefficient functions
are:
\begin{equation}
\begin{array}{l}
  \displaystyle C^{(0)}_{1,q}(x,\xi) = \delta(1-x)\,,\\
  \\ \displaystyle C^{(0)}_{2,q}(x,\xi) = \delta(1-x) \,,\\ \\
  \displaystyle C^{(0)}_{3,q}(x,\xi) = \delta(1-x) \,,\\ \\
  \displaystyle C^{(0)}_{L,q}(x,\xi) = (1-\lambda)\delta(1-x) \,.
\end{array}
\end{equation}

The NLO charged-current massive coefficient have been computed and
reported in Appendix A of Ref.~\cite{Gluck:1996ve}. However, before
being implemented in {\tt APFEL} they need some manipulation. We start
defining:
\begin{equation}
  K_A=\frac{1}{\lambda}(1-\lambda)\ln(1-\lambda)\,.
\end{equation}
In addition, in order to consider factorization scale variations, we
also need to consider the splitting function:
\begin{equation}
  P_{qq}^{(0)}(z) =
  C_F\left[\frac{1+z^2}{(1-z)_+}+\frac32\delta(1-z)\right]=C_F\left[\frac{2}{(1-z)_+}-(1+z)+\frac32\delta(1-z)\right]\,,
\end{equation}
and we also define:
\begin{equation}
  K_F^2 =\frac{Q^2}{\mu_F^2}\,.
\end{equation}

The explicit expressions of the NLO quark coefficient functions read:
\begin{equation}
\begin{array}{rcl}
  C^{(1)}_{1,q}&=&\displaystyle 2C_F \bigg\{
                   \bigg(-4-\frac{1}{2\lambda}-2\zeta_2-\frac{1+3\lambda}{2\lambda}K_A+\frac32
                   \ln\frac{K_F^2}{\lambda}\bigg)\delta(1-z)\\ \\ &-&\displaystyle
                                                                      \frac{(1+z^2)\ln z}{1-z} +
                                                                      \left(-\ln\frac{K_F^2}{\lambda}-2\ln(1-z)+\ln(1-\lambda
                                                                      z)\right)(1+z)+(3-z)+\frac{1}{\lambda^2}+\frac{z-1}{\lambda}\\ \\
               &+&\displaystyle 2 \left[\frac{2\ln(1-z)-\ln(1-\lambda
                   z)}{1-z}\right]_++
                   2\left(-1+\ln\frac{K_F^2}{\lambda}\right)\left[\frac{1}{1-z}\right]_+\\
  \\ &+& \displaystyle
         \frac{\lambda-1}{\lambda^2}\left[\frac{1}{1-\lambda z}\right]_+
         +\frac{1}{2}\left[\frac{1-z}{(1-\lambda z)^2}\right]_+\bigg\}
\end{array}\,,
\end{equation}

\begin{equation}
\begin{array}{rcl}
  C^{(1)}_{2,q}&=&\displaystyle 2C_F \bigg\{
                   \bigg(-4-\frac{1}{2\lambda}-2\zeta_2-\frac{1+\lambda}{2\lambda}K_A+\frac32
                   \ln\frac{K_F^2}{\lambda}\bigg)\delta(1-z)\\ \\ &-&\displaystyle
                                                                      \frac{(1+z^2)\ln z}{1-z} +
                                                                      \left(2-\ln\frac{K_F^2}{\lambda}-2\ln(1-z)+\ln(1-\lambda
                                                                      z)\right)(1+z)+\frac{1}{\lambda}\\ \\ &+&\displaystyle 2
                                                                                                                \left[\frac{2\ln(1-z)-\ln(1-\lambda z)}{1-z}\right]_++
                                                                                                                2\left(-1+\ln\frac{K_F^2}{\lambda}\right)\left[\frac{1}{1-z}\right]_+\\
  \\ &+& \displaystyle
         \frac{2\lambda^2-\lambda-1}{\lambda}\left[\frac{1}{1-\lambda
         z}\right]_+ +\frac{1}{2}\left[\frac{1-z}{(1-\lambda
         z)^2}\right]_+\bigg\}
\end{array}\,,
\end{equation}

\begin{equation}
\begin{array}{rcl}
  C^{(1)}_{3,q}&=&\displaystyle 2C_F \bigg\{
                   \bigg(-4-\frac{1}{2\lambda}-2\zeta_2-\frac{1+3\lambda}{2\lambda}K_A+\frac32
                   \ln\frac{K_F^2}{\lambda}\bigg)\delta(1-z)\\ \\ &-&\displaystyle
                                                                      \frac{(1+z^2)\ln z}{1-z} +
                                                                      \left(1-\ln\frac{K_F^2}{\lambda}-2\ln(1-z)+\ln(1-\lambda
                                                                      z)\right)(1+z)+\frac{1}{\lambda}\\ \\ &+&\displaystyle 2
                                                                                                                \left[\frac{2\ln(1-z)-\ln(1-\lambda z)}{1-z}\right]_++
                                                                                                                2\left(-1+\ln\frac{K_F^2}{\lambda}\right)\left[\frac{1}{1-z}\right]_+\\
  \\ &+& \displaystyle \frac{\lambda-1}{\lambda}\left[\frac{1}{1-\lambda
         z}\right]_+ +\frac{1}{2}\left[\frac{1-z}{(1-\lambda
         z)^2}\right]_+\bigg\}
\end{array}\,,
\end{equation}

\begin{equation}
\begin{array}{rcl}
  C^{(1)}_{L,q} &=&\displaystyle 2C_F
                    (1-\lambda)\bigg\{
                    \bigg(-4-\frac{1}{2\lambda}-2\zeta_2-\frac{1+\lambda}{2\lambda}K_A+\frac32
                    \ln\frac{K_F^2}{\lambda}\bigg)\delta(1-z)\\ \\ &-&\displaystyle
                                                                       \frac{(1+z^2)\ln z}{1-z} +
                                                                       \left(-\ln\frac{K_F^2}{\lambda}-2\ln(1-z)+\ln(1-\lambda
                                                                       z)\right)(1+z)+3\\ \\ &+&\displaystyle 2
                                                                                                 \left[\frac{2\ln(1-z)-\ln(1-\lambda z)}{1-z}\right]_++ 2
                                                                                                 \left(-1+\ln\frac{K_F^2}{\lambda}\right)\left[\frac{1}{1-z}\right]_+\\
  \\ &-& \displaystyle 2\left[\frac{1}{1-\lambda z}\right]_+
         +\frac{1}{2}\left[\frac{1-z}{(1-\lambda z)^2}\right]_+\bigg\} + 2C_F
         \left[\lambda K_A \delta(1-z) + (1+\lambda)z\right]
\end{array}\,.
\end{equation}

In order to proceed with our manipulations we need to define the
generalized or incomplete +-prescription:
\begin{equation}
\begin{array}{c}
  \displaystyle \int_x^1 dz\left[f(z)\right]_+g(z) =
  \int_x^1
  dz\,f(z)\left[g(z)-g(1)\right]-g(1)\underbrace{\int_0^xdz\,f(z)}_{-R_f(x)}=\\
  \\ \displaystyle \int_x^1
  dz\left\{\left[f(z)\right]_++R_f(x)\delta(1-z)\right\}g(z)\,.
\end{array}
\end{equation} 
where the +-prescription in the r.h.s of the equation above should be
understood in the usual way independently of the integration bounds.

Often the residual $R_f(x)$ function can be evaluated analytically by
performing the integral, however sometimes it need to be evaluated
numerically performing the integral in a numerical way. In particular
the $+$-prescripted functions that enter the expressions above give
rise to the following residual functions that can be computed
analytically:
\begin{equation}
  -\int_0^x\frac{dz}{1-z} = \ln(1-x)\,,
\end{equation}
\begin{equation}
  -\int_0^xdz\frac{\ln(1-z)}{1-z} = \frac12
  \ln^2(1-x)\,,
\end{equation}
\begin{equation}
  -\int_0^x\frac{dz}{1-\lambda z} =
  \frac{1}{\lambda}\ln(1-\lambda x)\,,
\end{equation}
\begin{equation}
  -\int_0^xdz\frac{1-z}{(1-\lambda z)^2} =
  \frac{1}{\lambda^2}\ln(1-\lambda
  x)+\frac{1-\lambda}{\lambda}\frac{x}{1-\lambda x}\,,
\end{equation}
while we do not know how to solve analytically the integral:
\begin{equation}
  R(x)=-\int_0^xdz\frac{\ln(1-\lambda z)}{1-z}
\end{equation}
therefore we will compute it numerically.

As a consequence, when convoluting the coefficient functions above
with PDFs in the point $x$, we can treat the $+$-prescripted functions
using the standard definition at the price of adding to the local
terms the following functions:
\begin{equation}
\begin{array}{rcl}
  C^{(1)}_{1,q}&\rightarrow& \displaystyle
                             C^{(1)}_{1,q} +
                             2C_F\bigg[2\ln^2(1-x)-2R(x)+2\left(-1+\ln\frac{K_F^2}{\lambda}\right)\ln(1-x)\\
  \\ &&\displaystyle +\frac{\lambda-1}{\lambda^3}\ln(1-\lambda
        x)+\frac{1}{2\lambda^2}\ln(1-\lambda
        x)+\frac{1-\lambda}{2\lambda}\frac{x}{1-\lambda x}\bigg]\delta(1-z)
\end{array}\,,
\end{equation}
\begin{equation}
\begin{array}{rcl}
  C^{(1)}_{2,q}&\rightarrow& \displaystyle
                             C^{(1)}_{2,q} +
                             2C_F\bigg[2\ln^2(1-x)-2R(x)+2\left(-1+\ln\frac{K_F^2}{\lambda}\right)\ln(1-x)\\
  \\ &&\displaystyle
        +\frac{2\lambda^2-\lambda-1}{\lambda^2}\ln(1-\lambda
        x)+\frac{1}{2\lambda^2}\ln(1-\lambda
        x)+\frac{1-\lambda}{2\lambda}\frac{x}{1-\lambda x}\bigg]\delta(1-z)
\end{array}\,,
\end{equation}
\begin{equation}
\begin{array}{rcl}
  C^{(1)}_{3,q}&\rightarrow& \displaystyle
                             C^{(1)}_{3,q} +
                             2C_F\bigg[2\ln^2(1-x)-2R(x)+2\left(-1+\ln\frac{K_F^2}{\lambda}\right)\ln(1-x)\\
  \\ &&\displaystyle +\frac{\lambda-1}{\lambda^2}\ln(1-\lambda
        x)+\frac{1}{2\lambda^2}\ln(1-\lambda
        x)+\frac{1-\lambda}{2\lambda}\frac{x}{1-\lambda x}\bigg]\delta(1-z)
\end{array}\,,
\end{equation}
\begin{equation}
\begin{array}{rcl}
  C^{(1)}_{L,q}&\rightarrow& \displaystyle
                             C^{(1)}_{L,q} +
                             2C_F(1-\lambda)\bigg[2\ln^2(1-x)-2R(x)+2\left(-1+\ln\frac{K_F^2}{\lambda}\right)\ln(1-x)\\
  \\ &&\displaystyle -\frac{2}{\lambda}\ln(1-\lambda
        x)+\frac{1}{2\lambda^2}\ln(1-\lambda
        x)+\frac{1-\lambda}{2\lambda}\frac{x}{1-\lambda x}\bigg]\delta(1-z)
\end{array}\,.
\end{equation}

Now let us consider the gluon coefficient functions, that read:
\begin{equation}
\begin{array}{rcl}
  C^{(1)}_{1,g}&=&\displaystyle
                   2T_R\bigg\{[z^2+(1-z)^2]\left[\ln\left(\frac{1-z}{z}\right)
                   -\frac12\ln(1-\lambda) +\frac12\ln\frac{K_F^2}{\lambda} \right]+\\ \\
               &&\displaystyle 4z(1-z) - 1+\\ \\ &&\displaystyle
                                                    (1-\lambda)\left[-4z(1-z) + \frac{z}{1-\lambda z} +2z(1-2\lambda
                                                    z)\ln\frac{1-\lambda z}{(1-\lambda)z}\right]\bigg\}\\
\end{array}\,,
\end{equation}
\begin{equation}
\begin{array}{rcl}
  C^{(1)}_{2,g}&=&\displaystyle 2T_R
                   \bigg\{[z^2+(1-z)^2]\left[\ln\left(\frac{1-z}{z}\right)
                   -\frac12\ln(1-\lambda) +\frac12\ln\frac{K_F^2}{\lambda} \right]+\\ \\
               &&\displaystyle 8z(1-z)- 1+\\ \\ &&\displaystyle
                                                   (1-\lambda)\left[-6(1+2\lambda) z(1-z)+\frac{1}{1-\lambda z} +
                                                   6\lambda z(1-2\lambda z)\ln\frac{1-\lambda
                                                   z}{(1-\lambda)z}\right]\bigg\}\\
\end{array}\,,
\end{equation}
\begin{equation}
\begin{array}{rcl}
  C^{(1)}_{3,g}&=&\displaystyle 2T_R
                   \bigg\{[z^2+(1-z)^2]\left[ 2\ln\left(\frac{1-z}{1-\lambda
                   z}\right)+\frac12\ln(1-\lambda)
                   +\frac12\ln\frac{K_F^2}{\lambda}\right]+\\ \\ &&\displaystyle
                                                                    (1-\lambda)\left[2 z(1-z) - 2z[1-(1+\lambda )z]\ln\frac{1-\lambda
                                                                    z}{(1-\lambda)z}\right]\bigg\}\,,
\end{array}
\end{equation}
\begin{equation}
\begin{array}{rcl}
  C^{(1)}_{L,g}&=&\displaystyle 2T_R
                   \bigg\{(1-\lambda)[z^2+(1-z)^2]\left[\ln\left(\frac{1-z}{z}\right)
                   -\frac12\ln(1-\lambda) +\frac12\ln\frac{K_F^2}{\lambda} \right]+\\ \\
               &&\displaystyle 4(2-\lambda)z(1-z)+\\ \\ &&\displaystyle
                                                           (1-\lambda)\left[-2(3+4\lambda) z(1-z)+ 4\lambda z(1-2\lambda
                                                           z)\ln\frac{1-\lambda z}{(1-\lambda)z}\right]\bigg\}\,.
\end{array}
\end{equation}
Since these functions do not contain any $+$-prescripted functions,
they can be implemented as they are.

We now consider the massless limit of the above massive coefficient
functions. We start considering that:
\begin{equation}
\begin{array}{l} 
  \lambda \rightarrow 1\\ K_A \rightarrow 0
\end{array}\,,
\end{equation}
as consequence we find that the quark coefficient functions tend to:
\begin{equation}
\begin{array}{rcl}
  C^{(1)}_{1,q}
  \displaystyle\mathop{\longrightarrow}_{m_H\rightarrow 0}
  C^{0,(1)}_{1,q} &=&\displaystyle 2C_F \bigg\{
                      -\left(\frac{9}{2}+2\zeta_2-\frac32 \ln K_F^2\right)\delta(1-z)\\ \\
                  &-&\displaystyle \frac{(1+z^2)\ln z}{1-z} -\left(\ln(1-z)+\ln
                      K_F^2\right)(1+z)+3\\ \\ &+&\displaystyle 2
                                                   \left[\frac{\ln(1-z)}{1-z}\right]_+ -\left(\frac{3}{2}-2\ln
                                                   K_F^2\right)\left[\frac{1}{1-z}\right]_+\bigg\}
\end{array}
\end{equation}
\begin{equation}
\begin{array}{rcl}
  C^{(1)}_{2,q}\displaystyle\mathop{\longrightarrow}_{m_H\rightarrow
  0}C^{0,(1)}_{2,q}&=&\displaystyle 2C_F \bigg\{
                       -\left(\frac{9}{2}+2\zeta_2-\frac32 \ln K_F^2\right)\delta(1-z)\\ \\
                   &-&\displaystyle \frac{(1+z^2)\ln z}{1-z} -\left(\ln(1-z)+\ln
                       K_F^2\right)(1+z)+2z+3\\ \\ &+&\displaystyle 2
                                                       \left[\frac{\ln(1-z)}{1-z}\right]_+ -\left(\frac{3}{2}-2\ln
                                                       K_F^2\right)\left[\frac{1}{1-z}\right]_+\bigg\}
\end{array}
\end{equation}
\begin{equation}
\begin{array}{rcl}
  C^{(1)}_{3,q}\displaystyle\mathop{\longrightarrow}_{m_H\rightarrow 0}
  C^{0,(1)}_{3,q}&=&\displaystyle 2C_F \bigg\{
                      -\left(\frac{9}{2}+2\zeta_2-\frac32 \ln K_F^2\right)\delta(1-z)\\ \\
                  &-&\displaystyle \frac{(1+z^2)\ln z}{1-z} -\left(\ln(1-z)+\ln
                      K_F^2\right)(1+z)+z+2\\ \\ &+&\displaystyle 2
                                                     \left[\frac{\ln(1-z)}{1-z}\right]_+ -\left(\frac{3}{2}-2\ln
                                                     K_F^2\right)\left[\frac{1}{1-z}\right]_+\bigg\}
\end{array}
\end{equation}
\begin{equation} C^{(1)}_{L,q}\mathop{\longrightarrow}_{m_H\rightarrow
0} C^{0,(1)}_{L,q} = 4C_F z
\end{equation}

The local term to be added to the quark coefficient functions,
considering that:
\begin{equation}
  R(x) \mathop{\longrightarrow}_{m_H\rightarrow 0}
  \frac12\ln(1-x)^2\,,
\end{equation}
are:
\begin{equation}
  C^{0,(1)}_{1,q}\rightarrow C^{0, (1)}_{1,q} +
  2C_F\left[\ln^2(1-x)-\left(\frac32-2\ln
      K_F^2\right)\ln(1-x)\right]\delta(1-z)\,,
\end{equation}
\begin{equation}
  C^{0, (1)}_{2,q}\rightarrow C^{0, (1)}_{2,q} +
  2C_F\left[\ln^2(1-x)-\left(\frac32-2\ln
      K_F^2\right)\ln(1-x)\right]\delta(1-z)\,,
\end{equation}
\begin{equation}
  C^{0, (1)}_{3,q}\rightarrow C^{0, (1)}_{3,q} +
  2C_F\left[\ln^2(1-x)-\left(\frac32-2\ln
      K_F^2\right)\ln(1-x)\right]\delta(1-z)\,,
\end{equation}
while no local term needs to be added to $C^{0, (1)}_{L,q}$.

Now we turn to the gluon coefficient functions where we need to know
that:
\begin{equation} 
  \ln(1-\lambda)
  \mathop{\longrightarrow}_{m_H\rightarrow 0}
  -\ln\left(\frac{Q^2}{m_H^2}\right)
\end{equation}
so that:
\begin{equation}
  C^{(1)}_{1,g}\mathop{\longrightarrow}_{m_H\rightarrow
    0} C^{0, (1)}_{1,g} =
  2T_R\left\{[z^2+(1-z)^2]\left[\ln\left(\frac{1-z}{z}\right) +\frac12
      \ln\left(\frac{Q^2}{m_H^2}\right) +\frac12\ln K_F^2\right]+ 4z(1-z) -
    1\right\}\,,
\end{equation}
\begin{equation}
  C^{(1)}_{2,g}\mathop{\longrightarrow}_{m_H\rightarrow
    0} C^{0, (1)}_{2,g} =
  2T_R\left\{[z^2+(1-z)^2]\left[\ln\left(\frac{1-z}{z}\right) +\frac12
      \ln\left(\frac{Q^2}{m_H^2}\right) +\frac12\ln K_F^2\right]+ 8z(1-z) -
    1\right\}\,,
\end{equation}
\begin{equation}
  C^{(1)}_{3,g}\mathop{\longrightarrow}_{m_H\rightarrow
    0} C^{0, (1)}_{3,g} = 2T_R[z^2+(1-z)^2]\left[-\frac12
    \ln\left(\frac{Q^2}{m_H^2}\right) +\frac12\ln K_F^2\right]\,,
\end{equation}
\begin{equation}
  C^{(1)}_{L,g}\mathop{\longrightarrow}_{m_H\rightarrow
    0} C^{0,(1)}_{L,g} = 2T_R\left[4z(1-z)\right]\,.
\end{equation}
We also note that in the limit $m_H\rightarrow 0$, the covolution
integrals in eqs.~(\ref{F1}), (\ref{F2}), (\ref{F3}) and~(\ref{FL})
will extend from $x$ to 1 rather than from $\chi$ to 1.

As clear from the definitions in eqs.~(\ref{F1}), (\ref{F2}),
(\ref{F3}) and~(\ref{FL}), in order to compute a give structure
functions for some given value of $x$, one needs to convolute the
coefficient functions that we have written above with PDFs in the
rescaled point $\chi=x/\lambda > x$, so in particular the convolution
integral extends from $\chi$ to 1. This is a kinematical consequence
of the mass of the heavy quark involved that reduces the phase space
available for the process.

From the point of view of the implementation of the FONLL scheme in
{\tt APFEL}, given that the massive scheme needs to be combined with
the massless and the massive-zero schemes whose convolution integrals
extend from $x$ to 1, it would be convinient to rewrite
eqs.~(\ref{F1}), (\ref{F2}), (\ref{F3}) and~(\ref{FL}) in such a way
that the lower interagration bound is $x$ rather than $\chi$. To this
end, let us consider the integral:
\begin{equation}
  I=\int_\chi^1\frac{dy}{y}
  C(y)f\left(\frac{\chi}{y}\right)\,,
\end{equation}
where $\chi=x/\lambda$. By the change of integration variable
$z = \lambda y$, we can rewrite the integral above as:
\begin{equation}
  I=\int_x^\lambda\frac{dz}{z}
  C\left(\frac{z}{\lambda}\right)f\left(\frac{x}{y}\right) =
  \int_x^1\frac{dz}{z}
  \widetilde{C}(z,\lambda)f\left(\frac{x}{y}\right)\,,
\end{equation}
where:
\begin{equation}
  \widetilde{C}(z,\lambda)=\theta(\lambda-z)C\left(\frac{z}{\lambda}\right)\,.
\end{equation}
In this way we have achived the goal of expressing the ``reduced''
convolution in eqs.~(\ref{F1}), (\ref{F2}), (\ref{F3}) and~(\ref{FL})
as a ``standard'' convolution between $x$ and 1. The price to pay is
to consider the massive coefficient functions during the integration
as explicit functions of the variable $z/\lambda$ and to cut off the
region $z>\lambda$ by means of the Heaviside $\theta$-function.  Of
course, this does not neet to be done in the massive-zero case as the
convolution already extends between $x$ and $1$.

\section{Target Mass Corrections}

Kinematic corrections due to the finite mass of the target proton
$M_p$ which recoils against the photon might be relevant in the
small-energy region. The leading contributions to such corrections
have been computed long time ago in Ref.~\cite{Georgi:1976ve} and,
denoting the target-mass corrected struture functions with
$\widetilde{\quad}$, they take the form:
\begin{equation}
\begin{array}{rcl}
  \displaystyle \widetilde{F}_2(x,Q) &=&
                                         \displaystyle \frac{x^2}{\xi^2 \tau^{3/2}} F_2(\xi,Q) + \frac{6\rho
                                         x^3}{\tau^2} I_2(\xi,Q)\,,\\ \\ \displaystyle \widetilde{F}_L(x,Q) &=&
                                                                                                                \displaystyle F_L(\xi,Q)+\frac{x^2(1-\tau)}{\xi^2 \tau^{3/2}}
                                                                                                                F_2(\xi,Q) + \frac{\rho x^3(6-2\tau)}{\tau^2} I_2(\xi,Q)\,,\\ \\
  \displaystyle x\widetilde{F}_3(x,Q) &=& \displaystyle \frac{x^2}{\xi^2
                                          \tau} \xi F_3(\xi,Q) + \frac{4\rho x^3}{\tau^{3/2}} I_3(\xi,Q)\,,
\end{array}
\end{equation}
where:
\begin{equation}
  \rho = \frac{M_p^2}{Q^2}\,,\qquad \tau = 1 + 4\rho
  x^2\,,\qquad \xi = \frac{2x}{1+\sqrt{\tau}}\,,
\end{equation}
and:
\begin{equation}
  I_2(\xi,Q) = \int_\xi^1dy \frac{F_2(y,Q)}{y^2}\,,
  \qquad I_3(\xi,Q) = \int_\xi^1dy \frac{yF_3(y,Q)}{y^2}\,.
\end{equation}
Using the interpolation formula, we have that:
\begin{equation}
  \frac{F_2(y,Q)}{y^2} = \sum_{\alpha}
  \frac{F_2(x_\alpha,Q)}{x_\alpha^2} w_\alpha^{(k)}(y)\,,
\end{equation}
therefore:
\begin{equation}
  I_2(\xi,Q) = \sum_{\alpha}
  \frac{F_2(x_\alpha,Q)}{x_\alpha^2} \int_\xi^1dy\,w_\alpha^{(k)}(y)\,,
\end{equation}
while:
\begin{equation}
  I_3(\xi,Q) = \sum_{\alpha} \frac{x_\alpha
    F_3(x_\alpha,Q)}{x_\alpha^2} \int_\xi^1dy\,w_\alpha^{(k)}(y)\,.
\end{equation} In turn, again using the interpolation formula, we can
write:
\begin{equation}
  J_\alpha(\xi)\equiv\int_\xi^1dy\,w_\alpha^{(k)}(y) =
  \sum_{\beta=0}^{N_x}
  \underbrace{\left[\int_{x_\beta}^1dy\,w_\alpha^{(k)}(y)\right]}_{J_{\beta\alpha}}
  w_\beta^{(k)}(\xi)\,.
\end{equation}
Considering the fact that:
\begin{equation}\label{nonzero}
  w_{\alpha}^{(k)}(y) \neq 0
  \quad\mbox{for}\quad x_{\alpha-k} < y < x_{\alpha+1}\,,
\end{equation}
it follows that:
\begin{equation}
  J_{\beta\alpha} = 0\quad\mbox{for}\quad \beta >
  \alpha
\end{equation}
and thus:
\begin{equation}
  J_\alpha(\xi)=\sum_{\beta=0}^\alpha J_{\beta\alpha}
  w_\beta^{(k)}(\xi)\,.
\end{equation}
In addition, we can simplify the integral as follows:
\begin{equation}
  J_{\beta\alpha} = \int_c^d dy\,w_\alpha^{(k)}(y)\,,
\end{equation}
where:
\begin{equation}
  c =
  \mbox{max}(x_\beta,x_{\alpha-k})\quad\mbox{and}\quad d =
  \mbox{min}(1,x_{\alpha+1})\,.
\end{equation}
In conclusion, we can treat $J_{\beta\alpha}$ exactly in the same
manner as the regular part of a massless coefficient function or a
splitting function and thus it can be precomputed and stored.

At the end of the day we have that:
\begin{equation}
  I_2(\xi,Q) = \sum_{\alpha=0}^{N_x}
  \sum_{\beta=0}^\alpha w_\beta^{(k)}(\xi) J_{\beta\alpha}
  \frac{F_2(x_\alpha,Q)}{x_\alpha^2} \,,
\end{equation}
that can also be written as:
\begin{equation}
  I_2(\xi,Q) = \sum_{\beta=0}^{N_x}
  \underbrace{\left[\sum_{\alpha=\beta}^{N_x} J_{\beta\alpha}
      \frac{F_2(x_\alpha,Q)}{x_\alpha^2}\right]}_{I_2(x_\beta,Q)}
  w_\beta^{(k)}(\xi) \,.
\end{equation}
Similarly:
\begin{equation}
  I_3(\xi,Q) = \sum_{\beta=0}^{N_x}
  \underbrace{\left[\sum_{\alpha=\beta}^{N_x} J_{\beta\alpha}
      \frac{x_\alpha F_3(x_\alpha,Q)}{x_\alpha^2}\right]}_{I_3(x_\beta,Q)}
  w_\beta^{(k)}(\xi) \,.
\end{equation}
Gathering all pieces we finally find:
\begin{equation}
\begin{array}{rcl}
  \displaystyle \widetilde{F}_2(x,Q) &=&
                                         \displaystyle \sum_{\beta=0}^{N_x} \left[\frac{x^2}{\xi^2 \tau^{3/2}}
                                         F_2(x_\beta,Q) + \frac{6\rho x^3}{\tau^2}
                                         I_2(x_\beta,Q)\right]w_\beta^{(k)}(\xi)\,,\\ \\ \displaystyle
  \widetilde{F}_L(x,Q) &=& \displaystyle \sum_{\beta=0}^{N_x} \left[
                           F_L(x_\beta,Q)+\frac{x^2(1-\tau)}{\xi^2 \tau^{3/2}} F_2(x_\beta,Q) +
                           \frac{\rho x^3(6-2\tau)}{\tau^2} I_2(x_\beta,Q)
                           \right]w_\beta^{(k)}(\xi)\,,\\ \\ \displaystyle x\widetilde{F}_3(x,Q)
                                     &=& \displaystyle \sum_{\beta=0}^{N_x} \left[ \frac{x^2}{\xi^2 \tau}
                                         x_\beta F_3(x_\beta,Q) + \frac{4\rho x^3}{\tau^{3/2}} I_3(x_\beta,Q)
                                         \right]w_\beta^{(k)}(\xi)\,.
\end{array}
\end{equation}

For the equations above is clear that in the case when $M_p = 0$, that
implies $\rho = 0$, $\tau = 1$ and $\xi = x$, all structure functions
reduce to the uncorrected formulas.

When considering the extraction of the DIS operator times the
evolution operator like in eq.~(\ref{CFtimesEvolution}), one should be
careful with $I_2$ and $I_3$. Condidering that:
\begin{equation}
  F(x_\alpha,Q) =
  \sum_{\gamma,\delta}\sum_{i,j}\Gamma_{i,\alpha\gamma}(Q)
  M_{ij,\gamma\delta}(Q,Q_0) \tilde{q}_j(x_{\delta},Q_0)
\end{equation}
we have that:
\begin{equation}
  I(x_\beta,Q) = \sum_{\alpha=\beta}^{N_x}
  J_{\beta\alpha} \frac{F(x_\alpha,Q)}{x_\alpha^2} =
  \sum_{\gamma,\delta}\sum_{i,j} \left[\sum_{\alpha=\beta}^{N_x}
    \frac{J_{\beta\alpha}\Gamma_{i,\alpha\gamma}(Q) }{x_\alpha^2}\right]
  M_{ij,\gamma\delta}(Q,Q_0)\tilde{q}_j(x_{\delta},Q_0)\,,
\end{equation}

\section{Renormalization and Factorization Scale Variation}

In the previous sections, when discussing the implementation of the
structure functions in {\tt APFEL}, we assumed that the
renormalization scale $\mu_R$ and the factorization scale $\mu_F$ were
identified to the scale pf the process $Q$. In this section, we want
to relax this assumption and to do so we the expansion of the DGLAP
and RG equation for $\alpha_s$ up to NLO, that is:
\begin{equation}
  \frac{\partial f_{i}}{\partial\ln\mu_F^2} =
  \frac{\alpha_s(\mu_F)}{4\pi}\left[ P_{ij}^{(0)}(x) +
    \frac{\alpha_s(\mu_F)}{4\pi} P_{ij}^{(1)}(x) + \dots\right]\otimes
  f_j(x,\mu_F)\,,
\end{equation}
and:
\begin{equation}
  \frac{\partial
  }{\partial\ln\mu_R^2}\left(\frac{\alpha_s}{4\pi}\right) =
  -\left(\frac{\alpha_s(\mu_R)}{4\pi}\right)^2\left[ \beta_0 +
    \frac{\alpha_s(\mu_R)}{4\pi}\beta_1 + \dots\right]\,.
\end{equation}
Defining:
\begin{equation}
  \xi_R\equiv\frac{\mu_R}{Q}\,,\quad\xi_F\equiv\frac{\mu_F}{Q}\quad\mbox{and}\quad
  a_s=\frac{\alpha_s}{4\pi}
\end{equation}
where $Q$ is constant, and defining:
\begin{equation}
  t_R \equiv \ln\xi_R^2\quad\mbox{and}\quad t_F \equiv
  \ln\xi_F^2\,,
\end{equation}
the equations above can be written as:
\begin{equation}\label{DGLAPsimp}
  \frac{\partial f_{i}}{\partial t_F}
  = a_s(t_F)\left[ P_{ij}^{(0)} + a_s(t_F) P_{ij}^{(1)} +
    \dots\right]\otimes f_j(t_F)\,,
\end{equation}
and:
\begin{equation}\label{BETAsimp}
  \frac{\partial a_s}{\partial t_R} =
  -a_s^2(t_R)\left[ \beta_0 + a_s(t_R)\beta_1 + \dots\right]\,.
\end{equation}
Now, expanding $f_i(t)$ around $t=t_F$ we have:
\begin{equation}\label{expDGLAP}
  f_i(t) = f_i(t_F)+\frac{\partial
    f_{i}}{\partial t}\bigg|_{t=t_F} (t-t_F) + \frac12 \frac{\partial^2
    f_{i}}{\partial t^2}\bigg|_{t=t_F} (t-t_F)^2 + \dots
\end{equation}
Using eqs.~(\ref{DGLAPsimp}) and~(\ref{BETAsimp}), we have that:
\begin{equation}
\begin{array}{l} 
  \displaystyle \frac{\partial f_{i}}{\partial
  t}\bigg|_{t=t_F} = \left[ a_s(t_F) P_{ij}^{(0)} + a_s^2(t_F)
  P_{ij}^{(1)}\right]\otimes f_j(t_F) + \mathcal{O}(a_s^3)\\ \\
  \displaystyle \frac{\partial^2 f_{i}}{\partial t^2}\bigg|_{t=t_F} =
  a_s^2(t_F)\left[ P_{il}^{(0)}\otimes P_{lj}^{(0)} - \beta_0
  P_{ij}^{(0)} \right]\otimes f_j(t_F) + \mathcal{O}(a_s^3)
\end{array}
\end{equation}
Chosing $t=0$ in eq.~(\ref{expDGLAP}), we finally have:
\begin{equation}\label{expDGLAP1}
  f_i(0) = \left\{1-a_s(t_F) t_F
    P_{ij}^{(0)} + a_s^2(t_F)\left[-t_F P_{ij}^{(1)}+ t_F^2 \frac12\left(
        P_{il}^{(0)}\otimes P_{lj}^{(0)} - \beta_0 P_{ij}^{(0)}
      \right)\right]\right\}\otimes f_j(t_F) + \mathcal{O}(a_s^3)\,.
\end{equation}
Now, using eq.~(\ref{BETAsimp}), we easily find:
\begin{equation}\label{BETAexp} 
  a_s(t_F)=
  a_s(t_R)\left[1+a_s(t_R)\beta_{0}(t_R-t_F)+\mathcal{O}(a_s^2)\right]\,,
\end{equation}
which can be plugged into eq.~(\ref{expDGLAP1}) to give:
\begin{equation}\label{expDGLAP2}
  f_i(0) = \left\{1-a_s(t_R) t_F
    P_{ij}^{(0)} + a_s^2(t_R)\left[-t_F P_{ij}^{(1)}+ t_F^2 \frac12\left(
        P_{il}^{(0)}\otimes P_{lj}^{(0)} + \beta_0 P_{ij}^{(0)} \right)-
      t_Ft_R\beta_{0}P_{ij}^{(0)}\right]\right\}\otimes f_j(t_F) +
  \mathcal{O}(a_s^3)\,.
\end{equation}
Finally, setting $t_F=0$ in eq.~(\ref{BETAexp}), we find:
\begin{equation}\label{BETAexp1} 
  a_s(0)=
  a_s(t_R)\left[1+a_s(t_R)\beta_{0}t_R+\mathcal{O}(a_s^2)\right]\,,
\end{equation}

Considering that and NNLO the ZM structure functions are written in
terms of PDFs and coefficient functions as:
\begin{equation}\label{NonZeroScales}
  F(t_R,t_F) / x = \left[\sum_{k=0}^{2} a_s^k(t_R)
    \widetilde{\mathcal{C}}_i^{(k)}(t_R,t_F)\right]\otimes f_i(t_F)
  +\mathcal{O}(a_s^3)\,,
\end{equation}
and that, up to subleading terms, the structure functions must be
renormalization and factorization scale independent, this requires
that:
\begin{equation}\label{invariance}
  F(t_R,t_F) = F(0,0)\,.
\end{equation} 
But since:
\begin{equation}\label{ZeroScales}
  F(0,0) / x = \left[\sum_{k=0}^{2}
    a_s^k(0) \widetilde{C}_i^{(k)}\right]\otimes
  f_i(0)+\mathcal{O}(a_s^3)\,,
\end{equation} 
where $\widetilde{C}_i^{(k)}$ are the well-know ZM coefficient
functions, using eqs.~(\ref{expDGLAP2}) and~(\ref{BETAexp1}) in
eq.~(\ref{ZeroScales}) and finally imposing the identity in
eq.~(\ref{invariance}), we can find the explicit espression of the
``generalized'' coefficient functions
$\widetilde{\mathcal{C}}_i^{(k)}(t_R,t_F)$. In fact:
\begin{equation}
  \begin{array}{rcl}
    F(0,0)/x &=&\bigg\{ \widetilde{C}_j^{(0)} \\ \\
             &+&\displaystyle a_s(t_R)\left[\widetilde{C}_j^{(1)}- t_F
                 \widetilde{C}_i^{(0)} \otimes P_{ij}^{(0)}\right]\\ \\
             &+&\displaystyle a_s^2(t_R)\bigg[\widetilde{C}_j^{(2)} + t_R\beta_0
                 \widetilde{C}_j^{(1)} -t_F \left(\widetilde{C}_i^{(0)} \otimes
                 P_{ij}^{(1)}+\widetilde{C}_i^{(1)} \otimes P_{ij}^{(0)}\right)\\ \\
             &+&\displaystyle \frac{t_F^2}2 \widetilde{C}_i^{(0)} \otimes \left(
                 P_{il}^{(0)}\otimes P_{lj}^{(0)} + \beta_0 P_{ij}^{(0)} \right)-
                 t_Ft_R\beta_{0}\widetilde{C}_i^{(0)} \otimes
                 P_{ij}^{(0)}\bigg]\bigg\}\otimes f_j(t_F)+\mathcal{O}(a_s^3)\,.
\end{array}
\end{equation}
Finally, using the identity in eq.~(\ref{invariance}), it is easy to
find that:
\begin{equation}\label{generalizedCF}
\begin{array}{rcl}
  \displaystyle
  \widetilde{\mathcal{C}}_j^{(0)}(t_R,t_F) &=& \displaystyle
                                               \widetilde{C}_j^{(0)} \\ \\ \displaystyle
  \widetilde{\mathcal{C}}_j^{(1)}(t_R,t_F) &=& \displaystyle
                                               \widetilde{C}_j^{(1)}-t_F \widetilde{C}_i^{(0)} \otimes P_{ij}^{(0)}
  \\ \\ \displaystyle \widetilde{\mathcal{C}}_j^{(2)}(t_R,t_F) &=&
                                                                   \displaystyle \widetilde{C}_j^{(2)} + t_R\beta_0 \widetilde{C}_j^{(1)}
                                                                   -t_F \left(\widetilde{C}_i^{(0)} \otimes
                                                                   P_{ij}^{(1)}+\widetilde{C}_i^{(1)} \otimes P_{ij}^{(0)}\right)\\ \\
                                           &+&\displaystyle\frac{t_F^2}2 \widetilde{C}_i^{(0)} \otimes \left(
                                               P_{il}^{(0)}\otimes P_{lj}^{(0)} + \beta_0 P_{ij}^{(0)} \right)-
                                               t_Ft_R\beta_{0}\widetilde{C}_i^{(0)} \otimes P_{ij}^{(0)}\,.
\end{array}
\end{equation}
Therefore, in the ZM-VFNS, in order to perform scale variation we need
to precompute the additional convolutions:
$\widetilde{C}_i^{(0)} \otimes P_{ij}^{(0)}$,
$\widetilde{C}_i^{(0)} \otimes P_{ij}^{(1)}$,
$\widetilde{C}_i^{(1)} \otimes P_{ij}^{(0)}$ and
$\widetilde{C}_i^{(0)} \otimes P_{il}^{(0)} \otimes P_{lj}^{(0)}$.

In order to proceed, it is opportune to specify the basis in which
PDFs are expressed. As usual, the most natural choice is the QCD
evolution basis
$\{\Sigma,g,V,V_{3},V_{8},V_{15},V_{24},V_{35},T_{3},T_{8},V_{15},T_{24},T_{35}\}$
and thus the indices $i$, $j$ and $l$ in eq.~(\ref{generalizedCF}) run
between 1 and 13 over this basis. The advantage of this basis is the
fact that the splitting function matrix $P_{ij}$ is almost completely
diagonalized. The starting point, is the usual definition that, up to
a factor $x$ and omitting the convolution symbol $\otimes$, can be
written as:
\begin{equation}\label{StructFuncDef}
F=\langle e_q^2 \rangle \left\{C_gg +\sum_{i=u}^{t}\underbrace{\theta(Q^2-m_i^2)\left[C_{\rm PS}+\frac{e_i^2}{\langle e_q^2
      \rangle}C_+\right]}_{\hat{C}_i}q_i^+\right\}\,,
\end{equation}
where:
\begin{equation}
\langle e_q^2 \rangle = \sum_{i=u}^t e_i^2\theta(Q^2-m_i^2)\,.
\end{equation}
Now, in order to express the structunre function in
eq.~(\ref{StructFuncDef}) in the evolution basis, we need to find the
tranformation such that:
\begin{equation}\label{Rotation}
q_i^+ = \sum_{j=1}^6T_{ij}f_j\,,
\end{equation}
where $f_j$ belongs to the evolution basis, that is: $f_1=\Sigma$,
$f_2=T_3$, $f_3=T_8$ and so on. One can show that the
trasformation matrix $T_{ij}$ can be written as:
\begin{equation}\label{TransDef}
\begin{array}{l}
\displaystyle T_{ij}=\theta(j-i)\frac{1-\delta_{ij}j}{j(j-1)}\quad j\geq 2\,,\\
\\
\displaystyle T_{i1} = \frac{1}{6}\,,
\end{array}
\end{equation}
with $\theta(j-i)=1$ for $j\geq i$ and zero otherwise.
In addition, one can show that:
\begin{equation}
\sum_{j=1}^6T_{ij} = 0\,,\quad\mbox{and}\quad \sum_{i=1}^6T_{ij} = \delta_{1j}\,.
\end{equation}
Now, we can plug eq.~(\ref{Rotation}) into eq.~(\ref{StructFuncDef})
and, using eq.~(\ref{TransDef}), we get:
\begin{equation}\label{StructFuncDefEvol}
F=\langle e_q^2 \rangle \left\{C_gg +\frac16\left(C_++n_f C_{\rm PS}\right)\Sigma+\sum_{j=2}^{6}\frac{1}{j(j-1)}\left[\sum_{i=1}^j\hat{C}_i-j\hat{C}_j\right]f_j\right\}\,,
\end{equation}
where we have transmuted the sum over $u$, $d$ and so on into a sum
between 1 and 6 and where we have defined:
\begin{equation}
n_f = \sum_{i=1}^6\theta(Q^2-m_i^2)\,.
\end{equation}
Now we need to express the term in sqauer brackets in terms of the
usual coefficient functions $C_+$ and $C_{\rm PS}$, in particular:
\begin{equation}
\sum_{i=1}^j\hat{C}_i-j\hat{C}_j = \sum_{i=1}^j \theta(Q^2-m_i^2)\left(C_{\rm PS}+\frac{e_i^2}{\langle e_q^2
      \rangle}C_+\right) - j \theta(Q^2-m_j^2)\left(C_{\rm PS}+\frac{e_j^2}{\langle e_q^2
      \rangle}C_+\right)\,.
\end{equation}
Here we can distinguish two case, the first is $Q^2<m_j^2$ and under
this assumption we have:
\begin{equation}
\sum_{i=1}^j\hat{C}_i-j\hat{C}_j = C_++n_fC_{\rm PS}\,.
\end{equation}
If instead $Q^2 \geq m_j^2$, then:
\begin{equation}
\sum_{i=1}^j\hat{C}_i-j\hat{C}_j = K_j C_+\,,
\end{equation}
with:
\begin{equation}
K_j=\frac{1}{\langle e_q^2\rangle}\left(\sum_{i=1}^{j}e_i^2-je_j^2\right)=\frac{1}{\langle e_q^2\rangle}\left(\sum_{i=1}^{j-1}e_i^2-(j-1)e_j^2\right)\,.
\end{equation}
We can express both cases in one single formula as:
\begin{equation}\label{SumCoef}
\sum_{i=1}^j\hat{C}_i-j\hat{C}_j = \theta(m_j^2-Q^2-\epsilon)\left[ C_++n_f C_{\rm PS}\right]
+\theta(Q^2-m_j^2)\left[K_jC_+\right]\,.
\end{equation}
where $\epsilon$ is a small parameter that ensures that the case
$Q^2=m_j^2$ is included in the second term of the r.h.s. of
eq.~(\ref{SumCoef}). Eq.~(\ref{SumCoef}) can aslo be written as:
\begin{equation}\label{SumCoef}
\sum_{i=1}^j\hat{C}_i-j\hat{C}_j = \theta(n_f-j)\left[K_jC_+\right]+\theta(j-n_f-1)\left[ C_++n_f C_{\rm PS}\right]\,.
\end{equation}
In addition, one can easily see that:
\begin{equation}\label{SingletReduction}
f_j = \theta(n_f-j)f_j+\theta(j-n_f-1)\Sigma\,,
\end{equation}
and thus:
\begin{equation}
\sum_{j=2}^{6}\frac{1}{j(j-1)}\left[\sum_{i=1}^j\hat{C}_i-j\hat{C}_j\right]f_j
= C_+\left[\sum_{j=2}^{nf}\frac{K_j}{j(j-1)}f_j\right]+\left[\sum_{j=n_f+1}^{6}\frac{1}{j(j-1)}\right] \left[ C_++n_f C_{\rm PS}\right]\Sigma\,,
\end{equation}
but:
\begin{equation}
\sum_{j=n_f+1}^{6}\frac{1}{j(j-1)}=\frac{1}{n_f}-\frac{1}{6},.
\end{equation}
Moreover:
\begin{equation}
\frac{K_j}{j(j-1)} = \frac{1}{\langle e_q^2\rangle}\frac{1}{j(j-1)}\left(\sum_{i=1}^je_i^2-je_j^2\right)=\frac{1}{\langle e_q^2\rangle}\underbrace{\frac{1}{j(j-1)}\sum_{i=1}^6e_i^2\left[\theta(j-i)-j\delta_{ij}\right]}_{d_j}\,.
\end{equation}
Finally, putting all pieces together, we find:
\begin{equation}\label{StructureFunctionEvol}
F=\langle e_q^2 \rangle \left[C_g g +\left(C_{\rm PS} +\frac1{n_f}C_+
  \right) \Sigma\right]+C_+\sum_{j=2}^{n_f} d_j f_j\,.
\end{equation}
It is interesting to separate the contributions coming from the
different flavors. To do so, we just need to separate the
contributions coming from, say the $k$-th charge $e_k^2$ and this is
easily done applying the following replacement:
\begin{equation}
e_i^2\rightarrow \delta_{ik}e_i^2\,.
\end{equation}
In this way we have that:
\begin{equation}
\langle e_q^2 \rangle \rightarrow \theta(Q^2-m_k^2) e_k^2\,,
\end{equation}
and:
\begin{equation}
d_j \rightarrow \frac{e_k^2\left[\theta(j-k)-j\delta_{kj}\right]}{j(j-1)}=\theta(Q^2-m_k^2) e_k^2\frac{\left[\theta(j-k)-j\delta_{kj}\right]}{j(j-1)}\,,
\end{equation}
so that the $k$-th component of the structure function $F$ is:
\begin{equation}
\begin{array}{rcl}
F^{(k)} &=&\displaystyle  \theta(Q^2-m_k^2) e_k^2\left\{\left[C_g g +\left(C_{\rm PS} +\frac1{n_f}C_+
  \right) \Sigma\right]+C_+\sum_{j=2}^{n_f}
            \frac{\left[\theta(j-k)-j\delta_{kj}\right]}{j(j-1)} f_j
            \right\}\\
\\
&=& \displaystyle \theta(Q^2-m_k^2) e_k^2\left\{\left[C_g g +\left(C_{\rm PS} +\frac1{n_f}C_+
  \right) \Sigma\right]-\frac{1}{k}C_+f_k+C_+\sum_{j=k+1}^{n_f}
            \frac{1}{j(j-1)} f_j
            \right\}
\end{array}
\end{equation}
and it is such that:
\begin{equation}
F = \sum_{k=1}^6 F^{(k)}\,.
\end{equation}

In {\tt APFEL} we split the total structure functions into a light
component and three heavy quark components. The light components is
defined as:
\begin{equation}
F^l = \sum_{k=1}^3 F^{(k)}= \langle e_l^2 \rangle \left[C_g g +\left(C_{\rm PS} +\frac1{n_f}C_+
  \right) \Sigma\right]+C_+\sum_{j=2}^{n_f} d_j^{(l)} f_j\,.
\end{equation}
where:
\begin{equation}
\langle e_l^2\rangle = \sum_{i=1}^3e_i^2\,,
\end{equation}
and:
\begin{equation}
d_j^{(l)}=
\frac{1}{j(j-1)}\sum_{i=1}^3e_i^2\left[\theta(j-i)-j\delta_{ij}\right]=
\left\{
\begin{array}{ll}
\frac{1}{2}(e_u^2-e_d^2)\,,\quad& j= 2 \\
\\
\frac{1}{6}(e_u^2+e_d^2-2e_s^2)\,,\quad& j= 3 \\
\\
\frac{\langle e_l^2\rangle}{j(j-1)}\,,\quad & j\geq 4
\end{array}
\right.\,,
\end{equation}
no need of the $\theta$-functions as the scale $Q$ will always be above
the strange threshold. This way the explicit form of $F^l$ is:
\begin{equation}\label{LightSF}
F^l = \langle e_l^2 \rangle \left[C_g g +\left(C_{\rm PS} +\frac1{n_f}C_+
  \right) \Sigma\right]+\frac{1}{2}(e_u^2-e_d^2)C_+T_3+\frac{1}{6}(e_u^2+e_d^2-2e_s^2)C_+T_8+\langle e_l^2 \rangle C_+\sum_{j=4}^{n_f} \frac{1}{j(j-1)} f_j\,.
\end{equation}

The heavy-quark components are instead defined as:
\begin{equation}\label{HeavySF}
\begin{array}{rcl}
F^c &=& \displaystyle \theta(Q^2-m_c^2) e_c^2\left\{\left[C_g g +\left(C_{\rm PS} +\frac1{n_f}C_+
  \right) \Sigma\right]-\frac{1}{4}C_+T_{15}+C_+\sum_{j=5}^{n_f}
            \frac{1}{j(j-1)} f_j
            \right\}\,,\\
\\
F^b &=& \displaystyle \theta(Q^2-m_b^2) e_b^2\left\{\left[C_g g +\left(C_{\rm PS} +\frac1{n_f}C_+
  \right) \Sigma\right]-\frac{1}{5}C_+T_{24}+C_+\sum_{j=6}^{n_f}
            \frac{1}{j(j-1)} f_j
            \right\}\,,\\
\\
F^t &=& \displaystyle \theta(Q^2-m_t^2) e_t^2\left\{\left[C_g g +\left(C_{\rm PS} +\frac1{n_f}C_+
  \right) \Sigma\right]-\frac{1}{6}C_+T_{35}\right\}\,.
\end{array}
\end{equation}

To conclude the treatment of all the structure functions, we should
adda that eq.~(\ref{StructureFunctionEvol}) is valid only for $F_2$
and $F_L$. However, $F_3$ can be easily derived following the very
same steps with the only differences are that: the distributions
$\{\Sigma,T_3,T_8,T_{15},T_{24},T_{35}\}$ must be replaced with
$\{V,V_3,V_8,V_{15},V_{24},V_{35}\}$, $C_+$ must be replaced with
$C_-$, the gluon and the pure-singlet coefficient functions are
identically zero and the squared electric charges replaced with the
appropriate electroweak charges $c_i$. Following this recipe, we find:
\begin{equation}\label{StructureFunctionEvolF3}
F_3=\langle c_q^2 \rangle \frac1{n_f}C_- V+C_-\sum_{j=2}^{n_f} d_j g_j\,,
\end{equation}
where $g_j$ belongs to $\{V,V_3,V_8,V_{15},V_{24},V_{35}\}$.

Eq.~(\ref{StructureFunctionEvol}), explicitly written in
eqs.~(\ref{LightSF}) and~(\ref{HeavySF}), is the final result that
allows us to implement the scale variation formulae given in
eq.~(\ref{generalizedCF}) in {\tt APFEL}. The good aspect of
eq.~(\ref{StructureFunctionEvol}) if the fact that it is written in
terms of the fundamental coefficient functions $C_g$, $C_+$ and
$C_{\rm PS}$ and PDFs appear in the evolution basis where the
splitting-function matrix diagonalizes. In particular, up to
$\mathcal{O}(\alpha_s^2)$, we have that:
\begin{equation}
\begin{array}{ll}
  P_{ij}^{(k)} \rightarrow P_{ij}^{(k)} &\quad i,j=g,q(\Sigma)\\
  P_{ij}^{(k)} \rightarrow \delta_{ij}P_+^{(k)} &\quad i,j=T_{3},T_{8},V_{15},T_{24},T_{35}\\
  P_{ij}^{(k)} \rightarrow \delta_{ij}P_-^{(k)} &\quad i,j=V,V_{3},V_{8},V_{15},V_{24},V_{35}
\end{array}
\end{equation}
Also, defining:
\begin{equation}
C_q=C_{\rm PS} +\frac{1}{n_f}C_+\,,
\end{equation}
we can connect eq.~(\ref{ZeroScales}) and
eq.~(\ref{StructureFunctionEvol}) by observing that:
\begin{equation}
\begin{array}{ll} 
  \widetilde{C}_j^{(k)} \rightarrow \langle e_q^2\rangle C_j^{(k)}
  &\quad j=g,q(\Sigma)\\
  \widetilde{C}_j^{(k)}
  \rightarrow d_jC_+^{(k)} &\quad j=T_{3},T_{8},T_{15},T_{24},T_{35}\\
  \widetilde{C}_j^{(k)}\rightarrow d_jC_-^{(k)} &\quad j=V_{3},V_{8},V_{15},V_{24},V_{35}
\end{array}
\end{equation}
where we have also considered the ``minus'' distributions that appear
in the $F_3$ structure function. Of course, the same relations must
hold also for eq.~(\ref{NonZeroScales}):
\begin{equation}
\begin{array}{ll}
  \widetilde{\mathcal{C}}_j^{(k)} \rightarrow
  \langle e_q^2\rangle \mathcal{C}_j^{(k)} &\quad j=g,q(\Sigma)\\
  \widetilde{\mathcal{C}}_j^{(k)} \rightarrow d_j\mathcal{C}_+^{(k)}
                         &\quad j=T_{3},T_{8},T_{15},T_{24},T_{35}\\
  \widetilde{\mathcal{C}}_j^{(k)} \rightarrow d_j\mathcal{C}_-^{(k)}
                         &\quad j=V_{3},V_{8},V_{15},V_{24},V_{35}
\end{array}
\end{equation}
with
\begin{equation}
\mathcal{C}_q=\mathcal{C}_{\rm PS} +\frac{1}{n_f}\mathcal{C}_+\,.
\end{equation}
In addition, in the following, we will make use of the following identity:
\begin{equation}
  P_-^{(0)}=P_+^{(0)}=P_{qq}^{(0)}\,.
\end{equation}
Now, considering that $C_{\rm PS}$ starts at
$\mathcal{O}(\alpha_s^2)$, we can write:
\begin{equation}
\begin{array}{l}
  \displaystyle C_-^{(0)}(x) = C_+^{(0)}(x) =
  \Delta_{\rm SF}\delta(1-x)\\
  \displaystyle C_j^{(0)}(x) = \left(\Delta_{\rm
  SF}/n_f\right)\delta_{qj}\delta(1-x) \quad \mbox{for } j=q,g
\end{array}
\end{equation}
where $\Delta_{\rm SF}=1$ for $F_2$ and $F_3$ and
$\Delta_{\rm SF} = 0$ for $F_L$.  From eq.~(\ref{generalizedCF}) it
follows that:
\begin{equation}\label{NonSingletCF}
\begin{array}{rcl}
  \displaystyle \mathcal{C}_\pm^{(0)}(t_R,t_F) &=&
                                                   \displaystyle \Delta_{\rm SF}\delta(1-x) \\ \\ \displaystyle
  \mathcal{C}_\pm^{(1)}(t_R,t_F) &=& \displaystyle
                                     C_\pm^{(1)}-\Delta_{\rm SF} t_F P_{qq}^{(0)} \\ \\ \displaystyle
  \mathcal{C}_\pm^{(2)}(t_R,t_F) &=& \displaystyle C_\pm^{(2)} +
                                     t_R\beta_0 C_\pm^{(1)} -t_F \left(\Delta_{\rm SF}
                                     P_\pm^{(1)}+C_\pm^{(1)} \otimes P_{qq}^{(0)}\right)\\ \\
                                               &+&\displaystyle\Delta_{\rm SF} \frac{t_F^2}2 \left(
                                                   P_{qq}^{(0)}\otimes P_{qq}^{(0)} + \beta_0 P_{qq}^{(0)} \right)-
                                                   \Delta_{\rm SF} t_Ft_R\beta_{0}P_{qq}^{(0)}\,,
\end{array}
\end{equation}
that can be rearranged as:
\begin{equation}\label{NonSingletCF1}
\begin{array}{rcl}
  \displaystyle \mathcal{C}_\pm^{(0)}(t_R,t_F) &=&
                                                   \displaystyle \Delta_{\rm SF}\delta(1-x) \\ \\ \displaystyle
  \mathcal{C}_\pm^{(1)}(t_R,t_F) &=& \displaystyle
                                     C_\pm^{(1)}-\Delta_{\rm SF} t_F P_{qq}^{(0)} \\ \\ \displaystyle
  \mathcal{C}_\pm^{(2)}(t_R,t_F) &=& \displaystyle C_\pm^{(2)} +
                                     t_R\beta_0 C_\pm^{(1)} -t_F C_\pm^{(1)} \otimes P_{qq}^{(0)}\\ \\
                                               &+&\displaystyle\Delta_{\rm SF}\frac{t_F^2}2 \left(
                                                   P_{qq}^{(0)}\otimes P_{qq}^{(0)} - \beta_0 P_{qq}^{(0)} \right)-
                                                   \Delta_{\rm SF} t_F\left[P_\pm^{(1)} - (t_F-t_R)\beta_{0}
                                                   P_{qq}^{(0)}\right]\,.
\end{array}
\end{equation}
The term in square brackets in the r.h.s. of the third line
corresponds to what we would call $\mathcal{P}_\pm^{(1)}(t_R,t_F)$,
that is the NLO splitting function in the presence of scale variations
($\mu_R\neq\mu_F$). This quantity is already evaluated by {\tt APFEL}
and thus does not need to be recomputed.

Now, let us consider the singlet sector that, cosidering the fact
that becomes:
\begin{equation}\label{SingletCF}
\begin{array}{rcl}
  \displaystyle {\mathcal{C}}_j^{(0)}(t_R,t_F) &=&
                                                   \displaystyle \frac{\Delta_{\rm SF}}{n_f}\delta_{qj}\delta(1-x) \\ \\
  \displaystyle {\mathcal{C}}_j^{(1)}(t_R,t_F) &=& \displaystyle
                                                   {C}_j^{(1)}- \frac{\Delta_{\rm SF}}{n_f} t_F P_{qj}^{(0)} \\ \\ \displaystyle
  {\mathcal{C}}_j^{(2)}(t_R,t_F) &=& \displaystyle {C}_j^{(2)} +
                                     t_R\beta_0 {C}_j^{(1)} -t_F {C}_i^{(1)} \otimes P_{ij}^{(0)}\\ \\
                                               &+&\displaystyle\frac{\Delta_{\rm SF}}{n_f} \frac{t_F^2}2 \left(
                                                   P_{qi}^{(0)}\otimes P_{ij}^{(0)} - \beta_0 P_{qj}^{(0)} \right)-
                                                   \frac{\Delta_{\rm SF}}{n_f} t_F\widetilde{P}_{qj}^{(1)}\,.
\end{array}
\end{equation}
for $j=g,q$. Taking into account eq.~(\ref{NonSingletCF1}) and
considering also the fact that $C_{\rm PS}^{(0)}=C_{\rm PS}^{(1)}=0$
($i.e.$ $C_{\rm PS}$ is $\mathcal{O}(\alpha_s^2)$), it is easy to see
that:
\begin{equation}\label{PureSingletCF}
\begin{array}{rcl}
  \displaystyle {\mathcal{C}}_{\rm PS}^{(0)}(t_R,t_F)
  &=& \displaystyle 0 \\ \\ \displaystyle {\mathcal{C}}_{\rm
  PS}^{(1)}(t_R,t_F) &=& \displaystyle 0 \\ \\ \displaystyle
  {\mathcal{C}}_{\rm PS}^{(2)}(t_R,t_F) &=& \displaystyle {C}_{\rm
                                            PS}^{(2)}-t_FC_g^{(1)}\otimes
  P_{gq}^{(0)}+\frac{\Delta_{\rm SF}}{n_f}\frac{t_F^2}{2}P_{qg}^{(0)}\otimes P_{gq}^{(0)}-\frac{\Delta_{\rm SF}}{n_f}t_F\left[\widetilde{P}_{qq}^{(1)}-\widetilde{P}_{+}^{(1)}\right]\\
\end{array}
\end{equation}
and also that:
\begin{equation}\label{SingletCFg}
\begin{array}{rcl}
  \displaystyle {\mathcal{C}}_g^{(0)}(t_R,t_F) &=&
                                                   \displaystyle 0 \\ \\ \displaystyle {\mathcal{C}}_g^{(1)}(t_R,t_F) &=&
                                                                                                                          \displaystyle {C}_g^{(1)}- \frac{\Delta_{\rm SF}}{n_f} t_F P_{qg}^{(0)} \\ \\
  \displaystyle {\mathcal{C}}_g^{(2)}(t_R,t_F) &=& \displaystyle
                                                   {C}_g^{(2)} + t_R\beta_0 {C}_g^{(1)} -t_F {C}_i^{(1)} \otimes
                                                   P_{ig}^{(0)}\\ \\ &+&\displaystyle\frac{\Delta_{\rm SF}}{n_f} \frac{t_F^2}2 \left(
                                                                         P_{qi}^{(0)}\otimes P_{ig}^{(0)} - \beta_0 P_{qg}^{(0)} \right)-
                                                                         \frac{\Delta_{\rm SF}}{n_f} t_F\widetilde{P}_{qg}^{(1)}\,,
\end{array}
\end{equation}
where the term ${C}_i^{(1)} \otimes P_{ig}^{(0)}$ in the r.h.s. of the
third line of eq.~(\ref{SingletCFg}) should be interpreted as:
\begin{equation} 
  {C}_i^{(1)} \otimes P_{ig}^{(0)} =
 \frac{1}{n_f}{C}_{+}^{(1)}\otimes P_{qg}^{(0)} + {C}_g^{(1)} \otimes
  P_{gg}^{(0)}\,.
\end{equation}

Now we consider the massive scheme. In the neutral-current sector the
leading-order coefficient functions $C_i^{(0)}$ are identically equal
to zero and this simplifies substantially the structure of the massive
coefficient functions in the presence of scale variations:
\begin{equation}
\begin{array}{rcl}
  \displaystyle \mathcal{C}_j^{(0)}(t_R,t_F) &=&
                                                 \displaystyle 0 \\ \\ \displaystyle \mathcal{C}_j^{(1)}(t_R,t_F) &=&
                                                                                                                      \displaystyle C_j^{(1)} \\ \\ \displaystyle
  \mathcal{C}_j^{(2)}(t_R,t_F) &=& \displaystyle C_j^{(2)} + t_R\beta_0
                                   C_j^{(1)} -t_F C_i^{(1)} \otimes P_{ij}^{(0)}\,.
\end{array}
\end{equation}
In addition, the factorization scale variation terms are already
present in the implementation of the massive coefficient functions in
{\tt APFEL}. As a consequence, only the renormalization variation
terms need to be implemented. This is a great facilitation because the
renormalization variation terms do not require any further convolution
and thus no additional terms need to be computed during the
initialization phase.

As far as the massive charged-current sector is concerned, no
$\mathcal{O}(a_s^2)$ are presently available and thus only the first
two lines of eq.~(\ref{generalizedCF}) are actually needed. Also in
this case the factorization scale variation terms are already present
in the implementation and again this avoids the precomputation of
additional terms.

Now let us discuss how to implement in {\tt APFEL} the additional
terms needed to perform scale variations. The only terms that are a
bit more complicated to implement are those that require a convolution
between two splitting functions of between a plitting functions and a
coefficient functions. More in particular, we only need to compute the
terms: $P_{ij}^{(0)}(x)\otimes P_{jk}^{(0)}(x)$ and
$C_{i}^{(1)}(x)\otimes P_{ij}^{(0)}(x)$. In pricinple, these terms
could be evaluated analitically by computing the explicit convolution
between the know expressions that are involved. However, it seems
easier in {\tt APFEL} to compute these terms numerically using the
ingredients that have already been evaluated in the initialization
stage. To show how to reduce these terms to known quantity, let us
cosider the following convolution:
\begin{equation}
  F(x_\alpha)=x_\alpha C(x_\alpha)\otimes Q(x_\alpha) =
  x_\alpha\int_{x_\alpha}^1\frac{dy}{y}C(y)Q\left(\frac{x_\alpha}{y}\right)
  =\int_{x_\alpha}^1\frac{dy}{y}yC(y)\frac{x_\alpha}{y}Q\left(\frac{x_\alpha}{y}\right)
  =
  \int_{x_\alpha}^1\frac{dy}{y}\widetilde{C}(y)\widetilde{Q}\left(\frac{x_\alpha}{y}\right)\,,
\end{equation}
where $x_\alpha$ is node of the $x$-space grid of {\tt APFEL} and
$\widetilde{C}(y)=yC(y)$ and $\widetilde{Q}(y)=yQ(y)$. Now, using the
well-known interpolation formula we can write:
\begin{equation}
  \int_{x_\alpha}^1\frac{dy}{y}\widetilde{C}(y)\widetilde{Q}\left(\frac{x_\alpha}{y}\right)
  =
  \sum_{\beta}\underbrace{\left[\int_{x_\alpha}^1\frac{dy}{y}\widetilde{C}(y)w_{\beta}^{(k)}\left(\frac{x_\alpha}{y}\right)\right]}_{\Gamma_{\alpha\beta}}\widetilde{Q}(x_\beta)\,,
\end{equation}
where $w_{\beta}^{(k)}$ are the usual interpolation functions of
degree $k$. Now suppose that in turn:
\begin{equation} 
  \widetilde{Q}(x_\beta)=x_\beta P(x_\beta)\otimes
  f(x_\beta) =
  \int_{x_\beta}^1\frac{dz}{z}\widetilde{P}(z)\widetilde{f}\left(\frac{x_\beta}{z}\right)=\sum_{\gamma}\underbrace{\left[\int_{x_\beta}^1\frac{dz}{z}\widetilde{P}(z)w_{\gamma}^{(k)}\left(\frac{x_\beta}{z}\right)\right]}_{\Pi_{\beta\gamma}}\widetilde{f}(x_\gamma)\,,
\end{equation}
it follows that:
\begin{equation}
  F(x_\alpha)=\widetilde{C}(x_\alpha)\otimes
  \widetilde{P}(x_\alpha)\otimes \widetilde{f}(x_\alpha) =
  \sum_{\beta,\gamma}
  \Gamma_{\alpha\beta}\Pi_{\beta\gamma}\widetilde{f}(x_\gamma)\,.
\end{equation} 
The formula above clearly shows that the missing pieces can be easily
obtained by properly multimplying the precomputed splitting function
matrices $\Pi_{ij,\alpha\beta}$ and the coefficient function matrices
$\Gamma_{i,\alpha\beta}$ accordind to the scale variation formulas
derived above.

As an alternative to the numerical convolution of the new pieces
arising when including renormalization- and factorization-scale
variations, one can try to compute the analytically the convolutions
above. In fact, all the terms involved in the new convolutions are
usually simple enough to make the analytic computation possible using,
for instance, {\tt Mathematica}. This is advantageous because it
avoids any inaccuracy of numerical origin coming from the numerical
convolution of the operators involved. In order to do so, we only need
to know how to treat some particular term that appear in the
combinations. In particular, we need to be able to treat terms in
which Dirac $\delta$-functions and $+$-prescripted functions are
present at the same time. The most trivial convolutions are those
involving one or two $\delta$-functions, that is:
\begin{equation}\label{ConvolutionDelta}
\begin{array}{l}
\displaystyle \delta(1-x)\otimes\delta(1-x) = \delta(1-x)\,,\\
\\
\displaystyle \left(\frac{\ln^n(1-x)}{1-x}\right)_+\otimes\delta(1-x)
  = \left(\frac{\ln^n(1-x)}{1-x}\right)_+\quad n\geq0\,,
\end{array}
\end{equation}
that can be easily proven in Mellin space where the convolution
$\otimes$ becomes a simple product and the $\delta$-function
corresponds to the unity. The Mellin-space method can be used also in
the cases where two $+$-prescripted functions are involved. Up to
$\mathcal{O}(\alpha_s^2)$ there are only two possible combinations,
that are:
\begin{equation}\label{ConvolutionPlus}
\begin{array}{l}
\displaystyle \left(\frac{1}{1-x}\right)_+\otimes
  \left(\frac{1}{1-x}\right)_+= 2
  \left(\frac{\ln(1-x)}{1-x}\right)_+-\frac{\ln(x)}{1-x}-\zeta(2)\delta(1-x)\,,\\
\\
\displaystyle \left(\frac{1}{1-x}\right)_+\otimes
  \left(\frac{\ln(1-x)}{1-x}\right)_+= \frac32\left(\frac{\ln^2(1-x)}{1-x}\right)_+-\zeta(2) \left(\frac{1}{1-x}\right)_+-\frac{\ln(x)\ln(1-x)}{1-x}+\zeta(3)\delta(1-x)\,.
\end{array}
\end{equation}
The relations in eq.~(\ref{ConvolutionPlus}) can be obtained
rearranging, in Mellin space, the terms is such a way to reconstruct
the Mellin-transform of well-known terms.

Now, given the LO splitting functions (with expansion parameter
$\alpha_s/4\pi$ and such that they can be used to evolve the singlet
combination $\{q^+,g\}$):
\begin{equation}
\begin{array}{l}
\displaystyle P_{qq}^{(0)}(x) = 2 C_F \left[2
  \left(\frac{1}{1-x}\right)_+ - (1 + x) +\frac{3}{2} \delta(1 -
  x)\right]\,,\\
\\
\displaystyle P_{qg}^{(0)}(x) = 4 n_f T_R \left[x^2 + (1 - x)^2\right]\,,\\
\\
\displaystyle P_{gq}^{(0)}(x) = 2 C_F \left[\frac{1+ (1 -
  x)^2}{x}\right]\,,\\
\\
\displaystyle P_{gg}^{(0)}(x) = 4 C_A \left[\left(\frac{1}{1-x}\right)_+
  - 2 + x - x^2 + \frac{1}{x}\right] + \frac{11 C_A - 4 n_f T_R}{3}
  \delta(1 - x)\,,
\end{array}
\end{equation}
we can compute the additional terms involving only combinations of
splitting functions. In particular, we need to compute:
\begin{equation}\label{NSP0P0}
P_{qq}^{(0)}(x) \otimes P_{qq}^{(0)}(x)\,,
\end{equation}
involved in the $\mathcal{O}(\alpha_s^2)$ non-singlet coefficient
functions, and:
\begin{equation}\label{SGP0P0}
\begin{array}{l}
P_{qg}^{(0)}(x) \otimes P_{gq}^{(0)}(x)\,,\\
\\
\displaystyle P_{qi}^{(0)}(x) \otimes P_{ig}^{(0)}(x)=P_{qq}^{(0)}(x) \otimes
  P_{qg}^{(0)}(x)+P_{qg}^{(0)}(x) \otimes P_{gg}^{(0)}(x)\,.
\end{array}
\end{equation}
present in the pure-singlet and in the gluon coefficient functions,
respectively.

The convolution in eq.~(\ref{NSP0P0}) can be easily computed by hand
using eqs.~(\ref{ConvolutionDelta}) and~(\ref{ConvolutionPlus}) and
the result is:
\begin{equation}
\begin{array}{rcl}
P_{qq}^{(0)}(x) \otimes P_{qq}^{(0)}(x) &=&\displaystyle 4
                                            C_F^2\bigg[8\left(\frac{\ln(1-x)}{1-x}\right)_++6\left(\frac{1}{1-x}\right)_+-4\frac{\ln(x)}{1-x}-4(1+x)\ln(1-x)\\
\\
&+&\displaystyle 3(1+x)\ln(x)-(x+5)+\left(\frac{9}{4}-4\zeta(2)\right)\delta(1-x)\bigg]\,.
\end{array}
\end{equation}
As for eq.~(\ref{SGP0P0}), where no convolutions of the kinds given in
eqs.~(\ref{ConvolutionDelta}) and~(\ref{ConvolutionPlus}) are present,
we can safely use {\tt Mathematica}, obtaining:
\begin{equation}
\begin{array}{rcl}
P_{qg}^{(0)}(x) \otimes P_{gq}^{(0)}(x)&=&\displaystyle C_F n_f
                                           T_R\left[\frac{8}{3} \left(-4 x^2-3 x+\frac{4}{x}+3\right)+16 (x+1) \ln(x)\right]\,,\\
\\
P_{qi}^{(0)}(x) \otimes P_{ig}^{(0)}(x)&=&\displaystyle n_f T_R C_A
                                           \left[16 \left(2 x^2-2
                                           x+1\right) \ln
                                           (1-x)+16(4 x+1) \ln
                                           (x)+\frac{4}{3} \left(-40
                                           x^2+26 x+17+\frac{8}{x}\right)
                                           \right]\\
\\
&+&\displaystyle n_f T_R C_F \left[16 \left(2 x^2-2 x+1\right) \ln
    (1-x)-8 \left(4 x^2-2 x+1\right) \ln (x)+4\left(4 x-1\right)\right]\\
\\
&+&\displaystyle n_f^2 T_R^2 \left[-\frac{16}{3} (2 x^2-2 x+1)\right]
\end{array}
\end{equation}

Now we need to consider the additional terms involving combinations of
splitting functions and coefficient functions. Let us start
considering $F_L$ and it is the easiest case. Here we have:
\begin{equation}
\begin{array}{rcl}
\displaystyle C_{L,\pm}^{(1)}(x) &=& \displaystyle 4 C_F x\,,\\
\\
\displaystyle C_{L,q}^{(1)}(x) &=& \displaystyle  \frac1{n_f} C_{L,\pm}^{(1)}(x)\,,\\
\\
\displaystyle C_{L,g}^{(1)}(x) &=& \displaystyle  4 T_Rx(1-x)\,,
\end{array}
\end{equation}
and for the non-singlet case we need to compute:
\begin{equation}
C_{L,\pm}^{(1)}(x)\otimes P_{qq}^{(0)}(x) = 4C_F^2 \left[ (x+2)+4 x \ln (1-x)-2 x \ln(x)\right]\,.
\end{equation}
For the pure-singlet and the gluon coefficient functions, instead, we
need to compute:
\begin{equation}
\begin{array}{rcl}
C_{L,g}^{(1)}(x)\otimes P_{gq}^{(0)}(x) &=& \displaystyle C_F T_R \left[\frac{32}{3} \left(2 x^2-3+\frac{1}{x}\right)-32 x \ln (x)\right]\,,\\
\\
C_{L,i}^{(1)}(x)\otimes P_{ig}^{(0)}(x) &=& \displaystyle  C_A T_R
                                            \left[64 x (1-x) \ln
                                            (1-x)-128 x \ln
                                            (x)+\frac{16}{3} \left(23
                                            x^2-19 x-6 + \frac{2}{x}\right)\right]\\
\\
&+&\displaystyle C_F T_R \left[\frac{16}{3} x \ln
                                            (x)-\frac{8}{3} \left(2
                                            x^2-x-1\right)\right]\\
\\
&+&\displaystyle n_fT_R^2\left[-\frac{64}{3}x(1-x)\right]\,.
\end{array}
\end{equation}

Now we consider $F_2$, for which we have:
\begin{equation}
\begin{array}{rcl}
\displaystyle C_{2,\pm}^{(1)}(x) &=& \displaystyle 
                                     2C_F \bigg[2 \left(\frac{\ln(1-x)}{1-x}\right)_+-\frac{3}{2}\left(\frac{1}{1-x}\right)_+
                                     -2\frac{\ln
                                     (x)}{1-x}-(x+1) \left[\ln (1-x)-\ln
                                     (x)\right] \\
\\
&+&\displaystyle 2 x+3 - \left(2 \zeta(2)+\frac{9}{2}\right) \delta(1-x)\bigg] \,,\\
\\
\displaystyle C_{2,q}^{(1)}(x) &=& \displaystyle  \frac1{n_f} C_{\pm,2}^{(1)}(x) \,,\\
\\
\displaystyle C_{2,g}^{(1)}(x) &=& \displaystyle  4 T_R \left[\left(x^2+(1-x)^2\right) [\ln
                                   (1-x)-\ln (x)]-8x^2+8 x-1\right] \,.
\end{array}
\end{equation}
Also in this case we need to compute $C_{2,\pm}^{(1)}(x)\otimes
P_{qq}^{(0)}(x)$ and $C_{2,i}^{(1)}(x)\otimes P_{ig}^{(0)}(x)$.

\section{Implementation of the Semi-Inclusive $e^+e^-$ Annihilation}

The implementation of the Semi-Inclusive $e^+e^-$ Annihilation (SIA)
in {\tt APFEL} is not very complicated. The reason for that is the
fact that SIA is structurally identical to DIS. In fact, we can regard
SIA as the time-like counterpart of DIS and the differences are only
at the level of coefficient functions and splitting functions.
Considering that {\tt APFEL} already implement the time-like
evolution~\cite{Bertone:2015cwa} ($i.e.$ the time-like splitting
functions), the only thing to do is implement the respective
coefficient functions. Presently the coefficient functions for SIA are
known up to $\mathcal{O}(\alpha_s^2)$ (NNLO) in the zero-mass scheme
and they have been computed in Ref.~\cite{Mitov:2006wy} and the
$x$-space expressions of interest for the implementation in {\tt
  APFEL} reported in Appendix C.

The way in which the SIA expressions are reported is slightly
different from the standard way in which we are used to see the DIS
expressions. We would like to reduce the SIA expressions to the same
form of DIS in such a way to use the DIS module of {\tt APFEL} also
for the SIA cross sections. In particular the SIA cross section is
Ref.~\cite{Mitov:2006wy} expressed in terms of the three structure
functions: $F_T$, $F_L$ and $F_A$. However, comparing the SIA cross
section with the DIS cross sections it is easy to realize that
defining:
\begin{equation}\label{SIAtoDIS}
\begin{array}{l} F_2(x,Q) = F_T(x,Q) + F_L(x,Q)\,,\\ F_L(x,Q) =
F_L(x,Q)\,,\\ F_A(x,Q) = xF_3(x,Q)\,,
\end{array}
\end{equation} the SIA cross section reduces to the same structure of
the DIS cross section.

Assuming that:
\begin{equation}
  F_k(x,Q) = \sum_{j=q,g} x\int_x^1\frac{dy}{y}
  c_{k,j}(\alpha_s(Q),x)\mathcal{D}_j\left(\frac{x}{y},Q\right)\,,\quad\mbox{with}\quad
  k = 2,L,3\,,
\end{equation} 
(note that, to uniform the notation, we understood the factor $x$ in
front of $F_3$) where $\mathcal{D}_j$ is the fragmentation function of
the flavour $j$ and where the coefficient functions $c_{k,j}$ allow
for the perturbative expansion:
\begin{equation}
  c_{k,j}(\alpha_s(Q),x) = \sum_{n=0}^N \alpha_s^n(Q)
  c_{k,j}^{(n)}(x)\,,
\end{equation}
we have that the leading-order cofficient functions are trivially:
\begin{equation}
\begin{array}{l}
  c_{k,g}^{(0)}(x) = 0 \,,\quad k = 2,L,3\,,\\ \\
  c_{L,q}^{(0)}(x) = 0\,, \\ \\ c_{2,q}^{(0)}(x) = c_{3,q}^{(0)}(x) =
  \delta(1-x)\,.
\end{array}
\end{equation}

Now we consider the NLO coefficient functions. Their explicit
expressions are give in eqs.~(C.13)-(C.17) of Ref.~\cite{Mitov:2006wy}
but, in order to write them in a form suitable for the implementation
in {\tt APFEL}, we need to isolate the regular, soft-divergent and
local terms and finally combine them according to
eq.~(\ref{SIAtoDIS}).

\begin{equation}
\begin{array}{lcl}
  \displaystyle c_{L,q}^{(1)}(x) = 2C_F & &\\ \\
  \displaystyle c_{L,g}^{(1)}(x) = 2C_F\frac{4(1-x)}{x} & &\\ \\
  \displaystyle c_{2,q}^{(1)}(x) = c_{T,q}^{(1)}(x) + c_{L,q}^{(1)}(x)
                                        &=& \displaystyle
                                            2C_F\bigg[2\left(\frac{\ln(1-x)}{1-x}\right)_+-\frac{3}{2}\left(\frac{1}{1-x}\right)_+
                                            - (1+x)\ln(1-x)\\ \\ & &\displaystyle +2\frac{1+x^2}{1-x}\ln x
                                                                     +\frac{5}{2} -\frac{3}{2} x
                                                                     +\left(4\zeta_2-\frac{9}{2}\right)\delta(1-x)\bigg]\\ \\ \displaystyle
  c_{2,g}^{(1)}(x) = c_{T,g}^{(1)}(x) + c_{L,g}^{(1)}(x) &=&
                                                             \displaystyle 4C_F\frac{1+(1-x)^2}{x} \ln[x^2(1-x)]\\ \\ \displaystyle
  c_{3,q}^{(1)}(x) &=& \displaystyle
                       2C_F\bigg[2\left(\frac{\ln(1-x)}{1-x}\right)_+-\frac{3}{2}\left(\frac{1}{1-x}\right)_+
                       - (1+x)\ln(1-x)\\ \\ & &\displaystyle +2\frac{1+x^2}{1-x}\ln x
                                                +\frac{1}{2} -\frac{1}{2} x
                                                +\left(4\zeta_2-\frac{9}{2}\right)\delta(1-x)\bigg]\\ \\ \displaystyle
  c_{3,g}^{(1)}(x) = 0
\end{array}
\end{equation}

It is interesting to notice that, as expected, the soft-singular part
of the quark coefficient functions is exactly the same as in DIS and
this allows us to reuse part of the DIS coefficient functions.

\section{Polarized DIS cross section and structure functions}

Let us consider the differential cross sections for unpolarized and
polarized Deep-Inelastic Scattering (DIS) (see {\it e.g.} Eq.~(19.16) of
Sec.~19 in Ref.~\cite{Agashe:2014kda}):
\begin{equation}
\begin{array}{lcl}
\displaystyle \frac{d^2\sigma^i}{dxdy}
& = &
\displaystyle \frac{2\pi\alpha^2}{xyQ^2}\eta^i
\left[
+Y_+ F_2^i \mp Y_- x F_3^i - y^2 F_L^i
\right]
\\ \\
\displaystyle \frac{d^2\Delta\sigma^i}{dxdy}
& = &
\displaystyle \frac{2\pi\alpha^2}{xyQ^2}\eta^i
\left[
-Y_+ g_4^i \mp Y_- 2x g_1^i + y^2 g_L^i
\right]
\mbox{\,,}
\end{array}
\end{equation}
where $i={\rm NC, CC}$, $Y_{\pm}=1 \pm (1-y)^2$, $\eta^{\rm NC}=1$,
$\eta^{\rm CC}=(1\pm \lambda)^2\eta_W$ (with $\lambda=\pm 1$ is the 
helicity of the incoming lepton and $\eta_W=\frac{1}{2}
\left(\frac{G_FM_W}{4\pi\alpha}\frac{Q^2}{Q^2+M_W^2} \right)^2$), and 
\begin{equation}
\begin{array}{lcl}
\displaystyle F_L^i & = & \displaystyle F_2^i - 2xF_1^i\\ \\
\displaystyle F_L^i & = & \displaystyle g_4^i - 2xg_5^i
\mbox{\,.}
\end{array}
\end{equation}
Because the same tensor structure occurs in the spin-dependent and 
spin-independent parts of the DIS hadronic tensor (in the limit $M^2/Q^2\to 0$),
the polarized cross section can be obtained from the unpolarized cross section
with the following replacement
\begin{equation}
\displaystyle F_2^i \rightarrow -2g_4^i
\ \ \ \ \ \ \ \ \ 
\displaystyle F_3^i \rightarrow +4g_1^i
\ \ \ \ \ \ \ \ \
\displaystyle F_L^i \rightarrow -2g_L^i 
\mbox{\,.}
\end{equation}
Note that the extra factor two is due to the fact that the total cross section 
is an average over initial-state polarizations.

The {\it polarized} structure functions $g_4$, $g_1$ and $g_L$ 
are expressed as a convolution of coefficient functions, $\Delta c_{k,j}$,
and polarized PDFs, $\Delta f_j$, (summed over all flavors $j$)
\begin{equation}
g_k(x,Q) 
=
\sum_{j=q,g}x \int_x^1 \frac{dy}{y} \Delta c_{k,j}(\alpha_s(Q),x)
\Delta f_j\left(\frac{x}{y},Q\right)
\mbox{\,,}
\ \ \ \ \ \ \ \ \ \ \
{\rm with} 
\ \ k=4,1,L
\mbox{\,.}
\end{equation} 
The coefficient functions $\Delta c_{k,j}$ allow for the usual perturbative
expansion
\begin{equation}
\Delta c_{k,j}(\alpha_s(Q),x) 
=
\sum_{n=0}^N\alpha_s^n(Q)\Delta c_{k,j}^{(n)}(x)
\,\mbox{,}
\end{equation}
where the coefficients $\Delta c_{k,j}^{(n)}(x)$ are known up to NLO,
{\it i.e.} $n=1$ (see {\it e.g.}~\cite{deFlorian:2012wk} and references 
therein). At LO they are 
\begin{equation}
\begin{array}{lcl}
\Delta c_{4,q}^{(0)}(x) = \Delta c_{1,q}^{(0)}(x) &=& \delta(1-x)
\\ \\
\Delta c_{L,q}^{(0)}(x) &=& 0
\,\mbox{,}
\\ \\
\Delta c_{k,g}^{(0)}(x) &=& 0
\ \ \ \ \ \ \ \ \ \
{\rm with} \ k=4,1,L
\,\mbox{.}
\end{array}
\end{equation}
At NLO they read 
\begin{equation}
\begin{array}{lcl}
  %C4q
  \displaystyle \Delta c_{4,q}^{(1)}(x) 
  &=& 
  \displaystyle 2C_F\,
  \bigg\{
  2\left[\frac{\ln(1-x)}{1-x}\right]_+ 
  - \frac{3}{2}\left[\frac{1}{1-x}\right]_+
  - (1+x)\ln(1-x)
  \\ \\ 
  & &\displaystyle - \frac{1+x^2}{1-x}\ln x
  + 3 + 2x
  -\left(\frac{9}{2} + 2\zeta_2\right)\delta(1-x)\bigg\}
  \,\mbox{,}
  \\ \\ 
  %C4g
  \displaystyle \Delta c_{4,g}^{(1)}(x) 
  &=& 
  0
  \,\mbox{,} 
  \\ \\
  %C1q
  \displaystyle \Delta c_{1,q}^{(1)}(x) 
  &=& 
  \displaystyle 2C_F\,
  \bigg\{
  2\left[\frac{\ln(1-x)}{1-x}\right]_+ 
  - \frac{3}{2}\left[\frac{1}{1-x}\right]_+
  - (1+x)\ln(1-x)
  \\ \\ 
  & &\displaystyle - \frac{1+x^2}{1-x}\ln x
  + 2 + x
  -\left(\frac{9}{2} + 2\zeta_2\right)\delta(1-x)\bigg\}
  \,\mbox{,}
  \\ \\ 
  %C1g
  \displaystyle \Delta c_{1,g}^{(1)}(x) 
  &=&
  \displaystyle 4T_R
  \bigg\{(2x - 1)\ln \frac{1-x}{x} - 4x +3 \bigg\} 
  \,\mbox{,}
  \\ \\ 
  %CLq
  \displaystyle \Delta c_{L,q}^{(1)}(x) 
  &=& 
  2C_F\, 2x 
  \,\mbox{,}
  \\ \\
  %CLg
  \displaystyle \Delta c_{L,g}^{(1)}(x) 
  &=& 
  0 
  \,\mbox{.}
\end{array}
\end{equation}

In the NC case the couplings can be written as:
\begin{equation}
\begin{array}{l}
\displaystyle B_q(Q^2) = -e_qA_q(V_e\pm \lambda A_e)P_Z+V_qA_q(V_e^2+A_e^2\pm2\lambda V_eA_e)P_Z^2 \,\mbox{,}\\
\\
\displaystyle D_q(Q^2) = \pm\frac12 \lambda e_q^2 - e_qV_q(A_e\pm\lambda V_e)P_Z +\frac12(V_q^2+A_q^2)\left[2V_eA_e\pm\lambda (V_e^2+A_e^2)\right]P_Z^2\,\mbox{.}
\end{array}
\end{equation}
where $\lambda$ corresponds to the polarization of the incoming
lepton. It should be stressed that $B_q$ multiplies $g_4$ and $g_L$
while $D_q$ multiplies $g_1$.

\section{The $\chi$ Prescription in FONLL}

As is well known, the original formulation of the FONLL matched scheme
gives rise to discontinuities in correspondence of the heavy quark
thresholds arising from uncontrolled subleading terms. Such subleading
terms can however be numerically important especially arond the charm
threshold where the numerical value of the strong coupling $\alpha_s$
is large. In order to remedy this unwanted feature different
prescriptions have been introduced and traditionally the FONLL schem
DIS has been implemented using the so-called damping factor which
directly suppresses the unwanted subleading terms by means of a
function that goes smoothly to zero at the threshold and below and
tends to one for energies much larger than the threshold itself.

As an alternative to the damping factor, one can damp the subleading
terms close to the threshold by mimicing in the subtraction terms the
phase-space suppression given by the presence of one or more heavy
quarks in the final state. This is easily done juct by introducing the
so-called slow-rescaling variable $\chi$, that in the NC case is:
\begin{equation}
\chi=x\left(1+\frac{4m_H^2}{Q^2}\right)=\frac{x}{\eta}\,,
\end{equation}
$m_H$ being the mass of the heavy quark, in the convolution between
coefficient functions and PDFs in the zero-mass and in the
massless-limit bits of the FONLL structure function. In other words,
the usual zero-mass Mellin convolution becomes:
\begin{equation}
x\int_x^1\frac{dy}{y}C\left(\frac{x}{y}\right)f(y)\rightarrow x\int_\chi^1\frac{dy}{y}C\left(\frac{\chi}{y}\right)f(y)=x\int_\chi^1\frac{dy}{y}C(y)f\left(\frac{\chi}{y}\right)\,.
\end{equation}
The question is how to treat the new integral on a discreet $x$-space
grid. What we have done so far for the massive integarls like that in
the r.h.s. of the equation above is re-express it in terms of the
physical Bjorken $x$ as:
\begin{equation}
x\int_\chi^1\frac{dy}{y}C(y)f\left(\frac{\chi}{y}\right)
\end{equation}


\newpage

\begin{thebibliography}{alp}

%\cite{Alekhin:2003ev}
\bibitem{Alekhin:2003ev}
  S.~I.~Alekhin and J.~Blumlein,
  %``Mellin representation for the heavy flavor contributions to deep inelastic structure functions,''
  Phys.\ Lett.\ B {\bf 594} (2004) 299
  [hep-ph/0404034].
  %%CITATION = HEP-PH/0404034;%%
  %63 citations counted in INSPIRE as of 12 Feb 2015

%\cite{Laenen:1992xs}
\bibitem{Laenen:1992xs}
  E.~Laenen, S.~Riemersma, J.~Smith and W.~L.~van Neerven,
  %``O(alpha-s) corrections to heavy flavor inclusive distributions in electroproduction,''
  Nucl.\ Phys.\ B {\bf 392} (1993) 229.
  %%CITATION = NUPHA,B392,229;%%
  %154 citations counted in INSPIRE as of 12 Feb 2015

%\cite{Forte:2010ta}
\bibitem{Forte:2010ta}
  S.~Forte, E.~Laenen, P.~Nason and J.~Rojo,
  %``Heavy quarks in deep-inelastic scattering,''
  Nucl.\ Phys.\ B {\bf 834} (2010) 116
  [arXiv:1001.2312 [hep-ph]].
  %%CITATION = ARXIV:1001.2312;%%
  %98 citations counted in INSPIRE as of 12 Feb 2015

%\cite{Buza:1995ie}
\bibitem{Buza:1995ie}
  M.~Buza, Y.~Matiounine, J.~Smith, R.~Migneron and W.~L.~van Neerven,
  %``Heavy quark coefficient functions at asymptotic values Q**2 >> m**2,''
  Nucl.\ Phys.\ B {\bf 472} (1996) 611
  [hep-ph/9601302].
  %%CITATION = HEP-PH/9601302;%%
  %164 citations counted in INSPIRE as of 12 Feb 2015

%\cite{Gluck:1996ve}
\bibitem{Gluck:1996ve}
  M.~Gluck, S.~Kretzer and E.~Reya,
  %``The Strange sea density and charm production in deep inelastic charged current processes,''
  Phys.\ Lett.\ B {\bf 380} (1996) 171
   [Erratum-ibid.\ B {\bf 405} (1997) 391]
  [hep-ph/9603304].
  %%CITATION = HEP-PH/9603304;%%
  %88 citations counted in INSPIRE as of 12 Feb 2015

%\cite{Georgi:1976ve}
\bibitem{Georgi:1976ve}
  H.~Georgi and H.~D.~Politzer,
  %``Freedom at Moderate Energies: Masses in Color Dynamics,''
  Phys.\ Rev.\ D {\bf 14} (1976) 1829.
  %%CITATION = PHRVA,D14,1829;%%
  %925 citations counted in INSPIRE as of 25 Feb 2015

%\cite{Bertone:2015cwa}
\bibitem{Bertone:2015cwa}
  V.~Bertone, S.~Carrazza and E.~R.~Nocera,
  %``Reference results for time-like evolution up to $\mathcal{O}(\alpha_s^3)$,''
  JHEP {\bf 1503} (2015) 046
  [arXiv:1501.00494 [hep-ph]].
  %%CITATION = ARXIV:1501.00494;%%
  %1 citations counted in INSPIRE as of 16 Mar 2015

%\cite{Mitov:2006wy}
\bibitem{Mitov:2006wy}
  A.~Mitov and S.~O.~Moch,
  %``QCD Corrections to Semi-Inclusive Hadron Production in Electron-Positron Annihilation at Two Loops,''
  Nucl.\ Phys.\ B {\bf 751} (2006) 18
  [hep-ph/0604160].
  %%CITATION = HEP-PH/0604160;%%
  %39 citations counted in INSPIRE as of 16 Mar 2015

%\cite{Agashe:2014kda}
\bibitem{Agashe:2014kda}
  K.~A.~Olive {\it et al.} [Particle Data Group Collaboration],
  %``Review of Particle Physics,''
  Chin.\ Phys.\ C {\bf 38} (2014) 090001.
  %doi:10.1088/1674-1137/38/9/090001
  %%CITATION = doi:10.1088/1674-1137/38/9/090001;%%
  %2477 citations counted in INSPIRE as of 10 Dec 2015

%\cite{deFlorian:2012wk}
\bibitem{deFlorian:2012wk}
  D.~de Florian and Y.~R.~Habarnau,
  %``Polarized semi-inclusive electroweak structure functions at next-to-leading-order,''
  Eur.\ Phys.\ J.\ C {\bf 73} (2013) 3,  2356
  %doi:10.1140/epjc/s10052-013-2356-3
  [arXiv:1210.7203 [hep-ph]].
  %%CITATION = doi:10.1140/epjc/s10052-013-2356-3;%%
  %5 citations counted in INSPIRE as of 10 Dec 2015

\end{thebibliography}

\end{document}
