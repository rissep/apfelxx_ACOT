%% LyX 2.0.3 created this file.  For more info, see http://www.lyx.org/.
%% Do not edit unless you really know what you are doing.
\documentclass[twoside,english]{paper}
\usepackage{lmodern}
\renewcommand{\ttdefault}{lmodern}
\usepackage[T1]{fontenc}
\usepackage[latin9]{inputenc}
\usepackage[a4paper]{geometry}
\geometry{verbose,tmargin=3cm,bmargin=2.5cm,lmargin=2cm,rmargin=2cm}
\usepackage{color}
\usepackage{babel}
\usepackage{float}
\usepackage{bm}
\usepackage{amsthm}
\usepackage{amsmath}
\usepackage{amssymb}
\usepackage{graphicx}
\usepackage{esint}
\usepackage[unicode=true,pdfusetitle,
 bookmarks=true,bookmarksnumbered=false,bookmarksopen=false,
 breaklinks=false,pdfborder={0 0 0},backref=false,colorlinks=false]
 {hyperref}
\usepackage{breakurl}
\usepackage{makeidx}

\makeatletter

%%%%%%%%%%%%%%%%%%%%%%%%%%%%%% LyX specific LaTeX commands.
%% Because html converters don't know tabularnewline
\providecommand{\tabularnewline}{\\}

%%%%%%%%%%%%%%%%%%%%%%%%%%%%%% Textclass specific LaTeX commands.
\numberwithin{equation}{section}
\numberwithin{figure}{section}

%%%%%%%%%%%%%%%%%%%%%%%%%%%%%% User specified LaTeX commands.
\usepackage{babel}

\@ifundefined{showcaptionsetup}{}{%
 \PassOptionsToPackage{caption=false}{subfig}}
\usepackage{subfig}
\makeatother

\usepackage{listings}


\begin{document}

\title{Precomputation of the integrals in APFEL++}

\author{Valerio Bertone}

\tableofcontents{}

\section{Structure of the integrals}

The general structure of the quantities to be computed in {\tt APFEL}
is the combination of terms $I$ that have the form of Mellin
convolutions between an operator $O$ and distribution function $d$,
that is:
\begin{equation}\label{eq:ZMconv}
I(x) = x\int_0^1dz\int_0^1dy\, O(z)d(y)\delta(x-yz) = x\int_x^1\frac{dz}{z} O \left(\frac{x}{z}\right) d(z) = x\int_x^1\frac{dz}{z} O(z)d\left(\frac{x}{z}\right)\,.
\end{equation}
However, very often, typically in the presence of mass effects, the
integration phase space is modified and the convolution in
eq.~(\ref{eq:ZMconv}), limiting ourselves to the leftmost identity, is
generalised as:
\begin{equation}\label{eq:Genconv}
I(x,\eta) = x\int_{x/\eta}^1\frac{dz}{z} O(z,\eta)d\left(\frac{x}{\eta
    z}\right)\,.
\end{equation}
where $\eta\leq1$, with $\eta = 1$ reproducing
eq.~(\ref{eq:ZMconv}). However, for purposes that will become clear
later, we want to write the integral in eq.~(\ref{eq:Genconv}) in the
form of eq.~(\ref{eq:ZMconv}), that is in such a way that lower bound
of the integral is not the rescaled variable $x/\eta$ but the physical
Bjorken $x$. To do so, one needs to perform the change of variable
$y =\eta z$, so that:
\begin{equation}\label{eq:Genconv1}
  I(x,\eta)= \int_x^\eta dy\,
  O\left(\frac{y}{\eta},\eta\right)\,\frac{x}{y}d\left(\frac{x}{y}\right)\,.
\end{equation}

In order to precompute the expensive part of the integral in
eq.~(\ref{eq:Genconv1}), we use the standard interpolation formula to
the distribution $d$:
\begin{equation}\label{eq:Interpolation}
\frac{x}{y} d\left(\frac{x}{y}\right) = \sum_{\alpha=0}^{N_x} x_\alpha
d(x_\alpha) w_{\alpha}^{(k)}\left(\frac{x}{y}\right)\,,
\end{equation}
where $\alpha$ runs over the node of a give grid in $x$ space and the
weights $w_{\alpha}$ are typically polynomials of degree $k$ ($i.e.$
Lagrange interpolants). Now we use eq.~(\ref{eq:Interpolation}) in
eq.~(\ref{eq:Genconv1}) and at the same time we limit the computation
of the integral $I$ to the point $x_\beta$ of the grid used in
eq.~(\ref{eq:Interpolation}). This way we get:
\begin{equation}\label{eq:Genconv1}
  I(x_\beta,\eta)=  \sum_{\alpha=0}^{N_x} \overline{d}_\alpha\int_{x_\beta}^\eta dy\,
  O\left(\frac{y}{\eta},\eta\right)\,w_{\alpha}^{(k)}\left(\frac{x_\beta}{y}\right)\,.
\end{equation}
where we have defined $\overline{d}_\alpha = x_\alpha d(x_\alpha)$.
In the particular case of the Lagrange interpolants (in the {\tt
  APFEL} procedure), one can show that:
\begin{equation}
w_{\alpha}^{(k)}\left(\frac{x_\beta}{y}\right)\neq 0\quad\mbox{for}\quad c < y < d\,,
\end{equation}
with:
\begin{equation}\label{eq:intlims}
  c =
  \mbox{max}(x_\beta,x_\beta/x_{\alpha+1}) \quad\mbox{and}\quad d =
  \mbox{min}(\eta,x_\beta/x_{\alpha-k}) \,,
\end{equation}
and thus eq.~(\ref{eq:Genconv1}) can be adjusted as:
\begin{equation}\label{eq:Genconv2}
  I(x_\beta,\eta)=  \sum_{\alpha=0}^{N_x} \overline{d}_\alpha\int_c^d dy\,
  O\left(\frac{y}{\eta},\eta\right)\,w_{\alpha}^{(k)}\left(\frac{x_\beta}{y}\right)\,.
\end{equation}
Finally, we change back the integration variable $z=y/\eta$ so that
eq.~(\ref{eq:Genconv2}) becomes:
\begin{equation}\label{eq:Genconv3}
  I(x_\beta,\eta)=  \sum_{\alpha=0}^{N_x} \overline{d}_\alpha \underbrace{\left[\eta\int_{c/\eta}^{d/\eta} dz\,
  O\left(z,\eta\right)\,w_{\alpha}^{(k)}\left(\frac{x_\beta}{\eta z}\right)\right]}_{\Gamma_{\alpha\beta}}\,.
\end{equation}
The quantity in squared brackets is the interesting bit.

The expressions for the operator $O$ that we have to deal with have
this general form:
\begin{equation}\label{eq:CoeffFuncs}
  O(z,\eta) = R(z,\eta)+\sum_{i}\left[P^{(i)}(z,\eta)\right]_+ S^{(i)}(z,\eta)+ L(\eta)\delta(1-z)\,,
\end{equation}
where $R$ and $S^{(i)}$ are a regular functions in $z=1$, that is:
\begin{equation}
R(1,\eta) =  \lim_{z\rightarrow 1} R(z,\eta) = K(\eta)\quad\mbox{and}\quad S^{(i)}(1,\eta) =  \lim_{z\rightarrow 1} S^{(i)}(z,\eta) = J^{(i)}(\eta)\,,
\end{equation}
being $K$, $J^{(i)}$, and $L$ a finite function of $\eta$. The
functions $P^{(i)}$ are instead plus-prescripted functions whose
behaviour in $z=1$ is singular in the limit $\eta\rightarrow1$. Notice
that the sum over $i$ in eq.~(\ref{eq:CoeffFuncs}) depends on the
perturbative order of the expressions. Plugging
eq.~(\ref{eq:CoeffFuncs}) into the definition of
$\Gamma_{\alpha\beta}$ in eq.~(\ref{eq:Genconv3}) and making use of
the definition of plus-prescription, we obtain:
\begin{equation}\label{eq:FinalExpression}
\begin{array}{rcl}
\Gamma_{\beta\alpha} &=& \eta\displaystyle \int_{c/\eta}^{d/\eta} dz\left\{R\left(z,\eta\right)w_{\alpha}^{(k)}\left(\frac{x_\beta}{\eta z}\right)+\sum_{i}P^{(i)}(z,\eta)\left[S^{(i)}\left(z,\eta\right)w_{\alpha}^{(k)}\left(\frac{x_\beta}{\eta z}\right)-S^{(i)}(1,\eta)w_{\alpha}^{(k)}\left(\frac{x_\beta}{\eta}\right)\right]\right\}\\
\\
&+&\displaystyle \eta\left[L(\eta)-\sum_{i}S^{(i)}(1,\eta)\int_0^{c/\eta}dz
    P^{(i)}(z,\eta)\right]w_{\alpha}^{(k)}\left(\frac{x_\beta}{\eta}\right)\,,
\end{array}
\end{equation}
that can be further manipulated changing the integration variable
back, defining $\eta z = y$:
\begin{equation}\label{eq:FinalExpression2}
\begin{array}{rcl}
  \Gamma_{\beta\alpha} &=& \displaystyle \int_{c}^{d} dy\left\{R\left(\frac{y}{\eta},\eta\right)w_{\alpha}^{(k)}\left(\frac{x_\beta}{y}\right)+\sum_{i}P^{(i)}\left(\frac{y}{\eta},\eta\right)\left[S^{(i)}\left(\frac{y}{\eta},\eta\right)w_{\alpha}^{(k)}\left(\frac{x_\beta}{y}\right)-S^{(i)}(1,\eta)w_{\alpha}^{(k)}\left(\frac{x_\beta}{\eta}\right)\right]\right\}\\
  \\
                       &+&\displaystyle \eta \left[L(\eta)+\sum_{i}S^{(i)}(1,\eta)Q^{(i)}\left(\frac{c}{\eta},\eta\right)\right]w_{\alpha}^{(k)}\left(\frac{x_\beta}{\eta}\right)\,.
\end{array}
\end{equation}
where we have defined:
\begin{equation}
  Q^{(i)}(a,\eta)\equiv-\int_0^{a}dz P^{(i)}\left(z,\eta\right).
\end{equation}
These integrals can, most of the times, be computed analytically.

Eqs.~(\ref{eq:FinalExpression})-(\ref{eq:FinalExpression2}) express
the full complexity of the task. However, there are a few remarks that
apply is some particular cases and that reduce the complexity. In the
case of the ($\overline{\mbox{MS}}$) splitting functions, there are
two main simplications: the first is that $\eta=1$, the second is that
there is one single term in the sum over $i$ ($i=0$) and the form of
the function $P$ is:
\begin{equation}
P^{(0)}(z,\eta)\rightarrow \frac1{1-z}\,,
\end{equation}
so that:
\begin{equation}
Q^{(0)}(a,\eta)=\ln(1-a)\,.
\end{equation}
Considering that:
\begin{equation}
w_\alpha^{(k)}(x_\beta) = \delta_{\alpha\beta}\,,
\end{equation}
and that the expressions can be manipulated in such a way that the
coefficient of the plus-prescripted term $S$ is a constant, we have that:
\begin{equation}\label{eq:SplittingFunctions}
\Gamma_{\beta\alpha} = \int_{c}^{d} dz\left\{R\left(z\right)w_{\alpha}^{(k)}\left(\frac{x_\beta}{z}\right)+\frac{S}{1-z}\left[w_{\alpha}^{(k)}\left(\frac{x_\beta}{z}\right)-\delta_{\alpha\beta}\right]\right\}+\displaystyle \left[S\ln\left(1-c\right)+L \right]\delta_{\alpha\beta}\,.
\end{equation}

The same kind of simplifications apply to the case of the Zero-Mass
(ZM) coefficient functions with the only exception that the sum over
$i$ extends to more terms. In particular, since we will use exact
expressions only up to $\mathcal{O}(\alpha_s)$, $i.e.$ NLO, we have
that:
\begin{equation}\label{eq:ZMCoeffFunctions}
\begin{array}{rcl}
  \Gamma_{\beta\alpha} &=&\displaystyle \int_{c}^{d}
                           dz\left\{R\left(z\right)w_{\alpha}^{(k)}\left(\frac{x_\beta}{z}\right)
  + \frac{S^{(0)}+S^{(1)}\ln(1-z)}{1-z}\left[w_{\alpha}^{(k)}\left(\frac{x_\beta}{z}\right)-\delta_{\alpha\beta}\right]\right\}\\
\\
&+&\displaystyle
    \left[S^{(0)}\ln\left(1-c\right)+\frac12
    S^{(1)}\ln^2\left(1-c\right)+L \right]\delta_{\alpha\beta}\,.
\end{array}
\end{equation}

As far as the massive coefficient functions up to
$\mathcal{O}(\alpha_s^2)$ are concerned, things can be more
complicated and we will discuss it later. In the next section we will
consider the case of hadronic observables in the ZM scheme showing how
to pre-compute the integral for the double-differential distributions
in Drell-Yan production.

\subsection{Advantage of a logarithmic grid}

Given the particular structure of the integral $I$ in
Eq.~(\ref{eq:ZMconv}), it turns out to be very convenient to use a
logarithmically distributed grid along with Lagrange interpolating
functions. Let us specifically consider the (massless) integrals:
\begin{equation}\label{eq:integralex}
I_{\beta\alpha} = \int_{x_\beta}^1dy\,O(y)w_{\alpha}\left(\frac{x_\beta}{y}\right)\,.
\end{equation}
If the grid is logarithmically spaced, \textit{i.e.}
$\ln x_{n+1}=\ln x_{n}+\delta x$ with $\delta x$ a positive constant,
and $\{w_{\alpha}(z)\}$ is a set of Lagrange interpolating functions of
degree $\kappa$ in $\ln z$, one has:
\begin{equation}
w_{\alpha}(z) =
\widetilde{w}\left(\ln\frac{z}{x_\alpha}\right)\quad\Rightarrow\quad
w_{\alpha}\left(\frac{x_\beta}{y}\right) =
\widetilde{w}\left(\ln\frac{x_\beta}{x_\alpha}-\ln y\right)
=\widetilde{w}\left((\beta-\alpha)\delta x-\ln y\right)\,.
\end{equation}
Therefore, the integrand of the integral in Eq.~(\ref{eq:integralex})
only depends on the difference $\beta-\alpha$ and not on $\beta$ and
$\alpha$ separately. Since the lower bound is $x_\beta$, this symmetry
seems to broken at the level of the integral.  However, the symmetry
is preserved thanks to the support properties of the interpolating
functions $w_\alpha$ and the fact that relevant functions (PDFs or FFs)
are zero at $x=1$. To see this, we consider the integration limits in
Eq.~(\ref{eq:intlims}) with $\eta=1$. They can be written as:
\begin{equation}\label{eq:intlims1}
  c =
  \mbox{max}(x_\beta,e^{(\beta-\alpha-1)\delta x}) \quad\mbox{and}\quad d =
  \mbox{min}(1,e^{(\beta-\alpha+\kappa)\delta x}) \,.
\end{equation}
While the limit $d$ is manifestly only dependent on the difference
$\beta-\alpha$, the limit $c$ is not. However, $c$ does not have this
symmetry only when $x_\beta$ is selected in place of
$e^{(\beta-\alpha-1)\delta x}$ and this can only happen when:
\begin{equation}\label{eq:ineq}
x_\beta > e^{(\beta-\alpha-1)\delta x}\,.
\end{equation}
Since the last point of the grid is $x_{N_x}=1$, being $N_x$ the
number of grid nodes minus one, one can write:
\begin{equation}
x_\beta = \frac{x_\beta}{x_{N_x}} = e^{(\beta-N_x)\delta x}.
\end{equation}
Finally, the inequality in Eq.~(\ref{eq:ineq}) becomes:
\begin{equation}
  \beta-N_x > \beta-\alpha-1\quad\Leftrightarrow\quad \alpha > N_x-1\quad\Leftrightarrow\quad \alpha = N_x\,.
\end{equation}
Therefore, the integrals $I_{\beta N_x}$ do not respect the
``$\beta-\alpha$'' symmetry.  However, as mentioned above,
$I_{\beta N_x}$ will always multiply a function computed in
$x_{N_x}=1$. If this function is a PDF or a FF, it is identically zero
at $x_{N_x}=1$ and thus the symmetry is effectively preserved. In
addition, $c$ in Eq.~(\ref{eq:intlims1}) is such that if
$\beta>\alpha$ one has $c\geq1$. But being $c$ the lower
integration bound of and since in Eq.~(\ref{eq:integralex}) the upper
bound is 1, one immediately has that $I_{\beta\alpha}=0$ for
$\beta>\alpha$. The consequence of these observations is that
computing the integrals $a_\alpha=I_{0\alpha}$ for
$\alpha=0,\dots,N_x$ is enough to reconstruct the full set of
$I_{\beta\alpha}$ because, in matricial representation, $I$ will look
like this:
\begin{equation}\label{eq:MatrixRep}
\displaystyle I_{\beta\alpha} = 
\begin{pmatrix}
a_0 &  a_1 & a_2 & \cdots & a_{N_x} \\
 0  & a_0 & a_1 & \cdots & a_{N_x-1} \\
 0  & 0   &  a_0 & \cdots & a_{N_x-2} \\
\vdots & \vdots & \vdots & \ddots & \vdots \\
 0  &   0  &   0 & \cdots & a_0 
\end{pmatrix}\,,
\end{equation}
In conclusion, adopting a logarithmically-spaced grid allows one to
compute $N_x+1$ integrals rather than $(N_x+1)(N_x+2)/2$ integrals.

There is another detail that matters in terms of numerical efficiency
of the computation of the integrals $I_{\beta\alpha}$. Given the
support region of the functions $w_\alpha$, the integral in
Eq.~(\ref{eq:integralex}) effectively reads:
\begin{equation}
I_{\beta\alpha} = \int_c^ddy\,O(y)w_{\alpha}\left(\frac{x_\beta}{y}\right)\,,
\end{equation}
with the integration limits given in Eq.~(\ref{eq:intlims}). These
limits can be rearranged as follows:
\begin{equation}\label{eq:intlimsind}
  c=\frac{x_\beta}{x_{{\rm min}[N_x,\alpha+1]}}\quad\mbox{and}\quad
  {d} = \frac{x_\beta}{x_{{\rm max}[\beta,\alpha-\kappa]}}\,,
\end{equation}
which makes it manifest the index range covered by the integration
range. The basic observation is that the functions $w_\alpha$ are
piecewise in correspondence of the grid nodes. This feature makes a
numerical integration over the full range defined in
Eq.~(\ref{eq:intlimsind}) hard to converge due to the cusps at the
grid nodes. However, the functions $w_\alpha$ are smooth between two
consecutive nodes. Therefore, it turns out that it is convenient to
compute the integrals $I_{\beta\alpha}$ by breaking the integration
range as follows:
\begin{equation}\label{eq:finalformula}
I_{\beta\alpha} = \sum_{j={\rm max}[0, \alpha + 1 - N_x]}^{{\rm min}[\kappa, \alpha - \beta]}  \int_{x_\beta/x_{\alpha-j+1}}^{x_\beta/x_{\alpha-j}} dy\,O(y)w_{\alpha}\left(\frac{x_\beta}{y}\right)\,,
\end{equation}
in such a way that the integrand of each single integrand is a smooth
function and thus easier to integrate. Despite the number of
intergrals to be computed increases, this procedure makes the
computation faster and more accurate.

Finally, if the grid is logarithmically distributed, and one defines:
\begin{equation}
s = \exp\left[\delta x\right]\,,
\end{equation}
Eq.~(\ref{eq:finalformula}) can also be written as:
\begin{equation}
I_{\beta\alpha} = \sum_{j={\rm max}[0, \alpha + 1 - N_x]}^{{\rm
    min}[\kappa, \alpha - \beta]}  \int_{s^{\beta -\alpha + j-1}}^{s^{\beta -\alpha + j}} dy\,O(y)w_{\alpha}\left(\frac{x_\beta}{y}\right)\,.
\end{equation}

\subsubsection{GPD-related integrals}

When considering, for example, GPD-related computations such as the
evolution of GPDs, another kind of integral structure comes into play,
that is:
\begin{equation}\label{eq:ZMconvERBL}
J(x) = x\int_0^1\frac{dz}{z} O \left(\frac{x}{z},x\right) d(z)\,,
\end{equation}
which differes from that in Eq.~(\ref{eq:ZMconv}) in two respects: by
the lower integration bound that is zero rather than $x$ and by the
fact that the operator $O$ may also depend explicitly on the external
variable $x$. Under the assumption that this integral is convergent,
we need to adapt the strategy for the computation of numerical
integrals discussed above to Eq.~(\ref{eq:ZMconvERBL}). As a first
step, we make the change of integration variable $z\rightarrow x/y$
and use the interpolation formula in Eq.~(\ref{eq:Interpolation}), to
obtain:
\begin{equation}
  J(x_\beta)=  \sum_{\alpha=0}^{N_x} \overline{d}_\alpha\int_{x_\beta}^\infty dy\,
  O(y,x_\beta)\,w_{\alpha}\left(\frac{x_\beta}{y}\right) =\sum_{\alpha=0}^{N_x} J_{\beta\alpha}\overline{d}_\alpha\,.
\end{equation}
The $\infty$ in the upper bound is a new feature. However, the support
region of the functions $w_{\alpha}$ is the same so that:
\begin{equation}
  J_{\beta\alpha}=  \int_{c}^{\overline{d}} dy\,
  O(y,x_\beta)\,w_{\alpha}\left(\frac{x_\beta}{y}\right)\,,
\end{equation}
with integration bounds:
\begin{equation}\label{eq:intlimsGPD}
  c=\frac{x_\beta}{x_{{\rm min}[N_x,\alpha+1]}}\quad\mbox{and}\quad
  \overline{d} = \frac{x_\beta}{x_{{\rm max}[0,\alpha-\kappa]}}\,.
\end{equation}
Following the procedure discussed in the previous section, the actual
computation is achieved by:
\begin{equation}\label{eq:ERBLints}
\begin{array}{rcl}
J_{\beta\alpha} &=&\displaystyle \sum_{j={\rm max}[0, \alpha + 1 -
                    N_x]}^{{\rm min}[\kappa,\alpha]}
                    \int_{x_\beta/x_{\alpha-j+1}}^{x_\beta/x_{\alpha-j}}
                    dy\,O(y,x_\beta)w_{\alpha}\left(\frac{x_\beta}{y}\right)\\
\\
&=&\displaystyle \sum_{j={\rm max}[0, \alpha + 1 - N_x]}^{{\rm min}[\kappa,\alpha]}  \int_{s^{\beta-\alpha+j-1}}^{s^{\beta-\alpha+j}} dy\,O(y,x_\beta)w_{\alpha}\left(\frac{x_\beta}{y}\right)\,.
\end{array}
\end{equation}
Therefore, at the level of the single integral, the only difference
with respect to the ``standard'' case is a change of the upper
summation bound. It should be clear, however, that this procedure
necessarily implies an approximation. This is due to the fact that the
interpolation grid, understood to be logarithmically distributed, does
not reach zero. This approximation manifests itself in the fact that
the upper bound of the intergrals in Eq.~(\ref{eq:ERBLints}) never
reaches infinity but it gets as large as $s^{\kappa}$. The size of
$s^{\kappa}$ can range widely depending on the lower bound of the grid
$x_0$, on the number of nodes $N_x$, and on the interpolation degree
$\kappa$, according to:
\begin{equation}
s^\kappa = \exp\left[-\frac{\kappa}{N_x}\ln x_0\right]=x_0^{-\frac{\kappa}{N_x}}\,.
\end{equation}
In particular, $s^{\kappa}$ becomes larger as: $x_0$ gets smaller,
$N_x$ gets smaller, and $\kappa$ gets larger. We will need to assess
the impact of this approximation numerically.

If the operator $O$ in Eq.~(\ref{eq:ZMconvERBL}) did not explicitly
depend on $x$, the integral $J_{\beta\alpha}$ would only depend on the
difference $\beta-\alpha$. However, under this assumption that we will
release below, at variance with the standard case,
$J_{\beta\alpha}\neq 0$ also for $\beta>\alpha$. The reason is that
the lower bound $c$ is now allowed to exceed 1. Using
Eq.~(\ref{eq:intlims1}), this implies that $\beta-\alpha$ can also be
larger than zero.  This means in turn that computing the integrals
$b_\alpha=J_{0\alpha}$ and $b_{-\alpha}=J_{\alpha0}$ for
$\alpha=0,\dots,N_x$ is enough to reconstruct the full set of
$J_{\beta\alpha}$. In matricial representation, $J$ takes the form:
\begin{equation}\label{eq:MatrixRepERBL}
\displaystyle J_{\beta\alpha} = 
\begin{pmatrix}
b_0 &  b_1 & b_2 & \cdots & b_{N_x} \\
 b_{-1}  & b_0 & b_1 & \cdots & b_{N_x-1} \\
 b_{-2}  & b_{-1}   &  b_0 & \cdots & b_{N_x-2} \\
\vdots & \vdots & \vdots & \ddots & \vdots \\
 b_{-N_x}  &   b_{-N_x+1}  &   b_{-N_x+2} & \cdots & b_0 
\end{pmatrix}\,,
\end{equation}
Therefore, adopting a logarithmically-spaced grid allows one to
compute $2N_x+1$ integrals rather than $N_x^2$ integrals.

\subsection{Joining grids}

As discussed above, logarithmically distributed grids are particularly
advantageous because the allow one to substantially reduce the number
of integrals to be compute. However, logarithmic grids have two main
drawbacks. First, they cannot get down to $x = 0$ and this is a
potential problem for integrals like those in
Eq.~(\ref{eq:ZMconvERBL}). Second, logarithmic grids that start from a
relatively low value of $x$ tend to be relatively sparse at large
values of $x$. This is also a problem because all integrals that we
are considering are such that the function being interpolated is
integrated up to $x=1$ and the interpolation there can thus
potentially degrade in accuracy.

A possible solution to this problem is to increase the density in a
stepwise fashion as $x$ gets closer to one. This produces locally
logarithmically distributed grids that allow to exploit the symmetry
discussed above while making the grid denser and thus more accurate at
large $x$. The implementation of this procedure in APFEL is achieved
by means of \textit{locked} subgrids. In practice, one starts with a
logarithmic grid with a given lower bound, \textit{e.g.}
$x_{\rm min}^{(0)}=10^{-5}$. Starting from a given node,
$x_{\rm min}^{(1)}$, the density of the grid is increased by some
integer factor. This procedure can be repeated an arbitrary number of
times as one moves towards large $x$, effectively defining logarithmic
subgrids that are increasingly denser and thus guarantee a better
interpolation accuracy.

When dealing with integrals such as that in Eq.~(\ref{eq:integralex}),
the simplest, but possibly not optimal, way to exploit the subgrid
structure is to switch to a denser grid at the level of the
\textit{integral} essentially using one grid when $x_\beta$ is below
the transition node and the other when it is above. The advantage of
this approach is that the integration procedure discussed above
applies verbatim with the only detail that according to the position
of $x_\beta$ one grid is used rather than another. The disadvantage of
this procedure is that integrals with low values of $x_\beta$ do not
take advantage of the denser grids at large $x$. When dealing with
functions like PDFs that vanish rapidly as $x$ tends to one, this is
typically fine because the bulk of the integrals is typically due to
the region at $x\gtrsim x_\beta$. This usually makes possible
interpolation inaccuracies at large $x$ negligible.

The procedure discussed above when applied to integrals of the kind of
Eq.~(\ref{eq:ZMconvERBL}) may lead to severe inaccuracies a large
values of $x_\beta$. This is due to the fact that the integral extends
down to zero and the naive procedure does not make use of the low-$x$
grids effectively truncating the integral to increasingly larger
values of $x$.

In the following we discuss a more convoluted but more effective way
to exploit the subgrids that solves the problem mentioned above,
\textit{i.e.}  the lack of exploitation of the large-$x$ grids when
$x_\beta$ is small and the lack of exploitation of the low-$x$ grids
when $x_\beta$ is large. In this approach the switch from one grid to
the other does not happen, as above, at the level of the integral but
at the level of the \textit{integrand}. 

In this respect, the basic observation is that, thanks to the locking
of the subgrids, the integral for a DGLAP-like convolution as in
Eq.~(\ref{eq:ZMconv}) computed in the point $x_\beta$ of the joint
grid can be decomposed as follows:
\begin{equation}\label{eq:decompDGLAP}
I(x_\beta) = \sum_{g=0}^{N_g=1}\int_{x_\beta/{\rm
    max}[x_\beta,x_{\gamma^{(g+1)}}]}^{x_\beta/{\rm max}[x_\beta,x_{\gamma^{(g)}}]}dy\,O(y)f\left(\frac{x_\beta}{y}\right) = \sum_{g=1}^{N_g}I^{(g)}(x_\beta)\,,
\end{equation}
while for and ERBL-like convolution as that in
Eq.~(\ref{eq:ZMconvERBL}) the decomposition reads:
\begin{equation}\label{eq:decompERBL}
  J(x_\beta) = \sum_{g=0}^{N_g-1}\int_{x_\beta/x_{\gamma^{(g+1)}}}^{x_\beta/x_{\gamma^{(g)}}}dy\,O(y)f\left(\frac{x_\beta}{y}\right)  = \sum_{g=1}^{N_g}J^{(g)}(x_\beta)\,,
\end{equation}
where $\gamma^{(g)}$ is the index corresponding to the lower bound of
the $g$-th grid on the joint grid with the conditions $\gamma^{(0)}=0$
and $\gamma^{(N_g)}=N_x$. It is interesting to observe that if
$x_{\gamma^{(g+1)}}\leq x_\beta$, the integration bounds of the
integral in the $g$-th term in Eq.~(\ref{eq:decompDGLAP}) coincide
giving a vanishing contribution. Therefore, only the subgrids having
$x_{\gamma^{(g+1)}}> x_\beta$ will eventually contribute to the full
integral. Conversely, for Eq.~(\ref{eq:decompERBL}) all grids
contribute to the full convolution integral no matter the position of
$x_\beta$ w.r.t. the subgrid bounds.

Let us first consider the $g$-th contribution to the integral in
Eq.~(\ref{eq:decompERBL}), $J^{(g)}$, and apply the usual
interpolation procedure on the $g$-th subgrid. To do so, we need to
take into account the fact that the $g$-th subgrid starts from the
index $\gamma^{(g)}$ of the joint grid and that in the range
$\gamma^{(g)}\leq \alpha \leq\gamma^{(g+1)}$ joint grid and $g$-th
subgrid coincide up to an index offset given by $\gamma^{(g)}$
(\textit{i.e.} $x_{\alpha} =x_{\alpha-\gamma^{(g)}}^{(g)}$). This
gives:
\begin{equation}\label{eq:decompERBLg}
  J^{(g)}(x_\beta) = \sum_{\alpha=\gamma^{(g)}}^{\gamma^{(g+1)}+\kappa^{(g)}-1}\overline{d}_\alpha \int_{\overline{c}}^{\overline{d}}dy\,O(y)w_{\alpha-\gamma^{(g)}}\left(\frac{x_{\beta}}{y}\right)=\sum_{\alpha=\gamma^{(g)}}^{\gamma^{(g+1)}+\kappa^{(g)}-1}J_{\beta\alpha}^{(g)}\,\overline{d}_\alpha\,,
\end{equation}
where we have exploited the fact that, if
$\alpha\geq \gamma^{(g+1)}+\kappa^{(g)}$, the lower bounds of the sum
above exceeds to upper one giving no contribution to set the upper
limit of the sum.  In addition, the integration bounds are defined as:
\begin{equation}
\begin{array}{l}
  \ln {\overline{c}} = (\beta-\alpha-1+ {\rm
  max}[0,\alpha+1-\gamma^{(g+1)}])\delta x^{(g)}\,,\\
\\
\ln \overline{d} =
  (\beta-\alpha+{\rm min}[\kappa^{(g)},\alpha-\gamma^{(g)}])\delta
  x^{(g)}\,.
\end{array}
\end{equation}
This allows us to write the factors $J_{\beta\alpha}^{(g)}$ in
Eq.~(\ref{eq:decompERBLg}) as follows:
\begin{equation}\label{eq:decompERBLgg}
J_{\beta\alpha}^{(g)}=\sum_{j={\rm max}[0,\alpha+1-\gamma^{(g+1)}]}^{{\rm min}[\kappa^{(g)},\alpha-\gamma^{(g)}]}\int_{s^{\beta-\alpha+j-1}}^{s^{\beta-\alpha+j}}dy\,O(y)w_{\alpha-\gamma^{(g)}}\left(\frac{x_\beta}{y}\right)\,,
\end{equation}
with $s=\exp(\delta x^{(g)})$. Therefore,
Eq.~(\ref{eq:decompERBL}) can finally be written as:
\begin{equation}\label{eq:decompERBL3}
  J(x_\beta) = \sum_{g=0}^{N_g-1}\sum_{\alpha=\gamma^{(g)}}^{\gamma^{(g+1)}+\kappa^{(g)}-1}J_{\beta\alpha}^{(g)}\,\overline{d}_\alpha \,.
\end{equation}
We now introduce the discreet $\theta$-function $\theta_{\rho,\sigma}$
defined such that:
\begin{equation}
\theta_{\rho,\sigma}=\left\{
\begin{array}{rr}
1 & \quad\rho\geq\sigma\\
0 & \quad\rho<\sigma
\end{array}
\right.\,,
\end{equation}
which allows to write:
\begin{equation}
  J(x_\beta) =
  \sum_{\alpha=0}^{N_x+\kappa^{(N_g-1)}-1} \overline{d}_\alpha \left(\sum_{g=0}^{N_g-1}\theta_{\alpha,\gamma^{(g)}}\theta_{\gamma^{(g+1)}+\kappa^{(g)}-1,\alpha}J_{\beta\alpha}^{(g)}\right)\,.
\end{equation}
Assuming that $\overline{d}_\alpha=0$ for $\alpha\geq N_x$, as is often
the case, and defining:
\begin{equation}\label{eq:tensor}
  J_{\beta\alpha} =
\sum_{g=0}^{N_g-1}\theta_{\alpha,\gamma^{(g)}}\theta_{\gamma^{(g+1)}+\kappa^{(g)}-1,\alpha}J_{\beta\alpha}^{(g)}\,,
\end{equation}
one finally obtains:
\begin{equation}
  J(x_\beta) =
  \sum_{\alpha=0}^{N_x-1} J_{\beta\alpha}\overline{d}_\alpha\,.
\end{equation}
In is useful to introduce the helper function $h$ that returns the
grid index given as an argument the index on the joint grid and that
can be represented as:
\begin{equation}
h(\alpha) = \sum_{g'=0}^{N_g-1}\theta_{\alpha,\gamma^{(g')}} - 1\,.
\end{equation}
It is easy to see that:
\begin{equation}
h(\gamma^{(g)}) = g \,.
\end{equation}
Making use on the monotonicity of $h$, Eq.~(\ref{eq:tensor}) can be
reduced to:
\begin{equation}
  J_{\beta\alpha} = \sum_{g={\rm max}[0,h(\alpha)-1]}^{h(\alpha)}J_{\beta\alpha}^{(g)}\,.
\end{equation}

We can now apply the same procedure to the integral $I$ in
Eq.~(\ref{eq:decompDGLAP}). As above, one finds:
\begin{equation}\label{eq:decompDGLAPg}
  I^{(g)}(x_\beta) = \sum_{\alpha=\gamma^{(g)}}^{\gamma^{(g+1)}+\kappa^{(g)}-1}\overline{d}_\alpha \int_c^ddy\,O(y)w_{\alpha-\gamma^{(g)}}\left(\frac{x_{\beta}}{y}\right)=\sum_{\alpha=\gamma^{(g)}}^{\gamma^{(g+1)}+\kappa^{(g)}-1}I_{\beta\alpha}^{(g)}\,\overline{d}_\alpha\,,
\end{equation}
where the logarithm of the integration bounds this time read:
\begin{equation}
\begin{array}{l}
  \ln {c} = (\beta-\alpha-1+ {\rm max}[0,\alpha+1-{\rm
  max}[\beta,\gamma^{(g+1)}]])\delta x^{(g)}\,,\\
\\
\ln d =
  (\beta-\alpha+{\rm min}[\kappa^{(g)},\alpha-{\rm
  max}[\beta,\gamma^{(g)}]])\delta x^{(g)}\,.
\end{array}
\end{equation}
which gives:
\begin{equation}\label{eq:decompERBLgg}
I_{\beta\alpha}^{(g)}=\sum_{j={\rm max}[0,\alpha+1-{\rm
    max}[\beta,\gamma^{(g+1)}]]}^{{\rm min}[\kappa^{(g)},\alpha-{\rm max}[\beta,\gamma^{(g)}]]}\int_{s^{\beta-\alpha+j-1}}^{s^{\beta-\alpha+j}}dy\,O(y)w_{\alpha-\gamma^{(g)}}\left(\frac{x_\beta}{y}\right)\,,
\end{equation}
allowing one to define:
\begin{equation}
  I_{\beta\alpha} = \sum_{g={\rm max}[0,h(\alpha)-1]}^{h(\alpha)}I_{\beta\alpha}^{(g)}\,,
\end{equation}
such that:
\begin{equation}
  I(x_\beta) =
  \sum_{\alpha=0}^{N_x-1} I_{\beta\alpha}\overline{d}_\alpha\,.
\end{equation}




\newpage
\section{Double convolution}

In the case of Drell-Yan or Semi-Inclusive DIS (SIDIS), cross sections
are the result of double convolutions between partonic cross sections
and a pair of non-perturbative distributions. In this case the kind of
expressions we are dealing with has the following form:
\begin{equation}\label{eq:DoubleZMconv}
\begin{array}{rcl}
J(x_\delta,x_\gamma) &=&\displaystyle
x_\delta x_\gamma\int_{x_\delta}^1\frac{dy_1}{y_1}\int_{x_\gamma}^1\frac{dy_2}{y_2}
O(y_1,y_2) d^{(1)}\left(\frac{x_\delta}{y_1}\right)
  d^{(2)}\left(\frac{x_\gamma}{y_2}\right)\\
\\
&=&\displaystyle
\int_{x_\delta}^1dy_1\int_{x_\gamma}^1dy_2\,
O(y_1,y_2) \left[\frac{x_\delta}{y_1}d^{(1)}\left(\frac{x_\delta}{y_1}\right)\right]
  \left[\frac{x_\gamma}{y_2}d^{(2)}\left(\frac{x_\gamma}{y_2}\right)\right]\\
\\
&=&\displaystyle
    \sum_{\alpha=0}^{N_x}\sum_{\beta=0}^{N_x} \overline{d}^{(1)}_\beta
    \overline{d}^{(2)}_\alpha \underbrace{\left[\int_{x_\delta}^1 dy_1 \int_{x_\gamma}^1 dy_2\,
    O(y_1,y_2)\, w_{\beta}^{(k)}\left(\frac{x_\delta}{y_1}\right) 
    w_{\alpha}^{(k)}\left(\frac{x_\gamma}{y_2}\right)\right]}_{\Theta^{\beta\delta,\alpha\gamma}}\,.
\end{array}
\end{equation}
In Eq.~(\ref{eq:DoubleZMconv}) we assume that there are no mass
corrections and thus the convolutions take the simplest form. In the
case of double convolutions, the partonic cross sections $O$ are functions
of two variables, $y_1$ and $y_2$, and, as in the case of the single
convolutions, they get three kinds of contributions in both these
variables: local terms proportional to $\delta$-functions, singular
terms proportional to plus-prescripted functions, and regular
terms. The complication here is that these contributions from the two
variables $y_1$ and $y_2$ mix and thus, for example, terms local in
$y_1$ and singular in $y_2$ appear. It is thus necessary to identify
the general structure of the function $O$ to see whether it is
possible to decompose the double operator $\Theta^{\beta\delta,\alpha\gamma}$
into products of single operators of the kind $\Gamma^{\beta\delta}$
and $\Gamma^{\alpha\gamma}$. 

In the case of the NLO corrections to SIDIS, the general structure of
the function $O$ can be inferred looking at Eqs.~(C.2)-(C.7) of
Ref.~\cite{deFlorian:1997zj}:
\begin{equation}\label{eq:DoubleFuncStruct}
\begin{array}{rclclcl}
  O(y_1,y_2) &=&\displaystyle {\rm LL}\,\delta(1-y_1)\delta(1-y_2) &+&\displaystyle {\rm LS}\,
                                                                       \delta(1-y_1)\left[\frac{\ln(1-y_2)}{1-y_2}\right]_+ &+&\displaystyle \delta(1-y_1)\,{\rm LR}(y_2) \\
  \\
             &+&\displaystyle {\rm SL}\,\left[\frac{\ln(1-y_1)}{1-y_1}\right]_+\delta(1-y_2)
                                                                   &+&\displaystyle {\rm SS}\,\left[\frac1{1-y_1}\right]_+
                                                                       \left[\frac1{1-y_2}\right]_+ &+&\displaystyle \left[\frac1{1-y_1}\right]_+\,{\rm
                                                                                                        SR}(y_2)\\
  \\
             &+&\displaystyle {\rm RL}(y_1)\,\delta(1-y_2)   &+&\displaystyle
                                                                 {\rm RS}(y_1)
                                                                 \,\left[\frac1{1-y_2}\right]_+&+&\displaystyle
                                                                                                   \sum_iK_i{\rm
                                                                                                   R}_i^{(1)}(y_1){\rm
                                                                                                   R}_i^{(2)}(y_2)\,.
\end{array}
\end{equation}
It is clear that in Eq.~(\ref{eq:DoubleFuncStruct}) all terms
factorise into a part that only depends on $y_1$ and a part that only
depends on $y_2$. This is crucial to use the same technology developed
above for the single convolutions. Plugging
Eq.~(\ref{eq:DoubleFuncStruct}) into Eq.~(\ref{eq:DoubleZMconv}), one
finds that:
\begin{equation}\label{eq:MasterFormula}
\begin{array}{rclclcl}
  \Theta^{\beta\delta,\alpha\gamma} &=& {\rm LL} \,
                                        \Gamma^{\rm L}_{\beta\delta}\Gamma^{\rm L}_{\alpha\gamma}
  &+& \displaystyle {\rm LS}\,\Gamma^{\rm L}_{\beta\delta}\Gamma_{\alpha\gamma}^{\rm S1}
  &+&\displaystyle \Gamma^{\rm L}_{\beta\delta} \Gamma_{\alpha\gamma}^{\rm LR}\\
  \\
                                    &+&{\rm SL}\,\displaystyle\Gamma_{\beta\delta}^{\rm S1}\Gamma^{\rm L}_{\alpha\gamma} 
  &+&
     {\rm SS}\,\Gamma_{\beta\delta}^{\rm
      S0}\Gamma_{\alpha\gamma}^{\rm S0}&+& \Gamma_{\beta\delta}^{\rm S0}\Gamma_{\alpha\gamma}^{\rm SR}\\
\\
&+& \displaystyle \Gamma_{\beta\delta}^{\rm RL}\Gamma^{\rm L}_{\alpha\gamma} &+& \Gamma_{\beta\delta}^{\rm RS}\Gamma_{\alpha\gamma}^{\rm
      S0} &+& \displaystyle \sum_i K_i \Gamma_{\beta\delta}^{R_i^{(1)}}\Gamma_{\alpha\gamma}^{R_i^{(2)}}
\end{array}
\end{equation}
with:
\begin{equation}\label{eq:PrecompOp}
\begin{array}{rcl}
  \Gamma_{\alpha\beta}^{\rm L} &=& \displaystyle  \int_{c_{\alpha\beta}}^{d_{\alpha\beta}}
                                       dz\,\delta(1-z)w_{\beta}^{(k)}\left(\frac{x_\alpha}{z}\right)
                                   =
                                   w_{\beta}^{(k)}\left(x_\alpha\right)
                                   =  \delta_{\alpha\beta}\\
\\
  \Gamma_{\alpha\beta}^{{\rm S}n} &=& \displaystyle  \int_{c_{\alpha\beta}}^{d_{\alpha\beta}}
                                       dz\frac{\ln^n(1-z)}{1-z}\left[w_{\beta}^{(k)}\left(\frac{x_\alpha}{z}\right)-\delta_{\alpha\beta}\right]+\frac1{(n+1)!}\ln^{n+1}\left(1-c_{\alpha\beta}\right)\delta_{\alpha\beta}\\
  \\
  \Gamma_{\alpha\beta}^{f} &=&\displaystyle \int_{c_{\alpha\beta}}^{d_{\alpha\beta}}
                                     dz\,f(z)\,w_{\beta}^{(k)}\left(\frac{x_\alpha}{z}\right)
\end{array}
\end{equation}
where $f$ is a regular function and the integration bounds are defined
as:
\begin{equation}
  c_{\alpha\beta} =
  \mbox{max}(x_\alpha,x_\alpha/x_{\beta+1}) \quad\mbox{and}\quad d_{\alpha\beta} =
  \mbox{min}(1,x_\alpha/x_{\beta-k}) \,.
\end{equation}
In general terms, it should always be possible to write an object of
the kind of $\Theta^{\beta\delta,\alpha\gamma}$ as series of bilinear
terms:
\begin{equation}\label{eq:OpSeries}
\Theta^{\beta\delta,\alpha\gamma} = \sum_j C_j\Gamma_j^{(1),\beta\delta}\Gamma_j^{(2),\alpha\gamma}
\end{equation}
where $C_j$ is a scalar, and $\Gamma_j^{(1),\beta\delta}$ and
$\Gamma_j^{(2),\alpha\gamma}$ are single operators that can be
computed using the technology discussed in the previous
section. Plugging Eq.~(\ref{eq:OpSeries}) into
Eq.~(\ref{eq:DoubleZMconv}), one finds that:
\begin{equation}\label{eq:DoubleZMconv2}
  J(x_\delta,x_\gamma) =
  \sum_j C_j  f_j^{(1),\delta} f_j^{(2),\gamma}\,,
\end{equation}
where we have defined:
\begin{equation}
f_j^{(1),\delta}\equiv \sum_{\beta=0}^{N_x} \overline{d}^{(1)}_\beta\Gamma_j^{(1),\beta\delta}\quad\mbox{and}\quad f_j^{(2),\gamma}\equiv \sum_{\alpha=0}^{N_x}\overline{d}^{(2)}_\alpha \Gamma_j^{(2),\alpha\gamma}\,.
\end{equation}

Eq.~(\ref{eq:DoubleZMconv2}) shows that, in the hypothesis that the
double function $O(y_1,y_2)$ can be expressed in terms of a series of
terms where the dependence on $y_1$ and $y_2$ factorizes\footnote{This
  is the case for SIDIS and DY up to NLO but we expect this feature to
  hold also beyond, despite complications due to the more involved
  plus-prescripted functions are also expected.}, the double
convolution in Eq.~(\ref{eq:DoubleZMconv}) can be expressed as a
series of bilinear terms of distributions ($f_j^{(1),\delta}$ and
$f_j^{(2),\gamma}$) singularly obtained as convolutions of a limited
number of single operators with distributions. This is a particularly
useful achievement that allows us to compute double convolutions
without the need of extending the integration and the interpolation
procedures with an obvious gain in accuracy and performance. As a
matter of fact, the same argument can be extended to a multiple
convolution of the function $O(\{y_i\})$, which again can be expressed
as a series of $n$-linear terms, with $i=1,\dots,n$, with $n$
distributions:
\begin{equation}
J(\{x_{\alpha_i}\}) = \sum_j C_j  \prod_{i=1}^nf_j^{(i),\alpha_i}\,,
\end{equation}
with:
\begin{equation}
f_j^{(i),\alpha_i} \equiv \sum_{\beta=0}^{N_x} \overline{d}^{(i)}_\beta\Gamma_j^{(i),\beta\alpha_i}\,.
\end{equation}
This technology could be useful for more complicated observables, like
cross sections in $pp$ collisions with an identified hadron in the
final state, that requires for example three convolutions.

The challenging part of the technology just presented resides in the
``pre-processing'' of the function $O(y_1,y_2)$, that is the
analytical work required to disentangle the different terms. This step
however has to be taken only once.

Before employing this procedure for any concrete application, it is
appropriate to connect Eq.~(\ref{eq:MasterFormula}) to
Eq.~(\ref{eq:OpSeries}) by identifying number and form of the
coefficients and operators involved. Specifically, assuming that the
series in the last term in the r.h.s. of Eq.~(\ref{eq:MasterFormula})
has $r$ terms, the series in Eq.~(\ref{eq:OpSeries}) will have $8+r$
terms, that is:
\begin{equation}\label{eq:OpSeriesCon}
\Theta^{\beta\delta,\alpha\gamma} = \sum_{j=1}^{8+r} C_j\Gamma_j^{(1),\beta\delta}\Gamma_j^{(2),\alpha\gamma}
\end{equation}
with:
\begin{equation}
\begin{array}{lclll}
j = 1 &:& C_1 = {\rm LL}, & \Gamma_1^{(1),\beta\delta} = \Gamma^{\rm L}_{\beta\delta}, & \Gamma_1^{(2),\alpha\gamma} = \Gamma^{\rm L}_{\alpha\gamma},\\
j = 2 &:& C_2 = {\rm LS}, & \Gamma_2^{(1),\beta\delta} = \Gamma^{\rm L}_{\beta\delta}, & \Gamma_2^{(2),\alpha\gamma} = \Gamma_{\alpha\gamma}^{\rm S1},\\
j = 3 &:& C_3 = 1, & \Gamma_3^{(1),\beta\delta} = \Gamma^{\rm L}_{\beta\delta}, & \Gamma_3^{(2),\alpha\gamma} = \Gamma_{\alpha\gamma}^{\rm LR},\\
j = 4 &:& C_4 = {\rm SL}, & \Gamma_4^{(1),\beta\delta} = \Gamma_{\beta\delta}^{\rm S1}, & \Gamma_4^{(2),\alpha\gamma} = \Gamma^{\rm L}_{\alpha\gamma},\\
j = 5 &:& C_5 = {\rm SS},& \Gamma_5^{(1),\beta\delta} = \Gamma_{\beta\delta}^{\rm S0}, & \Gamma_5^{(2),\alpha\gamma} = \Gamma_{\alpha\gamma}^{\rm S0},\\
j = 6 &:& C_6 = 1, & \Gamma_6^{(1),\beta\delta} = \Gamma_{\beta\delta}^{\rm S0}, & \Gamma_6^{(2),\alpha\gamma} = \Gamma_{\alpha\gamma}^{\rm SR},\\
j = 7 &:& C_7 = 1, & \Gamma_7^{(1),\beta\delta} = \Gamma_{\beta\delta}^{\rm RL}, & \Gamma_7^{(2),\alpha\gamma} = \Gamma^{\rm L}_{\alpha\gamma},\\
j = 8 &:& C_8 = 1, & \Gamma_8^{(1),\beta\delta} = \Gamma_{\beta\delta}^{\rm RS}, & \Gamma_8^{(2),\alpha\gamma} = \Gamma_{\alpha\gamma}^{\rm S0},\\
j = 9 &:& C_9 = K_1 & \Gamma_9^{(1),\beta\delta} = \Gamma_{\beta\delta}^{R_1^{(1)}}, & \Gamma_9^{(2),\alpha\gamma} = \Gamma_{\alpha\gamma}^{R_1^{(2)}},\\
\vdots & & & & \\
j = 8 + r &:& C_{8+r} = K_r & \Gamma_{8+r}^{(1),\beta\delta} = \Gamma_{\beta\delta}^{R_r^{(1)}}, & \Gamma_{8+r}^{(2),\alpha\gamma} = \Gamma_{\alpha\gamma}^{R_r^{(2)}}
\end{array}
\end{equation}
It should be noted that, despite the large number of terms in the
series in Eq.~(\ref{eq:OpSeriesCon}), the number of operators to be
precomputed it pretty limited. In addition, in many cases many of the
terms of the series are zero so the number of contributions is greatly
reduced.

We can now apply this procedure up to NLO in QCD to two specific
cases: SIDIS first and DY second.

\subsection{Semi-inclusive deep inelastic scattering (SIDIS)}

the structure of the SIDIS observables and the expressions for the
respective hard cross sections can be found in
Ref.~\cite{deFlorian:1997zj}. Following this paper, the SIDIS
differential cross section for the exchange of a virtual photon can be
written as:
\begin{equation}\label{eq:sidis}
  \frac{d^3\sigma}{dx\, dy\, dz} = 
  \frac{2\, \pi\alpha^2}{xyQ^2} 
  \left[ (1+(1-y)^2) 2xF_1(x,z,Q^2) + 
    2 (1-y) F_L(x,z,Q^2) \right]\,,
\end{equation}
with $Q^2 = - q^2$ the (negative) virtuality of the exchanged photon,
$x$ and $z$ the momentum fractions of PDFs and FFs, and $Q^2 = xys$
the definition of the inelasticity $y$ in terms of the squared
collision energy in the center of mass $s$. Notice that, as compared
to Ref.~\cite{deFlorian:1997zj}, we have absorbed into the definition
of $F_L$ a factor $x$ as customary in the definition of the
longitudinal structure function in inclusive DIS.

We now use the Callan-Gross relation:
\begin{equation}\label{eq:CallanGross}
F_2 = 2xF_1 + F_L
\end{equation}
to replace $2xF_1$ with $F_2$ in Eq.~(\ref{eq:sidis}):
\begin{equation}\label{eq:sidis2}
  \frac{d^3\sigma}{dx\, dy\, dz} = 
  \frac{2\, \pi\alpha^2}{xyQ^2} 
  \left[ Y_+ F_2(x,z,Q^2)
    -y^2 F_L(x,z,Q^2) \right]\,,
\end{equation}
where we have defined:
\begin{equation}
Y_+ = 1+(1-y)^2\,.
\end{equation}
It is also useful to write Eq.~(\ref{eq:sidis2}) as differential in
$x$, $Q^2$, and $z$:
\begin{equation}\label{eq:sidis3}
  \frac{d^3\sigma}{dx\, dQ^2\, dz} = 
  \frac{2\, \pi\alpha^2}{xQ^4} 
  \left[ Y_+ F_2(x,z,Q)
    -y^2 F_L(x,z,Q) \right]\,.
\end{equation}
The structure functions $F_2$ and $F_L$ are given at NLO by:
\begin{equation}\label{eq:f1sidis}
\begin{array}{rcl}
\displaystyle  F_{2,L}(x,z,Q) &=& \displaystyle x\sum_{q,\overline{q}} e_q^2 \bigg[ q(x,Q)
    \otimes  C^{2,L}_{qq}(x,z) \otimes D_q(z,Q)  \\
&+&\displaystyle q(x,Q)  \otimes  C^{2,L}_{gq}(x,z) \otimes D_g(z,Q)+  g(x,Q)  \otimes
  C^{2,L}_{qg}(x,z) \otimes D_q(z,Q) \bigg]\,,
\end{array}
\end {equation}
where $\{q,g\}$ are the quark and gluon PDFs and $\{D_q,D_g\}$ are the
quark and gluon FFs, $e_q$ is the electric charge of the quark $q$ and
$\{C^{2,L}_{qq},C^{2,L}_{qg},C^{2,L}_{gq}\}$ are the relevant partonic
cross sections. The partonic cross sections allow for a perturbative
expansion in power of $\alpha_s$:
\begin{equation}
C = \sum_{n=0} \left(\frac{\alpha_s}{4\pi}\right)^nC^{(n)}
\end{equation}
that we truncate to NLO, $i.e.$ to $n=1$. At LO ($n=0$) we have the simple
expression:
\begin{equation}
C^{2,(0)}_{qq}(x,z) = \delta(1-x)\delta(1-z)\quad\mbox{and all others zero.}
\end{equation}

At NLO ($n=1$) we take the expressions from Appendix C of
Ref.~\cite{deFlorian:1997zj} being careful to take into account an
additional factor two due to the difference in the expansion parameter
($\alpha_s/4\pi$ rather than $\alpha_s/2\pi$) and to combine the
expressions for $F_1$ and $F_L$ using Eq.~(\ref{eq:CallanGross}) to
obtain the partonic cross sections for $F_2$. We start with the
partonic cross sections for $F_L$ that read:
\begin{equation}\label{eq:cfFL}
\begin{array}{rcl}
C_{qq}^{L,(1)} &=& 8 C_F x z\,, \\
C_{gq}^{L,(1)} &=& 8 C_F x (1-z)\,, \\
C_{qg}^{L,(1)} &=& 8 x(1-x)\,, 
\end{array}
\end{equation}
while those for $F_2$ read:
\begin{equation}\label{eq:cfF2}
\begin{array}{rcl}
  \displaystyle \frac{C_{qq}^{2,(1)}}{2C_F} &=& \displaystyle -8\delta(1-x)\delta(1-z)+2\delta(1-x)
                                  \left(\frac{\ln
                                  (1-z)}{1-z}\right)_++ 
                                  \delta(1-x) \left[\frac{1+z^2}{1-z}\ln z+(1-z)-(1+z)\ln(1-z)\right] \\
  \\
                              &+&\displaystyle 
                                  2\left(\frac{\ln
                                  (1-x)}{1-x}\right)_+\delta(1-z)+2\left(\frac{1}{1-x}\right)_+\left(\frac{1}{1-z}\right)_+- \left(\frac{1}{1-x}\right)_+(1+z)\\
  \\
                              &+& \displaystyle \left[ 
                                  -\frac{1+x^2}{1-x}\ln x+(1-x)
                                  -(1+x)\ln(1-x)\right]\delta(1-z) -
                                  (1+x)\left(\frac{1}{1-z}\right)_++(2+6xz)\\
  \\
  \displaystyle \frac{C_{gq}^{2,(1)}}{2C_F} &=& \displaystyle \delta (1-x) \left[\frac{1+(1-z)^2}{z} \ln\left[
                                               z(1-z)\right]+z\right]\\
  \\
                                            &+&\displaystyle \left(\frac{1}{1-x}\right)_+\frac{1+(1-z)^2}{z}\\
  \\
                              &+&  \displaystyle 2(1+3x)-6xz-(1+x)\frac{1}{z}\\
  \\
  C_{qg}^{2,(1)} &=& \displaystyle \left[(x^2+(1-x)^2)
                     \ln\left(\frac{1-x}{x}\right)
                     +2x(1-x)\right]\delta (1-z) +(x^2+(1-x)^2)
                     \left(\frac{1}{1-z}\right)_+\\
\\
&+&\displaystyle 2(-1+6x-6x^2) + (x^2+(1-x)^2) \frac{1}{z} 
\end{array}
\end{equation}
By inspection of Eqs.~(\ref{eq:cfFL}) and~(\ref{eq:cfF2}) we can
deduce the various coefficients of Eq.~(\ref{eq:DoubleFuncStruct}). In
particular, as usual, $F_L$ involves only regular functions so that
all contributions are zero but the fully regular ones:
\begin{equation}
\begin{array}{rcl}
C_{qq}^{L,(1)}(x,z) &:& K_1 = 8 C_F\,,\quad R_1^{(1)}(x) = x\,,\quad
                        R_1^{(2)}(z) = z\,,\\
C_{gq}^{L,(1)}(x,z) &:& K_1 = 8 C_F\,,\quad R_1^{(1)}(x) = x\,,\quad
                        R_1^{(2)}(z) = 1-z\,,\\
C_{qg}^{L,(1)}(x,z) &:& K_1 = 8\,,\quad R_1^{(1)}(x) = x(1-x)\,,\quad
                        R_1^{(2)}(z) = 1\,.\\
\end{array}
\end{equation}

For $F_2$ the situation is more complicated but we can still identify
the different contributions:
\begin{equation}
\begin{array}{rcl}
C_{qq}^{2,(1)} &:& \displaystyle {\rm LL} = -16 C_F\,,\quad {\rm LS} =
                        4C_F\,,\quad {\rm LR}(z) = 2C_F
                        \left[\frac{1+z^2}{1-z}\ln
                        z+(1-z)-(1+z)\ln(1-z)\right]\\
\\
&& \displaystyle {\rm SL} = 4C_F\,,\quad {\rm SS} = 4C_F\,,\quad {\rm
   SR}(z) = -2C_F(1+z)\\
\\
&& \displaystyle {\rm RL}(x) = 2C_F \left[ 
                                  -\frac{1+x^2}{1-x}\ln x+(1-x)
                                  -(1+x)\ln(1-x)\right]\,,\quad {\rm
   RS}(x) = -2C_F(1+x)\,,\\
\\
&& \displaystyle \left\{K_1 =  4C_F,\, R_1^{(1)}(x) = 1,\,
   R_1^{(2)}(z) = 1\right\}\,,\\
\\
&& \displaystyle \left\{K_2 =  12C_F,\, R_2^{(1)}(x) = x,\,
   R_2^{(2)}(z) = z\right\}\,,\\
\\
C_{gq}^{2,(1)} &:& \displaystyle {\rm LR}(z) = 2C_F\left[\frac{1+(1-z)^2}{z} \ln\left[
                                               z(1-z)\right]+z\right]\,,\\
  \\
&& \displaystyle {\rm SR}(z) = 2C_F\left[\frac{1+(1-z)^2}{z}\right]\,,\\
\\
&& \displaystyle \left\{K_1 =  4C_F,\, R_1^{(1)}(x) = 1 + 3x,\,
   R_1^{(2)}(z) = 1\right\}\,,\\
\\
&& \displaystyle \left\{K_2 =  -12C_F,\, R_2^{(1)}(x) = x,\,
   R_2^{(2)}(z) = z\right\}\,,\\
\\
&& \displaystyle \left\{K_3 =  -2C_F,\, R_3^{(1)}(x) = 1+x,\,
   R_3^{(2)}(z) = \frac1{z}\right\}\,,\\
\\
C_{qg}^{2,(1)} &:&\displaystyle {\rm RL}(x) = \left[x^2+(1-x)^2\right]
                     \ln\left(\frac{1-x}{x}\right)
                     +2x(1-x)\,,\quad  {\rm RS}(x) = x^2+(1-x)^2\,,\\
\\
&& \displaystyle \left\{K_1 = 2,\, R_1^{(1)}(x) = - 1 + 6x-6x^2,\,
   R_1^{(2)}(z) = 1\right\}\,,\\
\\
&& \displaystyle \left\{K_2 =  1,\, R_2^{(1)}(x) = x^2+(1-x)^2,\,
   R_2^{(2)}(z) = \frac1{z}\right\}\,.
\end{array}
\end{equation}
Analogously, for the only LO partonic cross sections we find that:
\begin{equation}
\begin{array}{rcl}
C_{qq}^{2,(0)} &:& \displaystyle {\rm LL} =1 \,.
\end{array}
\end{equation}
All the coefficients that are not mentioned are equal to zero. We can
now implement explicitly Eq.~(\ref{eq:MasterFormula}). The one thing
that is left to sort out is the structure of the structure functions
in terms of the appropriate PDF and FF combinations. Considering
Eq.~(\ref{eq:f1sidis}), we observe that none of the coefficient
functions depends on the particular quark flavour (this is a feature
of the ZM scheme). Therefore, simplifying the notation, we can rewrite
Eq.~(\ref{eq:f1sidis}) as:
\begin{equation}\label{eq:structF2L}
\begin{array}{rcl}
  \displaystyle  F &=& \displaystyle C_{qq}  \sum_{q} e_q^2 \left[q  D_q +\overline{q}  D_{\overline{q}}\right]  + C_{gq}  \sum_{q} e_q^2
                       \left[q+\overline{q}\right]\,D_g  +  C_{qg}  g \sum_{q} e_q^2
                       \left [D_q +D_{\overline{q}}\right]\,,
\end{array}
\end {equation}
where now the sums run only over the quark flavours and not the
antiflavours.

\subsection{Drell Yan (DY)}

In this section we apply to the Drell-Yan (DY) process the same
procedure followed above for SIDIS. As a matter of fact, SIDIS and DY
are strictly connected in that DY can be regarded as the time-like
counterpart of SIDIS. As a consequence, the structure of the relevant
observables as well as the form of the expressions involved are very
similar. Therefore, the application of the method described above is
straightforward. 

% The SIDIS kinematics can be translated into the DY
% kinematics by identifying $\{x,z\}\rightarrow\{x_1,x_2\}$, where $x_1$
% and $x_1$ are the momentum fractions of the incoming partons and
% $Q^2\rightarrow M^2$, with $M^2$ the invariant mass of the partonic
% initial state which is thus positive 

% The inelasticity $y$ in DIS measures the energy 



\newpage

\begin{thebibliography}{alp}

%\cite{deFlorian:1997zj}
\bibitem{deFlorian:1997zj}
  D.~de Florian, M.~Stratmann and W.~Vogelsang,
  %``QCD analysis of unpolarized and polarized Lambda baryon production in leading and next-to-leading order,''
  Phys.\ Rev.\ D {\bf 57} (1998) 5811
  doi:10.1103/PhysRevD.57.5811
  [hep-ph/9711387].
  %%CITATION = doi:10.1103/PhysRevD.57.5811;%%
  %138 citations counted in INSPIRE as of 04 Jun 2017

\end{thebibliography}




\end{document}
